\subsection{Formal Definition of Riemann Surfaces}
Let \(X=(X,\tau)\) be a \textit{connected} topological space, and let \(x\in X\) be a fixed point. 

A surface is a Hausdorff space that is locally homeomorphic to \(\mathbb{R}^2\). In other words, for any point \(p\) on the surface, there exists a neighborhood \(U\) of \(p\) such that \(U\) is homeomorphic to an open subset of \(\mathbb{R}^2\).

Let \(\cbraces{U_\alpha}_{\alpha\in I}\) be an collection of open subsets of a surface \(X\) that covers \(X\). For each \(\alpha\), let \(\varphi_\alpha:U_\alpha\to\mathbb{C}\) be a homeomorphism between \(U_\alpha\) and an open subset of \(\mathbb{C}\cong\mathbb{R}^2\). The pair \(\qty(U_\alpha,\varphi_\alpha)\) is a \textit{coordinate chart} of \(X\). The collection \(\cbraces{\qty(U_\alpha, \varphi_\alpha)}_{\alpha\in I}\) is a \textit{coordinate atlas} of \(X\). Let \(U_\alpha\) and \(U_\beta\) be two arbitrary sets in the covering. If all \textit{transition maps} \(\varphi_\beta\circ\varphi_\alpha^{-1}:\varphi_\alpha\qty(U_\alpha\cap U_\beta)\to\varphi_\beta\qty(U_\alpha\cap U_\beta)\) are biholomorphic on their respective \(f_\alpha\qty(U_\alpha\cap U_\beta)\) (we don't need to explicitly consider \(\varphi_\alpha\circ\varphi_\beta^{-1}\) as it is the inverse of the biholomorphism, which is biholomorphic by definition), then the atlas is said to be \textit{holomorphic} (intuitively, transition maps convert Euclidean coordinates between two charts on an overlapping region). Such an \(X\) is formally known as a \textit{Riemann surface}.

In short, a Riemann surface is a locally Euclidean connected Hausdorff space with complex structure.

Riemann surfaces are the definition of one-dimensional complex manifolds. Complex manifolds of higher dimension can be defined similarly.

We will now give examples of Riemann surfaces (we will distinguish between \(\extcomplex\) and \(S^2\)).
\begin{example}
    The extended complex plane \(\extcomplex\) is a Riemann surface.
\end{example}
\begin{proof}
    The extended complex plane can be covered by two charts: \(\qty(\mathbb{C},\varphi_1)\) and \(\qty(\extcomplex\setminus\cbraces{0},\varphi_2)\), where \(\varphi_1(z)=z\) and \(\varphi_2(z)=\frac{1}{z}\). The transition map is given by \(\varphi_2\circ\varphi_1^{-1}(z)=\frac{1}{z}\), which is biholomorphic on \(\varphi_1\qty(\mathbb{C}\cap\extcomplex^*)=\mathbb{C}^*\). Hence, \(\extcomplex\) is a Riemann surface.
\end{proof}
\begin{example}
    The Riemann sphere \(S^2=\cbraces{\qty(x_1,x_2,x_3)\in\mathbb{R}^3}{x_1^2+x_2^2+x_3^2=1}\) is a Riemann surface.
\end{example}
\begin{proof}
    Consider the open sets \(U_N=S^2\setminus (0,0,1)\) and \(U_S=S^2\setminus (0,0,-1)\) that cover \(S^2\). Define their stereographic projection homeomorphisms \(\varphi_N:U_N\to\mathbb{C}\) and \(\varphi_S:U_S\to\mathbb{C}\) with centers \((0,0,1)\) and \((0,0,-1)\), respectively. By taking the conjugate of \(\varphi_S\) (this is necessary to ensure that the transition map is not anti-holomorphic), we form an atlas \(\cbraces{\qty(U_N,\varphi_N),\qty(U_S,\overline{\varphi_S})}\) with two charts. More explicitly, we can assume \[\varphi_N(x_1,x_2,x_3)=\frac{x_1+\ii x_2}{1-x_3}\qquad\overline{\varphi_S}(x_1,x_2,x_3)=\frac{x_1-\ii x_2}{1+x_3}.\]
    On \(\varphi_N\qty(S^2\setminus\qty((0,0,1)\cup(0,0,-1)))=\extcomplex\setminus\qty(\cbraces{0}\cup\cbraces{\infty})=\mathbb{C}^*\), the transition map \[\overline{\varphi_S}\circ\varphi_N^{-1}(z)=\frac{\Re(z)\qty(1-x_3)-\ii\Im(z)\qty(1-x_3)}{1+x_3}=\overline{z}\frac{1-x_3}{1+x^3}=\frac{\overline{z}}{\abs{z}^2}=\frac{1}{z}\] is biholomorphic. Hence, by the holomorphy of the atlas, \(S^2\) is a Riemann surface.
\end{proof}
TO BE CONTINUED
