\subsection{The Sheaf of Germs of Analytic Functions}
To facilitate future discussions, we introduce the concept of equivalence relations and classes:
\begin{definition}
    Let \(X\) be a set. A relation \(\sim\) on \(X\) an \textit{equivalence relation} if it satisfies the following properties for all \(x,y,z\in X\):
    \begin{enumerate}
        \item \(x\sim x\) (reflexivity).
        \item \(x\sim y\) implies \(y\sim x\) (symmetry).
        \item \(x\sim y\) and \(y\sim z\) together imply \(x\sim z\) (transitivity).
    \end{enumerate}
\end{definition}
\begin{definition}
    Let \(\sim\) be an equivalence relation on a set \(X\) and let \(x\in X\). The \textit{equivalence class} of \(x\) is the subset
    \[[x]=\cbraces{y\in X}{y\sim x}.\] Sometimes, the relation is made explicit and the equivalence class is then denoted \([x]_\sim\).
\end{definition}
If \(\qty(f,D)\) is an analytic function element, then for some \(a\in D\), define an equivalence relation \(\sim_a\) by \[(g,B)\sim (h,C)\equiv g\equiv h\qq{on}B\cap C\] given that \(a\in B\cap C\). Then each \(a\) defines an equivalence class, informally denoted by \([f]_a\), containing function elements \(\qty(g,B)\) such that \(g\equiv f\) on \(B\cap D\), known as a \textit{germ}.
% todo: do we make conway's distinction between riemann surface and analytic manifolds.
\begin{definition}
    Let \(U\subseteq\mathbb{C}\) be an open region. Then the \textit{sheaf of germs of analytic functions} over \(U\) is defined by \[\mathscr{J}=\cbraces{\qty(z,[f]_z)}{z\in U,f\text{ analytic at }z}.\] The \textit{projection map} \(\rho:\mathscr{J}\to U\) is then defined by \(\rho\qty(z,[f]_z)=z\). The sheaf is comprised by stalks at each point \(z\), given by \(\rho^{-1}\qty(\cbraces{z})\) or simply \[\rho^{-1}\qty(\cbraces{z})=\cbraces{\qty(z,[f]_z)}{f\text{ analytic at }z}.\]
\end{definition}
