\subsubsection{The Wiener--Ikehara Theorem}
Previously we have seen the subtle relation between \(\zeta\) and \(\psi\).
Given the properties of the Laplace transform of \(\psi\circ\exp\) and certain
conditions established above, we apply now prove what is known as a
\textit{Tauberian} theorem to extract sufficient information (such as
convergence) of the function itself.

For arbitrary \(x\in\mathbb{R}\) and \(\lambda>0\), define a triangular kernel
cutoff by \[K_\lambda(x)=
    \begin{dcases}
        1-\frac{\abs{x}}{2\lambda} & \abs{x}<2\lambda, \\
        0                          & \text{otherwise},
\end{dcases}\qquad\supp K_\lambda=\qty[-2\lambda,2\lambda],\] and a normalized Fejér kernel by
\[k_\lambda(x)=
    \begin{dcases}
        \frac{2\lambda}{\sqrt{2\uppi}}\qty(\frac{\sin(\lambda x)}{\lambda x})^2 & x\neq 0, \\
        \frac{2\lambda}{\sqrt{2\uppi}}                                          & x=0.
\end{dcases}\]
The two kernels are related by a angular unitary Fourier transform:
\begin{lemma}\label{lem:wienerikehara_kernels_fouriertransform}
    For every \(\lambda>0\), we have that \[\mathcal{F}\cbraces{K_\lambda}(x)=\mathcal{F}^{-1}\cbraces{K_\lambda}(x)=k_\lambda(x),\qquad\mathcal{F}\cbraces{k_\lambda}(x)=\mathcal{F}^{-1}\cbraces{k_\lambda}(x)=K_\lambda(x),\] where we define \[\mathcal{F}\cbraces{f}(x)=\frac1{\sqrt{2\uppi}}\int_{-\infty}^\infty f(t)\ee^{-\ii xt}\ddt,\qquad\mathcal{F}^{-1}\cbraces{f}(x)=\frac1{\sqrt{2\uppi}}\int_{-\infty}^\infty f(t)\ee^{\ii xt}\ddt.\]
\end{lemma}
\begin{proof}
    The equivalence between the Fourier and inverse Fourier transforms follows from the evenness of both kernels. We hence prove only the forward transform equivalences.

    Since \(\supp K_\lambda=\qty[-2\lambda,2\lambda]\), we have that
    \begin{align*}
        \mathcal{F}\cbraces{K_\lambda}(x) & =\frac1{\sqrt{2\uppi}}\int_{-2\lambda}^{2\lambda}\qty(1-\frac{\abs{t}}{2\lambda})\ee^{-\ii xt}\ddt=\frac{1}{\sqrt{2\uppi}}\int_0^{2\lambda}\qty(1-\frac{t}{2\lambda})\qty(\ee^{-\ii xt}+\ee^{\ii xt})\ddt \\
        & =\frac{\sqrt{2}}{\sqrt{\uppi}}\int_0^{2\lambda}\qty(1-\frac{t}{2\lambda})\cos\qty(xt)\ddt                                                                                                                 \\
        & =\frac{\sqrt{2}}{\sqrt{\uppi}}\qty[\eval{\frac{\sin(xt)}x}_0^{2\lambda}-\eval{\frac{t\sin(xt)}{2\lambda x}}_0^{2\lambda}+\int_0^{2\lambda}\frac{\sin(xt)}{2\lambda x}\ddt]                                \\
        & =\frac{\sqrt{2}}{\sqrt{\uppi}}\qty[\frac{\sin(2\lambda x)}x-\frac{\sin(2\lambda x)}{x}-\eval{\frac{\cos(xt)}{2\lambda x^2}}_0^{2\lambda}]                                                                 \\
        & =\frac{\sqrt{2}}{\sqrt{\uppi}}\qty[\frac{1-\cos(2\lambda x)}{2\lambda x^2}]=\frac{2\lambda}{\sqrt{2\uppi}}\qty(\frac{\sin(\lambda x)}{\lambda x})^2=k_\lambda(x).
    \end{align*}
    On the other hand,
    \begin{align*}
        \mathcal{F}\cbraces{k_\lambda}(x) & =\frac{2\lambda}{2\uppi}\int_{-\infty}^\infty\qty(\frac{\sin(\lambda t)}{\lambda t})^2\ee^{-\ii xt}\ddt=\frac2{\uppi\lambda}\int_{0}^\infty\frac{\sin^2(\lambda t)}{t^2}\cos(xt)\ddt \\
        & =\frac{1}{\uppi\lambda}\int_0^\infty\frac{1-\cos(2\lambda t)}{t^2}\cos(xt)\ddt.
    \end{align*}
    Let \(I(x)=\int_0^\infty\frac{1-\cos(2\lambda t)}{t^2}\cos(xt)\ddt\). Differentiation under the integral sign yields \[\dv{I}{x}=\int_0^\infty\pdv{x}\qty(\frac{1-\cos(2\lambda t)}{t^2}\cos(xt))\ddt=\int_0^\infty\qty(\frac{\cos(2\lambda t)-1}{t})\sin(xt)\ddt.\] % TODO: prove the conditions required for the commute
    By the product-to-sum formulas, this can be rewritten as \[\dv{I}{x}=\int_0^\infty\frac{\sin(xt+2\lambda t)}{2t}\ddt+\int_0^\infty\frac{\sin(xt-2\lambda t)}{2t}\ddt-\int_0^\infty\frac{\sin(xt)}t\ddt=\mathrm{I}+\mathrm{I\!I}-\mathrm{I\!I\!I}.\]
    By substituting \(u=t(x+2\lambda)\), we have that \[\mathrm{I}=\int_0^{\pm\infty}\frac{\sin u}{2u}\dd{u},\qfor\pm\infty=
        \begin{dcases}
            +\infty & x+2\lambda>0, \\
            -\infty & x+2\lambda<0,
    \end{dcases}\] and substituting \(u=t(x-2\lambda)\) gives \[\mathrm{I\!I}=\int_0^{\pm\infty}\frac{\sin u}{2u}\dd{u},\qfor\pm\infty=
        \begin{dcases}
            +\infty & x-2\lambda>0, \\
            -\infty & x-2\lambda<0.
    \end{dcases}\] By the Dirichlet integral \(\int_0^\infty\frac{\sin u}u\dd{u}=\frac\uppi2\), we
    have that \[\mathrm{I}=\frac\uppi4\operatorname{sgn}(x+2\lambda),\qquad\mathrm{I\!I}=\frac\uppi4\operatorname{sgn}(x-2\lambda).\]
    Similarly, \(\mathrm{I\!I\!I}=\frac\uppi2\operatorname{sgn}(x)\). Hence, \[\dv{I}{x}=\frac\uppi4\qty[\operatorname{sgn}(x+2\lambda)+\operatorname{sgn}(x-2\lambda)-2\operatorname{sgn}(x)]=\frac{\uppi}4\begin{dcases}
        0& \abs{x}>2\lambda, \\
        1& -2\lambda<x<0, \\
        -1& 0<x<2\lambda,
    \end{dcases}\] and \[I(0)+\int_0^x\dv{I}{t}\ddt=I(0)+
        \begin{dcases}
            \qty(\mathmakebox[\widthof{\(\int^-\)}][l]{\int_0^{-2\lambda}}+\int_{-2\lambda}^x)\dv{I}{t}\ddt & x<-2\lambda,   \\
            \int_0^x\dv{I}{t}\ddt                                   & -2\lambda<x<0, \\
            \int_0^x\dv{I}{t}\ddt & 0<x<2\lambda,  \\
            \qty(\mathmakebox[\widthof{\(\int^-\)}][l]{\int_0^{2\lambda}}+\int_{2\lambda}^x)\dv{I}{t}\ddt                           & x>2\lambda,
    \end{dcases}\] which after simplification, becomes \[I(x)=I(0)+\uppi
        \begin{dcases}
            -\lambda  & x<-2\lambda,   \\
            \frac x2  & -2\lambda<x<0, \\
            -\frac x2 & 0<x<2\lambda,  \\
            \lambda   & x>2\lambda,
    \end{dcases}.\]
    Since
    \begin{align*}
        I(0) & =\int_0^\infty\frac{1-\cos(2\lambda t)}{t^2}\ddt=\eval{\frac{\cos(2\lambda t)-1}{t}}_0^\infty+\int_0^\infty 2\lambda\frac{\sin(2\lambda t)}{t}\ddt \\
        & =\lim_{t\to 0}\frac{1-\cos(2\lambda t)}{t}+2\lambda\frac\uppi2=\uppi\lambda,
    \end{align*}
    we obtain \[I=
        \begin{dcases}
            0                           & x<-2\lambda,   \\
            \uppi\qty(\lambda+\frac x2) & -2\lambda<x<0, \\
            \uppi\qty(\lambda-\frac x2) & 0<x<2\lambda,  \\
            0                           & x>2\lambda,
        \end{dcases}\implies\mathcal{F}\cbraces{k_\lambda}(x)=
        \begin{dcases}
            0                         & x<-2\lambda,   \\
            \qty(1+\frac x{2\lambda}) & -2\lambda<x<0, \\
            \qty(1-\frac x{2\lambda}) & 0<x<2\lambda,  \\
            0                         & x>2\lambda,
    \end{dcases}\] which confirms the second equivalence.
\end{proof}
\begin{definition}
    A function \(f:\mathbb{R}\to\mathbb{R}\) is said to be \textit{slowly decreasing} if \[\liminf_{\delta\to0^+}\liminf_{x\to+\infty}\qty[f(x+\delta)-f(x)]\geq 0,\] or equivalently, for every \(\varepsilon>0\), \(\exists x_0\),
    \(\exists\delta>0\) such that \[\forall x,y>x_0,\quad0<y-x<\delta\implies f(y)-f(x)>-\varepsilon.\]
\end{definition}
\begin{proof}[Proof of equivalence]

\end{proof}
\begin{proposition}\label{prop:wienerikehara_intermediatetauberiantheorem}
    Let \(f:\mathbb{R}\to\mathbb{R}\) be a slowly decreasing function bounded by \(M>0\). If the limit of the convolution \(f*k_\lambda\) given by \[L\equiv\lim_{x\to+\infty}\qty(f*k_\lambda)(x)=\lim_{x\to+\infty}\frac1{\sqrt{2\uppi}}\int_{-\infty}^\infty f(t)k_\lambda(x-t)\ddt\] is independent of \(\lambda>0\), then \(\lim_{x\to+\infty}f(x)=L\).
\end{proposition}
\begin{proof}
    Let \(\varepsilon>0\) be arbitrary. Assume, for the sake of contradiction, that \(f\not\to L\) as \(x\to\infty\). Then there exists some sequence \(\cbraces{x_n}_{n\in\mathbb{N}}\) such that \(\abs{f(x_n)-L}>\varepsilon\) for all \(n\). From here, we may extract an infinite subsequence (continued to be denoted by \(\cbraces{x_n}\)) such that one of the two cases is assumed:
    \begin{enumerate}
        \item \(f(x_n)-L>\varepsilon\) for all \(n\in\mathbb{N}\). By the slow decrease of \(f\), there exist \(\delta>0\) and \(N\in\mathbb{N}\) such that \[f(y)>f\qty(x_n)-\frac{\varepsilon}{2}>L+\frac\varepsilon2\] for any \(n>N\) and \(0<y-x_n<2\delta\). Then we have
            \begin{align*}
                \qty(f*k_\lambda)\qty(x+\delta) & =\frac1{\sqrt{2\uppi}}\int_{-\infty}^\infty f(t)k_\lambda\qty(x+\delta-t)\ddt                                                                                                    \\
                & =\frac1{\sqrt{2\uppi}}\qty(\int_x^{x+2\delta}+\int_{-\infty}^x+\int_{x+2\delta}^\infty)f(t)k_\lambda\qty(x+\delta-t)\ddt                                                         \\
                & >\frac{1}{\sqrt{2\uppi}}\int_x^{x+2\delta}\qty(L+\frac{\varepsilon}2)k_\lambda\qty(x+\delta-t)\ddt                                                                               \\
                & \quad-\frac M{\sqrt{2\uppi}}\qty(\int_{-\infty}^x+\int_{x+2\delta}^\infty)k_\lambda\qty(x+\delta-t)\ddt                                                                          \\
                & >\frac{2L+\varepsilon}{2\sqrt{2\uppi}}\int_x^{x+2\delta}k_\lambda\qty(x+\delta-t)\ddt-\frac M{\sqrt{2\uppi}}\qty(\int_\delta^\infty+\int_{-\infty}^{-\delta})k_\lambda(u)\dd{u}.
            \end{align*}
            Because
            \begin{align*}
                \frac{1}{\sqrt{2\uppi}}\int_{-\infty}^\infty k_{\lambda}(t)\ddt & =\frac{\lambda}{\uppi}\int_{\mathbb{R}}\qty(\frac{\sin(\lambda t)}{\lambda t})^2\ddt=\frac{1}{\uppi}\int_{\mathbb{R}}\qty(\frac{\sin t}t)^2\ddt    \\
                & =-\eval{\frac{\sin^2 t}{\uppi t}}_{-\infty}^\infty+\int_{\mathbb{R}}\frac{\dd(\sin^2 t)}{\uppi t}=\int_{\mathbb{R}}\frac{\sin(2t)}{\uppi t}\ddt=1,
            \end{align*}
            we then have for any \(n>N\), that \[\lim_{n\to\infty}(f*k_\lambda)\qty(x_n+\delta)\geq\frac{2L+\varepsilon}{2\sqrt{2\uppi}}\int_{-\delta}^{\delta}\frac{2\lambda}{\sqrt{2\uppi}}\qty(\frac{\sin\qty(\lambda u)}{\lambda u})^2\dd{u}-\frac{2M}{\uppi}\int_{\delta\lambda}^\infty\qty(\frac{\sin t}{t})^2\ddt.\]
            Letting \(\lambda\to\infty\), we have \[\frac{2L+\varepsilon}{2\uppi}\int_{-\delta\lambda}^{\delta\lambda}\mathmakebox[\widthof{\(\qty(\frac{\sin t}{t})\)}][l]{\qty(\frac{\sin t}{t})^2}\ddt-\frac{2M}{\uppi}\int_{\delta\lambda}^\infty\mathmakebox[\widthof{\(\qty(\frac{\sin t}{t})\)}][l]{\qty(\frac{\sin t}{t})^2}\ddt\to\frac{2L+\varepsilon}{2\uppi}\int_{\mathbb{R}}\mathmakebox[\widthof{\(\qty(\frac{\sin t}{t})\)}][l]{\qty(\frac{\sin t}{t})^2}\ddt=L+\frac{\varepsilon}{2}.\]
            Because \(\lim_{n\to\infty}(f*k_\lambda)\qty(x_n+\delta)=L\geq
            L+\frac\varepsilon2\), we reach a contradiction.
        \item \(f(x_n)-L<-\varepsilon\). Then there exist \(\delta>0\), \(N\in\mathbb{N}\) such that \[f\qty(x_n)-f(y)>-\frac\varepsilon2\implies f(y)<f\qty(x_n)+\frac\varepsilon2<L-\frac\varepsilon2\] for any \(n>N\) and \(0<x_n-y<2\delta\). Splitting the convolution integral
            into \(\int_\infty^{x_n-2\delta}\), \(\int_{x_n-2\delta}^{x_n}\), and
            \(\int_{x_n}^\infty\), we have
            \begin{align*}
                \qty(f*k_\lambda)\qty(x_n-\delta) & =\frac1{\sqrt{2\uppi}}\int_{-\infty}^\infty f(t)k_\lambda\qty(x_n-\delta-t)\ddt                                                                                              \\
                & =\frac1{\sqrt{2\uppi}}\qty(\int_{-\infty}^{x_n-2\delta}+\int_{x_n-2\delta}^{x_n}+\int_{x_n}^\infty)f(t)k_\lambda\qty(x_n-\delta-t)\ddt                                       \\
                & <\frac M{\sqrt{2\uppi}}\qty(\int_{-\infty}^{x_n-2\delta}+\int_{x_n}^\infty)k_\lambda\qty(x_n-\delta-t)\ddt                                                                   \\
                & \quad+\frac{1}{\sqrt{2\uppi}}\int_{x_n-2\delta}^{x_n}\qty(L-\frac{\varepsilon}2)k_\lambda\qty(x_n-\delta-t)\ddt                                                              \\
                & <\frac M{\sqrt{2\uppi}}\qty(\int_{-\infty}^{-\delta}+\int_{\delta}^\infty)k_\lambda(u)\dd{u}+\frac{2L-\varepsilon}{2\sqrt{2\uppi}}\int_{-\delta}^{\delta}k_\lambda(u)\dd{u}.
            \end{align*}
            Letting \(\lambda\to\infty\), we have similarly that \[\lim_{n\to\infty}(f*k_\lambda)\qty(x_n-\delta)\leq L-\frac{\varepsilon}2,\] contradicting \(\lim_{n\to\infty}(f*k_\lambda)\qty(x_n-\delta)=L\).
    \end{enumerate}
    Hence, no such sequence exists in either case, and \(\lim_{x\to+\infty}f(x)=L\).
\end{proof}
\begin{theorem}[\textsc{Wiener--Ikehara}]\label{thm:wienerikehara}
    Let \(f:\mathbb{R}_{\geq0}\to\mathbb{R}\) be a nonnegative, monotically nondecreasing function such that the Laplace transform
    \begin{equation}
        \mathcal{L}\cbraces{f}(s)=\int_0^\infty f(x)\ee^{-sx}\ddx\label{eq:wienerikehara_laplacetransform}
    \end{equation} converges for \(\Re s>1\). Suppose that there exists some constant \(c>0\) such
    that
    \begin{equation}
        g(t)=\lim_{\sigma\to1^+}\qty[\mathcal{L}\cbraces{f}(s)-\frac c{s-1}],\qquad s=\sigma+\ii t\label{eq:wienerikehara_gfunction}
    \end{equation} converges locally uniformly with respect to \(t\) in \(\mathbb{R}\) and is continuously differentiable. Then
    \begin{equation}
        \lim_{x\to+\infty}\frac{f(x)}{\ee^x}=c.\label{eq:wienerikehara_conclusion}
    \end{equation}
\end{theorem}
\begin{proof}
    Let \[a(t)=
        \begin{dcases}
            f(t)\ee^{-t} & t\geq 0, \\
            0            & t<0,
        \end{dcases}\qand A(t)=
        \begin{dcases}
            c & t\geq 0, \\
            0 & t<0.
    \end{dcases}\]
    The convergence of \cref{eq:wienerikehara_laplacetransform} implies that for
    \(\varepsilon,\lambda>0\), the convolution \[I_{\lambda,\varepsilon}=\frac1{\sqrt{2\uppi}}\int_{-\infty}^\infty k_\lambda(x-t)\qty(\frac{a(t)-A(t)}{\ee^{\varepsilon t}})\ddt\] converges. Since \[\int_{-\infty}^\infty\frac{a(t)-A(t)}{\ee^{\qty(\varepsilon+\ii y)t}}\ddt\] converges uniformly for \(\abs{y}<2\lambda\), we have by virtue of
    \cref{lem:wienerikehara_kernels_fouriertransform}, that
    \begin{align*}
        I_{\lambda,\varepsilon}(x) & =\frac{1}{2\uppi}\int_{-\infty}^\infty\frac{a(t)-A(t)}{\ee^{\varepsilon t}}\int_{-2\lambda}^{2\lambda}K_\lambda(y)\ee^{\ii(x-t)y}\ddy\ddt             \\
        & =\frac{1}{2\uppi}\int_{-2\lambda}^{2\lambda}K_\lambda(y)\ee^{\ii xy}\int_0^\infty\frac{a(t)-A(t)}{\ee^{(\varepsilon+\ii y)t}}\ddt\ddy                 \\
        & =\frac{1}{2\uppi}\int_{-2\lambda}^{2\lambda}K_\lambda(y)\ee^{\ii xy}\qty[\mathcal{L}\cbraces{f}(1+\varepsilon+\ii y)-\frac c{\varepsilon+\ii y}]\ddy.
    \end{align*}
    Now \(\forall\varepsilon'>0\), by the local uniform convergence of \(g\) in \cref{eq:wienerikehara_gfunction}, there exists some \(\delta>0\) such that \[\forall\abs{y}<2\lambda,\quad\abs{\mathcal{L}\cbraces{f}(1+\varepsilon+\ii y)-\frac c{\varepsilon+\ii y}-g(y)}<\frac{\uppi\varepsilon'}{\lambda}.\] Hence, we have that \[\abs{I_{\lambda,\varepsilon}(x)-\frac1{2\uppi}\int_{-2\lambda}^{2\lambda}K_\lambda(y)\ee^{\ii xy}g(y)\ddy}\leq\frac{1}{2\uppi}\int_{-2\lambda}^{2\lambda}K_\lambda(y)\frac{\uppi\varepsilon'}{\lambda}\ddy=\varepsilon'.\] Therefore, \[\lim_{\varepsilon\to 0^+}I_{\lambda,\varepsilon}(x)=\frac1{2\uppi}\int_{-2\lambda}^{2\lambda} K_\lambda(y)\ee^{\ii xy}g(y)\ddy,\] or when the limit is exchanged with the integral. By parts, we have that \[\lim_{\varepsilon\to0^+}I_{\lambda,\varepsilon}(x)=\frac{1}{2\uppi}\qty(\eval{K_\lambda(y)g(y)\frac{\ee^{\ii x y}}{\ii x}}_{-2\lambda}^{2\lambda}-\frac{1}{\ii x}\int_{-2\lambda}^{2\lambda}\qty(K_\lambda g)'(y)\ee^{\ii xy}\ddy).\]
    This implies that \(\lim_{x\to+\infty}\lim_{\varepsilon\to0^+}I_{\lambda,\varepsilon}(x)\equiv 0\). On the other hand, manual calculation yields \[\lim_{\varepsilon\to0^+}I_{\lambda,\varepsilon}(x)=\lim_{\varepsilon\to0^+}\frac1{\sqrt{2\uppi}}\qty(\int_0^\infty k_\lambda(x-t)a(t)\ee^{-\varepsilon t}\ddt-\int_0^\infty k_\lambda(x-t)A(t)\ee^{-\varepsilon t}\ddt).\] The Lebesgue's Dominated Convergence Theorem then gives \[\lim_{\varepsilon\to0^+}I_{\lambda,\varepsilon}(x)=\frac1{\sqrt{2\uppi}}\int_0^\infty k_\lambda(x-t)\qty(a(t)-A(t))\ddt=(a*k_\lambda-A*k_\lambda)(x)\to0\] as \(x\to+\infty\). Since \((A*k_\lambda)(x)=c\) for all \(x\), we have that
    \begin{equation}
        \lim_{x\to+\infty}(a*k_\lambda)(x)=c.\label{eq:wienerikehara_convolutionlimit}
    \end{equation}
    Therefore, \(\exists x_0>0\) such that \[\qty(a*k_\lambda)(x)<c+1\implies\frac{1}{\uppi}\int_{-\infty}^\infty\qty(\frac{\sin t}{t})^2a\qty(x-\frac t\lambda)\ddt<c+1\] for all \(x>x_0\). Substituting \(x+\frac2{\sqrt{\lambda}}\) for \(x\) in the integral, we have (after further restricting the integration bounds, which preserves the inequality)
    \[\int_{-\sqrt{\lambda}}^{\sqrt{\lambda}}\qty(\frac{\sin t}t)^2a\qty(x+\frac{2\sqrt{\lambda}-t}\lambda)\ddt<\uppi(c+1)\]
    Since \(\ee^{u}a(u)=f(u)\) (for \(u=x+\frac{2\sqrt{\lambda}-t}\lambda>0\)) is
    nondecreasing in \(u\), it is bounded below by
    \(f\qty(x+\frac{\sqrt{\lambda}}\lambda)>f(x)\). Thus, we have \[\int_{-\sqrt{\lambda}}^{\sqrt{\lambda}}\qty(\frac{\sin t}t)^2\exp\qty(\frac{t-2\sqrt{\lambda}}{\lambda}-x)f\qty(x)\ddt<\uppi\qty(c+1),\] implying that \[\int_{-\sqrt{\lambda}}^{\sqrt{\lambda}}\qty(\frac{\sin t}t)^2\ee^{\frac{t-2\sqrt{\lambda}}{\lambda}}a\qty(x)\ddt<\uppi\qty(c+1)\implies a(x)\ee^{-\frac{3}{\sqrt{\lambda}}}\int_{-\sqrt{\lambda}}^{\sqrt{\lambda}}\qty(\frac{\sin t}t)^2\ddt<\uppi(c+1),\]
    which is satisfied for all \(\lambda>0\) and \(x>x_0\). Letting \(\lambda\to\infty\), we have that \(a(x)<c+1\). Compactness shows that \(a\)
    is bounded on \([0,x_0]\) (\cref{thm:continuousfunctionboundedoncompact}).
    Hence, \(a\) is bounded (above) on \(\mathbb{R}\) by some \(M>0\).

    The final hypothesis required is the slow decrease of \(a\): for any
    \(\varepsilon>0\), \(\exists\delta>0\) such that \(\forall x,y>0\) with
    \(0<y-x<\delta\), we have 
    \begin{align*}
        a(y)-a(x)&=\ee^{-x}\qty(\ee^{x-y}f(y)-f(x))>\ee^{-x}f(x)\qty(\ee^{-\delta}-1)\\
        &=a(x)\qty(\ee^{-\delta}-1)>M\qty(\ee^{-\delta}-1).
    \end{align*}
    If \(\delta\) is chosen so that \(M\qty(1-\ee^{-\delta})<\varepsilon\), namely
    \(0<\delta<\ln\qty(\frac M{M-\varepsilon})\), then \(a(y)-a(x)>-\varepsilon\),
    and \(a\) then exhibits slow decrease.

    The slow decrease, boundedness, and the condition in
    \cref{eq:wienerikehara_convolutionlimit} are sufficient by
    \cref{prop:wienerikehara_intermediatetauberiantheorem}, to show that
    \(\lim_{x\to+\infty}a(x)=c\), or equivalently,
    \(\lim_{x\to+\infty}\frac{f(x)}{\ee^x}=c\).
\end{proof}
\begin{theorem}[name=\textit{Prime Number Theorem},store=thm:primenumber]\label{thm:primenumber}
    The prime counting function \(\pi(x)\) is asymptotically equal to \(\frac x{\log x}\), i.e., \(\lim_{x\to+\infty}\frac{\pi(x)\log x}x=1\).
\end{theorem}
\begin{proof}
    By \cref{thm:chebyshevfunctions_limsup_inflim_equivalences}, it suffices to show that \(\lim_{x\to+\infty}\frac{\psi(x)}x=1\). Consider the Laplace transform of \(\psi\circ\exp\) in \cref{eq:primenumbertheorem_laplacetransformchebyshevfunction_statement}. By \cref{thm:primenumbertheorem_laplacetransformchebyshevfunction}, the function \(g\) defined therein converges uniformly on compact subsets of \(\mathbb{R}\) and is continuously differentiable. Hence, by the Wiener--Ikehara theorem (\cref{thm:wienerikehara}), we have that \[\lim_{x\to+\infty}\frac{\psi\qty(\ee^x)}{\ee^x}=\lim_{x\to+\infty}\frac{\psi\qty(x)}{x}=1,\] and the Prime Number Theorem follows.
\end{proof}