\subsubsection{Deficiency Relation}\label{sec:deficiencyrelation}
One of the reformulations of the Second Fundamental Theorem is characterized in a way so that it comprises the statements of both Picard theorems in a compact manner (of course, the conclusions are substantially more far-reaching than just the Picard theorems themselves). We first introduce the relevant terminology.
\begin{definition}
    For any \(a\in\extcomplex\) and a meromorphic function \(f:D(0,R)\to\extcomplex\), define the \textit{deficiency} (or \textit{defect}) \(a\) in \(f\) by \[\delta(a,f)=\liminf_{r\to R^-}\frac{m(r,f)}{T(r,f)}=1-\limsup_{r\to R^-}\frac{N(r,f)}{T(r,f)}.\]
    Then \(a\) is said to be a deficient value of \(f\) if its deficiency is nonzero.
\end{definition}
\begin{remark}
    Because \(N(r,f)\leq T(r,f)\) holds by definition, the deficiency will always be a nonnegative real value.
\end{remark}
The deficiency is the primary value we are interested in. We also have the \textit{Verzweigungsindex} (ramification index) \(\theta\) and the truncated deficiency \(\Theta\) (of order 1): \[\theta(a,f)=1-\limsup_{r\to R^-}\frac{\overline{N}(r,f)}{T(r,f)},\qquad\Theta(a,f)=\liminf_{r\to R^-}\frac{N(r,f)-\overline{N}(r,f)}{T(r,f)}.\]
\begin{proposition}
    If \(f:D(0,R)\to\extcomplex\) meromorphic and non-constant omits a value \(a\in\extcomplex\), then \(a\) has deficiency 1.
\end{proposition}
\begin{proof}
    The assertion is trivial from \(N(r,a)=0\).
\end{proof}
%admissible functions?
\begin{theorem}[\textsc{Deficiency Relation}]

\end{theorem}
