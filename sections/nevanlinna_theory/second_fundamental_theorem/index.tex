% decide whether or not to have log diff lemma
\subsection{The Second Fundamental Theorem}
Whereas the First Fundamental Theorem gives the relation between the general size of \(T\) relative to \(m\) and \(N\), efforts have been made to quantify the relative size of \(m\) and \(N\). This question was answered by R. Nevanlinna in July 1924 with considerable accuracy via the inequality \[T(r,f)\leq N(r,a,f)+N(r,b,f)+N(r,c,f)+S(r,f),\] where \(a,b,c\in\extcomplex\) are distinct and \(S(r,f)\) is a term that is generally small relative to \(T(r,f)\) (is \(o\qty(T(r,f))\) for most \(r\)). It also implies that in most cases, \(N(r,a,f)\) will be much larger than \(m(r,a,f)\). For example, assume \(f\)'s meromorphy is on all of \(\mathbb{C}\). Then if there are two values \(a,b\) where the proximity dwarfs the counting function, such as when \[\liminf_{r\to\infty}\frac{m(r,a,f)}{T(r,f)}=1,\qquad\liminf_{r\to\infty}\frac{m(r,b,f)}{T(r,f)}=1,\] then for all other values \(c\) on the Riemann sphere,
\begin{align*}
    1&\leq\varlimsup_{r_n\to\infty}\frac{N\qty(r_n,a,f)}{T\qty(r_n,f)}+\varlimsup_{r_n\to\infty}\frac{N\qty(r_n,b,f)}{T\qty(r_n,f)}+\varlimsup_{r_n\to\infty}\frac{N\qty(r_n,c,f)}{T(r_n,f)}+\varlimsup_{r_n\to\infty}\frac{S\qty(r_n,f)}{T\qty(r_n,f)}\\
    &\leq\varlimsup_{r\to\infty}\frac{N\qty(r,a,f)}{T\qty(r,f)}+\varlimsup_{r\to\infty}\frac{N\qty(r,b,f)}{T\qty(r,f)}+\varlimsup_{r\to\infty}\frac{N\qty(r,c,f)}{T(r,f)}+\varlimsup_{r_n\to\infty}\frac{S\qty(r_n,f)}{T\qty(r_n,f)}\\
    &=\varlimsup_{r\to\infty}\frac{N(r,c,f)}{T(r,f)},
\end{align*}
where the precise formulation allows for a sequence \(\qty{r_n}\) diverging to \(\infty\) to be chosen so that \(S(r,f)=o\qty(T(r,f))\) for \(r\to\infty\) in this sequence. This implies \(\textstyle\limsup_{r\to\infty}\frac{N(r,c,f)}{T(r,f)}=1\). This in turn then proves Picard's theorems (see \cref{sec:deficiencyrelation}). As we shall later see, the conclusions of the theorem have even further extensibility.

Regarding Nevanlinna's original paper containing the result, mathematician Hermann Weyl has stated in 1943 that ``the appearance of this paper has been one of the few great mathematical events of our century.''
\begin{theorem}[\textsc{Second Fundamental Theorem of Nevanlinna Theory}]\label{thm:nevanlinnasecondfundamentaltheorem}
    Let \(f:D(0,R)\to\extcomplex\) be meromorphic and non-constant (\(0<R\leq\infty\)). If \(a_1,\ldots,a_q\) are \(q\geq 2\) distinct finite complex numbers. Then for \(0<\delta\leq\min\cbraces{a_\mu-a_\nu}{1\leq\mu<\nu\leq q}\), \[m(r,f)+\sum_{\nu=1}^q m\qty(r,a_\nu,f)\leq 2T(r,f)-N_1(r,f)+S(r,f),\]
    where \[0<N_1(r)=N(r,0,f')+2N(r,f)-N(r,f')\]
    and
    \begin{equation}
        S(r,f)=m\qty(r,\frac{f'}f)+m\qty(r,\sum_{\nu=1}^q\frac{f'}{f-a_\nu})+q\log^+\frac{3q}{\delta}+\log 2-\log\abs{f'(0)}\label{eq:nevanlinnasecondfundamentaltheorem_smallerrorterm}
    \end{equation}
    (with modifications for the singular cases of \(f(0)=\infty,f'(0)=0\)). %figure out what these are
\end{theorem}
\begin{remark}
    For interpretive purposes, \(S(r,f)\) will serve the purpose of a generally unimportant small error term. The utility of this theorem is not realized until more useful conclusions are drawn on its estimation.
\end{remark}
\begin{proof}
    Define \[F(z)=\sum_{\nu=1}^q\frac1{f(z)-a_\nu}.\]
    For all \(\nu,z\) such that \(\textstyle\abs{f(z)-a_\nu}<\flatfrac{\delta}{3q}\), for any \(\mu\neq\nu\), we have \[\]
\end{proof}
