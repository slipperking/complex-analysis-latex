\subsubsection{The Estimation of \texorpdfstring{\(S(r,f)\)}{the Small Error Term}}
The second part of the Second Fundamental Theorem is typically a suitable estimate on \(S(r,f)\).
\getkeytheorem{thm:nevanlinnasecondfundamentaltheorempart2}
Modern results vastly improve said estimations, and the search for a sharp estimation is one of the most challenging questions in Nevanlinna Theory.
\begin{lemma}\label{lem:nevanlinnasmallerrortermestimation1}
    Let \(z\in\mathbb{C}\) be arbitrary and fixed and suppose \(0<r<\infty\). Let \[E_k=\cbraces{\theta}{-\uppi<\theta<\uppi\wedge\abs{z-r\ee^{\ii\theta}}<kr}\] for each \(0<k\leq 1\). It follows that \[I=\int_{E_k}\log\abs{\frac{r}{z-r\ee^{\ii\theta}}}\dd{\theta}<\uppi k\qty(1-\log k).\]
\end{lemma}
\begin{proof}
    Let \(z=\rho\ee^{\ii\psi}\). Performing a substitution of \(\vartheta=\theta-\psi\), we have \[I=\int_{E'_k}\log\abs{\frac{r}{\rho-r\ee^{\ii\vartheta}}}\dd{\vartheta},\] where \[E'_k=\cbraces{\vartheta}{-\uppi-\psi<\vartheta<\uppi-\psi\wedge\abs{\rho-r\ee^{\ii\vartheta}}<kr}.\]
    Because the exponential is periodic every \(2\uppi\ii\), we may assume that \(E'_k\) is restricted to \(-\uppi<\vartheta<\uppi\) instead of \(-\uppi-\psi<\vartheta<\uppi-\psi\). The expression \(\abs{\rho-r\ee^{\ii\vartheta}}=kr\) gives at most two symmetric solutions \(\pm\vartheta_0\) (let \(\vartheta_0\geq0\)).

    Evidently, for \(\vartheta>\flatfrac{\uppi}2\) or \(\vartheta<-\flatfrac{\uppi}2\), \(\abs{\rho-r\ee^{\ii\vartheta}}\geq r\geq kr\), so it follows that \(\vartheta_0\leq\flatfrac{\uppi}2\), and furthermore, all \(-\vartheta_0<\vartheta<\vartheta_0\) lie in \(E'_k\) by the geometry of a circle. Since \[\abs{\rho-r\ee^{\ii\vartheta}}\geq\abs{\Im\qty(\rho-r\ee^{\ii\vartheta})}=r\abs{\sin\vartheta},\]
    assuming that \(E'_k\neq\emptyset\), it follows that
    \begin{align*}
        I&=2\int_0^{\vartheta_0}\log\abs{\frac{r}{\rho-r\ee^{\ii\vartheta}}}\dd{\vartheta}\leq2\int_0^{\vartheta_0}\log\csc\vartheta\dd{\vartheta}\leq2\int_0^{\vartheta_0}\log\tfrac{\uppi}{2\vartheta}\dd{\vartheta}\\
        &=2\vartheta_0\log\tfrac{\uppi}2-2\mathop{\mathmakebox[\widthof{\(\int\)}][l]{\int_0^{\vartheta_0}}}\log\vartheta\dd{\vartheta}=2\vartheta_0\log\tfrac{\uppi}2-2\qty(\vartheta_0\log\vartheta_0-\vartheta_0)=2\vartheta_0\qty(\log\tfrac{\uppi}{2\vartheta_0}+1).
    \end{align*}
    Since \(\qty(\flatfrac{\uppi}2)\sin x\geq x\) for \(x\in\qty[0,\flatfrac{\uppi}2]\), it follows that \[\vartheta_0\leq\tfrac{\uppi}2\sin\vartheta_0=\tfrac{\uppi k}2.\]
    Moreover, \(\vartheta\mapsto2\vartheta\qty(\log\tfrac{\uppi}{2\vartheta}+1)\) is an increasing function of \(0\leq\vartheta\leq\flatfrac{\uppi}2\); thus we may replace all instances of \(\vartheta_0\) with \(\flatfrac{\uppi k}2\), proving the lemma.
\end{proof}
\begin{lemma}\label{lem:nevanlinnasmallerrortermestimation2}
    Let \(\qty{z_\nu}_{\nu=1}^n\) be \(n\geq1\) (possibly indistinct) complex numbers. Define \(\textstyle\delta(z)=\min\cbraces{\abs{z-z_\nu}}{\nu\in\mathbb{N}_{\leq n}}\). It follows that for any \(r>0\), \[\frac1{2\uppi}\int_0^{2\uppi}\logp\frac{r}{\delta\qty(r\ee^{\ii\theta})}\dd{\theta}\leq2\log n+\tfrac12\]
\end{lemma}
\begin{proof}
    Define the set \[E_\nu=\cbraces{\theta\in(-\uppi,\uppi)}{\abs{z_\nu-r\ee^{\ii\theta}}<\tfrac{r}{n}}\]
    and let \(E=\textstyle\bigcup_{\nu=1}^n E_\nu\). For \(\theta\in E\), \(\delta\qty(r\ee^{\ii\theta})<\flatfrac{r}{n}\), implying that \(\tfrac{r}{\delta\qty(r\ee^{\ii\theta})}>n\). Define \[\log_0x=
        \begin{cases}
            \log x&x\geq n,\\
            0&\text{otherwise},
    \end{cases}(\geq 0)\]
    so that for \(\theta\) in the prescribed range, we have \[\logp\frac{r}{\delta\qty(r\ee^{\ii\theta})}=\log_0\frac{r}{\delta\qty(r\ee^{\ii\theta})}=\log_0\frac{r}{\min_{\nu\in\mathbb{N}_{\leq n}}\abs{r\ee^{\ii\theta}-z_\nu}}\leq\sum_{\nu=1}^n\log_0\frac{r}{\abs{r\ee^{\ii\theta}-z_\nu}}\]
    Since \(0<\flatfrac1n\leq1\), by virtue of \cref{lem:nevanlinnasmallerrortermestimation1} for \(k=\flatfrac1n\),
    \begin{multline}
        \frac1{2\uppi}\int_E\logp\frac{r}{\delta\qty(r\ee^{\ii\theta})}\dd{\theta}\leq\frac1{2\uppi}\sum_{\nu=1}^n\int_0^{2\uppi}\log_0\frac{r}{\abs{r\ee^{\ii\theta}-z_\nu}}\dd{\theta}\\
        =\frac1{2\uppi}\sum_{\nu=1}^n\int_0^{2\uppi}\log\frac{r}{\abs{r\ee^{\ii\theta}-z_\nu}}\dd{\theta}<\frac1{2\uppi}\sum_{\nu=1}^n\frac{\uppi}{n}\qty(1+\log n)=\tfrac12+\tfrac{\log n}2.\label{eq:nevanlinnasmallerrortermestimation2_Eset}
    \end{multline}
    For \(\theta\notin E\), for each \(\nu\) we have \(\abs{z_\nu-r\ee^{\ii\theta}}\geq\flatfrac{r}{n}\) and thus \(\delta\qty(r\ee^{\ii\theta})\geq\flatfrac{r}{n}\). It follows that
    \begin{equation}
        \frac1{2\uppi}\int_{\theta\notin E}\logp\frac{r}{\delta\qty(r\ee^{\ii\theta})}\dd{\theta}\leq\frac1{2\uppi}\int_0^{2\uppi}\logp n\dd{\theta}\leq\log n.\label{eq:nevanlinnasmallerrortermestimation2_Ecomplement}
    \end{equation}
    Combining \cref{eq:nevanlinnasmallerrortermestimation2_Eset,eq:nevanlinnasmallerrortermestimation2_Ecomplement} gives the desired result.
\end{proof}
\begin{proposition}\label{prop:nevanlinnalogdiffproximityestimate}
    If \(f\) is meromorphic on a neighborhood of \(\overline{D(0,R)}\), then for \(0<r<R\), letting \(c\) be the first nonzero coefficient of the Laurent of expansion of \(f\) about the origin,
    \begin{multline}
        m\qty(r,\tfrac{f'}f)<4\logp T(R,f)+4\logp\circ\logp\abs{\tfrac1c}\\
        +5\logp R+6\logp\tfrac1{R-r}+\logp\tfrac1r+14.\label{eq:nevanlinnalogdiffproximityestimate_statement}
    \end{multline}
\end{proposition}
\begin{proof}
    By the Poisson--Jensen Formula (\cref{thm:poissonjensenformula}), letting \(\rho=\tfrac12(R+r)\), for each non-singular point \(z\in\partial D(0,r)\),
    \begin{align*}
        \Re\log f(z)&=\frac1{2\uppi}\int_0^{2\uppi}\log\abs{f\qty(\rho\ee^{\ii\theta})}\Re\qty(\frac{\rho\ee^{\ii\theta}+z}{\rho\ee^{\ii\theta}-z})\dd{\theta}\\
        &\quad+\sum_{k=1}^{n(\rho,0,f)}\Re\log\qty(\frac{\rho\qty(z-a_k)}{\rho^2-\overline{a_k}z})-\sum_{j=1}^{n(\rho,f)}\Re\log\qty(\frac{\rho\qty(z-b_j)}{\rho^2-\overline{b_j}z}),
    \end{align*}
    where \(\qty{a_k}\) and \(\qty{b_j}\) are the respective zeros and poles. By \cref{eq:wirtingerderivativeofrealpartofholomorphicfunction}, we have \(\textstyle\pdv{z}(\Re g(z))=\tfrac12g'(z)\) for all holomorphic \(g\). Applying this with differentiation under the integral sign, we have that
    \begin{align*}
        \frac{f'(z)}{f(z)}&=\frac1{2\uppi}\int_0^{2\uppi}\log\abs{f\qty(\rho\ee^{\ii\theta})}\frac{2\rho\ee^{\ii\theta}\dd{\theta}}{\qty(\rho\ee^{\ii\theta}-z)^2}\\
        &\quad+\sum_{k=1}^{n(\rho,0,f)}\qty(\frac{\overline{a_k}}{\rho^2-\overline{a_k}z}-\frac1{a_k-z})+\sum_{j=1}^{n(\rho,f)}\qty(\frac1{b_j-z}-\frac{\overline{b_j}}{\rho^2-\overline{b_j}z}).
    \end{align*}
    Define \(\delta(z)=\min\qty(\cbraces{\abs{z-a_k}}{k\in\mathbb{N}_{\leq n(\rho,0,f)}}\cup\cbraces{\abs{z-b_j}}{j\in\mathbb{N}_{\leq n(\rho,f)}})\). Evidently the number of zeros and poles is given by \[n=n(\rho,f)+n(\rho,0,f).\]
    Since \[\abs{\rho^2-\overline{a_k}z}\geq\rho^2-r\abs{a_k}\geq\rho^2-r\rho,\qquad\abs{\rho^2-\overline{b_j}z}\geq\rho^2-r\rho,\]
    it follows that
    \begin{equation}
        \abs{\frac{\overline{a_k}}{\rho^2-\overline{a_k}z}}\leq\frac{\rho}{\rho^2-r\rho}=\frac1{\rho-r},\quad\abs{\frac{\overline{b_j}}{\rho^2-\overline{b_j}z}}\leq\frac{\rho}{\rho^2-r\rho}=\frac1{\rho-r},\label{eq:nevanlinnalogdiffproximityestimate_mobiusbounds}
    \end{equation} and additionally,
    \begin{equation}
        \abs{\frac1{b_j-z}}\leq\frac1{\delta(z)},\qquad\abs{\frac1{a_k-z}}\leq\frac1{\delta(z)}.\label{eq:nevanlinnalogdiffproximityestimate_inversionbounds}
    \end{equation}
    Furthermore,
    \begin{multline}
        \abs{\frac1{2\uppi}\int_0^{2\uppi}\log\abs{f\qty(\rho\ee^{\ii\theta})}\frac{2\rho\ee^{\ii\theta}\dd{\theta}}{\qty(\rho\ee^{\ii\theta}-z)^2}}\leq\frac1{2\uppi}\frac{2\rho}{\qty(\rho-r)^2}\int_0^{2\uppi}\abs{\log\abs{f\qty(\rho\ee^{\ii\theta})}}\dd{\theta}\\
        =\frac{2\rho}{\qty(\rho-r)^2}\qty[m\qty(\rho,f)+m\qty(\rho,\tfrac1f)]\label{eq:nevanlinnalogdiffproximityestimate_integralbounds}
    \end{multline}
    by \cref{itm:lognonnegativepartproperties_sumofreciprocallogs} of \cref{prop:lognonnegativepartproperties}. Combining \cref{eq:nevanlinnalogdiffproximityestimate_mobiusbounds,eq:nevanlinnalogdiffproximityestimate_inversionbounds,eq:nevanlinnalogdiffproximityestimate_integralbounds}, we have
    \begin{align*}
        \abs{\frac{f'(z)}{f(z)}}&\leq\frac{2\rho}{\qty(\rho-r)^2}\qty[m\qty(\rho,f)+m\qty(\rho,\tfrac1f)]+\sum_{k=1}^{n(\rho,0,f)}\qty(\frac1{\rho-r}+\frac1{\delta(z)})+\sum_{j=1}^{n(\rho,f)}\qty(\frac1{\rho-r}+\frac1{\delta(z)})\\
        &=\frac{2\rho}{\qty(\rho-r)^2}\qty[m\qty(\rho,f)+m\qty(\rho,\tfrac1f)]+n\qty(\frac{1}{\rho-r}+\frac{1}{\delta(z)}).
    \end{align*}
    By \cref{prop:nevanlinnafirsttheorematzero}, we have \(T(\rho,0,f)=T(\rho,f)-\log\abs{c}\), where \(c\) is the first nonzero coefficient of the Laurent expansion of \(f\) about the origin. It follows that \(m\qty(\rho,\tfrac1f)=T(\rho,f)-\log\abs{c}-N\qty(\rho,0,f)\) and \(m\qty(\rho,f)=T(\rho,f)-N\qty(\rho,f)\), thus \[m\qty(\rho,\tfrac1f)+m(\rho,f)=2T(\rho,f)-\log\abs{c}-N\qty(\rho,\tfrac1f)-N\qty(\rho,f)\leq 2T(\rho,f)+2\logp\abs{\tfrac1{c}},\]
    and
    \begin{align*}
        \abs{\frac{f'(z)}{f(z)}}&\leq\frac{4\rho}{\qty(\rho-r)^2}\qty[T(\rho,f)+\logp\abs{\frac1{c}}]+\frac{n}{r}\qty(\frac{r}{\rho-r}+\frac{r}{\delta(z)}).
    \end{align*}
    By the subadditive properties of \cref{prop:lognonnegativepartproperties}, we have
    \begin{align*}
        \logp\abs{\tfrac{f'(z)}{f(z)}}&\leq\logp\qty[\tfrac{4\rho}{\qty(\rho-r)^2}\qty[T(\rho,f)+\logp\abs{\tfrac1{c}}]]+\logp\qty[\tfrac{n}{r}\qty(\tfrac{r}{\rho-r}+\tfrac{r}{\delta(z)})]+\log 2\\
        &\leq\logp\rho+\logp\qty[\qty(\tfrac{2}{\rho-r})^2]+\logp\qty[T(\rho,f)+\logp\abs{\tfrac1{c}}]+\logp\tfrac{n}{r}\\
        &\quad+\logp\qty(\tfrac{r}{\rho-r}+\tfrac{r}{\delta(z)})+\log2\\
        &\leq\logp\rho+2\logp\tfrac{1}{\rho-r}+\logp T(\rho,f)+\logp\circ\logp\abs{\tfrac1{c}}+\logp\tfrac{n}{r}\\
        &\quad+\logp\tfrac{r}{\rho-r}+\logp\tfrac{r}{\delta(z)}+5\log2.
    \end{align*}
    Integrating on \(\abs{z}=r\) with \cref{lem:nevanlinnasmallerrortermestimation2} gives
    \begin{align}
        m\qty(r,\tfrac{f'}f)&\leq\logp\rho+2\logp\frac{1}{\rho-r}+\logp T(\rho,f)+\logp\circ\logp\abs{\tfrac1{c}}+\logp\tfrac{n}{r}\nonumber\\
        &\quad+\logp\frac{r}{\rho-r}+\frac1{2\uppi}\int_0^{2\uppi}\logp\frac{r}{\delta\qty(r\ee^{\ii\theta})}\dd{\theta}+5\log2\nonumber\\
        &\leq\logp\rho+3\logp\frac{1}{\rho-r}+\logp T(\rho,f)+\logp\circ\logp\abs{\frac1{c}}+3\logp n+\logp\tfrac{1}{r}\nonumber\\
        &\quad+\logp r+\tfrac12+5\log2.\label{eq:nevanlinnalogdiffproximityestimate_proximityprimaryestimate}
    \end{align}
    To derive a more useful estimate for \(n\), we note that
    \begin{gather*}
        N(R,f)\geq\int_\rho^R\frac{n(x,f)\ddx}{x}\geq\int_\rho^R\frac{n(\rho,f)\ddx}{R}=n(\rho,f)\frac{R-\rho}{R},\\
        N(R,0,f)\geq n(\rho,0,f)\frac{R-\rho}R.
    \end{gather*}
    Thus,
    \begin{align*}
        n&=n(\rho,f)+n(\rho,0,f)\leq\tfrac{R}{R-\rho}\qty[N(R,f)+N(R,0,f)]\\
        &\leq\tfrac{R}{R-\rho}\qty[T(R,f)+T(R,0,f)-m(R,f)-m(R,0,f)]\\
        &\leq\tfrac{R}{R-\rho}\qty[2T(R,f)+\log\abs{\tfrac1{c}}]\leq\tfrac{2R}{R-\rho}\qty[T(R,f)+\logp\abs{\tfrac1{c}}].
    \end{align*}
    Using the \(\logp{}\) subadditivity properties (as with earlier), we have
    \begin{align*}
        \logp{n}&\leq\logp{\frac{2R}{R-\rho}}+\logp T(R,f)+\logp\circ\logp\abs{\frac1c}+\log 2\\
        &\leq\logp R+\logp{\frac{1}{R-\rho}}+\logp T(R,f)+\logp\circ\logp\abs{\frac1c}+2\log 2
    \end{align*}
    Substituting this into \cref{eq:nevanlinnalogdiffproximityestimate_proximityprimaryestimate}, and using \(R-\rho=\rho-r=\tfrac{R-r}2\) we have that (recognizing \(\logp r,\logp\rho<\logp R\))
    \begin{multline*}
        \begin{aligned}
            m\qty(r,\frac{f'}f)&\leq\logp\rho+3\logp\frac{2}{R-r}+\logp T(\rho,f)+\logp\circ\logp\abs{\frac1{c}}+\logp\frac{1}{r}\nonumber\\
            &\qquad+\logp r+\tfrac12+5\log2+3\logp R+3\logp{\frac{2}{R-r}}\\
            &\qquad\qquad+3\logp T(R,f)+3\logp\circ\logp\abs{\frac1c}+6\log 2
        \end{aligned}\\
        \begin{aligned}
            {\!}&<5\logp R+17\log 2+6\logp\frac1{R-r}+4\logp T(\rho,f)+4\logp\circ\logp\abs{\frac1c}+\logp\frac1r+\frac12\\
            {\!}&\leq 5\logp R+17\log 2+6\logp\frac1{R-r}+4\logp T(R,f)+4\logp\circ\logp\abs{\frac1c}+\logp\frac1r+\frac12,
        \end{aligned}
    \end{multline*}
    since \(T(\rho,f)\leq T(R,f)\) (\cref{thm:nevanlinnacharacteristicnondecreasingconvex}). This proves the proposition.
\end{proof}
%%%
\begin{theorem}[name=\textsc{Second Fundamental Theorem of Nevanlinna Theory, Part 2},store=thm:nevanlinnasecondfundamentaltheorempart2]\label{thm:nevanlinnasecondfundamentaltheorempart2}
    Let \(f:D\qty(0,R_0)\to\extcomplex\) (\(0<R_0\leq\infty\)) be meromorphic and non-constant.
    \begin{enumerate}
        \item If \(R_0=\infty\) and \(f\) has finite order, then
            \begin{equation}
                S(r,f)=\order{\log T(r,f)}+\order{\log r}\label{eq:nevanlinnasecondfundamentaltheorempart2_entirecomplexplane}
            \end{equation} as \(r\to\infty\) through all values.
        \item If \(R_0=\infty\) and \(f\) has infinite order, then \cref{eq:nevanlinnasecondfundamentaltheorempart2_entirecomplexplane} as \(r\to\infty\) through all values of \(r\) outside a set \(E\) of finite (linear) measure (satisfying \(\textstyle\int_E\dd{r}<\infty\) with respect to the Lebesgue measure\footnote{Heuristically, ``finite linear measure'' means its length is finite. Here, the condition is equivalent to \(\textstyle\inf\cbraces{\sum_k\abs{I_k}}{E\subseteq\bigcup_k I_k}<\infty\) where \(\cbraces{I_k}\) are finite open intervals.}).
        \item If \(R_0<\infty\), then
            \begin{equation}
                S(r,f)=\order{\logp T(r,f)+\log\frac1{R_0-r}}\label{eq:nevanlinnasecondfundamentaltheorempart2_largedisk}
            \end{equation}
            as \(r\to R_0^-\) outside a set \(E\) satisfying
            \begin{equation}
                \int_E\frac{\dd{r}}{R_0-r}<\infty.
            \end{equation}
            Moreover, for any \(R,\rho,\rho'\in\qty(0,R_0)\) such that \(R-\rho'<\ee^{-2}\qty(R-\rho)\) (implying that \(\rho<\rho'\)), \(\exists r\in(\rho,\rho')\) such that \(r\notin E\).
    \end{enumerate}
\end{theorem}
