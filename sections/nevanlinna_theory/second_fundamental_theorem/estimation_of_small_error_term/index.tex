\subsubsection{The Estimation of \texorpdfstring{\(S(r,f)\)}{the Small Error Term}}
\begin{lemma}
    Let \(z\in\mathbb{C}\) be arbitrary and fixed and suppose \(0<r<\infty\). Let \[E_k=\cbraces{\theta}{-\uppi<\theta<\uppi\wedge\abs{z-r\ee^{\ii\theta}}<kr}\] for each \(0<k\leq 1\). It follows that \[I=\int_{E_k}\log\abs{\frac{r}{z-r\ee^{\ii\theta}}}\dd{\theta}<\uppi k\qty(1-\log k).\]
\end{lemma}
\begin{proof}
    Let \(z=\rho\ee^{\ii\psi}\). Performing a substitution of \(\vartheta=\theta-\psi\), we have \[I=\int_{E'_k}\log\abs{\frac{r}{\rho-r\ee^{\ii\vartheta}}}\dd{\vartheta},\] where \[E'_k=\cbraces{\vartheta}{-\uppi-\psi<\vartheta<\uppi-\psi\wedge\abs{\rho-r\ee^{\ii\vartheta}}<kr}.\]
    Because the exponential is periodic every \(2\uppi\ii\), we may assume that \(E'_k\) is restricted to \(-\uppi<\vartheta<\uppi\) instead of \(-\uppi-\psi<\vartheta<\uppi-\psi\). The expression \(\abs{\rho-r\ee^{\ii\vartheta}}=kr\) gives at most two symmetric solutions \(\pm\vartheta_0\) (let \(\vartheta_0\geq0\)).

    Evidently, for \(\vartheta>\flatfrac{\uppi}2\) or \(\vartheta<-\flatfrac{\uppi}2\), \(\abs{\rho-r\ee^{\ii\vartheta}}\geq r\geq kr\), so it follows that \(\vartheta_0\leq\flatfrac{\uppi}2\), and furthermore, all \(-\vartheta_0<\vartheta<\vartheta_0\) lie in \(E'_k\) by the geometry of a circle. Since \[\abs{\rho-r\ee^{\ii\vartheta}}\geq\abs{\Im\qty(\rho-r\ee^{\ii\vartheta})}=r\abs{\sin\vartheta},\]
    assuming that \(E'_k\neq\emptyset\), it follows that 
    \begin{align*}
        I&=2\int_0^{\vartheta_0}\log\abs{\frac{r}{\rho-r\ee^{\ii\vartheta}}}\dd{\vartheta}\leq2\int_0^{\vartheta_0}\log\csc\vartheta\dd{\vartheta}\leq2\int_0^{\vartheta_0}\log\tfrac{\uppi}{2\vartheta}\dd{\vartheta}\\
        &=2\vartheta_0\log\tfrac{\uppi}2-2\mathop{\mathmakebox[\widthof{\(\int\)}][l]{\int_0^{\vartheta_0}}}\log\vartheta\dd{\vartheta}=2\vartheta_0\log\tfrac{\uppi}2-2\qty(\vartheta_0\log\vartheta_0-\vartheta_0)=2\vartheta_0\qty(\log\tfrac{\uppi}{2\vartheta_0}+1).
    \end{align*}
    Since \(\qty(\flatfrac{\uppi}2)\sin x\geq x\) for \(x\in\qty[0,\flatfrac{\uppi}2]\), it follows that \[\vartheta_0\leq\tfrac{\uppi}2\sin\vartheta_0=\tfrac{\uppi k}2.\]
    Moreover, \(\vartheta\mapsto2\vartheta\qty(\log\tfrac{\uppi}{2\vartheta}+1)\) is an increasing function of \(0\leq\vartheta\leq\flatfrac{\uppi}2\); thus we may replace all instances of \(\vartheta_0\) with \(\flatfrac{\uppi k}2\), proving the lemma.
\end{proof}
\begin{lemma}
    Let \(\qty{z_\nu}_{\nu=1}^n\) be \(n\geq1\) (possibly indistinct) complex numbers. Define \(\textstyle\delta(z)=\min\cbraces{\abs{z-z_\nu}}{\nu\in\mathbb{N}_{\leq n}}\). It follows that \[\frac1{2\uppi}\int_0^{2\uppi}\logp\frac{r}{\delta\qty(r\ee^{\ii\theta})}\dd{\theta}\leq2\log n+\tfrac12\]
\end{lemma}