\subsection{The First Fundamental Theorem}
\begin{theorem}[First Fundamental Theorem of Nevanlinna Theory]\label{thm:nevanlinnafirstfundamentaltheorem}
    Let \(f\) be (non-constant) meromorphic on \(D(0,R)\) where \(0<r\leq\infty\). Then for any \(a\in\mathbb{C}\),
    \[T(r,a,f)=T(r,f)-\log\abs{c}+\varepsilon(r,a,f),\]
    where \(\abs{\varepsilon(r,a,f)}\leq\log^+\abs{a}+\log 2\) for \(0<r<R\) and \(c\) is the first nonzero coefficient of the innermost Laurent series expansion of \(f(z)-a\) about the origin.
\end{theorem}
\begin{proof}
    Since for \(z\in\partial D(0,r)\), by the properties of \(\log^+\) as in \cref{prop:lognonnegativepartproperties},
    \[\log^+\abs{f(z)-a}\leq\log^+\qty(\abs{f(z)}+\abs{a})\leq\log 2+\log^+\abs{f(z)}+\log^+\abs{a}.\]
    Integrating over \(\partial D(0,r)\) implies \[m(r,f-a)\leq\log 2+m(r,f)+\log^+\abs{a}.\]
    From \[\log^+\abs{f(z)-a+a}\leq\log^+\qty(\abs{f(z)-a}+\abs{a})\leq\log 2+\log^+\abs{f(z)-a}+\log^+\abs{a},\] we obtain 
    \[m(r,f)\leq\log 2+m(r,f-a)+\log^+\abs{a}.\]
    Letting \(\varepsilon(r,a,f)=m(r,f-a)-m(r,f)\), the two proximity-related inequalities give \[\abs{\varepsilon(r,a,f)}\leq\log 2+\log^+\abs{a}.\]
    Then by \cref{prop:nevanlinnafirsttheorematzero}, and the fact that the poles of \(f\) match those of \(f-a\) (more importantly, \(N(r,f-a)=N(r,f)\)), 
    \begin{align*}
        T\qty(r,0,f-a)&=N(r,0,f-a)+m(r,0,f-a)\\
        &=N(r,f-a)+m(r,f-a)-\log\abs{c}\\
        &=N(r,f)+m(r,f-a)-\log\abs{c}\\
        &=N(r,f)+m(r,f)+\varepsilon(r,a,f)-\log\abs{c}\\
        &=T(r,f)+\varepsilon(r,a,f)-\log\abs{c}.\qedhere
    \end{align*}
\end{proof}
For \(R=\infty\), we have \[N(r,a,f)+m(r,a,f)=T(r,f)+\order{1},\] which shows that \(T\) is essentially independent of the choice of \(a\), except for a bounded \(\order{1}\) term. Thus, this motivates why \(T\) is \textit{characteristic} of \(f\).
\begin{corollary}
    Let \(f\) be (non-constant) meromorphic on \(D(0,R)\). Then for \(0<r\leq\infty\),
    \[\frac1{2\uppi}\int_0^{2\uppi}m\qty(r,\ee^{\ii\theta},f)\dd{\theta}\leq\log 2.\]
\end{corollary}
\begin{proof}
    By the First Fundamental Theorem (\cref{thm:nevanlinnafirstfundamentaltheorem}), we have \[T(r,f)=N\qty(r,\ee^{\ii\theta},f)+m\qty(r,\ee^{\ii\theta},f)+\log\abs{c_\theta}+\varepsilon\qty(r,\ee^{\ii\theta},f),\] where \(\abs{\varepsilon\qty(r,\ee^{\ii\theta},f)}\leq\log 2\) and \(c_\theta\) is the first nonzero coefficient of the Laurent expansion for \(f-\ee^{\ii\theta}\). Hence, 
    \begin{align*}
        T(r,f)&=\frac1{2\uppi}\int_0^{2\uppi}\qty[N\qty(r,\ee^{\ii\theta},f)+m\qty(r,\ee^{\ii\theta},f)+\log\abs{c_\theta}+\varepsilon\qty(r,\ee^{\ii\theta},f)]\dd{\theta}.
    \end{align*}
    By \cref{thm:nevanlinnacartanidentity}, we thus have 
    \begin{align*}
        T(r,f)&=T(r,f)-\frac1{2\uppi}\int_0^{2\uppi}\log\abs{c_\theta}\dd{\theta}+\frac1{2\uppi}\int_0^{2\uppi}m\qty(r,\ee^{\ii\theta},f)\dd{\theta}\\
        &\quad+\frac1{2\uppi}\int_0^{2\uppi}\log\abs{c_\theta}\dd{\theta}+\frac1{2\uppi}\int_0^{2\uppi}\varepsilon\qty(r,\ee^{\ii\theta},f)\dd{\theta}.
    \end{align*}
    This implies that \[\int_0^{2\uppi}m\qty(r,\ee^{\ii\theta},f)\dd{\theta}=-\int_0^{2\uppi}\varepsilon\qty(r,\ee^{\ii\theta},f)\dd{\theta}\leq2\uppi\log 2.\qedhere\]
\end{proof}