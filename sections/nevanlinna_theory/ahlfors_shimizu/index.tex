\subsection{The Ahlfors--Shimizu Characteristic}
We now provide a second formulation of the first fundamental theorem, given by Ahlfors and Shimizu, found independently of each other.
\begin{lemma}\label{lem:ahlforsshimizugreens}
    Let \(U\) be a positively oriented bounded region by a piecewise \(C^1\) simple closed boundary. Let \(f\) be a (not identically 0) holomorphic function on an open neighborhood of \(\overline{U}\), \(G\in C^2\qty(\cbraces{\abs{f(z)}}{z\in\overline{U}})\). Then
    \begin{equation}
        I=\oint_{\partial U}\nabla_{\vu{n}}G\qty(\abs{f(z)})\abs{\ddz}=\int_Ug\qty(\abs{f(z)})\abs{f'(z)}^2r\dd{r}\wedge\dd{\theta},\label{eq:ahlforsshimizugreens_statement}
    \end{equation}
    where \(z=r\ee^{\ii\theta}\), \(g(R)=G''(R)+\tfrac1RG'(R)\), and \(\vu{n}\) is the unit normal vector pointing towards the exterior of \(\partial U\).
\end{lemma}
\begin{proof}
    Let the zeros of \(f\) in \(U\) (which are finite, otherwise they accumulate) be \(\qty{z_k}_{k=1}^n\). Choose \(\varepsilon'>0\) so that for any \(0<\varepsilon<\varepsilon'\) the disks \(\qty{D\qty(z_k,\varepsilon)}_{k=1}^n\) each line in the open set \(U\) and are pairwise disjoint from one another.

    Applying Green's Theorem (\cref{thm:realgreen})\footnote{We define \(\vu{n}\) to be the normal vector pointing into the region, so for the \(\partial U\) integral, \(\vu{n}\) points outwards and for the summation integrals \(\vu{n}\) points into the disks.} \[\qty(\oint_{\partial U}+\sum_{k=1}^n\oint_{\partial D\qty(z_k,\varepsilon)})\symbf{v}\cdot\vu{n}\dd{s}=\iint_{U\setminus\bigcup D\qty(z_k,\varepsilon)}\divergence{\symbf{v}}\dd{A}\] to \(I\), we have
    \begin{equation}
        I=\qty(\oint_{\partial U}+\sum_{k=1}^n\oint_{\partial D\qty(z_k,\varepsilon)})\grad{G\qty(\abs{f(z)})}\cdot\vu{n}\abs{\ddz}=\mathop{\mathmakebox[\widthof{\(\iint\)}][l]{\iint_{U\setminus\bigcup D\qty(z_k,\varepsilon)}}}\laplacian{G\qty(\abs{f(z)})}r\dd{r}\dd{\theta}.\label{eq:ahlforsshimizugreens_greensapplication}
    \end{equation}
    Letting \(R=\abs{f(z)}=\sqrt{f(z)\overline{f(z)}}\) and adopting the elliptic notations \(\partial_z=\pdv*{z},\partial_{\overline{z}}=\pdv*{\overline{z}}\) etc.,
    \[\partial_zR=\frac{\partial_z\qty(f\overline{f})}{2R}=\frac{f'\overline{f}+f\overline{\partial_{\overline{z}}f}}{2R}=\frac{f'\overline{f}}{2R},\qquad\partial_{\overline{z}}R=\frac{\partial_{\overline{z}}\qty(f\overline{f})}{2R}=\frac{f\overline{f'}}{2R},\]
    and
    \begin{align*}
        \qty(\partial_zR)\qty(\partial_{\overline{z}}R) & =\frac{\abs{f}^2\abs{f'}^2}{4R^2}=\frac{\abs{f'}^2}4, & \partial_{z\overline{z}}R & =\partial_{\overline{z}}\qty(\frac{f'\overline{f}}{2R})=\frac{2f'\overline{f'}R-2f'\overline{f}\qty(\frac{f\overline{f'}}{2R})}{4R^2} \\
        &                                                       &                           & =\frac{2\abs{f'}^2R^2-\abs{f'f}^2}{4R^3}=\frac{\abs{f'}^2}{4R},
    \end{align*}
    since
    \begin{align*}
        \partial_{\overline{z}}G(R) & =G'(R)\partial_{\overline{z}}R,                  & \quad\partial_z\partial_{\overline{z}} G(R) & =\partial_zG'(R)\partial_{\overline{z}}R+G'(R)\partial_{\overline{z}z}R                   \\
        \partial_zG'(R)             & =G''(R)\partial_zR                               &                                             & =G''(R)\frac{f'\overline{f}}{2R}\cdot\frac{f\overline{f'}}{2R}+G'(R)\frac{\abs{f'}^2}{4R} \\
        & =\frac{\abs{f'}^2}4\qty(G''(R)+\frac{G'(R)}{R}),
    \end{align*}
    it follows that \(\laplacian{G\qty(\abs{f(z)})}=g(R)\abs{f'}^2\). Substituting this in \cref{eq:ahlforsshimizugreens_greensapplication} gives \[\qty(\oint_{\partial U}+\sum_{k=1}^n\oint_{\partial D\qty(z_k,\varepsilon)})\grad{G\qty(\abs{f(z)})}\cdot\vu{n}\abs{\ddz}=\mathop{\mathmakebox[\widthof{\(\iint\)}][l]{\iint_{U\setminus\bigcup D\qty(z_k,\varepsilon)}}}\laplacian{G\qty(\abs{f(z)})}r\dd{r}\dd{\theta}.\]
    As \(\varepsilon\to 0^+\), the right-hand side is simply the desired quantity in \cref{eq:ahlforsshimizugreens_statement}. By the continuous differentiability of \(G\), \(\exists M>0\) such that \(\abs{\grad{G}\cdot\vu{n}}\leq M\) on \(\textstyle\bigcup_k\overline{D\qty(z_k,\varepsilon)}\), thus
    \[\abs{\oint_{\partial D\qty(z_k,\varepsilon)}\grad{G}\cdot\vu{n}\dd{s}}\leq\int_0^{2\uppi}M\varepsilon\dd\theta\to 0\qq{as}\varepsilon\to 0^+.\]
    Thus, the left-hand side expression of \cref{eq:ahlforsshimizugreens_greensapplication} also tends to that of \cref{eq:ahlforsshimizugreens_statement}.
\end{proof}
We apply \cref{lem:ahlforsshimizugreens} to \[G(R)=\log\sqrt{1+R^2}\implies g(R)=\frac{1-R^2}{\qty(1+R^2)^2}+\frac1R\qty(\frac{R}{1+R^2})=\frac{2}{\qty(1+R^2)^2}.\]
Let \(f\) be meromorphic on \(\overline{D(0,r)}\) with no poles on the boundary, and let its poles in the interior be at each \(\qty{b_j}_{j=1}^n\) of respective orders \(\qty{k_j}_{j=1}^n\). Let \(\varepsilon'>0\) such that \(\forall 0<\varepsilon<\varepsilon'\), the disks \(\overline{D\qty(b_j,\varepsilon)}\) are disjoint and all lie in \(D(0,r)\). We aim to apply the lemma on this multiply connected region on which \(f\) is holomorphic.

In a prescribed small disk containing a pole (fix \(j\)), \(f\) has the Laurent series \[f(z)=\sum_{i=-k_j}^\infty c_i\qty(z-b_j)^i=\frac{\phi(z)}{\qty(z-b_j)^{k_j}},\] where \(\phi\) is analytic and non-vanishing in the closed disk (choose \(\varepsilon\) such that \(\frac1f\) is analytic therein too). On the boundary \(\partial D\qty(b_j,\varepsilon)\), \[\abs{f(z)}=\abs{\phi(z)}\varepsilon^{-k_j}.\]
This implies that \[G(\abs{f(z)})=\log\abs{f(z)}+\log\sqrt{1+\frac1{\abs{f(z)}^2}}=\log\abs{\phi(z)}-k_j\log\varepsilon+\log\sqrt{1+\frac1{\abs{f(z)}^2}}.\]
On the contour defined by \(\partial D(b_j,\varepsilon)\)\footnote{The outwards normal vector actually points directly into the interior, since the disk is removed from the pertinent region.},
\[\nabla_{\vu{n}}G(\abs{f(z)})=-\pdv{}{\varepsilon}G(\abs{f(z)})=\frac{k_j}{\varepsilon}-\frac{\pdv{\varepsilon}\abs{\phi(z)}}{\abs{\phi(z)}}-\frac{\pdv{\varepsilon}\qty(\frac{1}{\abs{f(z)}^2})}{2+\frac2{\abs{f(z)}^2}}=\frac{k_j}{\varepsilon}+\order{1}\]
as \(\varepsilon\to 0^+\), where \(\order{1}\) is uniform in \(t\) (for \(z=b_j+\varepsilon\ee^{\ii t}\)). By the lemma, { \newlength{\intlength}\newlength{\minuslength}\setlength{\intlength}{\widthof{\(\oint_{\partial D(0,r)}\)}}\setlength{\minuslength}{\widthof{\(-\)}}%%%%
    \[\qty(\mathmakebox[\dimexpr\intlength-\minuslength\relax][l]{\oint_{\partial D(0,r)}}+\sum_{j=1}^n\oint_{\partial D\qty(b_j,\varepsilon)})\nabla_{\vu{n}}{G\qty(\abs{f(z)})}\abs{\ddz}=\iint_{D(0,r)\setminus\bigcup D\qty(b_j,\varepsilon)}{\frac{2\abs{f'\qty(z)}^2}{\qty(1+\abs{f\qty(z)}^2)^2}}\dd{A}.\]
}
which requires the computation \[\oint_{\partial D\qty(b_j,\varepsilon)}\qty(\frac{k_j}\varepsilon+\order{1})\abs{\ddz}=\int_0^{2\uppi}\qty(k_j+\varepsilon\order{1})\dd{\theta}=2\uppi k_j+\order{\varepsilon}.\]
Taking the limit \(\varepsilon\to 0^+\) hence gives \[\frac{1}{2\uppi}\int_0^{2\uppi}\nabla_{\vu{n}}{G\qty(\abs{f\qty(r\ee^{\ii\theta})})}r\dd{\theta}+n(r,f)=\frac1{4\uppi}\int_0^{2\uppi}\int_0^r{f^\sharp\qty(\rho\ee^{\ii\theta})^2}\rho\dd{\rho}\dd{\theta},\]
where \(f^\sharp\) is the spherical derivative (\cref{def:sphericalderivative}). Let the expression on the right-hand side by denoted by \(A(r,f)\). We thus derive \[A(r,f)=\frac{r}{2\uppi}\int_0^{2\uppi}\pdv{r}\log\sqrt{1+\abs{f\qty(r\ee^{\ii\theta})}^2}\dd{\theta}+n(r,f).\]
By dividing by \(r\), changing variables, and integrating from 0 to \(r\), we have
\begin{align}
    \int_0^r\frac{A(t,f)\dd{t}}t & =\lim_{\varepsilon\to 0^+}\int_\varepsilon^r\frac{1}{2\uppi}\dv{t}\int_0^{2\uppi}\log\sqrt{1+\abs{f\qty(t\ee^{\ii\theta})}^2}\dd{\theta}\ddt+\int_\varepsilon^r\frac{n(0,f)}{t}\ddt\nonumber \\
    & \quad+N(r,f)-n(0,f)\log r.\nonumber                                                                                                                                                          \\
    & =\lim_{\varepsilon\to 0^+}\int_\varepsilon^r\frac{n(0,f)}{t}\ddt-\frac{1}{2\uppi}\int_0^{2\uppi}\log\sqrt{1+\abs{f\qty(\varepsilon\ee^{\ii\theta})}^2}\dd{\theta}\nonumber                   \\
    & \quad+N(r,f)-n(0,f)\log r+\frac{1}{2\uppi}\int_0^{2\uppi}\log\sqrt{1+\abs{f\qty(r\ee^{\ii\theta})}^2}\dd{\theta}\label{eq:ahlforsshimizuderivation_convergentintegral}                       \\
    & =\frac1{2\uppi}\int_0^{2\uppi}\log\sqrt{1+\abs{f\qty(r\ee^{\ii\theta})}^2}\dd{\theta}-\frac1{2\uppi}\int_0^{2\uppi}\log\sqrt{1+\abs{f(0)}^2}\dd{\theta}\nonumber                             \\
    & \quad+N(r,f)\qq{if}f(0)\neq\infty.\label{eq:ahlforsshimizuderivation_regularcase}
\end{align}
The limit expression of \cref{eq:ahlforsshimizuderivation_convergentintegral} is written in its present form to ensure convergence in the event of a pole. Thus assume a pole; let \(f(z)=cz^k+\order{z^{k+1}}\), where \(k<0\) and \(c\neq 0\). It follows that
\begin{gather*}
    \abs{f\qty(\varepsilon\ee^{\ii\theta})}=\abs{c}\varepsilon^k+\order{\varepsilon^{k+1}}=\abs{c}\varepsilon^k\qty(1+\order{\varepsilon})\\
    \log\abs{f\qty(\varepsilon\ee^{\ii\theta})}=\log\abs{c}+k\log\varepsilon+\log\qty(1+\order{\varepsilon})
\end{gather*} where the errors are uniform in \(\theta\). Since
\begin{align*}
    \log\sqrt{1+\abs{f\qty(\varepsilon\ee^{\ii\theta})}^2} & =\log\abs{f\qty(\varepsilon\ee^{\ii\theta})}+\log\sqrt{1+\frac1{\abs{f\qty(\varepsilon\ee^{\ii\theta})}^2}} \\
    & =\log\abs{f\qty(\varepsilon\ee^{\ii\theta})}+\order{\abs{f\qty(\varepsilon\ee^{\ii\theta})}^{-2}}           \\
    & =\log\abs{c}+k\log\varepsilon+\order{\varepsilon},
\end{align*}
it follows from \(k=-n(0,f)\), that
\begin{multline*}
    \int_\varepsilon^r\frac{n(0,f)}t\ddt-\frac1{2\uppi}\int_0^{2\uppi}\log\sqrt{1+\abs{f\qty(\varepsilon\ee^{\ii\theta})}^2}=-k\log r-\log\abs{c}+\order{\varepsilon}\\\to n(0,f)\log r-\log\abs{c}\qq{as}\varepsilon\to 0.
\end{multline*}
Hence, from \cref{eq:ahlforsshimizuderivation_convergentintegral}, we have
\begin{equation}
    \int_0^r\frac{A(t,f)\ddt}t=N(r,f)-\log\abs{c}+\frac1{2\uppi}\int_0^{2\uppi}\log\sqrt{1+\abs{f\qty(r\ee^{\ii\theta})}^2}\dd{\theta}\qif f(0)=\infty.\label{eq:ahlforsshimizuderivation_singularcase}
\end{equation}
For each \(a\in\extcomplex\), let \(W=\tfrac{1+\overline{a}w}{w-a}\) (for \(a=\infty\), \(W\equiv w\)), where \(w=f(z)\), and denote the function \(W=F(z)\). Set using limits on \(w\) and \(a\): \[k(w,a)=\frac1{\sqrt{1+\abs{W}^2}}=\frac{\abs{w-a}}{\sqrt{\abs{w-a}^2+\abs{1+\overline{a}w}^2}}=\frac{\abs{w-a}}{\sqrt{\qty(1+\abs{w}^2)\qty(1+\abs{a}^2)}}.\] For \(w=\infty\), taking the corresponding limit then gives
\[k(\infty,a)=\frac{1}{\sqrt{1+\abs{a}^2}}.\]
Moreover, \[\abs{\dv{W}{z}}=\frac{\abs{a}^2+1}{\abs{w-a}^2}\abs{\dv{w}{z}}.\]
Therefore, \[\frac{1}{1+\abs{W}^2}\abs{\dv{W}{z}}=\frac{\abs{a}^2+1}{\abs{w-a}^2}k(w,a)^2\abs{\dv{w}{z}}=\frac{1}{1+\abs{w}^2}\abs{\dv{w}{z}}.\]
Substituting in their corresponding functions, it follows that \[f^\sharp\equiv F^\sharp\implies A(r,f)=A(r,F).\]
Let \[T_0(r,f)=T_0(r,F)=\int_0^r\frac{A(t,f)\ddt}t=\int_0^r \frac{A(t,F)\ddt}t,\] which is known as the \textit{Ahlfors--Shimizu characteristic function} of \(f\) and \(F\).
Observe that if \(w=f(z)\) attains \(a\) with order \(k\), then \(1+\overline{a}f(z)=1+\abs{a}^2\geq 1\), and \(f(z)-a\) has a zero of multiplicity \(k\). Hence, \(F\) has a pole at \(z\) of order \(k\). More importantly,
\begin{equation}
    N(r,F)=N(r,a,f).\label{eq:ahlforsshimizuriemannsphererotationcountingfunction}
\end{equation}
Define the \textit{Ahlfors--Shimizu proximity function} to be equal to:
\begin{equation}
    m_0(r,F)=\frac1{2\uppi}\mathop{\mathmakebox[\widthof{\(\int_{\uppi}\)}][l]{\int_0^{2\uppi}}}\log\sqrt{1+\abs{F\qty(r\ee^{\ii\theta})}^2}\dd{\theta}=m_0(r,a,f)=\frac1{2\uppi}\mathop{\mathmakebox[\widthof{\(\int_{\uppi}\)}][l]{\int_0^{2\uppi}}}\log\qty(\frac1{k\qty(f\qty(r\ee^{\ii\theta}),a)})\dd{\theta}\label{eq:ahlforsshimizuproximity}
\end{equation}
Assume \(F(0)\neq\infty\), or when \(f(0)\neq a\), Applying \cref{eq:ahlforsshimizuderivation_regularcase} on the function \(F\), we have
\begin{equation}
    T_0(r,F)=N(r,F)+m_0(r,F)-m_0(0,F).\label{eq:ahlforsshimizufirstfundamentaltheorem_regularcase}
\end{equation}
If \(a\neq\infty\) and \(f(z)=a+c_kz^k+c_{k+1}z^{k+1}+\cdots\) (\(k\geq1\), \(c=c_k\)), since \(F(z)=\tfrac{1+\abs{a}^2}{cz^k}+\order{\tfrac1{z^{k-1}}}\), applying \cref{eq:ahlforsshimizuderivation_singularcase} on \(F\) yields
\begin{equation}
    T_0(r,F)=N(r,F)+m_0(r,F)-\log\abs{\frac{1+\abs{a}^2}{c}}.\label{eq:ahlforsshimizufirstfundamentaltheorem_singularcase}
\end{equation}
\begin{theorem}[First Fundamental Theorem in Ahlfors--Shimizu Form]\label{thm:nevanlinnafirstfundamentaltheoremahlforsshimizu}
    Let \(f\) be meromorphic on \(D(0,R)\) (where \(0<R\leq\infty\)). For \(0<r<R\) and \(a\in\extcomplex\) such that \(f(0)\neq a\),
    \[T_0(r,f)=N(r,a,f)+m_0(r,a,f)-m_0(0,a,f).\]
    If \(f(0)=a\neq\infty\), then \[T_0(r,f)=N(r,a,f)+m_0(r,a,f)-\log\abs{\frac{1+\abs{a}^2}{c}},\] where \(c\) is the first nonzero coefficient of the Laurent expansion of \(f-a\). If \(f(0)=a=\infty\), then \[T_0(r,f)=N(r,a,f)+m_0(r,a,f)-\log\abs{c}\] where \(c\) is the first non-zero coefficient of the Laurent series of \(f\).
\end{theorem}
\begin{proof}
    The first two cases follow from substituting \cref{eq:ahlforsshimizuriemannsphererotationcountingfunction,eq:ahlforsshimizuproximity} into \cref{eq:ahlforsshimizufirstfundamentaltheorem_regularcase,eq:ahlforsshimizufirstfundamentaltheorem_singularcase}.

    The final case is simply a rewriting of \cref{eq:ahlforsshimizuderivation_singularcase}.
\end{proof}
\begin{theorem}
    Let \(f:D(0,R)\to\extcomplex\) be meromorphic. Then for each \(0<r<R\), the Nevanlinna characteristic \(T\) and the Ahlfors--Shimizu characteristic \(T_0\) differ by a term that is uniformly bounded in \(r\).
\end{theorem}
\begin{proof}
    Observe that for \(z\in\partial D(0,r)\), \[\log^+\abs{f(z)}\leq\log\sqrt{1+\abs{f(z)}^2}=\frac12\log\qty(1+\abs{f(z)}^2).\]
    Since \(1+x^2\leq2\max\qty{1,x^2}\) for any real \(x\), \[\frac12\log\qty(1+\abs{f(z)}^2)\leq\frac12\log2+\log^+\abs{f(z)}.\] Integrating and adding \(N\), we have \[T(r,f)\leq m_0(r,f)+N(r,f)\leq\frac12\log 2+T(r,f).\]
    By the First Fundamental Theorem (\cref{thm:nevanlinnafirstfundamentaltheoremahlforsshimizu}) with \(a=\infty\), let \[C=
        \begin{cases}
            m_0(0,a,f)&\qif*f(0)\neq\infty\\
            \log\abs{c}
    \end{cases},\] where \(c\) is the first nonzero coefficient of the Laurent expansion of \(f\).
    It follows that \[T(r,f)-C\leq T_0(r,f)\leq T(r,f)+\frac12\log2-C,\] where \(C\) does not depend on \(R\).
\end{proof}
Let \(S\) be the Riemann sphere but centered at \(\qty(0,0,\tfrac12)\), diameter 1, with a stereographic projection with center \((0,0,1)\). Letting \[\symbf{\sigma}\qty(w)=\frac1{\abs{w}^2+1}\qty(\Re{w},\Im{w},\abs{w}^2),\]
for two points \(w_1,w_2\in\extcomplex\), their spherical points are \(\symbf{\sigma}\qty(w_1),\symbf{\sigma}\qty(w_2)\). The Euclidean distance between the two points on the sphere is verifiable (after manual simplification) to be \(k\qty(w_1,w_2)\). Thus, \(k\) is the \textit{chordal} distance function, and is thus geometrically invariant under rotations of the sphere. The mapping \(W\) of \(w\) is precisely such a transformation: geometrically it rotates points on the Riemann sphere. 

Let \(w=u+\ii v\) be a point in the projected plane and let \(\dd{u}\dd{v}\) be an area element on the plane. Then the corresponding area element on the sphere is equal to:
\begin{align*}
    \dd{A}_S&=\norm{\symbf{\sigma}'_u\times\symbf{\sigma}'_v}\dd{u}\dd{v}=\frac{\dd{u}\dd{v}}{\qty(\abs{w}^2+1)^4}\norm{\mqty(-u^2+v^2+1\\-2uv\\2u)\times\mqty(-2uv\\u^2-v^2+1\\2v)}\\
    &=\frac{\dd{u}\dd{v}}{\qty(\abs{w}^2+1)^4}\norm{\qty(-2uv^2-2u^3-2u,-2u^2v-2v^3-2v,1-\qty(u^2+v^2)^2)}\\
    &=\frac{\dd{u}\dd{v}}{\qty(\abs{w}^2+1)^3}\norm{\qty(-2u,-2v,1-\abs{w}^2)}=\frac{\dd{u}\dd{v}}{\qty(\abs{w}^2+1)^3}\sqrt{2\abs{w}^2+1+\abs{w}^4}\\
    &=\frac{\dd{u}\dd{v}}{\qty(\abs{w}^2+1)^2}
\end{align*}
The surface area of a surface \(E\subseteq S\) is given by \[\int_E\dd{A}_S=\int_{\symbf{\sigma}^{-1}(E)}\frac{\dd{u}\dd{v}}{\qty(\abs{w}^2+1)^2}.\]
For \(w=f(z)\), pulling back to \(z\) gives \[\uppi A(r)=\int_{\symbf{\sigma}(f\qty(D(0,r)))}\dd{A},\] or the area of the image of \(f\) on the Riemann sphere, counted according to multiplicities (referring to overlaps when not univalent). 

This yields another explanation for \cref{thm:nevanlinnafirstfundamentaltheoremahlforsshimizu}. This area is invariant under rotations of \(S\). In particular, we rotate the north pole \((0,0,1)\) corresponding to \(\infty\) to the point corresponding to \(a\).