\subsection{The Ahlfors--Shimizu Characteristic}
We now provide a second formulation of the first fundamental theorem, given by Ahlfors and Shimizu, found independently of each other.
\begin{lemma}\label{lem:ahlforsshimizugreens}
    Let \(U\) be a positively oriented bounded region by a piecewise \(C^1\) simple closed boundary. Let \(f\) be a (not identically 0) holomorphic function on an open neighborhood of \(\overline{U}\), \(G\in C^2\qty(\cbraces{\abs{f(z)}}{z\in\overline{U}})\). Then
    \begin{equation}
        I=\oint_{\partial U}\nabla_{\vu{n}}G\qty(\abs{f(z)})\abs{\ddz}=\int_Ug\qty(\abs{f(z)})\abs{f'(z)}^2r\dd{r}\wedge\dd{\theta},\label{eq:ahlforsshimizugreens_statement}
    \end{equation}
    where \(z=r\ee^{\ii\theta}\), \(g(R)=G''(R)+\tfrac1RG'(R)\), and \(\vu{n}\) is the unit normal vector pointing towards the exterior of \(\partial U\).
\end{lemma}
\begin{proof}
    Let the zeros of \(f\) in \(U\) (which are finite, otherwise they accumulate) be \(\qty{z_k}_{k=1}^n\). Choose \(\varepsilon'>0\) so that for any \(0<\varepsilon<\varepsilon'\) the disks \(\qty{D\qty(z_k,\varepsilon)}_{k=1}^n\) each line in the open set \(U\) and are pairwise disjoint from one another.

    Applying Green's Theorem (\cref{thm:realgreen})\footnote{We define \(\vu{n}\) to be the normal vector pointing into the region, so for the \(\partial U\) integral, \(\vu{n}\) points outwards and for the summation integrals \(\vu{n}\) points into the disks.} \[\qty(\oint_{\partial U}+\sum_{k=1}^n\oint_{\partial D\qty(z_k,\varepsilon)})\vb{v}\cdot\vu{n}\dd{s}=\iint_{U\setminus\bigcup D\qty(z_k,\varepsilon)}\divergence{\vb{v}}\dd{A}\] to \(I\), we have
    \begin{equation}
        I=\qty(\oint_{\partial U}+\sum_{k=1}^n\oint_{\partial D\qty(z_k,\varepsilon)})\grad{G\qty(\abs{f(z)})}\cdot\vu{n}\abs{\ddz}=\mathop{\mathmakebox[\widthof{\(\iint\)}][l]{\iint_{U\setminus\bigcup D\qty(z_k,\varepsilon)}}}\laplacian{G\qty(\abs{f(z)})}r\dd{r}\dd{\theta}.\label{eq:ahlforsshimizugreens_greensapplication}
    \end{equation}
    Letting \(R=\abs{f(z)}=\sqrt{f(z)\overline{f(z)}}\) along with the notation \(\partial_z=\pdv*{z},\partial_{\overline{z}}=\pdv*{\overline{z}}\) etc.,
    \[\partial_zR=\frac{\partial_z\qty(f\overline{f})}{2R}=\frac{f'\overline{f}+f\overline{\partial_{\overline{z}}f}}{2R}=\frac{f'\overline{f}}{2R},\qquad\partial_{\overline{z}}R=\frac{\partial_{\overline{z}}\qty(f\overline{f})}{2R}=\frac{f\overline{f'}}{2R},\]
    and
    \begin{align*}
        \qty(\partial_zR)\qty(\partial_{\overline{z}}R)&=\frac{\abs{f}^2\abs{f'}^2}{4R^2}=\frac{\abs{f'}^2}4,&\partial_{z\overline{z}}R&=\partial_{\overline{z}}\qty(\frac{f'\overline{f}}{2R})=\frac{2f'\overline{f'}R-2f'\overline{f}\qty(\frac{f\overline{f'}}{2R})}{4R^2}\\
        &&&=\frac{2\abs{f'}^2R^2-\abs{f'f}^2}{4R^3}=\frac{\abs{f'}^2}{4R},
    \end{align*}
    since
    \begin{align*}
        \partial_{\overline{z}}G(R)&=G'(R)\partial_{\overline{z}}R,&\quad\partial_z\partial_{\overline{z}} G(R)&=\partial_zG'(R)\partial_{\overline{z}}R+G'(R)\partial_{\overline{z}z}R\\
        \partial_zG'(R)&=G''(R)\partial_zR&&=G''(R)\frac{f'\overline{f}}{2R}\cdot\frac{f\overline{f'}}{2R}+G'(R)\frac{\abs{f'}^2}{4R}\\
        &=\frac{\abs{f'}^2}4\qty(G''(R)+\frac{G'(R)}{R}),
    \end{align*}
    it follows that \(\laplacian{G\qty(\abs{f(z)})}=g(R)\abs{f'}^2\). Substituting this in \cref{eq:ahlforsshimizugreens_greensapplication} gives \[\qty(\oint_{\partial U^+}-\sum_{k=1}^n\oint_{\partial D\qty(z_k,\varepsilon)})\grad{G\qty(\abs{f(z)})}\cdot\vu{n}\abs{\ddz}=\mathop{\mathmakebox[\widthof{\(\iint\)}][l]{\iint_{U\setminus\bigcup D\qty(z_k,\varepsilon)}}}\laplacian{G\qty(\abs{f(z)})}r\dd{r}\dd{\theta}.\]
    As \(\varepsilon\to 0^+\), the right-hand side is simply the desired quantity in \cref{eq:ahlforsshimizugreens_statement}. By the continuous differentiability of \(G\), \(\exists M>0\) such that \(\abs{\grad{G}\cdot\vu{n}}\leq M\) on \(\textstyle\bigcup_k\overline{D\qty(z_k,\varepsilon)}\), thus
    \[\abs{\oint_{\partial D\qty(z_k,\varepsilon)}\grad{G}\cdot\vu{n}\dd{s}}\leq\int_0^{2\uppi}M\varepsilon\dd\theta\to 0\qq{as}\varepsilon\to 0^+.\]
    Thus, the left-hand side expression of \cref{eq:ahlforsshimizugreens_greensapplication} also tends to that of \cref{eq:ahlforsshimizugreens_statement}.
\end{proof}
We apply \cref{lem:ahlforsshimizugreens} to \[G(R)=\log\sqrt{1+R^2}\implies g(R)=\frac{1-R^2}{\qty(1+R^2)^2}+\frac1R\qty(\frac{R}{1+R^2})=\frac{2}{\qty(1+R^2)^2}.\]
Let \(f\) be meromorphic on \(\overline{D(0,r)}\) with no poles on the boundary, and let its poles in the interior be at each \(\qty{b_j}_{j=1}^n\) of respective orders \(\qty{k_j}_{j=1}^n\). Let \(\varepsilon'>0\) such that \(\forall 0<\varepsilon<\varepsilon'\), the disks \(\overline{D\qty(b_j,\varepsilon)}\) are disjoint and all lie in \(D(0,r)\). We aim to apply the lemma on this multiply connected region on which \(f\) is holomorphic.

In a prescribed small disk containing a pole (fix \(j\)), \(f\) has the Laurent series \[f(z)=\sum_{i=-k_j}^\infty c_i\qty(z-b_j)^i=\frac{\phi(z)}{\qty(z-b_j)^{k_j}},\] where \(\phi\) is analytic and non-vanishing in the closed disk (choose \(\varepsilon\) such that \(\frac1f\) is analytic therein too). On the boundary \(\partial D\qty(b_j,\varepsilon)\), \[\abs{f(z)}=\abs{\phi(z)}\varepsilon^{-k_j}.\]
This implies that \[G(\abs{f(z)})=\log\abs{f(z)}+\log\sqrt{1+\frac1{\abs{f(z)}^2}}=\log\abs{\phi(z)}-k_j\log\varepsilon+\log\sqrt{1+\frac1{\abs{f(z)}^2}}.\]
On the contour defined by \(\partial D(b_j,\varepsilon)\)\footnote{The outwards normal vector actually points directly into the interior, since the disk is removed from the pertinent region.},
\[\nabla_{\vu{n}}G(\abs{f(z)})=-\pdv{}{\varepsilon}G(\abs{f(z)})=\frac{k_j}{\varepsilon}-\frac{\pdv{\varepsilon}\abs{\phi(z)}}{\abs{\phi(z)}}+\frac{\pdv{\varepsilon}\qty(\frac{1}{\abs{f(z)}^2})}{2+\frac2{\abs{f(z)}^2}}=\frac{k_j}{\varepsilon}+\order{1}\] as \(\varepsilon\to 0^+\), where \(\order{1}\rightrightarrows0\) uniformly in \(t\) (\(z=b_j+\varepsilon\ee^{\ii t}\)).
By the lemma, {
    \newlength{\intlength}%
    \newlength{\minuslength}%
    \setlength{\intlength}{\widthof{\(\oint_{\partial D(0,r)}\)}}%
    \setlength{\minuslength}{\widthof{\(-\)}}%
    \[\qty(\mathmakebox[\dimexpr\intlength-\minuslength\relax][l]{\oint_{\partial D(0,r)}}+\sum_{j=1}^n\oint_{\partial D\qty(b_j,\varepsilon)})\nabla_{\vu{n}}{G\qty(\abs{f(z)})}\abs{\ddz}=\iint_{D(0,r)\setminus\bigcup D\qty(b_j,\varepsilon)}{\frac{2\abs{f'\qty(z)}^2}{\qty(1+\abs{f\qty(z)}^2)^2}}\dd{A}.\]
}
which requires the computation \[\oint_{\partial D\qty(b_j,\varepsilon)}\qty(\frac{k_j}\varepsilon+\order{1})\abs{\ddz}=\int_0^{2\uppi}\qty(k_j+\varepsilon\order{1})\dd{\theta}=2\uppi k_j+\order{\varepsilon}.\]
Taking the limit \(\varepsilon\to 0^+\) hence gives \[\frac{1}{2\uppi}\int_0^{2\uppi}\nabla_{\vu{n}}{G\qty(\abs{f(z)})}\abs{\ddz}+n(r,f)=\frac1{4\uppi}\int_0^{2\uppi}\int_0^r{f^\sharp\qty(\rho\ee^{\ii\theta})^2}\rho\dd{\rho}\dd{\theta}.\]