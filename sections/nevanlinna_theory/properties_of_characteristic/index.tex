\subsection{Properties of \texorpdfstring{\(T(r,f)\)}{the Nevanlinna Characteristic} and the First Fundamental Theorem}
\begin{proposition}\label{prop:nevanlinnathreefunctionsmiscproperties}
    If \(f_k\) (\(k=1,\dots,n\)) are meromorphic in \(D(0,R)\), then for \(0<r<R\),
    \begin{enumerate}
        \item \(\textstyle N\qty(r,\prod_{k=1}^nf_k)\leq\sum_{k=1}^n N\qty(r,f_k)\).\label{itm:nevanlinnathreefunctionsmiscproperties_Nprodsubadd}
        \item \(\textstyle N\qty(r,\sum_{k=1}^nf_k)\leq\sum_{k=1}^n N\qty(r,f_k)\) (subadditivity).\label{itm:nevanlinnathreefunctionsmiscproperties_Nsubadd}
        \item \(\textstyle m\qty(r,\prod_{k=1}^nf_k)\leq\sum_{k=1}^nm\qty(r,f_k)\).\label{itm:nevanlinnathreefunctionsmiscproperties_mprodsubadd}
        \item \(\textstyle m\qty(r,\sum_{k=1}^nf_k)\leq\log n+\sum_{k=1}^nm\qty(r,f_k)\).\label{itm:nevanlinnathreefunctionsmiscproperties_msubaddconst}
        \item \(\textstyle T\qty(r,\prod_{k=1}^nf_k)\leq\sum_{k=1}^nT\qty(r,f_k)\).\label{itm:nevanlinnathreefunctionsmiscproperties_Tprodsubadd}
        \item \(\textstyle T\qty(r,\sum_{k=1}^nf_k)\leq\log n+\sum_{k=1}^nT\qty(r,f_k)\).\label{itm:nevanlinnathreefunctionsmiscproperties_Tsubaddconst}
    \end{enumerate}
\end{proposition}
\begin{proof}
    Let \(z\) be an arbitrary point in the disk. If \(m_k\) is the pole order of \(f_k\) at \(z\) (0 if it is not a pole), then the pole order at \(z\) of \(\textstyle\sum_kf_k\) does not exceed \(\textstyle\max_k m_k\). It is hence clear that \(n(r,\textstyle\sum_kf_k)\leq\textstyle\sum_kn\qty(r,f_k)\).

    Moreover, if \(\textstyle \qty{b_k^{(j)}}_j\) are the poles of \(f_k\) in the disk counted according to orders, and \(\textstyle\qty{b^{(j)}}_j\) are the poles of \(\textstyle\sum_kf_k\), then \[\sum_{k=1}^n\sum_j\log\abs{\frac r{b_k^{(j)}}}\geq\sum_j\log\abs{\frac r{b^{(j)}}}.\]
    Recognizing that
    \begin{gather*}
        N\qty(r,\sum_kf_k)=\sum_j\log\abs{\frac r{b^{(j)}}}+n\qty(0,\sum_kf_k)\log r,\\
        \sum_{k=1}^n N\qty(r,f_k)=\sum_{k=1}^n\sum_j\log\abs{\frac r{b_k^{(j)}}}+\sum_kn\qty(0,f_k)\log r,
    \end{gather*} \cref{itm:nevanlinnathreefunctionsmiscproperties_Nsubadd} then follows. Since the pole order at \(z\) of \(\textstyle\prod_k f_k\) does not exceed \(\textstyle\sum_k m_k\), it also follows that \(n(r,\textstyle\prod_kf_k)\leq\textstyle\sum_kn\qty(r,f_k)\). The same logic can be applied henceforth to prove \cref{itm:nevanlinnathreefunctionsmiscproperties_Nprodsubadd}.

    \Cref{itm:nevanlinnathreefunctionsmiscproperties_mprodsubadd,itm:nevanlinnathreefunctionsmiscproperties_msubaddconst} are clear from \cref{prop:lognonnegativepartproperties}. \Cref{itm:nevanlinnathreefunctionsmiscproperties_Tprodsubadd,itm:nevanlinnathreefunctionsmiscproperties_Tsubaddconst} follow the previous inequalities.
\end{proof}
\begin{theorem}\label{thm:nevanlinnacartanidentity}
    Let \(f:D(0,R)\to\extcomplex\) be meromorphic such that \(\textstyle f(z)=cz^k+\order{z^{k+1}}\). For \(0<r<R\),
    \begin{align*}
        T\qty(r,f)&=\frac1{2\uppi}\int_0^{2\uppi}N\qty(r,\ee^{\ii\theta},f)\dd{\theta}+\frac1{2\uppi}\int_0^{2\uppi}\log\abs{c_\theta}\dd{\theta}\\
        &=\frac1{2\uppi}\int_0^{2\uppi}N\qty(r,\ee^{\ii\theta},f)\dd{\theta}+
        \begin{cases}
            \log\abs{c}&\qif*{k<0}\\
            \logp \abs{c}&\qif*{k=0}\\
            0&\qif*k>0,
        \end{cases}
    \end{align*} where \(c_\theta\) is the first nonzero coefficient of the Laurent expansion of \(f-\ee^{\ii\theta}\).
\end{theorem}
\begin{proof}
    For \(\theta\in[0,2\uppi]\), from \cref{eq:nevanlinnacountingjensensformulaexposition2} on \(f-\ee^{\ii\theta}\) (let \(c_\theta\) be the first nonzero coefficient of the Laurent series) we obtain \[\log\abs{c_\theta}=N(r,f)-N\qty(r,\ee^{\ii\theta},f)+\frac1{2\uppi}\int_0^{2\uppi}\log\abs{f\qty(r\ee^{\ii\phi}-e^{\ii\theta})}\dd{\phi}.\]
    Therefore, by integrating in \(\theta\),
    \begin{align}
        \frac1{2\uppi}\int_0^{2\uppi}\log\abs{c_\theta}\dd{\theta}&=N(r,f)-\frac1{2\uppi}\int_0^{2\uppi}N\qty(r,\ee^{\ii\theta},f)\dd{\theta}\nonumber\\
        &\quad+\frac1{2\uppi}\int_0^{2\uppi}\qty[\int_0^{2\uppi}\log\abs{f\qty(r\ee^{\ii\phi}-e^{\ii\theta})}\dd{\phi}]\dd{\theta}.\label{eq:nevanlinnacartanidentity_intermediate}
    \end{align}
    For any \(w\in\mathbb{C}^*\),
    \begin{equation}
        \logp \abs{w}=\frac1{2\uppi}\int_0^{2\uppi}\log\abs{w-\ee^{\ii\theta}}\dd{\theta},\label{eq:jensensformulalinearcase}
    \end{equation} which follows directly from Jensen's formula on the function \(z\mapsto w-z\) (by considering when \(w\in\overline{\mathbb{D}}\) and \(\abs{w}>1\)). Letting \(\phi\in[0,2\uppi]\), we have by setting \(w=f\qty(r\ee^{\ii\phi})\),
    \[\logp \abs{f\qty(r\ee^{\ii\phi})}=\frac1{2\uppi}\int_0^{2\uppi}\log\abs{f\qty(r\ee^{\ii\phi})-\ee^{\ii\theta}}\dd{\theta}.\]
    Hence, from absolute convergence we have
    \begin{align*}
        m\qty(r,f)&=\frac1{2\uppi}\int_0^{2\uppi}\qty[\frac1{2\uppi}\int_0^{2\uppi}\log\abs{f\qty(r\ee^{\ii\phi})-\ee^{\ii\theta}}\dd{\theta}]\dd{\phi}\\
        &=\frac1{2\uppi}\int_0^{2\uppi}\qty[\frac1{2\uppi}\int_0^{2\uppi}\log\abs{f\qty(r\ee^{\ii\phi})-\ee^{\ii\theta}}\dd{\phi}]\dd{\theta}.
    \end{align*}
    From \cref{eq:nevanlinnacartanidentity_intermediate}, it follows that \[T(r,f)=\frac1{2\uppi}\int_0^{2\uppi}N\qty(r,\ee^{\ii\theta},f)\dd{\theta}+\frac1{2\uppi}\int_0^{2\uppi}\log\abs{c_\theta}\dd{\theta}.\]
    If \(f(0)=\infty\), then subtracting \(\ee^{\ii\theta}\) from \(f\) does not modify the first nonzero coefficient of the Laurent series (\(c=c_\theta\)). Thus, in this case \[T(r,f)=\frac1{2\uppi}\int_0^{2\uppi}N\qty(r,\ee^{\ii\theta},f)\dd{\theta}+\log\abs{c}\]
    Otherwise, \(c_\theta=f(0)-\ee^{\ii\theta}\) unless this quantity is 0 (cancels), which can happen for at most one \(\theta\) value, which is negligible when integrated. Thus by \cref{eq:jensensformulalinearcase}, \[\frac1{2\uppi}\int_0^{2\uppi}\log\abs{c_\theta}\dd{\theta}=\frac1{2\uppi}\int_0^{2\uppi}\log\abs{f(0)-\ee^{\ii\theta}}\dd{\theta}=
        \begin{cases}
            \logp \abs{c}&\qif* f(0)\neq 0\\
            0&\text{otherwise}.
    \end{cases}\qedhere\]
\end{proof}
\begin{theorem}
    For (non-constant) meromorphic \(f\) in \(D(0,R)\), \(T(r,f)\) is an increasing convex (not necessarily strictly convex) function of \(\log r\) (\(0<r<R\)).
\end{theorem}
\begin{proof}
    Applying \(\dv*{}{\log r}\) to \cref{thm:nevanlinnacartanidentity}, we have
    \begin{align*}
        \dv{T(r,f)}{\log r}&=\frac1{2\uppi}\dv{}{\log r}\int_0^{2\uppi}N\qty(r,\ee^{\ii\theta},f)\dd{\theta}\\
        &=\frac1{2\uppi}\int_0^{2\uppi}\qty[\frac{n\qty(r,\ee^{\ii\theta},f)-n\qty(0,\ee^{\ii\theta},f)}{r}\dv{r}{\log r}+n\qty(0,\ee^{\ii\theta},f)]\dd{\theta}\\
        &=\frac1{2\uppi}\int_0^{2\uppi}n\qty(r,\ee^{\ii\theta},f)\dd{\theta}>0
    \end{align*}
    by Lebesgue's Dominated Convergence Theorem. Because the counting function \(n\) is nondecreasing in \(r\), it is also nondecreasing in \(\log r\). 
\end{proof}
\begin{theorem}[\textsc{First Fundamental Theorem of Nevanlinna Theory}]\label{thm:nevanlinnafirstfundamentaltheorem}
    Let \(f\) be (non-constant) meromorphic on \(D(0,R)\) where \(0<r\leq\infty\). Then for any \(a\in\mathbb{C}\),
    \[T(r,a,f)=T(r,f)-\log\abs{c}+\varepsilon(r,a,f),\]
    where \(\abs{\varepsilon(r,a,f)}\leq\logp \abs{a}+\log 2\) for \(0<r<R\) and \(c\) is the first nonzero coefficient of the innermost Laurent series expansion of \(f(z)-a\) about the origin.
\end{theorem}
\begin{proof}
    Since for \(z\in\partial D(0,r)\), by the properties of \(\logp \) as in \cref{prop:lognonnegativepartproperties},
    \[\logp \abs{f(z)-a}\leq\logp \qty(\abs{f(z)}+\abs{a})\leq\log 2+\logp \abs{f(z)}+\logp \abs{a}.\]
    Integrating over \(\partial D(0,r)\) implies \[m(r,f-a)\leq\log 2+m(r,f)+\logp \abs{a}.\]
    From \[\logp \abs{f(z)-a+a}\leq\logp \qty(\abs{f(z)-a}+\abs{a})\leq\log 2+\logp \abs{f(z)-a}+\logp \abs{a},\] we obtain
    \[m(r,f)\leq\log 2+m(r,f-a)+\logp \abs{a}.\]
    Letting \(\varepsilon(r,a,f)=m(r,f-a)-m(r,f)\), the two proximity-related inequalities give \[\abs{\varepsilon(r,a,f)}\leq\log 2+\logp \abs{a}.\]
    Then by \cref{prop:nevanlinnafirsttheorematzero}, and the fact that the poles of \(f\) match those of \(f-a\) (more importantly, \(N(r,f-a)=N(r,f)\)),
    \begin{align*}
        T\qty(r,0,f-a)&=N(r,0,f-a)+m(r,0,f-a)\\
        &=N(r,f-a)+m(r,f-a)-\log\abs{c}\\
        &=N(r,f)+m(r,f-a)-\log\abs{c}\\
        &=N(r,f)+m(r,f)+\varepsilon(r,a,f)-\log\abs{c}\\
        &=T(r,f)+\varepsilon(r,a,f)-\log\abs{c}.\qedhere
    \end{align*}
\end{proof}
For \(R=\infty\), we have \[N(r,a,f)+m(r,a,f)=T(r,f)+\order{1},\] which shows that \(T\) is essentially independent of the choice of \(a\), except for a bounded \(\order{1}\) term. Thus, this motivates why \(T\) is \textit{characteristic} of \(f\).

As we have seen before, it is often pedantic and somewhat annoying to account for cases where \(f(0)\) is either zero or infinity, which is a minor grievance in Nevanlinna theory. Expositions from~\cite{hayman1964meromorphic} and~\cite{holland1973introduction} tend to leave out this tedious treatment. However, here we will make this distinction whenever necessary for completeness.
\begin{corollary}
    Let \(f\) be (non-constant) meromorphic on \(D(0,R)\). Then for \(0<r\leq\infty\),
    \[\frac1{2\uppi}\int_0^{2\uppi}m\qty(r,\ee^{\ii\theta},f)\dd{\theta}\leq\log 2.\]
\end{corollary}
\begin{proof}
    By the First Fundamental Theorem (\cref{thm:nevanlinnafirstfundamentaltheorem}), we have \[T(r,f)=N\qty(r,\ee^{\ii\theta},f)+m\qty(r,\ee^{\ii\theta},f)+\log\abs{c_\theta}+\varepsilon\qty(r,\ee^{\ii\theta},f),\] where \(\abs{\varepsilon\qty(r,\ee^{\ii\theta},f)}\leq\log 2\) and \(c_\theta\) is the first nonzero coefficient of the Laurent expansion for \(f-\ee^{\ii\theta}\). Hence,
    \begin{align*}
        T(r,f)&=\frac1{2\uppi}\int_0^{2\uppi}\qty[N\qty(r,\ee^{\ii\theta},f)+m\qty(r,\ee^{\ii\theta},f)+\log\abs{c_\theta}+\varepsilon\qty(r,\ee^{\ii\theta},f)]\dd{\theta}.
    \end{align*}
    By \cref{thm:nevanlinnacartanidentity}, we thus have
    \begin{align*}
        T(r,f)&=T(r,f)-\frac1{2\uppi}\int_0^{2\uppi}\log\abs{c_\theta}\dd{\theta}+\frac1{2\uppi}\int_0^{2\uppi}m\qty(r,\ee^{\ii\theta},f)\dd{\theta}\\
        &\quad+\frac1{2\uppi}\int_0^{2\uppi}\log\abs{c_\theta}\dd{\theta}+\frac1{2\uppi}\int_0^{2\uppi}\varepsilon\qty(r,\ee^{\ii\theta},f)\dd{\theta}.
    \end{align*}
    This implies that \[\int_0^{2\uppi}m\qty(r,\ee^{\ii\theta},f)\dd{\theta}=-\int_0^{2\uppi}\varepsilon\qty(r,\ee^{\ii\theta},f)\dd{\theta}\leq2\uppi\log 2.\qedhere\]
\end{proof}
\begin{remark}
    Since \(m\) is bounded in the integrated sense, if \(T\) is generally large, then for most values, \(N\) will be nearly equal to \(T\).
\end{remark}
\begin{proposition}\label{prop:nevanlinnacharacteristicproperties}
    Let \(f,g\) be meromorphic on \(D(0,R)\) and suppose \(m\in\mathbb{N}\). For any \(0<r<R\), the following properties hold:
    \begin{enumerate}
        \item \(T\qty(r,fg)\leq T(r,f)+T(r,g)\).\label{itm:nevanlinnacharacteristicproperties_prodsubadd}
        \item \(T\qty(r,f+g)\leq T(r,f)+T(r,g)+\order{1}\).\label{itm:nevanlinnacharacteristicproperties_subadd}
        \item \(T\qty(r,\tfrac1f)=T(r,f)+\order{1}\).\label{itm:nevanlinnacharacteristicproperties_inversion}
        \item \(T\qty(r,f^n)=nT(r,f)\).\label{itm:nevanlinnacharacteristicproperties_power}
    \end{enumerate}
\end{proposition}
\begin{proof}
    Since \[N(r,fg),N(r,f+g)\leq N(r,f)+N(r,g),\]
    and by \[\logp \abs{fg}\leq\logp \abs{f}+\logp \abs{g},\qquad\logp \abs{f+g}\leq\log 2+\logp \abs{f}+\logp \abs{g},\]
    it follows that \[m(r,fg)\leq m(r,f)+m(r,g),\quad m(r,f+g)\leq m(r,f)+m(r,g)+\order{1},\] and the conclusions of \cref{itm:nevanlinnacharacteristicproperties_prodsubadd,itm:nevanlinnacharacteristicproperties_subadd}. \Cref{itm:nevanlinnacharacteristicproperties_inversion} is a corollary of the First Fundamental Theorem (\cref{thm:nevanlinnafirstfundamentaltheorem}). Lastly, because a pole of \(f^n\) has order \(n\) times of that of \(f\), it follows that \(N\qty(r,f^n)=nN(r,f)\). If \(\abs{f}\geq1\), it follows that \(\logp \abs{f^n}=n\logp \abs{f}\) exactly. If \(\abs{f}<1\), \(\logp \abs{f^n}=n\logp \abs{f}=0\). Hence, \(m\qty(r,f^n)=nm\qty(r,f)\).

    Adding the two quantities \(N\) and \(m\) together gives \cref{itm:nevanlinnacharacteristicproperties_power}.
\end{proof}
We now provide some classical examples of the Nevanlinna characteristic:
\begin{enumerate}
    \item Let \[f(z)=c\frac{z^p+a_1z^{p-1}+\cdots+a_p}{z^q+b_1z^{q-1}+\cdots+b_q},\qquad c\neq0\] be an arbitrary rational function in reduced form. If \(p>q\), then for sufficiently large \(r\), \(f\to\infty\) in every direction. This implies \(\logp \abs{\tfrac1{f(z)-a}}=0\) for any complex \(a\) and sufficiently large \(r\) (\(\abs{z}>r\)). This implies \(m(r,a,f)=0\). Additionally, the equation \(f(z)=a\) has \(p\) complex solutions by the Fundamental Theorem of Algebra (\cref{thm:fundamentaltheoremofalgebra}). Thus, for sufficiently large \(r\) (\(r>r'\)), any \(\abs{z}>r\) has the property that \(n(r,a,f)=n(\abs{z},a,f)\),
    \begin{align*}
        N(r,a,f)&=\qty(\int_0^{r'}+\int_{r'}^r)\frac{n(x,a,f)-n(0,a,f)}{x}+n(0,a,f)\log r\\
        &=\order{1}+\qty(n\qty(r',a,f)-n(0,a,f))\qty(\log r-\log{r'})+n(0,a,f)\log r\\
        &=p\log r+\order{1},\qquad a\neq\infty.
    \end{align*}
    Hence, \[T(r,f)=p\log r+\order{1}.\] Similarly, \(T\qty(r,\tfrac1f)=T\qty(r,f)+\order{1}\) by a preceding result. It then follows that if \(p<q\), 
    \[T(r,f)=q\log r+\order{1}.\]
    Lastly, the case where \(p=q\) (derived with \(a\neq c\)) is resolvable with \[T(r,f)=p\log r+\order{1}=q\log r+\order{1}.\]
    In all cases, \[T(r,f)=d\log r+\order{1},\] where \(d=\max\qty{p,q}\) is defined to be the \textit{degree of the rational function} \(f\).
    
    Unless \(f(\infty)=a\), the proximity function vanishes for sufficiently large \(r\). In this case, let \(\alpha\) be the multiplicity at which \(f\) attains \(a\) at \(z=\infty\). If \(a\) is finite, 
    \begin{gather*}
        m\qty(r,a,f)=\frac1{2\uppi}\int_0^{2\uppi}\logp \abs{\frac1{\paren{r\ee^{\ii\theta}}^{-\alpha}\order{1}}}\dd{\theta}=\alpha\log r+\order{1}\\
        N(r,a,f)=\qty(d-\alpha)\log r+\order{1}.
    \end{gather*}
    If \(f\) has a pole of order \(\alpha\) at \(\infty\), we have likewise that
    \begin{gather*}
        m\qty(r,a,f)=\alpha\log r+\order{1},\qquad N(r,a,f)=\qty(d-\alpha)\log r+\order{1}.
    \end{gather*}
    \item Consider \[\exp(z)=\exp\qty(r\cos\theta+\ii r\sin\theta),\qquad z=r\ee^{\ii\theta}.\]
    By its entireness, \(N(r,f)=0\) for any \(r>0\), and the proximity is given by \[m(r,f)=\frac1{2\uppi}\int_0^{2\uppi}\logp \abs{\ee^{r\ee^{\ii\theta}}}\dd{\theta}=\frac1{2\uppi}\qty(\int_0^{\frac{\uppi}2}+\int_{\frac{3\uppi}2}^{2\uppi})\qty(r\cos\theta)^+\dd{\theta}=\frac{r}{\uppi}.\]
    Therefore, \[T(r)=m(r,f)+N(r,f)=\frac{r}{\uppi}.\]
    Moreover, \(m(r,0,f)=\tfrac{r}{\uppi}\) and \(N(r,0,f)=0\), whereas for \(a\in\mathbb{C}^*\), the solutions to \(e^z=a\) are in the form of \(z_k=z_0+2k\uppi\ii\), where \(z_0=x_0+\ii y_0\). The condition \(\abs{z_k}\leq r\) is equivalent to \[\abs{y_0+2k\uppi}^2+\abs{x_0}^2\leq r^2\implies\abs{y_0+2k\uppi}\leq r+\order{1}.\]
    It follows that \[n(r,a,f)=\frac{r}{\uppi}+\order{1}\implies N(r,a,f)=\frac{r}{\uppi}+\order{1},\quad m(r,a,f)=\order{1}.\]
\end{enumerate}
\subimport{classifying_meromorphic_growth/}{index.tex}
\subimport{factorization_of_meromorphic_function/}{index.tex}
