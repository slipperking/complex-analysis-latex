\subsection{Properties of \texorpdfstring{\(T(r,f)\)}{the Nevanlinna Characteristic} and the First Fundamental Theorem}
\begin{proposition}\label{prop:nevanlinnathreefunctionsmiscproperties}
    If \(f_k\) (\(k=1,\dots,n\)) are meromorphic in \(D(0,R)\), then for \(0<r<R\),
    \begin{enumerate}
        \item \(\textstyle N\qty(r,\sum_{k=1}^nf_k)\leq\sum_{k=1}^n N\qty(r,f_k)\) (subadditivity).\label{itm:nevanlinnathreefunctionsmiscproperties_Nsubadd}
        \item \(\textstyle m\qty(r,\prod_{k=1}^nf_k)\leq\sum_{k=1}^nm\qty(r,f_k)\).\label{itm:nevanlinnathreefunctionsmiscproperties_mprodsubadd}
        \item \(\textstyle m\qty(r,\sum_{k=1}^nf_k)\leq\log n+\sum_{k=1}^nm\qty(r,f_k)\).\label{itm:nevanlinnathreefunctionsmiscproperties_msubaddconst}
        \item \(\textstyle T\qty(r,\prod_{k=1}^nf_k)\leq\sum_{k=1}^nT\qty(r,f_k)\).\label{itm:nevanlinnathreefunctionsmiscproperties_Tprodsubadd}
        \item \(\textstyle T\qty(r,\sum_{k=1}^nf_k)\leq\log n+\sum_{k=1}^nT\qty(r,f_k)\).\label{itm:nevanlinnathreefunctionsmiscproperties_Tsubaddconst}
    \end{enumerate}
\end{proposition}
\begin{proof}
    Let \(z\) be an arbitrary point in the disk. If \(m_k\) is the pole order of \(f_k\) at \(z\) (0 if it is not a pole), then the pole order at \(z\) of \(\textstyle\sum_kf_k\) does not exceed \(\textstyle\max_k m_k\). It is hence clear that \(n(r,\textstyle\sum_kf_k)\leq\textstyle\sum_kn\qty(r,f_k)\). Moreover, if \(\qty{b_k^{(j)}}_j\) are the poles of \(f_k\) in the disk counted according to orders, and \(\qty{b^{(j)}}_j\) are the poles of \(\textstyle\sum_kf_k\), then \[\sum_{k=1}^n\sum_j\log\abs{\frac r{b_k^{(j)}}}\geq\sum_j\log\abs{\frac r{b^{(j)}}}.\]
    Recognizing that
    \begin{gather*}
        N\qty(r,\sum_kf_k)=\sum_j\log\abs{\frac r{b^{(j)}}}+n\qty(0,\sum_kf_k)\log r,\\
        \sum_{k=1}^n N\qty(r,f_k)=\sum_{k=1}^n\sum_j\log\abs{\frac r{b_k^{(j)}}}+\sum_kn\qty(0,f_k)\log r,
    \end{gather*} \cref{itm:nevanlinnathreefunctionsmiscproperties_Nsubadd} then follows.

    \Cref{itm:nevanlinnathreefunctionsmiscproperties_mprodsubadd,itm:nevanlinnathreefunctionsmiscproperties_msubaddconst} are clear from \cref{prop:lognonnegativepartproperties}. \Cref{itm:nevanlinnathreefunctionsmiscproperties_Tprodsubadd,itm:nevanlinnathreefunctionsmiscproperties_Tsubaddconst} follow the previous inequalities.
\end{proof}
\begin{theorem}\label{thm:nevanlinnacartanidentity}
    Let \(f:D(0,R)\to\extcomplex\) be meromorphic such that \(\textstyle f(z)=cz^k+\order{z^{k+1}}\). For \(0<r<R\),
    \begin{align*}
        T\qty(r,f)&=\frac1{2\uppi}\int_0^{2\uppi}N\qty(r,\ee^{\ii\theta},f)\dd{\theta}+\frac1{2\uppi}\int_0^{2\uppi}\log\abs{c_\theta}\dd{\theta}\\
        &=\frac1{2\uppi}\int_0^{2\uppi}N\qty(r,\ee^{\ii\theta},f)\dd{\theta}+
        \begin{cases}
            \log\abs{c}&\qif*{k<0}\\
            \log^+\abs{c}&\qif*{k=0}\\
            0&\qif*k>0,
        \end{cases}
    \end{align*} where \(c_\theta\) is the first nonzero coefficient of the Laurent expansion of \(f-\ee^{\ii\theta}\).
\end{theorem}
\begin{proof}
    For \(\theta\in[0,2\uppi]\), from \cref{eq:nevanlinnacountingjensensformulaexposition2} on \(f-\ee^{\ii\theta}\) (let \(c_\theta\) be the first nonzero coefficient of the Laurent series) we obtain \[\log\abs{c_\theta}=N(r,f)-N\qty(r,\ee^{\ii\theta},f)+\frac1{2\uppi}\int_0^{2\uppi}\log\abs{f\qty(r\ee^{\ii\phi}-e^{\ii\theta})}\dd{\phi}.\]
    Therefore, by integrating in \(\theta\),
    \begin{align}
        \frac1{2\uppi}\int_0^{2\uppi}\log\abs{c_\theta}\dd{\theta}&=N(r,f)-\frac1{2\uppi}\int_0^{2\uppi}N\qty(r,\ee^{\ii\theta},f)\dd{\theta}\nonumber\\
        &\quad+\frac1{2\uppi}\int_0^{2\uppi}\qty[\int_0^{2\uppi}\log\abs{f\qty(r\ee^{\ii\phi}-e^{\ii\theta})}\dd{\phi}]\dd{\theta}.\label{eq:nevanlinnacartanidentity_intermediate}
    \end{align}
    For any \(w\in\mathbb{C}^*\),
    \begin{equation}
        \log^+\abs{w}=\frac1{2\uppi}\int_0^{2\uppi}\log\abs{w-\ee^{\ii\theta}}\dd{\theta},\label{eq:jensensformulalinearcase}
    \end{equation} which follows directly from Jensen's formula on the function \(z\mapsto w-z\) (by considering when \(w\in\overline{\mathbb{D}}\) and \(\abs{w}>1\)). Letting \(\phi\in[0,2\uppi]\), we have by setting \(w=f\qty(r\ee^{\ii\phi})\),
    \[\log^+\abs{f\qty(r\ee^{\ii\phi})}=\frac1{2\uppi}\int_0^{2\uppi}\log\abs{f\qty(r\ee^{\ii\phi})-\ee^{\ii\theta}}\dd{\theta}.\]
    Hence, from absolute convergence we have
    \begin{align*}
        m\qty(r,f)&=\frac1{2\uppi}\int_0^{2\uppi}\qty[\frac1{2\uppi}\int_0^{2\uppi}\log\abs{f\qty(r\ee^{\ii\phi})-\ee^{\ii\theta}}\dd{\theta}]\dd{\phi}\\
        &=\frac1{2\uppi}\int_0^{2\uppi}\qty[\frac1{2\uppi}\int_0^{2\uppi}\log\abs{f\qty(r\ee^{\ii\phi})-\ee^{\ii\theta}}\dd{\phi}]\dd{\theta}.
    \end{align*}
    From \cref{eq:nevanlinnacartanidentity_intermediate}, it follows that \[T(r,f)=\frac1{2\uppi}\int_0^{2\uppi}N\qty(r,\ee^{\ii\theta},f)\dd{\theta}+\frac1{2\uppi}\int_0^{2\uppi}\log\abs{c_\theta}\dd{\theta}.\]
    If \(f(0)=\infty\), then subtracting \(\ee^{\ii\theta}\) from \(f\) does not modify the first nonzero coefficient of the Laurent series (\(c=c_\theta\)). Thus, in this case \[T(r,f)=\frac1{2\uppi}\int_0^{2\uppi}N\qty(r,\ee^{\ii\theta},f)\dd{\theta}+\log\abs{c}\]
    Otherwise, \(c_\theta=f(0)-\ee^{\ii\theta}\) unless this quantity is 0 (cancels), which can happen for at most one \(\theta\) value, which is negligible when integrated. Thus by \cref{eq:jensensformulalinearcase}, \[\frac1{2\uppi}\int_0^{2\uppi}\log\abs{c_\theta}\dd{\theta}=\frac1{2\uppi}\int_0^{2\uppi}\log\abs{f(0)-\ee^{\ii\theta}}\dd{\theta}=
        \begin{cases}
            \log^+\abs{c}&\qif* f(0)\neq 0\\
            0&\text{otherwise}.
    \end{cases}\qedhere\]
\end{proof}
\begin{theorem}
    For (non-constant) meromorphic \(f\) in \(D(0,R)\), \(T(r,f)\) is an increasing function of \(\log r\) (\(0<r<R\)).
\end{theorem}
\begin{proof}
    Applying \(\dv*{}{\log r}\) to \cref{thm:nevanlinnacartanidentity}, we have
    \begin{align*}
        \dv{T(r,f)}{\log r}&=\frac1{2\uppi}\dv{}{\log r}\int_0^{2\uppi}N\qty(r,\ee^{\ii\theta},f)\dd{\theta}\\
        &=\frac1{2\uppi}\int_0^{2\uppi}\qty[\frac{n\qty(r,\ee^{\ii\theta},f)-n\qty(0,\ee^{\ii\theta},f)}{r}\dv{r}{\log r}+n\qty(0,\ee^{\ii\theta},f)]\dd{\theta}\\
        &=\frac1{2\uppi}\int_0^{2\uppi}n\qty(r,\ee^{\ii\theta},f)\dd{\theta}>0.
    \end{align*}
    by Lebesgue's Dominated Convergence Theorem.
\end{proof}
\begin{theorem}[First Fundamental Theorem of Nevanlinna Theory]\label{thm:nevanlinnafirstfundamentaltheorem}
    Let \(f\) be (non-constant) meromorphic on \(D(0,R)\) where \(0<r\leq\infty\). Then for any \(a\in\mathbb{C}\),
    \[T(r,a,f)=T(r,f)-\log\abs{c}+\varepsilon(r,a,f),\]
    where \(\abs{\varepsilon(r,a,f)}\leq\log^+\abs{a}+\log 2\) for \(0<r<R\) and \(c\) is the first nonzero coefficient of the innermost Laurent series expansion of \(f(z)-a\) about the origin.
\end{theorem}
\begin{proof}
    Since for \(z\in\partial D(0,r)\), by the properties of \(\log^+\) as in \cref{prop:lognonnegativepartproperties},
    \[\log^+\abs{f(z)-a}\leq\log^+\qty(\abs{f(z)}+\abs{a})\leq\log 2+\log^+\abs{f(z)}+\log^+\abs{a}.\]
    Integrating over \(\partial D(0,r)\) implies \[m(r,f-a)\leq\log 2+m(r,f)+\log^+\abs{a}.\]
    From \[\log^+\abs{f(z)-a+a}\leq\log^+\qty(\abs{f(z)-a}+\abs{a})\leq\log 2+\log^+\abs{f(z)-a}+\log^+\abs{a},\] we obtain
    \[m(r,f)\leq\log 2+m(r,f-a)+\log^+\abs{a}.\]
    Letting \(\varepsilon(r,a,f)=m(r,f-a)-m(r,f)\), the two proximity-related inequalities give \[\abs{\varepsilon(r,a,f)}\leq\log 2+\log^+\abs{a}.\]
    Then by \cref{prop:nevanlinnafirsttheorematzero}, and the fact that the poles of \(f\) match those of \(f-a\) (more importantly, \(N(r,f-a)=N(r,f)\)),
    \begin{align*}
        T\qty(r,0,f-a)&=N(r,0,f-a)+m(r,0,f-a)\\
        &=N(r,f-a)+m(r,f-a)-\log\abs{c}\\
        &=N(r,f)+m(r,f-a)-\log\abs{c}\\
        &=N(r,f)+m(r,f)+\varepsilon(r,a,f)-\log\abs{c}\\
        &=T(r,f)+\varepsilon(r,a,f)-\log\abs{c}.\qedhere
    \end{align*}
\end{proof}
For \(R=\infty\), we have \[N(r,a,f)+m(r,a,f)=T(r,f)+\order{1},\] which shows that \(T\) is essentially independent of the choice of \(a\), except for a bounded \(\order{1}\) term. Thus, this motivates why \(T\) is \textit{characteristic} of \(f\).

As we have seen before, it is often pedantic and somewhat annoying to account for cases where \(f(0)\) is either zero or infinity, which is a minor grievance in Nevanlinna theory. Expositions in~\cite{hayman1964meromorphic} and~\cite{holland1973introduction} tend to leave out this tedious treatment. However, here we will make this distinction whenever necessary for completeness.
\begin{corollary}
    Let \(f\) be (non-constant) meromorphic on \(D(0,R)\). Then for \(0<r\leq\infty\),
    \[\frac1{2\uppi}\int_0^{2\uppi}m\qty(r,\ee^{\ii\theta},f)\dd{\theta}\leq\log 2.\]
\end{corollary}
\begin{proof}
    By the First Fundamental Theorem (\cref{thm:nevanlinnafirstfundamentaltheorem}), we have \[T(r,f)=N\qty(r,\ee^{\ii\theta},f)+m\qty(r,\ee^{\ii\theta},f)+\log\abs{c_\theta}+\varepsilon\qty(r,\ee^{\ii\theta},f),\] where \(\abs{\varepsilon\qty(r,\ee^{\ii\theta},f)}\leq\log 2\) and \(c_\theta\) is the first nonzero coefficient of the Laurent expansion for \(f-\ee^{\ii\theta}\). Hence,
    \begin{align*}
        T(r,f)&=\frac1{2\uppi}\int_0^{2\uppi}\qty[N\qty(r,\ee^{\ii\theta},f)+m\qty(r,\ee^{\ii\theta},f)+\log\abs{c_\theta}+\varepsilon\qty(r,\ee^{\ii\theta},f)]\dd{\theta}.
    \end{align*}
    By \cref{thm:nevanlinnacartanidentity}, we thus have
    \begin{align*}
        T(r,f)&=T(r,f)-\frac1{2\uppi}\int_0^{2\uppi}\log\abs{c_\theta}\dd{\theta}+\frac1{2\uppi}\int_0^{2\uppi}m\qty(r,\ee^{\ii\theta},f)\dd{\theta}\\
        &\quad+\frac1{2\uppi}\int_0^{2\uppi}\log\abs{c_\theta}\dd{\theta}+\frac1{2\uppi}\int_0^{2\uppi}\varepsilon\qty(r,\ee^{\ii\theta},f)\dd{\theta}.
    \end{align*}
    This implies that \[\int_0^{2\uppi}m\qty(r,\ee^{\ii\theta},f)\dd{\theta}=-\int_0^{2\uppi}\varepsilon\qty(r,\ee^{\ii\theta},f)\dd{\theta}\leq2\uppi\log 2.\qedhere\]
\end{proof}
\begin{remark}
    Since \(m\) is bounded in the integrated sense, if \(T\) is generally large, then for most values, \(N\) will be nearly equal to \(T\).
\end{remark}
\begin{proposition}

\end{proposition}
\subimport{classifying_growth_factorization/}{index.tex}
