\subsubsection{The Factorization of Transcendental Meromorphic Functions}
We now give a meromorphic analog of the Hadamard Factorization Theorem:
\begin{theorem}
    Let \(f:\mathbb{C}\to\extcomplex\) be non-constant and meromorphic of finite order \(\rho\) such that \(f(0)\neq 0,\infty\). If \(\qty{a_n}_{n=1}^\infty\) and \(\qty{b_n}_{n=1}^\infty\) are its zeros and poles in \(\mathbb{C}^*\) listed in order of nondecreasing moduli, counting multiplicities and zeros, then the products \[P_1(z)=\prod_{n=1}^\infty E_{\floor{\rho}}\qty(\frac{z}{a_n}),\quad P_2(z)=\prod_{n=1}^\infty E_{\floor{\rho}}\qty(\frac{z}{b_n})\] locally uniformly converge in \(\mathbb{C}\). Then \(f(z)=\ee^{\varphi(z)}\flatfrac{P_1(z)}{P_2(z)}\), where \(\varphi\) is a polynomial of degree \(\leq\rho\).
\end{theorem}
\begin{proof}
    By the convergence of \(\textstyle\sum_n\frac1{a_n^{\rho+\varepsilon}}\) and \(\textstyle\sum_n\frac1{b_n^{\rho+\varepsilon}}\) given by \cref{thm:meromorphicfunctionfiniteorderestimatessum}, the two products converge by the Weierstrass Product Theorem (\cref{thm:weierstrassproduct}).

    Let \(f_1(z)=f(z)P_2(z)\), which is entire. Since \(N\qty(r,P_2)\equiv 0\), we have for sufficiently large \(r\) that
    \begin{align*}
        T\qty(r,P_2)&=m(r,P_2)\leq\log^+ M\qty(r,P_2)=\log M\qty(r,P_2).
    \end{align*}
    % prove P2 order
\end{proof}
