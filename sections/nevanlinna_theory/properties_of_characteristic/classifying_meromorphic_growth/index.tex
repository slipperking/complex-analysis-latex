\subsubsection{Classifying Growth of Meromorphic Functions}
\begin{theorem}\label{thm:nevanlinnaentirefunctionmaximummodulussandwich}
    Let \(f\) be holomorphic on \(\overline{D(0,R)}\) for \(R>0\) and define \(M(r,f)\) to be \(\max_{\abs{z}\leq r}\abs{f(z)}=\max_{\abs{z}=r}\abs{f(z)}\) for \(0<r<R\); then \[T(r,f)\leq\logp M(r,f)\leq\frac{R+r}{R-r}T(R,f).\]
\end{theorem}
\begin{proof}
    Since \(f\) is holomorphic in \(D(0,r)\), \(N(r,f)\equiv 0\) and hence \[T(r,f)=m(r,f)=\frac1{2\uppi}\int_0^{2\uppi}\logp \abs{f\qty(r\ee^{\ii\theta})}\dd{\theta}\leq\frac1{2\uppi}\int_0^{2\uppi}\logp M(r,f)\dd{\theta},\] which proves the first inequality. Since \(T\geq0\) trivially, if \(M\leq 1\) the second assertion holds trivially. Hence, assume \(M>1\); by the Poisson--Jensen formula (\cref{thm:poissonjensenformula}), letting \(z\in\partial D(0,r)\) be where \(\abs{f}\) attains \(M\), we have
    \begin{align*}
        \logp M(r,f)&=\frac1{2\uppi}\int_0^{2\uppi}\log\abs{f\qty(R\ee^{\ii\theta})}\frac{R^2-r^2}{\abs{R\ee^{\ii\theta}-z}}\dd{\theta}\\
        &\quad+\sum_{j=1}^m\log\abs{\frac{R\qty(z-a_j)}{R^2-\overline{a_j}z}}-\sum_{k=1}^n\log\abs{\frac{R\qty(z-b_k)}{R^2-\overline{b_k}z}}\\
        &\leq\frac1{2\uppi}\int_0^{2\uppi}\log\abs{f\qty(R\ee^{\ii\theta})}\frac{R^2-r^2}{\abs{R\ee^{\ii\theta}-z}^2}\dd{\theta}\\
        &\leq\frac1{2\uppi}\int_0^{2\uppi}\log\abs{f\qty(R\ee^{\ii\theta})}\frac{R^2-r^2}{(R-r)^2}\dd{\theta}\\
        &\leq\frac1{2\uppi}\int_0^{2\uppi}\logp \abs{f\qty(R\ee^{\ii\theta})}\frac{R+r}{R-r}\dd{\theta}=\frac{R+r}{R-r}T(R,f).
    \end{align*}
    (The Möbius transformation-like terms of the zeros \(\leq 0\) since they map to the unit disk, and the second summation vanishes since there are no \(b_k\) by holomorphy).
\end{proof}
\begin{definition}\label{def:orderofmeromorphicfunction}
    The \textit{order of a meromorphic function} \(f:\mathbb{C}\to\extcomplex\), denoted \(\rho=\rho(f)\), is given by \[\rho=\varlimsup_{r\to\infty}\frac{\logp T(r,f)}{\log r},\]
    equivalent to the condition that \(T(r,f)=\order{r^{\rho+\varepsilon}}\) for any \(\varepsilon>0\) but not for \(\varepsilon<0\).
\end{definition}
\begin{proof}[Proof of equivalent definitions]
    For any prescribed \(\varepsilon>0\) there exists \(r'>0\) such that for any \(r>r'\), \[\frac{\logp T(r,f)}{\log r}\leq\rho+\varepsilon\implies T(r,f)\leq r^{\rho+\varepsilon}.\]
    Assume there exists some \(\varepsilon<0\) such that \[T(r,f)=\order{r^{\rho+\varepsilon}}.\] Then there exist finite \(c,r'\) such that \(\forall r>r'\), \[T(r,f)\leq cr^{\rho+\varepsilon}\implies\logp T(r,f)\leq\order{1}+\qty(\rho+\varepsilon)\log r\implies\frac{\logp T(r,f)}{\log r}\leq\rho+\varepsilon<\rho,\]
    which is a contradiction. The converse follows similarly to the case of entire functions in \cref{sec:classifyinggrowthofentirefunctions}.
\end{proof}
\begin{theorem}
    Let \(\rho_M\) be the order of a non-constant entire function \(f\) as defined in \cref{sec:classifyinggrowthofentirefunctions} and let \(\rho_T\) be the order as defined in \cref{def:orderofmeromorphicfunction}. Then \(\rho_M=\rho_T\).
\end{theorem}
\begin{proof}
    Since \(f\) is non-constant, for sufficiently large \(r\), \(\logp  M(r,f)=\log M(r,f)\). For the remainder of this proof we will assume the two are equivalent. From \cref{thm:nevanlinnaentirefunctionmaximummodulussandwich} it is apparent that (under \(R=2r\) for sufficiently large \(r\)) \[\frac{\logp  T(r,f)}{\log r}\leq\frac{\logp \log M(r,f)}{\log r}\leq\frac{\log3}{\log r}+\frac{\logp  T(2r,f)}{\log(2r)}\cdot\frac{\log(2r)}{\log r}.\]
    Letting \(r\to\infty\) in the limit superior gives that \[\rho_T\leq\rho_M\leq\rho_T\implies\rho_T=\rho_M.\qedhere\]
\end{proof}
\begin{proposition}\label{prop:meromorphicfunctionfiniteorderestimates}
    Let \(f:\mathbb{C}\to\extcomplex\) be a meromorphic function of finite order \(\rho\). Then for every \(\varepsilon>0\) and every \(a\in\extcomplex\),
    \begin{enumerate}
        \item \(m(r,a,f)=\order{r^{\rho+\varepsilon}}\).\label{itm:meromorphicfunctionfiniteorderestimates_proximity}
        \item \(N(r,a,f)=\order{r^{\rho+\varepsilon}}\).\label{itm:meromorphicfunctionfiniteorderestimates_counting}
        \item \(n(r,a,f)=\order{r^{\rho+\varepsilon}}\).\label{itm:meromorphicfunctionfiniteorderestimates_discretecounting}
    \end{enumerate}
\end{proposition}
\begin{proof}
    By the First Fundamental Theorem of Nevanlinna theory (\cref{thm:nevanlinnafirstfundamentaltheorem}),
    \[m(r,a)=T(r,f)+\order{1}-N(r,a)\]
    and since \(m(r,a)\ge0\) and \(N(r,a)\ge0\), we obtain
    \[m(r,a)\le T(r,f)+\order{1}\qand N(r,a)\le T(r,f)+\order{1}.\]
    Because \(f\) has order \(\rho\), we have
    \[T(r,f)=\order{r^{\rho+\varepsilon}},\]
    which proves \cref{itm:meromorphicfunctionfiniteorderestimates_proximity,itm:meromorphicfunctionfiniteorderestimates_counting}. Since \[N(2r,a,f)-N(r,a,f)=\int_r^{2r}\frac{n(x,a,f)}{x}\ddx\geq n(r,a,f)\int_r^{2r}\frac{\ddx}x=\log2\cdot n(r,a,f),\]
    it follows from \cref{itm:meromorphicfunctionfiniteorderestimates_counting} that \[n(r,a,f)\leq\frac{N(2r,a,f)-N(r,a,f)}{\log 2}\leq\frac{N(2r,a,f)}{\log 2}=\order{\qty(2r)^{\rho+\varepsilon}}=\order{r^{\rho+\varepsilon}}\] and hence \cref{itm:meromorphicfunctionfiniteorderestimates_discretecounting} follows.
\end{proof}
\begin{theorem}\label{thm:meromorphicfunctionfiniteorderestimatessum}
    Let \(f:\mathbb{C}\to\extcomplex\) be a meromorphic function of finite order \(\rho\). For \(a\in\mathbb{C}\), let \(r_n(a)\) denote the moduli of the zeros of \(f(z)-a\) (counted with multiplicity) in non-decreasing order, and let \(r_n(\infty)\) denote the moduli of the poles of \(f\) in non-decreasing order. Then for every \(\varepsilon>0\) and every \(a\in\extcomplex\),
    \[\sum_n\frac{1}{r_n(a)^{\rho+\varepsilon}}<\infty.\]
\end{theorem}
\begin{proof}
    Observe that \[\rho+\varepsilon>\limsup_{r\to\infty}\frac{\logp T(r,f)}{\log r}\geq\limsup_{r\to\infty}\frac{\log N(r,f)}{\log r}.\]
    Since \[N(\ee r,f)-N(r,f)=\int_r^{\ee r}\frac{n(x,f)\ddx}{x}\geq\int_r^{\ee r}\frac{n(r,f)\ddx}{x}=n(r,f),\]
    it follows that \(n\leq N\), and \[\rho+\varepsilon>\limsup_{r\to\infty}\frac{\log n(r,f)}{\log r},\] and hence the conclusion follows for \(a=\infty\) by \cref{thm:nonzerosequencepowersummationconvergence}. Assume \(a\) is finite; observe that \[T(r,f)=T(r,a,f)+\order{1}=T\qty(r,\frac1{f-a})+\order{1}\] by the First Fundamental Theorem (\cref{thm:nevanlinnafirstfundamentaltheorem}). It follows that \(\rho\) is the order of \(\flatfrac1{(f-a)}\). Applying the previous result to this function, whose poles are precisely at \(\cbraces{r_n(a)}\), the conclusion follows.
\end{proof}
