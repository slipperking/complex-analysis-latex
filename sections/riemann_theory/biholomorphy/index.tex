\subsection{Biholomorphy}
In \cref{sec:conformalityintroduction}, it was asserted that for a holomorphic function \(f(z)\), the map \(w=f(z)\) is conformal when \(f'(z)\neq 0\).

We have the following immediate assertion:
\begin{theorem}[\textsc{Open Mapping Theorem}]\label{thm:openmapping}
    Suppose \(U\subseteq\mathbb{C}\) is a region (open, nonempty, and connected). Then the image of any holomorphic and non-constant function \(f:U\to\mathbb{C}\), \(f(U)\), is a region.
\end{theorem}
\begin{proof}
    The nonemptiness of \(f(U)\) is an immediate conclusion from the fact that \(U\) is nonempty and \(f\) is defined on all of \(U\).

    Let \(w_0\) be an arbitrary point in \(f(U)\). Then \(\exists z_0\in U\) such that \(f\qty(z_0)=w_0\). Since \(f\) is non-constant, the function \(f-w_0\) has an isolated zero at \(z_0\). Thus for sufficiently small \(\rho>0\), the only zero of \(f-w_0\) in \(\overline{D\qty(z_0,\rho)}\) is at \(z_0\).

    By \cref{thm:hurwitzshifts}, then there exists \(\delta>0\) such that \(\forall\varepsilon\in D(0,\delta)\), \(f(z)-w_0-\varepsilon\) has exactly one zero in \(\overline{D\qty(z_0,\rho)}\). In other words, \(\forall w_0\in f(U)\), \(\exists\delta>0\) such that \(\forall w\in D\qty(w_0,\delta)\), \(\exists! z\in\overline{D\qty(z_0,\rho)}\) such that \(f(z)=w\). Thus, \(f\qty(D\qty(w_0,\delta))\subseteq f(U)\). Thus, \(f(U)\) is an open set since each contained point has a fully contained open neighborhood.

    Let \(w_1,w_2\in f(U)\) be arbitrary and distinct. Then there exist \(z_1, z_2\in U\) such that \(f\qty(z_1)=w_1\) and \(f\qty(z_2)=w_2\). By the connectivity of \(U\), there exists a path \(\gamma\subset U\) that connects \(z_1\) and \(z_2\). Then \(f(\gamma)\subset f(U)\) is a curve that joins \(w_1\) and \(w_2\). Thus, \(f(U)\) is connected.
\end{proof}
Holomorphic injectivity, or univalence, satisfies the proceeding assertion:
\begin{lemma}\label{lem:univalentnonvanishingderivative}
    Let \(U\subseteq\mathbb{C}\) be a region and suppose \(f:U\to\mathbb{C}\) is univalent. Then \(f'\) is non-vanishing on \(U\).
\end{lemma}
\begin{proof}
    Suppose, for the sake of contradiction, that \(f\) is univalent on \(U\) such that \(\exists z_0\in U\) such that \(f'\qty(z_0)=0\). Let \(w_0=f\qty(z_0)\). The previous statement is equivalent to: \(f(z)-w_0\) has a zero at \(z_0\) with multiplicity \(m\geq 2\).

    Since this zero is isolated, let \(\rho>0\) be chosen such that \(z_0\) is the only zero of \(f-w_0\) contained in \(\overline{D\qty(z_0,\rho)}\subset U\). By \cref{thm:hurwitzshifts}, \(\exists \delta>0\) such that \(\forall w\in D\qty(w_0,\delta)\), the equation \(f(z)=w\) has \(m\) solutions in \(\overline{D\qty(z_0,\rho)}\). This contradicts the univalence of \(f\).
\end{proof}
Conversely, we have the following statement on local univalence and invertibility.
\begin{theorem}\label{thm:nonvanishingderivativeunivalentonneighborhood}
    Let \(U\subseteq\mathbb{C}\) be a region and suppose \(f:U\to\mathbb{C}\) is holomorphic. If \(f'\qty(z_0)\neq0\) for some \(z_0\in U\), then there exists an open neighborhood of \(z_0\) on which \(f\) is univalent.
\end{theorem}
\begin{proof}
    Let \(w_0=f\qty(z_0)\). Since \(\lim_{z\to z_0}f(z)-w_0=0\) and \(\lim_{z\to z_0}\frac{f(z)-w_0}{z-z_0}\neq0\), it follows that \(z_0\) is a simple zero of \(f(z)-w_0\). Let \(V\) be an open neighborhood (relatively compact in \(U\)) of \(z_0\) whose closure does not contain other zeros of \(f-w_0\). By \cref{thm:hurwitzshifts}, \(\exists\delta>0\) such that \(\forall w\in D\qty(w_0,\delta)\), \(f(z)=w\) has only one solution for \(z\) satisfying \(z\in V\). Therefore, we can choose a relatively compact open subset \(W\) of \(V\) such that \(f\qty(W)\subseteq D\qty(w_0,\delta)\), on which \(f\) is univalent.
\end{proof}
Moreover, if \(w=f(z)\) is univalent and surjective, mapping \(U\) to \(G\), then its inverse \(z=f^{-1}(w)\) is univalent on \(G\). Such bijective holomorphic functions are known as \textit{biholomorphisms} or \textit{biholomorphic} functions.

We will now study holomorphic functions from a more geometric perspective.
\begin{theorem}\label{thm:boundaryofconformalmap}
    Let \(\Omega\subseteq\mathbb{C}\) be a region, and let \(\gamma\subset\Omega\) be a rectifiable simple closed counterclockwise-oriented curve that is null-homotopic in \(\Omega\). Denote \(\mathrm{int}(\gamma)\) by \(U\). If \(f:\Omega\to\mathbb{C}\) is holomorphic and maps \(\gamma\) injectively to a simple closed curve \(\Gamma\), then \(w=f(z)\) is univalent in \(U\), \(f(U)=\mathrm{int}(\Gamma)\), and \(\Gamma\) is traversed counterclockwise.
\end{theorem}
\begin{proof}
    Let \(w_0\in\mathbb{C}\). First assume that \(w_0\) does not lie on \(\Gamma\) itself. By the Argument Principle (\cref{thm:argumentprincipleholomorphic}), the number of zeros of \(f-w_0\) enclosed by \(\gamma\) is equal to \[k=\frac{1}{2\uppi \ii}\oint_\gamma \frac{f'(z)}{f(z)-w_0}\ddz=\frac{1}{2\uppi \ii}\oint_\Gamma\frac{\dd{w}}{w-w_0}=\operatorname{Ind}_{\Gamma}\qty(\omega_0).\] If \(w_0\in\mathrm{ext}(\Gamma)\), the integral vanishes by the Cauchy--Goursat Theorem (\cref{thm:cauchygoursattheorem}). On the contrary, if \(w_0\in\mathrm{int}(\Gamma)\), then \(\Gamma\) winds around \(w_0\) exactly once, and hence, in other words, \(\forall w_0\in\mathrm{int}(\Gamma)\), \(f(z)=w_0\) has a unique solution in \(U\). This verifies the univalence of \(f\) in \(U\).

    If \(w_0\) lies on \(\Gamma\), \(f-w_0\) has no zeros in \(U\). Indeed, for the sake of contradiction, assume that \(\exists z_0\in U\) such that \(f\qty(z_0)=w_0\). By the Open Mapping Theorem (\cref{thm:openmapping}) and \cref{thm:hurwitzshifts}, \(\exists\delta>0\) such that \(D\qty(w_0,\delta)\subseteq f(U)\) and \(\forall w'\in D\qty(w_0,\delta)\), \(f-w'\) has zeros in \(U\). Since \(w_0\) lies on \(\Gamma\), a subset of \(D\qty(w_0,\delta)\) lies in the exterior of \(\Gamma\). It was previously established that \(f-w'\) has no zeros if \(w'\in D\qty(w_0,\delta)\cap\mathrm{ext}(\Gamma)\). Thus, we have a contradiction. We then have \[k=
        \begin{dcases}
            0 & \qif* w_0\in\overline{\mathrm{ext}(\Gamma)}, \\
            1 & \qif* w_0\in \mathrm{int}(\Gamma).
        \end{dcases}\]
    Hence, for any arbitrary \(z_0\in U\), \(w_0=f\qty(z_0)\) must lie in \(\mathrm{int}(\Gamma)\). Indeed, if \(f\qty(z_0)\) lies in \(\mathrm{ext}(\Gamma)\) or \(\Gamma\), then \(k\) would be nonzero in those areas. It follows that \(f(U)=\mathrm{int}(\Gamma)\).
\end{proof}
We will now give examples of biholomorphisms.
\begin{example}
    The only biholomorphisms which map \(\mathbb{D}\) to itself are in the form of
    \begin{equation}
        w=\ee^{\ii\theta}\frac{z-a}{1-\overline{a}z},\quad a\in\mathbb{D},\theta\in\mathbb{R}.\label{eq:biholomorphismunitdiskautomorphism}
    \end{equation}
    This follows directly from \cref{thm:holomorphicautomorphismgrouponunitdisk}.
\end{example}
\begin{example}\label{ex:biholomorphismsupperhalfplanetounitdisk}
    The only biholomorphisms which map \(\mathbb{H}^+\) to \(\mathbb{D}\) are in the form of
    \begin{equation}
        w=\ee^{\ii\theta}\frac{z-a}{z-\overline{a}},\quad a\in\mathbb{H}^+,\theta\in\mathbb{R}.\label{eq:biholomorphismunitdisktoupperhalfplane}
    \end{equation}
\end{example}
\begin{proof}
    First assume \(y=\Im(z)>0\). It follows that \[\abs{w}=\abs{\frac{z-a}{z-\overline{a}}}=\sqrt{\frac{\qty(x-\Re(a))^2+\qty(y-\Im(a))^2}{\qty(x-\Re(a))^2+\qty(y+\Im(a))^2}}<1.\] Therefore, this transformation maps \(\mathbb{H}^+\) to \(\mathbb{D}\). The inverse mapping is equal to
    \begin{equation}
        z=\frac{w\overline{a}-a\ee^{\ii\theta}}{w-\ee^{\ii\theta}}.\label{eq:biholomorphismunitdisktoupperhalfplane_inverse}
    \end{equation}
    Assume \(w\in\mathbb{D}\). We then have
    \begin{align*}
        \Im(z)=\Im\frac{\qty(w\overline{a}-a\ee^{\ii\theta})\qty(\overline{w}-\ee^{-\ii\theta})}{\abs{w-\ee^{\ii\theta}}^2} & =\frac{\abs{w}^2\Im\qty(\overline{a})-\Im\qty(a\ee^{\ii\theta}\overline{w})-\Im\qty(w\overline{a}\ee^{-\ii\theta})+\Im(a)}{\abs{w-\ee^{\ii\theta}}^2} \\
                                                                                                                            & =\frac{\qty(1-\abs{w}^2)\Im\qty(a)}{\abs{w-\ee^{\ii\theta}}^2}>0.
    \end{align*}
    Hence, \(z\) maps \(\mathbb{D}\) to \(\mathbb{H}^+\) univalently and surjectively since it is also an element in \(\Aut\qty(\extcomplex)\).

    Let \(\psi(z)\) be the biholomorphism from \(\mathbb{H}^+\) to \(\mathbb{D}\) in the form of \(\psi(z)=\frac{z-\ii}{z+\ii}\) (for \(\theta=0\) and \(a=\ii\), known as the \textit{Cayley transform}). Let \(f\) be an arbitrary biholomorphism from \(\mathbb{H}^+\) to \(\mathbb{D}\). It follows that \(\varphi=f\circ\psi^{-1}\) is a holomorphic automorphism on \(\mathbb{D}\). Since \(\varphi\in\Aut(\mathbb{D})\), we have \[f(z)=\varphi\circ\psi(z)=\ee^{\ii\theta}\frac{z\qty(1-a)-\ii\qty(a+1)}{z\qty(1-\overline{a})+\ii\qty(\overline{a}+1)}=\ee^{\ii\theta}\frac{z\frac{1-a}{1-\overline{a}}+\ii\frac{a+1}{\overline{a}-1}}{z-\ii\frac{\overline{a}+1}{\overline{a}-1}}=\ee^{\ii\theta}\frac{1-a}{1-\overline{a}}\frac{z-\ii\frac{a+1}{1-a}}{z-\overline{\ii\frac{a+1}{1-a}}}.\]
    Obviously, \(\ee^{\ii\theta}\frac{1-a}{1-\overline{a}}\) attains every value on the unit disk for varying \(a\) and \(\theta\). Similarly, the values attained by \(\ii\frac{a+1}{1-a}\) cover the upper half-plane for \(a\in\mathbb{D}\) (since it is in the form of \cref{eq:biholomorphismunitdisktoupperhalfplane_inverse}). Thus, all biholomorphisms from \(\mathbb{H}^+\) to \(\mathbb{D}\) are in the form of \cref{eq:biholomorphismunitdisktoupperhalfplane}.
\end{proof}
Let us now introduce some important properties of linear fractional transformations. By \cref{prop:mobiustransformationcompositionmatrixmultiplication}, it follows that the composition of two linear fractional transformations is also a linear fractional transformation.
\begin{theorem}\label{thm:linearfractionaltransformationmapscirclestocircles}
    Let \(\mathcal{C}\) be the collection of subsets of \(\extcomplex\) that are circles or \(L\cup\cbraces{\infty}\), where \(L\) is a straight line in \(\mathbb{C}\) (known as generalized circles). Then every linear fractional transformation \(f:\extcomplex\to\extcomplex\) maps elements of \(\mathcal{C}\) to elements of \(\mathcal{C}\).
\end{theorem}
\begin{proof}
    Since each linear fractional transformation is a composition of maps in the form of \(z\mapsto az\), \(z\mapsto z+b\), and \(z\mapsto \frac{1}{z}\), it suffices to show that these maps preserve the property of being a circle or a straight line. Consider a circle defined implicitly with \[\alpha\qty(x^2+y^2)+\beta x+\gamma y+\delta=0,\quad{x,y\in\mathbb{R},\alpha,\beta,\gamma,\delta\in\mathbb{R}}\] For \(z=x+\ii y\), this can be rewritten as
    \begin{equation}
        \alpha z\overline{z}+\beta\frac{z+\overline{z}}{2}+\gamma\frac{z-\overline{z}}{2\ii}+\delta=\alpha z\overline{z}+\xi z+\overline{\xi}\overline{z}+\delta=0\qfor\xi=\frac{\beta}{2}+\frac{\gamma}{2\ii}.\label{eq:linearfractionaltransformationmapscirclestocircles_circlecomplexform}
    \end{equation}
    If \(\alpha=0\), the equation represents a straight line. It is easy to see that a complex dilation or a translation will preserve the property of being a straight line or a circle. Indeed, by substituting \(z=a\zeta\) for nonzero \(a\) into \cref{eq:linearfractionaltransformationmapscirclestocircles_circlecomplexform}, we have
    \[\alpha a^2\zeta\overline{\zeta}+\xi a\zeta+\overline{\xi} \overline{a}\overline{\zeta}+\delta=0,\] which is trivially in the form of \cref{eq:linearfractionaltransformationmapscirclestocircles_circlecomplexform}. Similarly, if we substitute \(z=\zeta+b\), we have
    \begin{gather*}
        \alpha(\zeta+b)\qty(\overline{\zeta}+\overline{b})+\xi(\zeta+b)+\overline{\xi}(\overline{\zeta}+\overline{b})+\delta=0\\
        \alpha\zeta\overline{\zeta}+\qty(\xi+\alpha\overline{b})\zeta+\qty(\overline{\xi}+\alpha b)\overline{\zeta}+\alpha\abs{b}^2+2\Re(\xi b)+\delta=0.
    \end{gather*}
    If we substitute \(z=\frac{1}{\zeta}\), we have \[\delta\zeta\overline{\zeta}+\xi\overline{\zeta}+\overline{\xi}\zeta+\alpha=0,\]
    which is in the form of \cref{eq:linearfractionaltransformationmapscirclestocircles_circlecomplexform}.
\end{proof}
\begin{remark}
    As in \cref{ex:biholomorphismsupperhalfplanetounitdisk}, we can consider extended straight lines in the form of \(L\cup\cbraces{\infty}\) as generalized circles in the Riemann sphere. In other words, the extended line can be geometrically visualized by a circle with infinite radius. In fact, when a circle on the Riemann sphere is projected stereographically onto the complex plane, the result is always either a circle or a straight line.
\end{remark}
\begin{definition}[Cross-Ratio]\label{def:crossratio}
    Let \(z_1,z_2,z_3,z_4\in\extcomplex\) be points such that at least three of them are distinct. The \textit{cross-ratio} of these points is defined as \[\qty(z_1,z_2;z_3,z_4)=\frac{\qty(z_1-z_3)\qty(z_2-z_4)}{\qty(z_1-z_4)\qty(z_2-z_3)}.\] If at least one of the four points is \(\infty\), then the cross-ratio is defined by the limit:
    \begin{align*}
        \qty(\infty,z_2;z_3,z_4) & =\frac{z_2-z_4}{z_2-z_3}, & \qty(z_1,\infty;z_3,z_4) & =\frac{z_1-z_3}{z_1-z_4} \\
        \qty(z_1,z_2;\infty,z_4) & =\frac{z_2-z_4}{z_1-z_4}, & \qty(z_1,z_2;z_3,\infty) & =\frac{z_1-z_3}{z_2-z_3}
    \end{align*}
\end{definition}
One important property of the cross-ratio is that it is invariant under linear fractional transformations. In other words, if \(f\) is a linear fractional transformation, then \[\qty(f\qty(z_1),f\qty(z_2);f\qty(z_3),f\qty(z_4))=\qty(z_1,z_2;z_3,z_4).\] The proof is trivial and can be verified by substituting the definition of the linear fractional transformation into the definition of the cross-ratio.

Furthermore, if a function \(f\qty(z_1,z_2,z_3,z_4)\) is invariant under the group of linear fractional transformations, then it is a function of the cross-ratio. In other words, the cross-ratio is the only invariant under the group of linear fractional transformations \(\Aut\qty(\extcomplex)\). Indeed, suppose that \[f\qty(\varphi\qty(z_1),\varphi\qty(z_2),\varphi\qty(z_3),\varphi\qty(z_4))=f\qty(z_1,z_2,z_3,z_4).\] We aim to show that \(f\) is a function of a cross-ratio. Let \[\varphi(z)=\frac{\qty(z-z_3)\qty(z_2-z_4)}{\qty(z-z_4)\qty(z_2-z_3)}\] be a linear fractional transformation. Then we have \[f\qty(\varphi\qty(z_1),\varphi\qty(z_2),\varphi\qty(z_3),\varphi\qty(z_4))=f\qty(\qty(z_1,z_2;z_3,z_4),1,0,\infty),\] which is a function of the cross-ratio.