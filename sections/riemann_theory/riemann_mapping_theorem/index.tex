\subsection{The Riemann Mapping Theorem}
The Riemann Mapping Theorem is one of the most profound results in complex analysis; in the case of one dimension, it establishes sufficient conditions for the biholomorphic equivalence between two open subsets of the complex plane.

If there exists a biholomorphism \(f\) between two regions, then the two regions are said to be \textscsl{conformally equivalent}, \textscsl{holomorphically equivalent}, or \textscsl{biholomorphically equivalent}.
\begin{theorem}[\textsc{Riemann Mapping Theorem}]\label{thm:riemannmapping}
    Let \(U\subset\mathbb{C}\) (a proper subset, in other words, \(U\neq\mathbb{C}\)) be a simply connected (nonempty) open region. Let \(z_0\in U\) be arbitrary. Then there exists a unique biholomorphism \(f:U\to\mathbb{D}\) such that \(f\qty(z_0)=0\) and \(f'\qty(z_0)\in\mathbb{R}_{>0}\).
\end{theorem}
\begin{proof}
    First consider the case for when \(U\) is a bounded region. In other words, \(\exists R>0\) such that \(U\subseteq D(0,R)\).

    Define \(\mathcal{F}\) to be the family of all univalent functions \(\alpha:U\to\mathbb{D}\) (not necessarily surjective) such that \(\alpha\qty(z_0)=0\). This family is well-defined and nonempty.

    To prove this assertion, observe that since \(z_0\in D(0,R)\), it follows that \(\forall z\in U\subseteq D(0,R)\), \(\abs{z-z_0}<2R\), and consequently, \(\abs{\frac{z-z_0}{2R}}<1\). Therefore, \[\alpha(z)=\frac{z-z_0}{2R}\] maps \(U\) to \(\mathbb{D}\), and it is linear and univalent. This shows that \(\alpha\in\mathcal{F}\). It is easy to prove that \(\mathcal{F}\) is infinite; any function in the form of \(z\mapsto\frac{z-z_0}{\zeta}\) for \(\zeta\geq2R\) also lies in \(\mathcal{F}\).

    Since \(\mathcal{F}\) is uniformly bounded on \(U\), by Montel's Theorem (\cref{thm:montel}), \(\mathcal{F}\) is a normal family. Let \(r>0\) satisfy \(\overline{D\qty(z_0,r)}\subset U\). Then by Cauchy's Estimate (\cref{thm:cauchysestimate}), \(\forall \alpha\in\mathcal{F}\), \(\abs{\alpha'}\leq\frac{1}{r}\) on \(\overline{D\qty(z_0,r)}\). Hence, we have
    \begin{equation}
        0<M=\sup_{\alpha\in\mathcal{F}}\abs{\alpha'\qty(z_0)}\leq\frac{1}{r},\label{eq:riemannmapping_fixedpointderivativesupremum}
    \end{equation}
    where we can assure that \(M\) is positive since each \(\alpha\in\mathcal{F}\) is univalent at \(z_0\) and by \cref{lem:univalentnonvanishingderivative}.

    If \(M\) is an accumulation point of \(\cbraces{\abs{\alpha'\qty(z_0)}}_{\alpha\in\mathcal{F}}\), there exists a sequence \(\cbraces{\alpha_n}_{n\in\mathbb{N}}\subseteq\mathcal{F}\) such that \(\cbraces{\abs{\alpha'_n\qty(z_0)}}_{n\in\mathbb{N}}\) converges to \(M\). If \(M\) is attained as a maximum or that \(\abs{\alpha'(z_0)}=M\) for some \(\alpha\in\mathcal{F}\), we may let each \(\alpha_n\equiv\alpha\).

    By the normality of \(\mathcal{F}\), there exists a subsequence \(\cbraces{\alpha_{n_k}(z)}_{k\in\mathbb{N}}\subseteq\cbraces{\alpha_n(z)}_{n\in\mathbb{N}}\) such that \(\cbraces{\alpha_{n_k}(z)}_{k\in\mathbb{N}}\) is locally uniformly convergent in \(U\) to a function \(\doubletilde{\alpha}(z)\) (holomorphy of which is provided by \cref{thm:weierstrassconvergence}). By definition, \(\abs{\doubletilde{\alpha}'\qty(z_0)}=M\), and define a function sequence with \(\widetilde{\alpha}_{n_k}=\alpha_{n_k}\frac{\abs{\doubletilde{\alpha}'\qty(z_0)}}{\doubletilde{\alpha}'\qty(z_0)}\in\mathcal{F}\), whose locally uniform limit is \(f\). It follows that \(f\) is a rotation of \(\doubletilde{\alpha}\) such that \(f'\qty(z_0)=M\).

    Let \(\zeta_1,\zeta_2\in U\) be arbitrary and different. Choose \(r'>0\) to satisfy \(0<r'<\abs{\zeta_1-\zeta_2}\), and let \(\psi_k(z)=\widetilde{\alpha}_{n_k}(z)-\widetilde{\alpha}_{n_k}\qty(\zeta_2)\). Since each \(\widetilde{\alpha}_{n_k}\) is univalent in \(U\), it follows that each \(\psi_k\) is non-vanishing in \(U\setminus\cbraces{\zeta_2}\) and consequently, in \(\overline{D\qty(\zeta_1,r')}\). By \cref{thm:hurwitzsimplecase}, it follows that the locally uniform limit of \(\psi_k\), or \(\psi=f(z)-f\qty(\zeta_2)\), is either non-vanishing or is identically zero in \(\overline{D\qty(\zeta_1,r')}\). The latter is an impossibility since \(\psi'\qty(z_0)=M>0\). Hence, \(f(z)=f\qty(\zeta_2)\) has no solutions for \(z\in\overline{D\qty(\zeta_1,r')}\). In particular, \(f\qty(\zeta_1)\neq f\qty(\zeta_2)\). By the arbitrariness of \(\zeta_1\) and \(\zeta_2\), the univalence of \(f\) follows.

    Additionally, since \(\forall k\in\mathbb{N}\), \(\abs{\widetilde{\alpha}_{n_k}}<1\) in \(U\), it follows that \(f(U)\subseteq\overline{\mathbb{D}}\). By the Open Mapping Theorem (\cref{thm:openmapping}), the condition becomes \(f(U)\subseteq\mathbb{D}\). Since \(\widetilde{\alpha}_{n_k}\qty(z_0)=0\) for all \(k\in\mathbb{N}\) and \(\widetilde{\alpha}_{n_k}\qty(z_0)\to 0=f\qty(z_0)\), it follows that \(f\in\mathcal{F}\).

    Suppose \(\Phi:U\to\mathbb{C}^*=\mathbb{C}\setminus\cbraces{0}\) is holomorphic. Define the \textscsl{holomorphic logarithm} of \(\Phi(z)\) to be a branch of \[\log\qty(\Phi(z))=\int_{\gamma}\frac{\Phi'(\zeta)}{\Phi(\zeta)}\ddzeta+\log\qty(\Phi\qty(z_0))\] for any \(z_0\in U\), where \(\gamma\subset U\) is any piecewise \(C^1\) curve from \(z_0\) to \(z\), and path independence is provided by the simple connectivity of \(U\). The result is the heuristic concatenation of several different branches of the complex logarithm, unique up to an additive factor of \(2\piup\ii k\). Similarly, we can define the \textscsl{holomorphic powers} of \(\Phi(z)\) to be branches of \(\Phi^\alpha(z)=\ee^{\alpha\log\qty(\Phi(z))}\), where \(\log\qty(\Phi(z))\) is the holomorphic logarithm.

    Lastly, we aim to prove that \(f\) maps \(U\) to \(\mathbb{D}\) surjectively. For the sake of contradiction, assume that \(\exists\xi\in\mathbb{D}^*\) such that \(\xi\notin f(U)\). Consider the unit disk automorphism \(\varphi_\xi(z)=\frac{z-\xi}{1-z\overline{\xi}}\). Since \(\varphi_\xi(z)\) vanishes when \(z=\xi\), and since \(f(z)=\xi\) has no solutions in \(U\), there exists a holomorphic square root \[\mu(z)=\sqrt{\varphi_\xi\circ f(z)}\in\mathbb{D}\] for \(z\in U\). Let \(\tau=\mu\qty(z_0)=\sqrt{-\xi}\), and let \[\eta(z)=\varphi_\tau\circ\mu(z),\]
    where \(\varphi_\tau=\frac{z-\tau}{1-z\overline{\tau}}\). Since \(\eta\qty(z_0)=\varphi_\tau\qty(\tau)=0\), it follows that \(\eta\in\mathcal{F}\). Let \(\widetilde{\eta}=\frac{\abs{\eta'\qty(z_0)}}{\eta'\qty(z_0)}\eta\), which is also in \(\mathcal{F}\). However, since \(\widetilde{\eta}'=\frac{\abs{\eta'\qty(z_0)}}{\eta'\qty(z_0)}\eta'\), we have
    \begin{align*}
        \widetilde{\eta}'\qty(z_0)=\abs{\frac{f'\qty(z_0)\qty(\varphi'_\tau\circ\tau)\qty(\varphi'_\xi\circ 0)}{2\sqrt{\varphi_\xi\circ 0}}} & =\frac{M}{2\sqrt{\abs{\xi}}}\eval{\frac{1-\overline{\tau}\tau}{\qty(1-z\overline{\tau})^2}}_{z=\tau}\eval{\frac{1-\overline{\xi}\xi}{\qty(1-z\overline{\xi})^2}}_{z=0} \\
                                                                                                                                             & =\frac{M}{2\sqrt{\abs{\xi}}}\frac{1-\abs{\xi}^2}{1-\abs{\xi}}=\frac{M\qty(1+\abs{\xi})}{2\sqrt{\abs{\xi}}}.
    \end{align*}
    Additionally, since
    \[\qty(\sqrt{\abs{\xi}}-1)^2>0\Longleftrightarrow 1+\abs{\xi}>2\sqrt{\abs{\xi}}\Longleftrightarrow\frac{1+\abs{\xi}}{2\sqrt{\abs{\xi}}}>1,\]
    it follows that \(\widetilde{\eta}'\qty(z_0)>M\), which is a contradiction of \cref{eq:riemannmapping_fixedpointderivativesupremum}.

    Hence, \(f:U\to\mathbb{D}\) is biholomorphic. To prove the uniqueness of \(f\), suppose \(g:U\to\mathbb{D}\) is an arbitrary biholomorphism such that \(g\qty(z_0)=0\) and \(g'\qty(z_0)>0\). Then, \(\varphi=f\circ g^{-1}\in\Aut\qty(\mathbb{D})\), and by \cref{thm:holomorphicautomorphismgrouponunitdisk}, \(\varphi(z)=\varphi_a(z\exp(\ii\theta))\) for some \(a\in\mathbb{D}\) and \(0\leq\theta<2\piup\). Since \(\varphi(0)=0\), it follows that \(a=0\). Since \(\varphi'(0)=f'\qty(z_0)\qty(g^{-1})'(0)=\frac{f'\qty(z_0)}{g'(z_0)}>0\), and \(\varphi'(0)=\varphi'_0(0)\exp(\ii\theta)=\exp(\ii\theta)>0\), it follows that \(\theta=0\). Hence, we have \(\varphi(z)=z\) and \(f\equiv g\).

    Next, assume that \(U\) is unbounded. It is easy to show that the boundary \(\partial U\) contains at least two points. Indeed, if \(\partial U=\varnothing\), \(U\) would be closed because \(\partial U\subseteq U\) and open by assumption. By \cref{thm:connectedtopologicalspaceclopensets}, \(U\) would either be equal to \(\varnothing\) or \(\mathbb{C}\), both of which are impossibilities. Additionally, if \(\partial U\) comprises exactly one point \(a\in\mathbb{C}\), then in subspace defined by \(X=\mathbb{C}\setminus\cbraces{a}\), \(U\) is clopen (by the same reason as before, open by assumption and closed because \(X\setminus U=\mathbb{C}\setminus\overline{U}\) is open). It follows that \(U=X=\mathbb{C}\setminus\cbraces{a}\), which is not simply connected.

    Suppose \(\xi_1\) and \(\xi_2\) are two distinct points in \(\partial U\). Let us apply the linear transformation \(\rho(z)=\frac{z-\xi_1}{\xi_2-\xi_1}\) to \(U\), and denote the resulting region by \(U'=\rho(U)\). It follows that \(0,1\in\partial U'\). Consider a branch \(\psi(z)\) of the holomorphic square root \(z\mapsto\sqrt{z-1}\) (existent by simple connectivity and the fact that \(1\notin U'\)). Trivially, \(\psi\) is univalent in \(U'\).

    In addition, we assert that \(\psi\qty(U')\cap\qty(-\psi\qty(U'))=\varnothing\). If not, then \(\exists \xi\in \psi\qty(U')\) such that \(-\xi\in\psi\qty(U')\). By definition, \(\exists z_1,z_2\in U'\) such that \(\psi\qty(z_1)=\xi\) and \(\psi\qty(z_2)=-\xi\). It would then follow that \(\sqrt{z_1-1}=\sqrt{z_2-1}\) and \(z_1=z_2\). It follows that \(\xi=0\), which is obtained when \(z_1=z_2=\psi^{-1}(\xi)=1\). Since \(1\in\partial U'\) and \(U\) is open, this is an impossibility.

    Fix \(\xi\in\psi(U')\) to be arbitrary. By the Open Mapping Theorem (\cref{thm:openmapping}), there exists an open neighborhood \(D(\xi,\varepsilon)\subseteq \psi(U')\). It follows that \(D\qty(-\xi,\varepsilon)\cap\psi\qty(U')=\varnothing\). Therefore, \(\forall z\in U'\), \(\abs{\psi(z)+\xi}\geq\varepsilon\), and consequently, \(\abs{\frac{1}{\psi(z)+\xi}}\leq\frac{1}{\varepsilon}\). Hence, the function \(\varphi(z)=\frac{1}{z+\xi}\) maps \(U'\) to a bounded region that lies within the compact disk \(\overline{D\qty(0,\frac{1}{\varepsilon})}\). Denote \(\varphi\circ\psi(U')\) by \(\widetilde{U}\).

    It is easy to see that \(\widetilde{U}\) is simply connected. Let \(\widetilde{U}=\varphi\circ\psi\circ\rho(U)\). To prove this, it suffices to show that the line integral of any holomorphic function over any closed curve in \(\widetilde{U}\) vanishes. Let \(g:\widetilde{U}\to\mathbb{C}\) be holomorphic, and let \(\Gamma\subset\widetilde{U}\) be a closed piecewise \(C^1\) curve. Then \[\oint_\Gamma g(z)\ddz=\oint_{\rho^{-1}\circ\psi^{-1}\circ\varphi^{-1}(\Gamma)}g\circ\varphi\circ\psi\circ\rho(z)\ddz=0,\]
    by \cref{lem:cauchyintegraltheoremoversimplyconnectedset}, since \(U\) is simply connected by assumption, \(\rho^{-1}\circ\psi^{-1}\circ\varphi^{-1}(\Gamma)\) is a closed piecewise \(C^1\) curve in \(U\), and \(g\circ\varphi\circ\psi\circ\rho\) is holomorphic on \(U\). Therefore, \(\widetilde{U}\) is simply connected.

    Hence, we may use our previous result and establish a biholomorphism \(\widetilde{f}:\widetilde{U}\to\mathbb{D}\), unique up to a transformation in \(\Aut(\mathbb{D})\). Let \(f=\widetilde{f}\circ\varphi\circ\psi\circ\rho\), which is a biholomorphism from \(U\) to \(\mathbb{D}\). Similarly, it is unique up to a transformation in \(\Aut(\mathbb{D})\), and the same assertion follows.
\end{proof}
\begin{remark}
    It is natural that we require \(U\neq\mathbb{C}\); if there exists a univalent function \(f:\mathbb{C}\to\mathbb{D}\), then by Liouville's Theorem (\cref{thm:liouville}), \(f\) would be a constant function.

    As we will see in \cref{sec:multivariatecomplexanalysis}, this theorem and many other properties of one-variable holomorphic functions do not extend to functions of several complex variables.
\end{remark}