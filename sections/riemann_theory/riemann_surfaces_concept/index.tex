\subsection{The Concept of Riemann Surfaces}\label{sec:riemannsurfaces}
Let \(X=(X,\tau)\) be a \textscsl{connected} topological space, and let \(x\in X\) be a fixed point. If \(\exists U\in\tau\) such that \(x\in U\), then \(U\) is a \textscsl{neighborhood} of \(x\). If any two points \(x,y\in X\) have disjoint neighborhoods, then \(X\) is a \textscsl{Hausdorff space}.

For two topological spaces \(X\) and \(Y\), a function \(f:X\to Y\) is a \textscsl{homeomorphism} (also known as a \textscsl{bicontinuous function}) if it is a bijection such that both \(f\) and \(f^{-1}\) are continuous. If such a function exists, then \(X\) and \(Y\) are \textscsl{homeomorphic}.

The continuity of the inverse is essential. Consider the two topological spaces \(S^1=\cbraces{(x_1,x_2)\in\mathbb{R}^2}{x_1^2+x_2^2=1}\) and \([0,2\piup)=\cbraces{t\in\mathbb{R}}{0\leq t<2\piup}\) under the standard topology.

The function \(f:[0,2\piup)\to S^1\) with \(f(t)=\qty(\cos(t),\sin(t))\) is indeed continuous, but the inverse \(f^{-1}\qty(x_1,x_2)\) is discontinuous at \(\qty(x_1,x_2)=(1,0)\).

A surface is a Hausdorff space that is locally homeomorphic to \(\mathbb{R}^2\). In other words, for any point \(p\) on the surface, there exists a neighborhood \(U\) of \(p\) such that \(U\) is homeomorphic to an open subset of \(\mathbb{R}^2\).

Let \(\cbraces{U_\alpha}_{\alpha\in I}\) be an collection of open subsets of a surface \(X\) that covers \(X\). For each \(\alpha\), let \(\varphi_\alpha:U_\alpha\to\mathbb{C}\) be a homeomorphism between \(U_\alpha\) and an open subset of \(\mathbb{C}\cong\mathbb{R}^2\). The pair \(\qty(U_\alpha,\varphi_\alpha)\) is a \textscsl{coordinate chart} of \(X\). The collection \(\cbraces{\qty(U_\alpha, \varphi_\alpha)}_{\alpha\in I}\) is a \textscsl{coordinate atlas} of \(X\). Let \(U_\alpha\) and \(U_\beta\) be two arbitrary sets in the covering. If all \textscsl{transition maps} \(\varphi_\beta\circ\varphi_\alpha^{-1}:\varphi_\alpha\qty(U_\alpha\cap U_\beta)\to\varphi_\beta\qty(U_\alpha\cap U_\beta)\) are biholomorphic on their respective \(f_\alpha\qty(U_\alpha\cap U_\beta)\) (we don't need to explicitly consider \(\varphi_\alpha\circ\varphi_\beta^{-1}\) as it is the inverse of the biholomorphism, which is biholomorphic by definition), then the atlas is said to be \textscsl{holomorphic} (intuitively, transition maps convert Euclidean coordinates between two charts on an overlapping region). Such an \(X\) is formally known as a \textscsl{Riemann surface}.

In short, a Riemann surface is a locally Euclidean connected Hausdorff space with complex structure.

Riemann surfaces are the definition of one-dimensional complex manifolds. Complex manifolds of higher dimension can be defined similarly.

We will now give examples of Riemann surfaces (we will distinguish between \(\extcomplex\) and \(S^2\)).
\begin{example}
    The extended complex plane \(\extcomplex\) is a Riemann surface.
\end{example}
\begin{proof}
    The extended complex plane can be covered by two charts: \(\qty(\mathbb{C},\varphi_1)\) and \(\qty(\extcomplex\setminus\cbraces{0},\varphi_2)\), where \(\varphi_1(z)=z\) and \(\varphi_2(z)=\frac{1}{z}\). The transition map is given by \(\varphi_2\circ\varphi_1^{-1}(z)=\frac{1}{z}\), which is biholomorphic on \(\varphi_1\qty(\mathbb{C}\cap\extcomplex^*)=\mathbb{C}^*\). Hence, \(\extcomplex\) is a Riemann surface.
\end{proof}
\begin{example}
    The Riemann sphere \(S^2=\cbraces{\qty(x_1,x_2,x_3)\in\mathbb{R}^3}{x_1^2+x_2^2+x_3^2=1}\) is a Riemann surface.
\end{example}
\begin{proof}
    Consider the open sets \(U_N=S^2\setminus (0,0,1)\) and \(U_S=S^2\setminus (0,0,-1)\) that cover \(S^2\). Define their stereographic projection homeomorphisms \(\varphi_N:U_N\to\mathbb{C}\) and \(\varphi_S:U_S\to\mathbb{C}\) with centers \((0,0,1)\) and \((0,0,-1)\), respectively. By taking the conjugate of \(\varphi_S\) (this is necessary to ensure that the transition map is not anti-holomorphic), we form an atlas \(\cbraces{\qty(U_N,\varphi_N),\qty(U_S,\overline{\varphi_S})}\) with two charts. More explicitly, we can assume \[\varphi_N(x_1,x_2,x_3)=\frac{x_1+\ii x_2}{1-x_3}\qquad\overline{\varphi_S}(x_1,x_2,x_3)=\frac{x_1-\ii x_2}{1+x_3}.\]
    On \(\varphi_N\qty(S^2\setminus\qty((0,0,1)\cup(0,0,-1)))=\extcomplex\setminus\qty(\cbraces{0}\cup\cbraces{\infty})=\mathbb{C}^*\), the transition map \[\overline{\varphi_S}\circ\varphi_N^{-1}(z)=\frac{\Re(z)\qty(1-x_3)-\ii\Im(z)\qty(1-x_3)}{1+x_3}=\overline{z}\frac{1-x_3}{1+x^3}=\frac{\overline{z}}{\abs{z}^2}=\frac{1}{z}\] is biholomorphic. Hence, by the holomorphy of the atlas, \(S^2\) is a Riemann surface.
\end{proof}
Riemann surfaces are particularly useful when considering multi-valued functions. These functions cannot be defined \emph{globally} as single-valued holomorphic maps, but they can be made into well-defined holomorphic functions if we enlarge the domain, passing on to a Riemann surface.

Consider \(\log(z)=\log\abs{z}=\ii\arg(z)\) on \(\mathbb{C}^*\), where \(\arg\) is multi-valued. The most common solution to this problem is achieved by restricting the range of \(\arg\), say to \((-\piup,\piup]\), effectively selecting a branch of \(\log\), inducing a discontinuity along the branch cut (here, along \(\mathbb{R}_{\leq0}\)).

The geometric solution is to construct a new topological space \(X\), where the branches or sheets of the logarithm are \emph{glued} together. More explicitly, we take infinite copies of \(\mathbb{C}^*\), each copy indexed by an integer \(k\in\mathbb{Z}\), and glue them along the consecutive branch cuts such that after traversing a counterclockwise loop around the origin, one moves from the \(k\)-th sheet to the \((k+1)\)-th sheet (also called a branch).

Note that the concatenation along a different branch cut (for example, along \(\ii\mathbb{R}_{\leq0}\)) yields an identical Riemann surface, although it is important to specify the branch cut before the construction.

A point that connects multiple branches or sheets is known as a \textscsl{branch point}. In the case of the above construction, the origin is a branch point. In general, a branch point does not need to connect \emph{all} sheets; if it connects \(n\) sheets, it is known as a branch point of order \(n-1\).

We now briefly touch on the importance of \(\mathbb{D}\), \(\mathbb{C}\), and \(\extcomplex\) below.
\begin{theorem}[\textsc{Uniformization}]\label{thm:uniformization}
    Any simply connected Riemann surface is biholomorphically equivalent to either \(\mathbb{D}\), \(\mathbb{C}\), or \(\extcomplex\).
\end{theorem}
We are yet to define generalizations such as analyticity over Riemann surfaces. However, the theory of Riemann surfaces relates closely to the generalizations of \cref{sec:analyticcontinuation} via germs, sheaves, and germs, as well as topological concepts such as covering spaces, which is explained in detail in~\cite{functionsofonecomplexvariablei}.