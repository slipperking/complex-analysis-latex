\subsection{The Reflection Principle}
We have previously considered analytic continuations over two regions with an intersection. Under certain conditions, analytic continuations can be derived across a curve, given by the following theorem.
\begin{theorem}[\textsc{Painlevé}]\label{thm:painleve}
    Let \(U_1\) and \(U_2\) be two disjoint simply connected open regions in \(\mathbb{C}\) such that \(\partial U_1\cap\partial U_2\) is a simple curve \(\gamma\) without its endpoints. Let \(f_1:U_1\to\mathbb{C}\) and \(f_2:U_2\to\mathbb{C}\) be two holomorphic functions that are continuous on \(U_1\cup\gamma\) and \(U_2\cup\gamma\), respectively, such that \(f_1\equiv f_2\) on \(\gamma\). Then there exists a unique holomorphic function \[f=
        \begin{dcases}
            f_1           & \qq*{on}U_1,    \\
            f_2           & \qq*{on}U_2,    \\
            f_1\equiv f_2 & \qq*{on}\gamma,
        \end{dcases}\] on \(U_1\cup U_2\cup\gamma\).
\end{theorem}
\begin{proof}
    We aim to prove that the constructed function \(f\) is holomorphic on \(U_1\cup U_2\cup\gamma\). In particular, we only need to prove that \(f\) is holomorphic on (a neighborhood of) \(\gamma\), after which the Identity Theorem (\cref{thm:identity}) applies.

    Let \(z\in\gamma\) be fixed, and choose \(R=R_z>0\) such that \(D(z,R)\subseteq U_1\cup U_2\cup\gamma\). Let \(\Gamma\) be any simple closed curve in \(D(z,R)\). If \(\Gamma\) is fully contained in \(U_1\cup\gamma(\cap D(z,R))\), then by Cauchy--Goursat (\cref{thm:cauchygoursattheorem}), \[\oint_{\Gamma}f(z)\ddz=\oint_{\Gamma}f_1(z)\ddz=0.\]
    Similarly, if \(\Gamma\) is fully contained in \(U_2\cup\gamma\), then \[\oint_{\Gamma}f(z)\ddz=\oint_{\Gamma}f_2(z)\ddz=0.\]
    If \(\Gamma\) intersects \(\gamma\), then we can decompose \(\Gamma=\Gamma_1\cup\Gamma_2\), where \(\Gamma_1\) is the part of \(\Gamma\) that lies in \(U_1\cup\gamma\) and \(\Gamma_2\) is the part of \(\Gamma\) that lies in \(U_2\cup\gamma\). The set \(\widetilde{\Gamma}=\gamma\cap\mathrm{int}\qty(\Gamma)\) closes \(\Gamma_1\) and \(\Gamma_2\) in the sense that \(\widetilde{\Gamma}_1=\Gamma_1\cup\widetilde{\Gamma}\) and \(\widetilde{\Gamma}_2=\Gamma_2\cup\widetilde{\Gamma}\) are both simple closed curves, (where \(\widetilde{\Gamma}\) in each of the two curves have opposite orientations). By Cauchy--Goursat (\cref{thm:cauchygoursattheorem}), we have \[\ointctrclockwise_\Gamma f(z)\ddz=\qty(\int_{\Gamma_1}+\int_{\Gamma_2}+\int_{\widetilde{\Gamma}}-\int_{\widetilde{\Gamma}})f(z)\ddz=\qty(\ointctrclockwise_{\widetilde{\Gamma}_1}+\ointctrclockwise_{\widetilde{\Gamma}_2})f(z)\ddz=0.\] Hence, by Morera's Theorem (\cref{thm:morera}), \(f\) is holomorphic on \(\bigcup_{z\in\gamma}D\qty(z,R_z)\), and the assertion follows.
\end{proof}
A consequent result was discovered by Schwarz, known as the \textit{reflection principle}, is a unique result derived from the above theorem for when the shared boundary curve lies in the real axis under certain conditions.
\begin{theorem}[Schwarz Reflection Principle]\label{thm:riemannschwarzreflection}
    Let \(U\subseteq\mathbb{C}\) be a connected region on one side of the real axis such that there exists a non-degenerate curve \(\gamma\subseteq\partial U\) such that \(\gamma\in\mathbb{R}\). Let \(f:U\to\mathbb{C}\) be holomorphic with continuity up to \(U\cup\gamma\) such that \(f\) is real-valued on \(\gamma\), and let \(\widetilde{U}=\cbraces{\overline{z}}{z\in U}\) be the reflection of \(U\) across the real axis. Then there exists a unique holomorphic function \[\widetilde{f}(z)=
        \begin{dcases}
            f(z)                                 & \qif* z\in U,            \\
            \overline{f(\overline{z})}           & \qif* z\in\widetilde{U}, \\
            f(z)\equiv\overline{f(\overline{z})} & \qif* z\in\gamma,
        \end{dcases}\] on \(U\cup\widetilde{U}\cup\gamma\).
\end{theorem}
\begin{proof}
    If \(z\in\mathbb{R}\), then \(\overline{z}=z\), and since \(f\) is real on \(\gamma\), it follows that \(f(z)=\overline{f\qty(\overline{z})}\) for \(z\in\gamma\). Thus, we are left to prove that \(z\mapsto\overline{f\qty(\overline{z})}\) is holomorphic on \(\widetilde{U}\). Let \(z_0\in\widetilde{U}\). It follows that \[\lim_{\substack{z\to z_0\\ z\in\widetilde{U}}}\frac{\overline{f\qty(\overline{z})}-\overline{f\qty(\overline{z_0})}}{z-z_0}=\lim_{\substack{z\to z_0\\ z\in\widetilde{U}}}\overline{\qty[\frac{f\qty(\overline{z})-f\qty(\overline{z_0})}{\overline{z}-\overline{z_0}}]}=\overline{f'\qty(\overline{z_0})}.\]
    Since this limit exists, it follows that \(\overline{f\qty(\overline{z})}\) is holomorphic on \(\widetilde{U}\). Assume that \(z_0\in\gamma\). Since \[\lim_{z\to z_0}\overline{f\qty(\overline{z})}=\overline{f\qty(\lim_{z\to z_0}\overline{z})}=\overline{f\qty(z_0)}=f\qty(z_0),\] it follows that \(\overline{f\qty(\overline{z})}\) is continuous on \(\widetilde{U}\cup\gamma\). Therefore, by the Painlevé Theorem, \(\widetilde{f}\) is holomorphic on \(U\cup\widetilde{U}\cup\gamma\).
\end{proof}
This conjugate-symmetry can be generalized by transforming \(\gamma\):
\begin{theorem}[\textsc{Symmetry Principle}]
    Let \(L\subset\mathbb{C}\) be an (infinite) straight line, and let \(U\subset\mathbb{C}\) be an open region lying entirely on one side of \(L\). Suppose \(\gamma\subseteq L\) is a non-degenerate open curve contained in \(\partial U\). If \(f\) is holomorphic on \(U\), continuous on \(U\cup\gamma\), and satisfies \(f(\gamma)\subseteq\Gamma\), where \(\Gamma\subset\mathbb{C}\) is a straight line, then there exists a unique holomorphic function \(\widetilde{f}:U\cup\widetilde{U}\cup\gamma\to\mathbb{C}\) such that \(\widetilde{f}\equiv f\) on \(U\), where \(\widetilde{U}\) is the reflection of \(U\) across \(L\). Moreover, for any pair \(z_1,z_2\in U\cup\widetilde{U}\cup\gamma\) symmetric with respect to \(L\), the values \(\widetilde{f}\qty(z_1)\) and \(\widetilde{f}\qty(z_2)\) are symmetric with respect to \(\Gamma\).
\end{theorem}
\begin{proof}
    There exist \(a,c\in\mathbb{C}^*\) and \(b,d\in\mathbb{C}\) such that \(\phi(z)=az+b\) maps \(L\) to \(\mathbb{R}\) and \(\psi(z)=cz+d\) maps \(\Gamma\) to \(\mathbb{R}\). Let \(U'=\phi(U)\), which lies entirely on one side of the real axis, and let \(\gamma'=\phi(\gamma)\), a curve on the real axis. The function \(\varphi=\psi\circ f\circ\phi^{-1}\) is holomorphic on \(U'\) and continuous on \(U'\cup\gamma'\). By the Schwarz Reflection Principle (\cref{thm:riemannschwarzreflection}), there exists a unique holomorphic function \(\widetilde{\varphi}:U'\cup\widetilde{U}'\cup\gamma'\to\mathbb{C}\) such that \(\widetilde{\varphi}\equiv\varphi\) on \(U'\), where \(\widetilde{U'}\) is the reflection of \(U'\) across the real axis. Then \(\widetilde{f}=\psi^{-1}\circ\widetilde{\varphi}\circ\phi\) is a holomorphic function on \(U\cup\widetilde{U}\cup\gamma\) such that \(\widetilde{f}\equiv f\) on \(U\). Since linear transformations preserves symmetry, for any pair \(z_1,z_2\in U\cup\widetilde{U}\cup\gamma\) symmetric with respect to \(L\), we have \(\phi\qty(z_1)=\overline{\phi\qty(z_2)}\), and thus \(\widetilde{\varphi}\circ\phi\qty(z_1)\) and \(\widetilde{\varphi}\circ\phi\qty(z_2)\) are symmetric with respect to \(\mathbb{R}\). Hence, \(f\qty(z_1)\) and \(f\qty(z_2)\) are symmetric with respect to \(\psi^{-1}\qty(\mathbb{R})=\Gamma\).
\end{proof}