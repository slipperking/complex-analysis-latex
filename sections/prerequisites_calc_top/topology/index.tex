\subsection{Topological Preliminaries}
The following definitions are subject to the assumption where the topological space is defined to be \(X=\mathbb{C}^n\). This is satisfactory to the main purpose of our proceeding passage, but it is noteworthy that it can be generalized to more abstract sets.
\begin{definition}[Accumulation Point]\label{def:accumulationpoint}
    A point \(z\in\mathbb{C}^n\) is an \textscsl{accumulation point} of \(X\) if for any open set \(U\) containing \(z\), \((U\setminus\cbraces{z})\cap X\neq\varnothing\)
\end{definition}
\begin{definition}[Closure]\label{def:closure}
    For a set \(X\in\mathbb{C}^n\), define the \textscsl{closure} of \(X\), or \(\overline{X}\) to be the intersection of all closed sets containing \(X\). In other words, it is the union of \(X\) and its accumulation points.
\end{definition}
\begin{definition}[Interior]\label{def:interior}
    For a set \(X\in\mathbb{C}^n\), the \textscsl{interior} of \(X\), denoted \(\interior{X}\), is the union of all open sets contained in \(X\), or the set of points \(z\in\mathbb{C}^n\) such that there exists an open neighborhood of \(z\) that is fully contained in \(X\).
\end{definition}
\begin{definition}[Compact Set]\label{def:compactsets}
    A set \(X\in\mathbb{C}^n\) is compact iff \(X\) is closed and bounded.
\end{definition}
\begin{definition}[Set Covering]
    A cover \(\mathcal{C}\) of a set \(X\) is a collection of sets \(\cbraces{U_n}\) such that \[\bigcup_{n\in\mathbb{N}}U_n\supseteq X.\] A cover is \textscsl{open} if every set in the collection is open.
\end{definition}
\begin{theorem}[\textsc{Bolzano--Weierstrass}]\label{thm:bolzanoweierstrass}
    Every infinite subset \(A\) of a compact set \(X\subset\mathbb{C}^n\) has an accumulation point in \(X\).
\end{theorem}
\begin{proof}
    Since \(X\) is bounded, there exists a closed cube \(Q\subset\mathbb{C}^n\) such that \(A\subseteq X\subset Q\).

    Bisect \(Q_0=Q\) into \(2^{2n}\) congruent sub-cubes. Since \(A\) is infinite and the sub-cubes are finite in number, at least one of the sub-cubes contains infinitely many points of \(A\), and choose one to be \(Q_1\).

    Bisect \(Q_1\) into \(2^{2n}\) sub-cubes, and choose a sub-cube \(Q_2\subset Q_1\) that contains infinitely many points of \(A\). We then obtain the recursive sequence \[Q_0\supset Q_1\supset Q_2\supset\cdots.\]

    Because the side lengths shrink to zero and the cubes are nested, the intersection
    \[\bigcap_{k=0}^{\infty} Q_k\]
    consists of exactly one point. Call this point \(z_\infty\in\mathbb{C}^n\).

    For each \(k\), \(Q_k\) contains infinitely many points of \(A\). Because the side length of \(Q_k\) tends to zero, for any \(\varepsilon>0\), \(\exists N\in\mathbb{N}\) such that \(\forall k\geq N\), \(Q_k\subset B^n(z_\infty,\varepsilon)\) where \(B^n(a,r)\subset\mathbb{C}^n\) is the \(n\)-dimensional \textscsl{ball} with radius \(r\) centered at \(a=\qty(a_1,a_2,\ldots,a_n)\in\mathbb{C}^n\), or \[B^n(a,r)=\cbraces{\qty(z_1,z_2,\ldots,z_n)\in\mathbb{C}^n}{\sum_{j=1}^n\abs{z_j-a_j}^2<r^2}.\]
    Then, \(B^n(z_\infty, \varepsilon)\) also contains infinitely many points of \(A\). Therefore, \(z_\infty\) is an accumulation point of \(A\).

    We now show that \(z_\infty\in X\). Suppose for contradiction that \(z_\infty\notin X\). Since \(X\) is closed, \(\mathbb{C}^n\setminus X\) is open, and \(\exists\delta>0\) such that \[B^n(z_\infty,\delta)\subset\mathbb{C}^n\setminus X.\]
    But then, for sufficiently large \(k\), we have \(Q_k \subset B^n\qty(z_\infty,\delta)\), and hence \(Q_k\cap X=\varnothing\). This contradicts the construction of \(Q_k\), which ensures that \(Q_k\) contains infinitely many points of \(A \subset X\).
\end{proof}
\begin{theorem}[\textsc{Heine--Borel}]\label{thm:heineborel}
    A set \(X\in\mathbb{C}^n\) is compact iff if every open cover has a finite subcover.
\end{theorem}
\begin{proof}
    We will first show that any set satisfying the condition is compact.

    First we will show that \(X\) is bounded. Suppose that \(\forall R>0\), \(\exists z\in X\) where \(\norm{z}>R\). Consider the collection of open sets \[\mathcal{U}=\cbraces{B^n(0,k)}{k\in\mathbb{N}}.\] \(\mathcal{U}\) forms an open cover of \(X\). Then by the assumption, there exists a finite subcover in \(\mathcal{U}\), namely \(\cbraces{B^n(0,k_1),\ldots,B^n(0,k_m)}\) which covers \(X\). Then, \[X\subseteq\bigcup_{i=1}^mB^n(0,k_i)=B^n\qty(0,\max\cbraces{k_1,\ldots k_m}).\] By contradiction, \(X\) must be bounded.

    \(X\) must also be a closed set. For the sake of contradiction, assume that there exists a point \(z_0\in\overline{X}\setminus X\). Since \(z_0\notin X\), the following open collection of sets covers \(X\):
    \[\mathcal{U}=\cbraces{\mathbb{C}^n\setminus\overline{B^n\qty(z_0,\frac{1}{k})}}{\forall k\in\mathbb{N}}.\]
    There then exists a finite subcover \(\mathcal{C}=\cbraces{\mathbb{C}^n\setminus\overline{B^n\qty(z_0,\frac{1}{k_j})}}{j\in\mathbb{N}_{\leq m}}\). Then, \[X\subseteq\mathbb{C}^n\setminus\overline{B^n\qty(z_0,\frac{1}{\max\cbraces{k_1,\ldots,k_m}})},\]
    and that \(X\cap\overline{B^n\qty(z_0,\frac{1}{\max\cbraces{k_1,\ldots,k_m}})}=\varnothing\). However, by the definition of the accumulation point, every open neighborhood of the accumulation point must intersect \(X\). Therefore, by contradiction, \(X\) is closed.

    We then prove the converse. By the assumption that \(X\) is bounded, \(\exists R>0\) such that the \(X\) is contained within the closed cube \[Q=\cbraces{z\in\mathbb{C}^n}{\max_{j\in\mathbb{N}_{\leq n}}\abs{\Re(z_j)}\le R\wedge\max_{j\in\mathbb{N}_{\leq n}}\abs{\Im(z_j)}\le R}.\]

    Assume that there exists an infinite open cover \(\mathcal{U}\) of \(X\) without finite subcovering. Bisect \(Q_0=Q\) into \(2^{2n}\) sub-cubes (for real and complex parts). Choose \(Q_1\) such that \(Q_1\cup X\) has no finite subcover of \(\mathcal{U}\). Under the previous assumptions, this is possible since if every \(\text{sub-cube}\cap X\) had finite subcovering, then \(Q_0\cap X=X\) would have finite subcovering. Similarly, choose \(Q_2\) by bisecting \(Q_1\) similarly, and recursively obtain a sequence of cubes:
    \[Q_0\supset Q_1\supset Q_2\supset\cdots\]
    Since the side length of each cube tends to 0, \(\bigcap_{j=0}^\infty Q_j\) consists of a single point \(z_{\infty}\in\mathbb{C}^n\). By the Bolzano-Weierstrass Theorem (\cref{thm:bolzanoweierstrass}), because \(\forall j\in\mathbb{N}\), \(Q_j\cap X\neq\varnothing\), select a point \(z_{j}\in Q_j\cap X\), forming a sequence \({z_k}\in X\) convergent to \(z_\infty\in X\) as \(X\) is closed. Therefore, \(\exists U\in\mathcal{U}\) where \(z_\infty\in U\). Since \(U\) is open, \(\exists\varepsilon>0\) such that \(B^n(z_\infty,\varepsilon)\subset U\).\ \(\exists N\in\mathbb{N}\) such that \(\forall k>N\), \(Q_k\subset B^n(z_\infty,\varepsilon)\). Then taking the intersection with \(X\) on both sides, \[Q_k\cap X\subseteq B^n(z_\infty,\varepsilon)\cap X\subset U.\] Our original assumption said that for every \(k\), \(Q_k\cap X\) has no finite subcovering. However, \(U\) covers \(Q_k\cap X\), which is a single open set that covers a nonempty subset. Therefore by contradiction, every open cover has finite subcovering.
\end{proof}
\begin{definition}[Support of a Function]\label{def:support}
    For a set \(X\) and a function \(f:X\to\mathbb{C}\), the \textscsl{support}, denoted by \(\supp(f)=\overline{\cbraces{z\in X}{f(z)\neq 0}}\), is the closure of the set for which \(f\) is nonzero.
\end{definition}
\begin{remark}
    A notable classification of functions comes from the compactness of support---more specifically, its boundedness. Compactly supported functions in \(C^\infty\) are commonly referred to as \textscsl{bump functions} (see \cref{sec:partitionsofunity}).
\end{remark}