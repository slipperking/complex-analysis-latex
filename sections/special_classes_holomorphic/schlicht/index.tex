\subsection{Schlicht Functions}
\begin{definition}
    A holomorphic function \(f:\mathbb{D}\to\mathbb{C}\) is called \textit{schlicht} iff it is univalent and satisfies \(f(0)=0\) and \(f'(0)=1\).
\end{definition}
The power series expansion of schlicht \(f\) around 0 is of the form
\[f(z)=\sum_{n=0}^{\infty}\frac{f^{(n)}(0)}{n!}z^n,\qquad z\in\mathbb{D},\] which by assumption, simplifies to
\[f(z)=z+\sum_{n=2}^{\infty}a_nz^n,\qquad z\in\mathbb{D},\] where \(a_n=\frac{f^{(n)}(0)}{n!}\) for \(n\geq 2\). Two historic results regarding schlicht functions state that
\begin{enumerate}
    \item The only schlicht functions satisfying \(\exists n\in\mathbb{N}\) with \(\abs{a_n}=n\) are the Koebe functions in the form of \[z\mapsto\frac{z}{\qty(1-\ee^{\ii\theta}z)^2}.\] The Koebe functions are more extremal in that it satisfies the conclusion \(\abs{a_n}=n\) for all \(n\in\mathbb{N}\); its power series is given by \[z\mapsto\sum_{n=1}^\infty n\ee^{\ii\theta(n-1)}z^n.\]
    \item For any schlicht \(f\) and \(\forall n\in\mathbb{N}\), it holds that \(\abs{a_n}\leq n\).
\end{enumerate}
Both were solved in the 20th century by Louis de Branges in the affirmative. The second statement is known as the \textit{Bieberbach Conjecture} as it was originally hypothesized by Ludwig Bieberbach.

We will introduce two of the oldest results regarding schlicht functions, namely \(\abs{a_2}\leq 2\) and the Koebe Quarter Theorem (\cref{thm:koebequarter}). There are many applications of the latter, for instance, we used it in the proof of Mergelyan's Theorem (\cref{thm:mergelyan}).
\begin{lemma}
    Let \(f:\mathbb{D}\to\mathbb{C}\) be schlicht and define \(h(z)=\frac1{f(z)}=\frac1z+\sum_{n=0}^\infty b_nz^n\). Let \[A_r=\cbraces{z\in\mathbb{C}}{r<\abs{z}<1}\] be an annulus for \(0<r<1\). Then \(\exists\eta>0\) such that \(\forall r\in(0,1)\) (\(\eta\) independent of \(r\)), \(h\qty(A_r)\) lies in an ellipse with a semi-major axis \(\alpha=\qty(\frac1r+\abs{b_1}\eta)\sqrt{1+\eta r^3}\) and a semi-minor axis \(\beta=\qty(\frac1r-\abs{b_1}\eta)\sqrt{1+\eta r^3}\).
\end{lemma}
\begin{proof}
    Let \(\widetilde{h}(z)=\sqrt{\frac{\overline{b_1}}{\abs{b_1}}}h\qty(z\frac{\overline{b_1}}{\abs{b_1}})-\sqrt{\frac{\overline{b_1}}{b_1}}b_0\), so that \[\widetilde{h}(z)=\frac1z+\sum_{n=1}^\infty\qty(\frac{\overline{b_1}}{\abs{b_1}})^{\frac{1+n}{2}}b_nz^n.\]
    Heuristically, we apply a rigid transformation to \(h\) so that \(\widetilde{h}\) satisfies the theorem hypothesis. Trivially, if the conclusions of the statement are satisfied for \(\widetilde{h}\), then they naturally follow for \(h\). Hence, without loss of generality, we consider only \(h(z)\) such that \(b_0=0\) and \(b_1\in\mathbb{R}\), or when \(h(z)=\frac1z+b_1z+\phi(z)\) where \(\phi(z)=\order{z^2}\). For \(z=r\ee^{\ii\theta}\), 
    \begin{align*}
        h(z)&=\frac1{r}\qty(\cos\theta-\ii\sin\theta)+b_1r\qty(\cos\theta+\ii\sin\theta)+\phi\qty(r\ee^{\ii\theta})\\
        &=\qty(\qty(\frac1r+b_1r)\cos\theta+\Re\phi)+\ii\qty(\qty(b_1r-\frac1r)\sin\theta+\Im\phi).
    \end{align*}
\end{proof}
\begin{theorem}\label{thm:schlichta2leq2}
    If \(f:\mathbb{D}\to\mathbb{C}\) is schlicht and expands to \(\sum_{n=1}^\infty a_nz^n\), then \(\abs{a_2}\leq 2\).
\end{theorem}
\begin{theorem}[\textsc{Koebe Quarter Theorem}]\label{thm:koebequarter}
    If \(f:\mathbb{D}\to\mathbb{C}\) is schlicht, then the image \(f(\mathbb{D})\) contains the open disk of radius \(\frac{1}{4}\) centered at \(f(0)\).
\end{theorem}
\subimport{bieberbach_conjecture/}{index.tex}
