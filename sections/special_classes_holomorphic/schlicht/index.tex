\subsection{Schlicht Functions}
\begin{definition}
    A holomorphic function \(f:\mathbb{D}\to\mathbb{C}\) is called \textit{schlicht} iff it is univalent and satisfies \(f(0)=0\) and \(f'(0)=1\).
\end{definition}
The power series expansion of schlicht \(f\) around 0 is of the form
\[f(z)=\sum_{n=0}^{\infty}\frac{f^{(n)}(0)}{n!}z^n,\qquad z\in\mathbb{D},\] which by assumption, simplifies to
\[f(z)=z+\sum_{n=2}^{\infty}a_nz^n,\qquad z\in\mathbb{D},\] where \(a_n=\frac{f^{(n)}(0)}{n!}\) for \(n\geq 2\). Two historic results regarding schlicht functions state that
\begin{enumerate}
    \item The only schlicht functions satisfying \(\exists n\in\mathbb{N}\) with \(|a_n|=1\) are the Koebe functions.
    \item For any schlicht \(f\) and \(\forall n\in\mathbb{N}\), it holds that \(|a_n|\leq n\).
\end{enumerate} Both were solved in the 20th century by Louis de Branges in the affirmative. The second statement is known as the \textit{Bieberbach Conjecture} as it was originally hypothesized by Ludwig Bieberbach.
\begin{theorem}[\textsc{Koebe Quarter Theorem}]\label{thm:koebequarter}

\end{theorem}
\subimport{bieberbach_conjecture/}{index.tex}
