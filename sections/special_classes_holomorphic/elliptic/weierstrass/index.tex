\subsubsection{Weierstrass Elliptic Functions}
Since any (non-constant) elliptic function has order greater than one, it is natural to next consider elliptic functions of order two.

Within a fundamental parallelogram, by \cref{thm:ellipticfunctionnumberofzerosandpoles}, such a function either has two simple poles or a single double pole. This distinction is the key difference between the Weierstrass theory of elliptic functions (the former case) and the Jacobi theory (the latter case).

In practice, elliptic functions derived from the Jacobi formulation often have more practical use cases, whereas the Weierstrass theory tends to be more convenient in theoretical analysis.
\begin{proposition}\label{prop:weierstrasspfunctionintermediateseriesconvergence}
    Let \(\omega_1\), \(\omega_2\) be a fundamental pair of periods generating the lattice \(\Lambda\). Then the series
    \begin{equation}
        \sum_{\omega\in\Lambda\setminus\cbraces{0}}\frac{1}{\abs{\omega}^\alpha}\label{eq:weierstrasspfunctionintermediateseriesconvergence_statement}
    \end{equation} is (absolutely) convergent for \(\alpha>2\).
\end{proposition}
\begin{proof}
    \begin{figure}
        \centering
        \begin{tikzpicture}
            \pgfmathsetmacro{\n}{4}
            \pgfmathsetmacro{\ax}{1.5}
            \pgfmathsetmacro{\ay}{0.3}
            \pgfmathsetmacro{\bx}{0.5}
            \pgfmathsetmacro{\by}{1.0}
            \pgfmathsetmacro{\over}{0.3}
            \foreach \j in {0,...,\n} {
                    \draw[thin]
                    ({-\over*\ax+\j*\bx},{-\over*\ay+\j*\by})
                    -- ({(\n+\over)*\ax+\j*\bx},{(\n+\over)*\ay+\j*\by});
                }
            \foreach \i in {0,...,\n} {
                    \draw[thin]
                    ({-\over*\bx+\i*\ax},{-\over*\by+\i*\ay}) -- ({\i*\ax+(\n+\over)*\bx},{\i*\ay+(\n+\over)*\by});
                }
            \foreach \i in {-1,1} {
                    \draw[thick] ({\ax*(\n/2+\i)+\bx*(\n/2-1)},{\ay*(\n/2+\i)+\by*(\n/2-1)}) -- ({\ax*(\n/2+\i)+\bx*(\n/2+1)},{\ay*(\n/2+\i)+\by*(\n/2+1)});
                    \draw[thick] ({\ax*(\n/2-1)+\bx*(\n/2+\i)},{\ay*(\n/2-1)+\by*(\n/2+\i)}) -- ({\ax*(\n/2+1)+\bx*(\n/2+\i)},{\ay*(\n/2+1)+\by*(\n/2+\i)});
                }
            \node[shift={(5pt,-5pt)}] at ({\ax*(\n/2)+\bx*(\n/2)},{\ay*(\n/2)+\by*(\n/2)}) {\(0\)};
            \node[shift={(5pt,-5pt)}] at ({\ax*(\n/2+1)+\bx*(\n/2)},{\ay*(\n/2+1)+\by*(\n/2)}) {\(\omega_1\)};
            \node[shift={(9pt,-5pt)}] at ({\ax*(\n/2-1)+\bx*(\n/2)},{\ay*(\n/2-1)+\by*(\n/2)}) {\(-\omega_1\)};
            \node[shift={(5pt,-5pt)}] at ({\ax*(\n/2)+\bx*(\n/2+1)},{\ay*(\n/2)+\by*(\n/2+1)}) {\(\omega_2\)};
            \node[shift={(9pt,-5pt)}] at ({\ax*(\n/2)+\bx*(\n/2-1)},{\ay*(\n/2)+\by*(\n/2-1)}) {\(-\omega_2\)};
            \node[shift={(16pt,-5pt)}] at ({\ax*(\n/2+1)+\bx*(\n/2+1)},{\ay*(\n/2+1)+\by*(\n/2+1)}) {\(\omega_1+\omega_2\)};
            \node[shift={(16pt,-5pt)}] at ({\ax*(\n/2-1)+\bx*(\n/2-1)},{\ay*(\n/2-1)+\by*(\n/2-1)}) {\(-\omega_1-\omega_2\)};
            \node[shift={(16pt,-5pt)}] at ({\ax*(\n/2-1)+\bx*(\n/2+1)},{\ay*(\n/2-1)+\by*(\n/2+1)}) {\(\omega_2-\omega_1\)};
            \node[shift={(16pt,-5pt)}] at ({\ax*(\n/2+1)+\bx*(\n/2-1)},{\ay*(\n/2+1)+\by*(\n/2-1)}) {\(\omega_1-\omega_2\)};
            \coordinate (O) at ({\ax*(\n/2)+\bx*(\n/2)},{\ay*(\n/2)+\by*(\n/2)});

            \pgfmathsetmacro{\lengtha}{sqrt(\ax*\ax+\ay*\ay)}
            \pgfmathsetmacro{\lengthb}{sqrt(\bx*\bx+\by*\by)}
            \pgfmathsetmacro{\sineanglebetween}{sqrt(1-((\ax*\bx+\ay*\by)/(\lengtha*\lengthb))^2)}
            \coordinate (H) at ([shift={(\ay*\lengthb*\sineanglebetween/\lengtha,-\ax*\lengthb*\sineanglebetween/\lengtha)}] O);
            \draw[dotted, thick] (O) -- (H);
            \node[xshift=-4, yshift=-2] at ($(O)!0.5!(H)$) {\(\delta\)};
        \end{tikzpicture}
        \caption{The parallelogram \(P_1\) with 8 periods on its boundary with lattice \(\Lambda\).}\label{fig:weierstrasspfunctionintermediateseriesconvergence_parallelogram}
    \end{figure}Let \(P_n\) be a parallelogram whose center is 0 and has \(n\qty(\omega_1+\omega_2)\) as a vertex (the specific case of \(n=1\) is illustrated in \cref{fig:weierstrasspfunctionintermediateseriesconvergence_parallelogram}). For each \(n\in\mathbb{N}\), there exist \(8n\) periods (points in \(\Lambda\)) on \(\partial P_n\).

    Let \(\delta\) be the distance from 0 to \(\partial P_1\). Hence, the distance from 0 to \(\partial P_n\) is \(n\delta\). Since each \(\omega\in\Lambda^*=\Lambda\setminus\cbraces{0}\) lies in a unique \(\partial P_n\), it follows that \(\abs{\omega}^\alpha\geq n^\alpha \delta^\alpha\) for all \(\alpha>0\). Hence, \[\sum_{\omega\in\Lambda^*}\frac{1}{\abs{\omega}^\alpha}\leq\sum_{n=1}^\infty\frac{8n}{n^\alpha\delta^\alpha}=\frac{8}{\delta^\alpha}\sum_{n=1}^\infty\frac{1}{n^{\alpha-1}},\] which is a \(p\)-series that converges for \(\alpha>2\).
\end{proof}
\begin{proposition}\label{prop:weierstrasspfunctionconvergence}
    Let \(\omega_1\), \(\omega_2\) be a fundamental pair of periods generating the lattice \(\Lambda\). Then the series
    \begin{equation}
        \sum_{\omega\in\Lambda\setminus\cbraces{0}}\qty[\frac{1}{\qty(z-\omega)^2}-\frac{1}{\omega^2}]\label{eq:weierstrasspfunctionconvergence_statement}
    \end{equation}
    locally uniformly converges on \(\mathbb{C}\setminus\Lambda\).
\end{proposition}
\begin{proof}
    Let \(R>0\) be arbitrary. Then \(\forall z\in D(0,R)\), and for \(\abs{\omega}>2R\), we have \[\frac{\abs{z}}{\abs{\omega}}<\frac12,\quad\abs{2-\frac{z}{\omega}}\leq 2+\frac{\abs{z}}{\abs{\omega}}<\frac52,\quad\abs{1-\frac{z}{\omega}}^2\geq\qty(1-\abs{\frac{z}{\omega}})^2>\frac14.\]
    It follows that
    \[\abs{\frac{1}{\qty(z-\omega)^2}-\frac{1}{\omega^2}}=\abs{\frac{2\omega z-z^2}{\omega^2(z-\omega)^2}}=\abs{\frac{z\qty(2-\frac{z}{\omega})}{\omega^3\qty(\frac{z}{\omega}-1)^2}}<\frac{\frac52z}{\frac14\omega^3}\leq\frac{10R}{\omega^3}.\]
    Hence, by \cref{prop:weierstrasspfunctionintermediateseriesconvergence},
    \begin{equation}
        \sum_{\substack{\omega\in\Lambda\\
                \omega\notin\overline{D\qty(0,2R)}}}\qty[\frac{1}{\qty(z-\omega)^2}-\frac{1}{\omega^2}]\label{eq:weierstrasspfunctionconvergence_intermediateseries}
    \end{equation} is termwise bounded by a series \(\sum_{\substack{\omega\in\Lambda\\
            \omega\notin\overline{D\qty(0,2R)}}}\frac{10R}{\omega^3}\), which is convergent by \cref{prop:weierstrasspfunctionintermediateseriesconvergence}. Weierstrass \(M\)--Test (\cref{thm:weierstrassmtest}) gives the uniform convergence of \cref{eq:weierstrasspfunctionconvergence_intermediateseries} on \(D(0,R)\). Since we have omitted only finitely many terms, \cref{eq:weierstrasspfunctionintermediateseriesconvergence_statement} converges uniformly on \(D(0,R)\setminus\Lambda\).

    Let \(K\subset\mathbb{C}\setminus\Lambda\) be compact and arbitrary. By boundedness, \(\exists R>0\) such that \(K\subset D(0,R)\setminus\Lambda\), on which it uniformly converges.
\end{proof}
\begin{definition}[Weierstrass \texorpdfstring{\(\wp\)}{p}-Function]\label{def:weierstrasspfunction}
    Let \(\omega_1\), \(\omega_2\) be a fundamental pair of periods generating the lattice \(\Lambda\). The Weierstrass \(\wp\)-function with period lattice \(\Lambda\) is defined by
    \begin{equation}
        \wp(z)=\frac{1}{z^2}+\sum_{\omega\in\Lambda\setminus\cbraces{0}}\qty[\frac{1}{\qty(z-\omega)^2}-\frac{1}{\omega^2}],\qquad z\in\mathbb{C}\setminus\Lambda.\label{eq:weierstrasspfunction}
    \end{equation}
\end{definition}
By \cref{prop:weierstrasspfunctionconvergence,thm:weierstrassconvergence}, \(\wp\) is well-defined and meromorphic on \(\mathbb{C}\). By \cref{thm:weierstrassconvergence}, we can use termwise differentiation to get \[\wp'(z)=-\frac{2}{z^3}-\sum_{\omega\in\Lambda\setminus\cbraces{0}}\frac{2}{\qty(z-\omega)^3}=-2\sum_{\omega\in\Lambda}\frac1{\qty(z-\omega)^3},\] which is absolutely convergent for \(z\notin\Lambda\). It follows that \(\wp'\) is also meromorphic on \(\mathbb{C}\). Hence, the series expression for \(\wp'(z)\) may be rearranged to \(\wp'(z+\omega_1)\), and hence, \[\wp'(z)=\wp'(z+\omega_1)=\wp'(z+\omega_2).\] Hence, \(\wp'\) is also an elliptic function with period lattice \(\Lambda\). Hence, we must have \[\wp(z)+c_1=\wp\qty(z+\omega_1),\qquad\wp(z)+c_2=\wp\qty(z+\omega_2).\] At \(z=-\frac{\omega_1}{2},-\frac{\omega_2}{2}\), we have \[\wp\qty(-\frac{\omega_1}{2})+c_1=\wp\qty(\frac{\omega_1}{2}),\qquad\wp\qty(-\frac{\omega_2}{2})+c_2=\wp\qty(\frac{\omega_2}{2})\] By evenness of \(\wp\), we must have \(c_1=c_2=0\). Therefore, \(\wp\) is also an elliptic function with period lattice \(\Lambda\). Moreover, each isolated singularity \(\omega\in\Lambda\) is a pole of order two with residue 0 (by \cref{prop:ellipticfunctionresiduesum}). The quotient group \(\mathbb{C}/\Lambda\), may be represented by a fundamental parallelogram \(P\) of \(\Lambda\) with vertices at \(0\), \(\omega_1\), \(\omega_2\), and \(\omega_1+\omega_2\), which by identification, are homeomorphic to a single point on the torus.
\begin{proposition}
    At \(z=0\), \(\wp\) has a double pole with Laurent expansion \[\frac{1}{z^2}+\sum_{n=1}^\infty c_{2n}z^{2n},\qquad c_{2n}=\sum_{\omega\in\Lambda^*}\frac{2n+1}{\omega^{2n+2}}\] whose convergence annulus centered at \(0\) has an outer radius equal to the distance from \(0\) to the nearest period in \(\Lambda^*\).
\end{proposition}
\begin{proof}
    Observe that for \(\abs{z}<R=\min_{\omega\in\Lambda^*}\abs{\omega}\), \(\omega\in\Lambda^*\), \[\frac{1}{\omega-z}=\frac{\frac1\omega}{1-\frac z\omega}=\frac1\omega\sum_{n=0}^\infty\qty(\frac z\omega)^n\implies\frac1{\qty(z-\omega)^2}=\frac1{\omega^2}\qty[\sum_{n=0}^\infty\qty(\frac z\omega)^n]^2.\] Therefore,
    \begin{align*}
        \wp(z) & =\frac{1}{z^2}+\sum_{\omega\in\Lambda^*}\qty[\frac{1}{\qty(z-\omega)^2}-\frac{1}{\omega^2}]=\frac{1}{z^2}+\sum_{\omega\in\Lambda^*}\qty[\frac{1}{\omega^2}\qty(\sum_{m=0}^\infty\qty(\frac z\omega)^m)^2-\frac{1}{\omega^2}] \\
               & =\frac{1}{z^2}+\sum_{\omega\in\Lambda^*}\frac{1}{\omega^2}\qty[\sum_{m=0}^\infty(m+1)\qty(\frac z\omega)^m-1]=\frac{1}{z^2}+\sum_{\omega\in\Lambda^*}\sum_{m=1}^\infty\frac{m+1}{\omega^{m+2}}z^m                            \\
               & =\frac{1}{z^2}+\sum_{m=1}^\infty\qty(\sum_{\omega\in\Lambda^*}\frac{m+1}{\omega^{m+2}})z^m=\frac1{z^2}+\sum_{m=1}^\infty c_mz^m,
    \end{align*}
    where \(c_m=\sum_{\omega\in\Lambda^*}\frac{m+1}{\omega^{m+2}}\). Indeed, since \[\abs{\sum_{\omega\in\Lambda^*}\sum_{m=1}^\infty\frac{m+1}{\omega^{m+2}}z^m}\leq\sum_{\omega\in\Lambda^*}\frac{2\abs{z\omega}-\abs{z}^2}{\qty(\abs{\omega}-\abs{z})^2\abs{\omega}^2}=\sum_{\omega\in\Lambda^*}\frac{\abs{z}\qty(2-\abs{\frac z\omega})}{\abs{\omega}^3\qty(\abs{\frac z\omega}-1)^2},\] which for \(\abs{\omega}>2R\) (which comprises all \(\omega\) except for finitely many) is bounded by \(\sum_{\substack{\omega\in\Lambda\\\abs{\omega}>2R}}\frac{10R}{\abs{\omega}^3}\) (estimates derived in the proof of \cref{prop:weierstrasspfunctionconvergence}), which is convergent by \cref{prop:weierstrasspfunctionintermediateseriesconvergence}. Hence, the series converges absolutely for \(\abs{z}<R\) and the summation exchange is valid. By the symmetry of the period lattice, it is trivial that \(c_{2m+1}=0\) for all \(m\in\mathbb{N}\). Hence, the assertion follows.
\end{proof}
Termwise differentiation (provided by \cref{thm:weierstrassconvergence}) now gives \[\wp'(z)=-\frac{2}{z^3}+\sum_{n=1}^\infty 2n c_{2n} z^{2n-1}.\] Letting \[b_m=\sum_{\omega\in\Lambda^*}\frac1{\omega^m},\] we have \[\wp(z)=\frac1{z^2}+3b_4z^2+5b_6z^4+7b_8z^6+\ldots,\quad\wp'(z)=-\frac2{z^3}+6b_4z+20b_6z^3+42b_8z^5+\ldots.\]

\begin{definition}[Weierstrass \(\zeta\)-Function]

\end{definition}
\begin{definition}[Weierstrass \(\sigma\)-Function]

\end{definition}
