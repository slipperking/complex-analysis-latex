\subsubsection{At the \texorpdfstring{\(\infty\)}{Infinity} Point}
Given the one-point compactification of \(\mathbb{C}\), \(\extcomplex\), we can now define and analyze the behavior of functions near the point at infinity. Similar to the classification of isolated singularities in \(\mathbb{C}\), we can classify \(\infty\) as a removable singularity, a pole, or an essential singularity of a holomorphic function.

Let \(f:\mathbb{C}\setminus\overline{D(0,R)}\to\mathbb{C}\) be holomorphic for
some \(R>0\). Then \(z=\infty\) is an \textscsl{isolated singularity} of \(f\).
To analyze the nature of the singularity, let \(\zeta=\frac{1}{z}\). We define
a new function \(g(\zeta)=f\qty(\frac{1}{\zeta})=f(z)\), which is holomorphic
on \(D^*\qty(0,\frac{1}{R})\). Then at \(\zeta=0\), \(g(\zeta)\) has the
Laurent expansion of \[g(\zeta)=\sum_{n=-\infty}^\infty c_{-n}\zeta^n=\sum_{n=0}^\infty c_{-n}\zeta^n+\sum_{n=1}^\infty c_{n}\zeta^{-n}=\varphi(\zeta)+\psi(\zeta),\]
where \(\varphi\) and \(\psi\) are the holomorphic and principal parts of
\(g\), respectively. Let \(\widetilde{\varphi}(z)=\varphi\qty(\frac{1}{z})\),
\(\widetilde{\psi}(z)=\psi\qty(\frac{1}{z})\). At \(z=0\), \(f\) then has the
Laurent expansion of \[f(z)=\sum_{n=-\infty}^\infty c_nz^n=\sum_{n=0}^\infty c_{-n}z^{-n}+\sum_{n=1}^\infty c_nz^n=\widetilde{\varphi}(z)+\widetilde{\psi}(z).\]

The classification of the singularity at \(\infty\) is then reduced to the
classification of the singularity of \(g\) at \(0\):
\begin{enumerate}
    \item If \(z=\infty\) is a removable singularity of \(f(z)\), then \(f(z)\) has the
        form of \[f(z)=c_0+\frac{c_{-1}}{z}+\frac{c_{-2}}{z^2}+\frac{c_{-3}}{z^3}+\cdots.\]
    \item If \(z=\infty\) is a pole of \(f(z)\) with degree \(m\in\mathbb{N}\), then
        \(f(z)\) can be written as \[f(z)=c_{m}z^m+c_{m-1}z^{m-1}+\cdots+c_0+\frac{c_{-1}}{z}+\cdots,\] where \(c_{m}\neq0\).
    \item If \(z=\infty\) is an essential singularity of \(f(z)\), then \(f(z)\) can be
        expanded as \[f(z)=\sum_{n=-\infty}^\infty c_{n}z^n,\]
        where \(\forall N\in\mathbb{N}\), \(\exists n>N\) such that \(c_{n}\neq0\)
        (infinitely many coefficients of \(\psi\) or \(\widetilde{\psi}\) are nonzero).
\end{enumerate}
\begin{remark}
    Under stereographic projection from the point \(\mqty(0\\0\\1)\) of the unit sphere \(S_2\), a neighborhood of that point maps to a subset of the extended complex plane of the form \(\extcomplex\setminus K\), where \(K\) is a compact and connected subset of \(\mathbb{C}\). Such sets are referred to as \textscsl{neighborhoods of \(\infty\)} in the Riemann sphere.
\end{remark}
\begin{example}
    The function \(z\mapsto\frac{1}{z}\) has a removable singularity at \(z=\infty\), the function \(z\mapsto z^2\) has a pole at \(z=\infty\), and \(z\mapsto \ee^z\) has an essential singularity at \(z=\infty\).
\end{example}
