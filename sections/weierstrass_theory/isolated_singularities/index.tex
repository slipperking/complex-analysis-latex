\subsection{Isolated Singularities}
An \textscsl{isolated singularity} of a complex function is a point
\(a\in\mathbb{C}\) where a function \(f\) is holomorphic on some open punctured
neighborhood of \(a\) (namely, for some \(r>0\), the punctured disk
\(D^*(a,r)\)), but not necessarily defined or holomorphic at \(a\) itself. The
nature of this isolated singularity is characterized by the principal part
\(\psi(z)\) (let \(\varphi(z)\) be the holomorphic part) of the Laurent series
of \(f\) at the point \(a\). Specifically, we can analyze the behavior of
\(f(z)\) as \(z\to a\).
\begin{enumerate}
    \item\label{itm:isolatedsingularities_removable} If \(\lim_{z\to a}f(z)\) exists and is finite, then \(z=a\) is a removable singularity and can be analytically continued to \(D(a,r)\) by \cref{thm:riemannremovablesingularities}. Consequently, \(f(z)\) has a convergent Taylor expansion and the principal part of its Laurent expansion vanishes, and \(f(z)=\varphi(z)\).
    \item\label{itm:isolatedsingularities_pole} If \(\lim_{z\to a}f(z)=\infty\), then \(z=a\) is a \textscsl{pole} of \(f\) (from the stereographic projection and the Riemann sphere, the \(\infty\) is a single point in \(\extcomplex\), and approaching \(\infty\) does not distinguish between different directions, unlike the use of \(+\infty\) and \(-\infty\)).
        \begin{theorem}\label{thm:isolatedsingularities_pole_laurentexpansion}
            The condition \(\lim_{z\to a}f(z)=\infty\) is equivalent to there being a finite number of nonzero \(c_{-n}\)'s, where \(n\in\mathbb{N}\).
        \end{theorem}
        In other words the principal part of \(f\) is equal to \[\psi(z)=\frac{c_{-1}}{z-a}+\cdots+\frac{c_{-m}}{(z-a)^m}\quad c_{-m}\neq0\] for some \(m\in\mathbb{N}\). Therefore,
        \(f(z)=\varphi(z)+\psi(z)=\sum_{n=-m}^\infty c_n(z-a)^n=\frac{g(z)}{(z-a)^m}\)
        on the punctured disk \(D^*(a,r)\), where \(g(z)=\sum_{n=0}^\infty
        c_{n-m}(z-a)^n\) is holomorphic on \(D(a,r)\) and does not attain a zero at
        \(z=a\). Then \(f(z)\) has a pole at \(z=a\) with order \(m\). If \(m=1\), the
        pole is also called a \textscsl{simple pole}.
        \begin{proof}
            Obviously, under the assumption of a finite, nonempty number of non-negative terms in the principal part of the Laurent expansion coefficients, \(\lim_{z\to a}f(z)\to\infty\). Now we will prove the converse. Let \(g(z)=\frac{1}{f(z)}\). Then \(\lim_{z\to a}g(z)=0\). There exists a \(\delta>0\) such that \(f\) is nonzero on \(D^*(a,\delta)\). Then \(g(z)\) is holomorphic on \(D^*(a,\delta)\) and has a removable singularity at \(z=a\). By \cref{thm:riemannremovablesingularities}, \(g\) can be analytically continued to \(D^*(a,\delta)\). Let the multiplicity of the zero at \(z=a\) be \(m\). Then \(g(z)=\phi(z)(z-a)^m\), where \(\phi(z)\) is holomorphic and nonzero at \(z=a\). Then there exists a \(\delta'>0\) such that \(\phi\) is nonzero on \(D(a,\delta')\). It follows that \(\frac{1}{\phi}\) is holomorphic and nonzero on \(D(a,\delta')\). We can then write its Taylor expansion as \[\frac{1}{\phi(z)}=c_{-m}+c_{1-m}(z-a)+\cdots,\]
            where \(c_{-m}\neq0\). It follows that \[f(z)=\frac{1}{g(z)}=\frac{(z-a)^{-m}}{\phi(z)}={c_{-m}}(z-a)^{-m}+c_{1-m}(z-a)^{1-m}+\cdots+c_0+\cdots.\]
            By the uniqueness of the Laurent series (by
            \cref{thm:laurentexpansionofholomorphicfunction}), the conclusion follows.
        \end{proof}
    \item\label{itm:isolatedsingularities_essential} If \(\lim_{z\to a}f(z)\) is nonexistent, then \(a\) is known as an \textscsl{essential singularity}.
        \begin{example}\label{ex:isolatedsingularities_essential_exp1z}
            The function \(\ee^{\frac{1}{z}}\) has an essential singularity at \(z=0\).
        \end{example}
        \begin{proof}
            Observe that \(\lim_{\substack{z\to 0\\z\in\mathbb{R}_{>0}}}=\infty\). Similarly, \(\lim_{\substack{z\to0\\z\in\mathbb{R}_{<0}}}=0\), and \(\lim_{\substack{z\to0\\z\in \ii\mathbb{R}_{>0}}}\) is divergent. Therefore, the limit does not exist.
        \end{proof}
        The implication on its Laurent expansion at \(a\) is:
        \begin{theorem}
            The necessary and sufficient for \(\lim_{z\to a} f(z)\) to not exist is that infinitely many of \(c_{-n}\) (where \(n\in\mathbb{N}\)) are nonzero.
        \end{theorem}
        This follows by elimination from the established trichotomy; if the limit as \(z\to a\) does not exist, then the singularity is neither removable nor a pole (results from \cref{itm:isolatedsingularities_removable} and \cref{itm:isolatedsingularities_pole}). Similar logic can be applied to the coefficients of the Laurent expansion.

        Indeed, in \cref{ex:isolatedsingularities_essential_exp1z}, the Laurent
        expansion is equal to:
        \[\ee^{\frac{1}{z}}=\sum_{n=0}^\infty\frac{z^{-n}}{n!},\]
        which has infinitely many nonzero coefficients of negative powers.
\end{enumerate}
A function with an essential singularity exhibits striking behavior. We will first introduce the following famous result.
\begin{theorem}[name=\textsc{Casorati--Sokhotski--Weierstrass},store=thm:casoratiweierstrass]\label{thm:casoratiweierstrass}
    Let \(a\in\mathbb{C}\) and \(U\subseteq\mathbb{C}\) be an open region. Suppose \(f:U\setminus\cbraces{a}\to\mathbb{C}\) is holomorphic with an essential singularity at \(a\). Then the set of values that \(f\) attains on any open punctured neighborhood of \(a\) is dense. In other words, \(\forall\varepsilon,\delta>0\), \(\forall w\in\mathbb{C}\), \(\exists z\in D^*(a,\delta)\) such that \(|f(z)-w|<\varepsilon\).
\end{theorem}
\begin{proof}
    Assume for the sake of contradiction that \(\exists\varepsilon,\delta>0\), and \(\exists w\in\mathbb{C}\) such that \(\forall z\in D^*(a,\delta)\), \(|f(z)-w|>\varepsilon\). Define the auxiliary function \(g(z)=\frac{f(z)-w}{z-a}\), which is holomorphic and non-vanishing on the punctured neighborhood of \(a\). Since as \(z\to a\), \(g(z)\to\infty\), it follows that \(g(z)\) has a pole at \(a\). Let the order of the pole be \(m\in\mathbb{N}\). By \cref{thm:isolatedsingularities_pole_laurentexpansion}, \(g(z)\) has the Laurent expansion of \[\frac{c_{-m}}{(z-a)^m}+\cdots c_0+c_1(z-a)+\cdots\]
    for some \(m\in\mathbb{N}\). It follows that \[f(z)=\frac{c_{-m}}{(z-a)^{m-1}}+\cdots+c_{-1}+w+c_0(z-a)+\cdots.\] If \(m=1\), then \(f\) has a removable singularity at \(a\). If \(m\geq2\),
    then \(f\) has a pole at \(a\). Hence, we have a contradiction.
\end{proof}
An analogous proof yields the following result for entire functions.
\begin{theorem}\label{thm:casoratiweierstrassentire}
    The set of values that a non-constant entire function \(f\) assumes is dense in \(\mathbb{C}\).
\end{theorem}
\begin{proof}
    For the sake of contradiction, assume there exists \(w\in\mathbb{C}\) and \(\varepsilon>0\) such that \(D(w,\varepsilon)\notin f\qty(\mathbb{C})\). Define \(g(z)=\frac{1}{f(z)-w}\). It follows that \(\abs{g}\leq\frac{1}{\varepsilon}\) on \(\mathbb{C}\). By Liouville's Theorem (\cref{thm:liouville}), \(g\) is a constant function, and hence, \(f\) is also constant, which is a contradiction of the statement.
\end{proof}
In \cref{sec:differentialgeometry}, we will prove a profound generalization of the two results (\cref{thm:greatpicard} and \cref{thm:littlepicard}), which was first proved by Émile Picard in 1879:
\getkeytheorem{thm:littlepicard}
\getkeytheorem{thm:greatpicard}
\subimport{at_infinity/}{index.tex}
