\subsection{Further Properties of Meromorphic and Entire Functions}
\begin{theorem}\label{thm:generalizedargumentprinciple}
    Let \(U\subseteq\mathbb{C}\) be a region and \(f:U\to\mathbb{C}\) be meromorphic. Let \(\gamma\subset U\) be a positively oriented Jordan curve that is null-homotopic in \(U\). If \(f\) has no zeros on \(\gamma\), then \(f\) has finitely many zeros and poles in the region bounded by \(\gamma\). Denote the zeros of \(f\) in the bounded region by \(a_1,\ldots,a_k\) with respective multiplicities \(\alpha_1,\ldots,\alpha_k\), and the poles by \(b_1,\ldots,b_m\) with respective orders \(\beta_1,\ldots,\beta_m\). Let \(\psi\) be any function holomorphic on a neighborhood of the closure of the bounded region. Then
    \[\frac{1}{2\muppi\ii}\oint_{\gamma}\frac{\psi(z)f'(z)}{f(z)}\dd z=\sum_{i=1}^k\alpha_i\psi\qty(a_i)-\sum_{j=1}^m\beta_j\psi\qty(b_j).\]
\end{theorem}
\begin{proof}
    Choose disks \(D\qty(a_i,\varepsilon_i)\) (with pairwise disjoint closures) around each zero \(a_i\) and \(D\qty(b_j,\varepsilon'_j)\) around each pole \(b_j\), with \(\varepsilon_i,\varepsilon'_j>0\) sufficiently small so that these disks are contained in \(\mathrm{int}(\gamma)\), disjoint from \(\gamma\), and contained in the neighborhood where \(\psi\) is holomorphic. The function
    \[g(z)=\frac{\psi(z)f'(z)}{f(z)}\]
    is holomorphic on
    \[\operatorname{int}(\gamma)\setminus\qty(\bigcup_{i=1}^k D\qty(a_i,\varepsilon_i)\cup\bigcup_{j=1}^m D\qty(b_j,\varepsilon'_j)),\]
    since \(\psi\) is holomorphic there, \(f\) is meromorphic with no other singularities, and \(f\neq0\) on \(\gamma\). The oriented boundary of this punctured domain is \(\gamma^+\cup\bigcup_{i=1}^k\partial D\qty(a_i,\varepsilon_i)^-\cup\bigcup_{j=1}^m\partial D\qty(b_j,\varepsilon'_j)^-\). By Cauchy--Goursat (\cref{thm:cauchygoursattheorem}), \[\oint_{\gamma^+}g(z)\ddz+\sum_{i=1}^k\oint_{\partial D\qty(a_i,\varepsilon_i)^-}g(z)\ddz+\sum_{j=1}^m\oint_{\partial D\qty(b_j,\varepsilon'_j)^-}g(z)\ddz=0.\]
    Thus,
    \[-\oint_{\gamma}g(z)\ddz=\sum_{i=1}^k \oint_{\partial D\qty(a_i,\varepsilon_i)^+}g(z)\ddz+\sum_{j=1}^m\oint_{\partial D\qty(b_j,\varepsilon'_j)^+}g(z)\ddz.\]
    Near each zero \(a_i\), write \(f(z)=\qty(z-a_i)^{\alpha_i}h(z)\) where \(h\) is holomorphic at \(a_i\) with \(h\qty(a_i)\neq0\). Then
    \[\frac{f'(z)}{f(z)}=\frac{\alpha_i}{z-a_i}+\frac{h'(z)}{h(z)},\]
    so
    \[g(z)=\psi(z)\qty(\frac{\alpha_i}{z-a_i}+\frac{h'(z)}{h(z)}).\]
    Then,
    \begin{align*}
        \oint_{\partial D\qty(a_i,\varepsilon_i)}g(z)\ddz
         & =\oint_{\partial D\qty(a_i,\varepsilon_i)}\psi(z)\qty(\frac{\alpha_i}{z-a_i}+\frac{h'(z)}{h(z)})\ddz=2\muppi\ii\alpha_i\psi\qty(a_i),
    \end{align*} where the first term has been reduced by the Cauchy--Goursat Formula (\cref{eq:cauchygoursatformula}) and the second integral vanishes by the Cauchy--Goursat Theorem (\cref{thm:cauchygoursattheorem}).

    Near a pole \(b_j\), write \(f(z)=\qty(z-b_j)^{-\beta_j}k(z)\) where \(k\) is holomorphic at \(b_j\) with \(k\qty(b_j)\neq0\). Then
    \[\frac{f'(z)}{f(z)}=-\frac{\beta_j}{z-b_j}+\frac{k'(z)}{k(z)},\]
    so
    \[g(z)=\psi(z)\qty(-\frac{\beta_j}{z-b_j}+\frac{k'(z)}{k(z)}).\]
    A similar calculation yields that
    \[\oint_{\partial D\qty(b_j,\varepsilon'_j)}g(z)\ddz=-2\muppi\ii\beta_j\psi\qty(b_j).\]
    Combining these,
    \begin{align*}
        \oint_{\gamma}\frac{\psi(z)f'(z)}{f(z)}\ddz
         & = \sum_{i=1}^k 2\muppi\ii\alpha_i \psi\qty(a_i)-\sum_{j=1}^m-2\muppi\ii\beta_j\psi\qty(b_j)      \\
         & = 2\muppi\ii\qty(\sum_{i=1}^k \alpha_i\psi\qty(a_i)-\sum_{j=1}^m \beta_j\psi\qty(b_j)).\qedhere
    \end{align*}
\end{proof}
\begin{theorem}[Argument Principle]\label{thm:argumentprinciplemeromorphic}
    Let \(U\subseteq\mathbb{C}\) be a region and \(f:U\to\mathbb{C}\) be meromorphic. Let \(\gamma\subset U\) be a simple, closed, positively oriented curve that is null-homotopic in \(U\). If \(f\) has no zeros or poles on \(\gamma\), then \(f\) has finitely many zeros and poles in the region bounded by \(\gamma\), and the number of zeros, \(k\), minus the number of poles, \(k'\), counting multiplicities and orders, is given by
    \[k-k'=\frac{1}{2\muppi\ii}\oint_{\gamma}\frac{f'(z)}{f(z)}\ddz.\]
    Let \(\Gamma\) be the image of \(\gamma\) under the map \(w=f(z)\). Then \(k-k'=\operatorname{Ind}_{\Gamma}(0)\).
\end{theorem}
\begin{proof}
    By \cref{thm:generalizedargumentprinciple} for \(\psi\equiv 1\),
    \[\frac{1}{2\muppi\ii}\oint_{\gamma}\frac{f'(z)}{f(z)}\ddz=k-k'\]

    Parametrize \(\Gamma\) by \(w=f(z)\). Then \(\dd{w}=f'(z)\ddz\), and
    \[k-k'=\frac{1}{2\muppi\ii}\oint_{\Gamma}\frac{\dd{w}}{w}=\operatorname{Ind}_{\Gamma}(0).\qedhere\]
\end{proof}
\subimport{complex_plane_holomorphic_automorphisms/}{index.tex}
\subimport{extended_plane_holomorphic_automorphisms/}{index.tex}
\subimport{construction_of_entire_and_meromorphic/}{index.tex}
\subimport{growth_hadamard_factorization/}{index.tex}
