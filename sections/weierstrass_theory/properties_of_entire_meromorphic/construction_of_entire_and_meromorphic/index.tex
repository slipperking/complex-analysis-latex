\subsubsection{The Construction of Entire and Meromorphic Functions}\label{sec:constructionofentireandmeromorphicfunctions}
It is common knowledge in algebra that any polynomial can be factored into linear factors. When can this factorization be extended to transcendental entire functions?

We will start by introducing the concept of \textit{infinite products}. Let \(\prod_{k=1}^n\qty(1+u_k)\) be an infinite product. If the limit \(\lim_{n\to\infty}\prod_{k=1}^n\qty(1+u_k)\) exists and is finite, then the infinite product is said to be \textit{convergent}.

For \(x\in\mathbb{R}_{\geq0}\), since \(\ee^x\geq x\) and \(\ee^0=1\), we can integrate over \([0,x]\) to get that \(\ee^x\geq x+1\). Therefore,
\begin{align*}
    \exp(\sum_{k=1}^n\abs{u_k}) & \geq\prod_{k=1}^n \qty(1+\abs{u_k})=1+\sum_{k=1}^n\abs{u_k} \\
                                & \quad+\sum_{\substack{j,k\in\cbraces{1,\ldots,n}            \\j<k}}\abs{u_j u_k}+\cdots+\prod_{k=1}^n\abs{u_k}>\sum_{k=1}^n\abs{u_k}.
\end{align*}
Since the convergence of \(\sum_{k=1}^\infty\abs{u_k}\) is the same as that of \(\exp(\sum_{k=1}^\infty\abs{u_k})\), it follows that the convergence of \(\sum_{k=1}^\infty\abs{u_k}\) is equivalent to that of \(\prod_{k=1}^\infty\qty(1+\abs{u_k})\). If \(\sum_{k=1}^\infty\abs{u_k}\) is convergent, then \(\prod_{k=1}^\infty\qty(1+u_k)\) is \textit{absolutely convergent}. As with the order of summing an absolutely convergent series is unimportant, we may also rearrange terms in an absolutely convergent infinite product.

Similar to series, absolute convergence is a stronger condition than convergence:
\begin{lemma}
    An absolutely convergent infinite product is convergent.
\end{lemma}
\begin{proof}
    Let \(\cbraces{u_k}_{k\in\mathbb{N}}\) be a complex sequence such that \(\sum_{k=1}^\infty\qty|u_k|\) is convergent. Then \(\prod_{k=1}^\infty\qty(1+u_k)\) is absolutely convergent. Let \(Q_n(z)\) denote the partial products of \(\prod_{k=1}^\infty\qty(1+\abs{u_k})\) and let \(P_n(z)\) denote the partial products of \(\prod_{k=1}^n\qty(1+u_k)\). By the Cauchy Criterion (\cref{thm:cauchycriterionsequenceconvergence}), we have that \(\forall \varepsilon>0\), \(\exists N\in\mathbb{N}\) such that \(\forall n>m>N\), \(\abs{Q_m(z)-Q_n(z)}<\varepsilon\). Let us now analyze the absolute difference between the \(P_n\) and \(P_m\):
    \begin{align*}
        \abs{P_n-P_m} & =\abs{\prod_{k=1}^n\qty(1+u_k)-\prod_{k=1}^m\qty(1+u_k)}                                 \\
                      & =\abs{\prod_{k=1}^{m}\qty(1+u_k)\prod_{k=m+1}^{n}\qty(1+u_k)-\prod_{k=1}^{m}\qty(1+u_k)} \\
                      & =\prod_{k=1}^{m}\qty|1+u_k|\cdot\qty|\prod_{k=m+1}^{n}\qty(1+u_k)-1|                     \\
                      & \leq\prod_{k=1}^{m}\qty(1+\qty|u_k|)\cdot\qty|\prod_{k=m+1}^{n}\qty(1+\abs{u_k})-1|      \\
                      & =\abs{Q_n-Q_m}<\varepsilon,
    \end{align*}
    which therefore satisfies \cref{thm:cauchycriterionsequenceconvergence}.
\end{proof} We will now provide the following assertions on the \textit{locally uniform convergence} of infinite products:
\begin{lemma}\label{lem:infiniteproductlocallyuniformconvergencecriterion}
    Let \(U\subseteq\mathbb{C}\) be open and connected. Suppose \(\sum_{k=1}^\infty f_k(z)\) uniformly converges on compact subsets of \(U\) such that each \(f_k\) is holomorphic on \(U\). Then the infinite product \(\prod_{k=1}^\infty \exp\qty[f_k(z)]\) is uniformly convergent on compact subsets of \(U\).
\end{lemma}
\begin{proof}
    Let \(K\) be an arbitrary compact subset of \(U\). Since \(\sum_{k=1}^\infty f_k(z)\) converges uniformly on \(K\), it follows that \(\forall\varepsilon>0\), \(\exists N\in\mathbb{N}\) such that \(\forall n>m>N\), \(\abs{\sum_{k=m+1}^n f_k(z)}<\varepsilon\) for all \(z\in K\). Additionally, we have: \[\abs{\prod_{k=1}^n \exp\qty[f_k(z)]-\prod_{k=1}^m\exp\qty[f_k(z)]}=\abs{\exp[\sum_{k=1}^n f_k(z)]-\exp[\sum_{k=1}^m f_k(z)]}.\]
    By \cref{thm:weierstrassconvergence}, the uniform limit \(\sum_{k=1}^\infty f_k(z)\) is holomorphic on \(U\). By continuity and \cref{thm:continuousfunctionboundedoncompact}, this limit is bounded on \(K\). It follows that each partial sum is uniformly bounded on \(K\). Since the exponential function is Lipschitz continuous on compact subsets of \(\mathbb{C}\), there exists a finite constant \(M>0\) such that \[\abs{\exp[\sum_{k=1}^n f_k(z)]-\exp[\sum_{k=1}^m f_k(z)]}\leq M\abs{\sum_{k=m+1}^n f_k(z)}<M\varepsilon.\qedhere\]
\end{proof}
\begin{remark}
    Uniform convergence on compact subsets is also known as \textit{compact convergence}. In the case of \(\mathbb{C}\) (or in any topological space such that every point has a compact neighborhood), compact convergence is equivalent to \textit{locally uniform convergence}.
\end{remark}
We also have:
\begin{lemma}\label{lem:infiniteproductlocallyuniformconvergencecriterion2}
    Let \(U\subseteq\mathbb{C}\) be open and connected. Suppose \(\sum_{k=1}^\infty \abs{f_k(z)}\) is uniformly convergent on compact subsets of \(U\) such that each \(f_k\) is holomorphic on \(U\). Then the infinite product \(\prod_{k=1}^\infty \qty[1+f_k(z)]\) is uniformly convergent on compact subsets of \(U\) to a holomorphic function, which vanishes only at a point \(z\) iff \(f_k(z)=-1\) for some \(k\in\mathbb{N}\). The multiplicity at each such zero \(z\) is the sum of the multiplicities of \(1+f_k\) at \(z\) for all \(k\) satisfying \(f_k(z)=-1\).
\end{lemma}
\begin{proof}
    Let \(K\subset U\) be an arbitrary compact set. By the uniform convergence of \(\sum_{k=1}^\infty \abs{f_k(z)}\) on \(K\), it follows that the uniform limit is continuous by the Uniform Limit Theorem (\cref{thm:uniformlimit}). By continuity on a compact set, it follows that the limit is bounded by some constant \(M'\). Additionally, \(\forall\varepsilon>0\), \(\exists N\in\mathbb{N}\) such that \(\forall n>N\), \(\sum_{k=1}^n \abs{f_k(z)}<M'+\varepsilon\). It follows that the partial sums are uniformly bounded in \(K\) by \(M=\max\cbraces{\max_{k=1}^N\qty(\max_{z\in K}f_k(z)), M'+\varepsilon}\). Similarly, by earlier discussion of infinite products, we have \[F_n(z)=\prod_{k=1}^n\qty(1+\abs{f_k(z)})\leq\exp\qty(\sum_{k=1}^n\abs{f_k(z)})\leq \ee^{M},\]
    or in other words, the partial products are uniformly bounded on \(K\). Let \(0<\varepsilon<1\) be arbitrary. By definition, there exists \(N\in\mathbb{N}\) such that \(\forall n>m>N\), \(\abs{\sum_{k=m+1}^n f_k(z)}<\varepsilon\) for all \(z\in K\). The difference between the non-absolute partial products satisfies
    \begin{align*}
        \abs{\prod_{k=1}^n\qty(1+{f_k(z)})-\prod_{k=1}^m\qty(1+{f_k(z)})} & \leq \abs{\prod_{k=1}^m\qty(1+{f_k(z)})}\abs{\prod_{k=m+1}^n\qty(1+{f_k(z)})-1} \\
                                                                          & \leq\abs{F_m(z)}\abs{\exp\qty(\sum_{k=m+1}^n\abs{f_k(z)})-1}                    \\
                                                                          & \leq \ee^M\qty(\ee^\varepsilon -1),
    \end{align*}
    where the second inequality can be easily verified by expanding the product \(\prod_{k=m+1}^n\qty(1+{f_k(z)})-1\) and the triangle inequality.

    Since \(\ee^{\varepsilon}-1\to 0^+\), it follows that \(F(z)=\prod_{k=1}^\infty \qty[1+f_k(z)]\) is uniformly convergent on \(K\). Let \(\xi\in U\) be a point satisfying \(F(\xi)=0\). Since there exists an \(m\in\mathbb{N}\) such that \[\prod_{k=m+1}^\infty \qty(1+f_k(z))\] is non-vanishing at \(z=\xi\), and from the fact that \[F(z)=\prod_{k=1}^m \qty(1+f_k(z))\cdot\prod_{k=m+1}^\infty \qty(1+f_k(z)),\] we can analyze the zeros of the finite product to obtain the conclusion.
\end{proof}
We will now study the construction of an entire function \(f(z)\) via its zeros. We have the following cases:
\begin{enumerate}
    \item If \(f\) has no zeros in \(\mathbb{C}\), then the function defined by \(z\mapsto\frac{f'(z)}{f(z)}\) is entire, it is the derivative of an entire function \(\varphi(z)\). Therefore, the function defined by \(z\mapsto f(z)\ee^{-\varphi(z)}\) has the vanishing derivative \(\dv{z}(f(z)\ee^{-\varphi(z)})=f'(z)\ee^{-\varphi(z)}-\varphi'(z)f(z)\ee^{-\varphi(z)}=0\). It follows that \(f(z)\ee^{-\varphi(z)}\) is constant, and therefore \(f(z)=c\exp(\varphi(z))\) for some constant \(c\in\mathbb{C}\). Since \(\varphi(z)\) is entire, it follows that \(f(z)\) is also entire. Absorb the constant \(c\) into \(\varphi(z)\), and \(f(z)=\exp(\varphi(z))\).\label{itm:entirefunctionconstructednonvanishing}
    \item If \(f\) is entire and has finitely many zeros in \(\mathbb{C}\), namely \(a_0=0,a_1,a_2,\ldots,a_n\) with respective multiplicities \(m_0,m_1,\ldots m_n\) (if \(0\) is not a zero, treat \(m_0=0\)), then at each zero \(a_k\), it has the local Taylor expansion of
          \[f(z)=\sum_{j=m_k}^\infty c_j{\qty(z-a_k)}^{j},\] where \(c_{m_k}\neq 0\). Therefore, we can divide \(f(z)\) by \({\qty(z-a_k)}^j\) to obtain a new entire function with no additional zeros and no zero at \(a_k\). Repeating this for every zero, we can define \(\psi(z)=\frac{f(z)}{p(z)}\), which is entire and has no zeros, where \[p(z)=z^{m_0}{\qty(1-\frac{z}{a_1})}^{m_1}\cdots{\qty(1-\frac{z}{a_n})}^{m_n}.\] We write \(p(z)\) in the above form rather than that of \(z^{m_0}\prod_{k=1}^n{\qty(z-a_k)}^{m_k}\) as we aim to generalize the construction to infinite products to study convergence. By \cref{itm:entirefunctionconstructednonvanishing}, \(\psi(z)=\exp(\varphi(z))\) for some entire function \(\varphi(z)\). Therefore, we can write
          \begin{equation}
              f(z)=p(z)\exp(\varphi(z)),\label{eq:weierstrassfactorization_finitezeros}
          \end{equation} where \(p(z)\) is a polynomial with zeros at \(a_k\) with respective multiplicities of \(m_k\). The entire functions \(p(z)\) and \(f(z)\) both have the same zeros with matching multiplicities.
    \item If \(f(z)\) is entire and has infinitely many zeros such that \(f\) is not identically zero. It follows that \(f\) has countably many zeros (since the zeros are isolated). Let the zeros be indexed by \(\mathbb{N}\), namely \(a_1,a_2,\ldots\). Without loss of generality, assume that \(\forall n\in\mathbb{N}\), \(0<\abs{a_n}\leq\abs{a_{n+1}}\) (repeated elements representing multiplicities), and \(\lim_{n\to\infty}a_n=\infty\). The case for a zero at \(0\) will be treated differently.

          There exists a positive integer sequence \(p_1,p_2,\ldots\) such that for every positive and finite \(R\), \(\sum_{n=1}^\infty{\qty|\frac{R}{a_n}|}^{p_n+1}\) converges. For example, let \(p_n=n\), and for sufficiently large \(n\), \(\frac{R}{\abs{a_n}}<1\) and the series is convergent. Consider the infinite product
          \begin{equation}
              \prod_{n=1}^\infty{\qty(1-\frac{z}{a_n})}\exp\qty[\frac{z}{a_n}+\frac{1}{2}\qty(\frac{z}{a_n})^2+\cdots+\frac{1}{p_n}{\qty(\frac{z}{a_n})}^{p_n}].\label{eq:infiniteproductweierstrassfactorizationintermediate}
          \end{equation}
          Let
          \begin{gather}
              P_p(z)=z+\frac{1}{2}z^2+\cdots+\frac{1}{p}z^p\nonumber\\
              Q_p(z)=\log(1-z)+P_p(z)\nonumber\\
              E_p(z)=\exp\qty[Q_p(z)]=(1-z)\exp\qty[P_p(z)].\label{eq:weierstrasselementaryfactor}
          \end{gather}
          Therefore, we can rewrite \cref{eq:infiniteproductweierstrassfactorizationintermediate} as
          \begin{equation*}
              \prod_{n=1}^\infty E_{p_n}\qty(\frac{z}{a_n}).
          \end{equation*}
          The expression in \cref{eq:weierstrasselementaryfactor} is known as the \(p\)-th \textit{Weierstrass elementary factor}.

          By \(\varepsilon\)--\(N\), for a fixed \(R>0\), \(\exists N\in\mathbb{N}\) such that \(\forall n\geq N\), \(\abs{a_n}>2R\) (the coefficient is an arbitrary constant greater than one). Consider the product \(\prod_{n=N}^\infty E_{p_n}\qty(\frac{z}{a_n})\). For \(z\in \overline{D(0,R)}\) and \(n\ge N\), we have \(\qty|\frac{z}{a_n}|\le\frac{1}{2}\). The Taylor expansion \(\log(1-w)=-\sum_{k=1}^\infty\frac{w^k}{k}\) has a convergence disk of \(D(0,1)\). Then,
          \begin{align}
              \abs{Q_{p_n}\qty(\frac{z}{a_n})} & =\abs{-\sum_{k=1}^\infty\frac{1}{k}\qty(\frac{z}{a_n})^k+\sum_{j=1}^{p_n}\frac{1}{k}\qty(\frac{z}{a_n})^k}\leq\sum_{k=p_n+1}^\infty\frac{1}{k}\abs{\frac{z}{a_n}}^k\nonumber                                      \\
                                               & \leq\sum_{k=p_n+1}^\infty\abs{\frac{z}{a_n}}^k=\abs{\frac{z}{a_n}}^{p_n+1}\frac{1}{1-\abs{\frac{z}{a_n}}}\leq 2\abs{\frac{R}{a_n}}^{p_n+1}(\leq 1).\label{eq:infiniteproductweierstrassfactorizationuniformbound}
          \end{align}
          By the definition of \(\cbraces{p_n}_{n\in\mathbb{N}}\), the series \(2\sum_{n=1}^\infty\abs{\frac{R}{a_n}}^{p_n+1}\) converges. Therefore, \(\sum_{n=1}^\infty Q_{p_n}\qty(\frac{z}{a_n})\) is uniformly and absolutely convergent on \(\overline{D(0,R)}\) by the Weierstrass \(M\)--Test (\cref{thm:weierstrassmtest}). We then get that \(\prod_{n=N}^\infty E_{p_n}\qty(\frac{z}{a_n})=\exp(\sum_{n=N}^\infty Q_{p_n}\qty(\frac{z}{a_n}))\), and it uniformly converges on \(\overline{D(0,R)}\) to a nonzero holomorphic function \(f(z)\) on \({D(0,R)}\) by \cref{lem:infiniteproductlocallyuniformconvergencecriterion}, \cref{thm:weierstrassconvergence}, and \cref{thm:hurwitzsimplecase}.

          The zeros of \[\prod_{n=1}^\infty E_{p_n}\qty(\frac{z}{a_n})\] are \(a_1,\ldots, a_{N-1}\) and lie in \(\overline{D(0,2R)}\). To prove the absolute convergence of \(\prod_{n=N}^\infty E_{p_n}\qty(\frac{z}{a_n})\) on \(\overline{D(0,R)}\), we will show that \(\sum_{n=N}^\infty\abs{E_{p_n}\qty(\frac{z}{a_n})-1}\) is convergent. Trivially, when \(\zeta\in\overline{\mathbb{D}}\), we have \(\abs{\exp(\zeta)-1}\leq\exp\qty|\zeta|-1\leq\paren{\ee-1}\abs{\zeta}\). By \cref{eq:infiniteproductweierstrassfactorizationuniformbound} above, we get that \(\qty|Q_{p_n}\qty(\frac{z}{a_n})|\leq 1\) when \(n\geq N\).

          Therefore, we have
          \begin{align*}
              \abs{E_{p_n}\qty(\frac{z}{a_n})-1} & =\abs{\exp(Q_{p_n}\qty(\frac{z}{a_n}))-1}                                            \\
                                                 & \leq (\ee-1)\abs{Q_{p_n}\qty(\frac{z}{a_n})}\leq2(\ee-1)\abs{\frac{R}{a_n}}^{p_n+1},
          \end{align*}
          which has a convergent series by definition. Therefore, \(\prod_{n=N}^\infty E_{p_n}\qty(\frac{z}{a_n})\) is absolutely convergent on \(\overline{D(0,R)}\).

          Letting \(R\to\infty\), for any such sequence \(\cbraces{a_n}_{n\in\mathbb{N}}\), we can define the entire function
          \begin{equation}
              f(z)=\prod_{n=1}^\infty E_{p_n}\qty(\frac{z}{a_n})\label{eq:entirefunctionconstructedfrominfinitelymanyzeros}
          \end{equation} to have zeros at each element of the sequence, and for any compact disk \(\overline{D(0,R)}\), the product above is uniformly convergent. We can then formulate:
          \begin{theorem}[\textsc{Weierstrass Factorization Theorem}]\label{thm:weierstrassfactorization}
              Suppose \(f(z)\) is an entire function. Let \(\cbraces{a_n}_{n\in\mathbb{N}}\) be the sequence of all nonzero zeros of \(f\) satisfying \(a_n\to\infty\) as \(n\to\infty\) and \(0<\qty|a_n|\leq\qty|a_{n+1}|\) (equality of \(a_n\) and \(a_{n+1}\) treated as multiplicities) for all \(n\). Let \(m\) be the multiplicity of \(f(z)\) at \(z=0\) (let \(m=0\) if there is no zero at \(0\)). Then there exists a sequence \(\cbraces{p_n}_{n\in\mathbb{N}}\) of nonnegative integers such that \(\forall R>0\), \(\sum_{n=1}^\infty{\abs{\frac{R}{a_n}}}^{p_n+1}\) converges. Then, we can write
              \begin{equation}
                  f(z)=z^m\ee^{\varphi(z)}\prod_{n=1}^\infty E_{p_n}\qty(\frac{z}{a_n})\label{eq:weierstrassfactorization_statement}
              \end{equation} on \(D(0,R)\), where \(E_p(z)\) is the \(p\)-th Weierstrass elementary factor defined in \cref{eq:weierstrasselementaryfactor} and \(\varphi(z)\) is an entire function. The infinite product converges uniformly on \(\overline{D(0,R)}\) and converges absolutely on \(\mathbb{C}\). If we let \(p_n=n\), we can write \[f(z)=z^m\ee^{\varphi(z)}\prod_{n=1}^\infty\qty(1-\frac{z}{a_n})\exp\qty[\frac{z}{a_n}+\frac{1}{2}\qty(\frac{z}{a_n})^2+\cdots+\frac{1}{n}\qty(\frac{z}{a_n})^n].\]
          \end{theorem}
          \begin{proof}
              By the argument of \cref{eq:entirefunctionconstructedfrominfinitelymanyzeros}, construct \(\psi(z)\) to be entire and have zeros at \(\cbraces{a_n}_{n\in\mathbb{N}}\). Thus, \(z^m\psi(z)\) and \(f(z)\) have the same zeros and corresponding multiplicities. Then the function \(z^m\frac{\psi(z)}{f(z)}\) has removable singularities on all of \(\cbraces{a_n}_{n\in\mathbb{N}}\cup\cbraces{0}\) and has an analytic continuation (\cref{thm:riemannremovablesingularities}) to an entire and non-vanishing function. Therefore, it can be written as \(z^m\frac{\psi(z)}{f(z)}=\ee^{\phi(z)}\), where \(\phi\). Let \(\varphi=-\phi\), and from rearrangement, we obtain \cref{eq:weierstrassfactorization_statement}
          \end{proof}
\end{enumerate}
By the construction above, we have:
\begin{corollary}
    Let \(f\) be meromorphic on \(\mathbb{C}\). Then \(f\) can be written as the quotient of two entire functions.
\end{corollary}
\begin{proof}
    Let \(\varphi(z)\) be any entire function with zeros only at each pole of \(f\) (with multiplicities matching the order of each pole). If there are infinitely many poles, we can explicitly construct such a \(\varphi\) by the Weierstrass Factorization Theorem (\cref{thm:weierstrassfactorization}). If there are finitely many poles, construct \(\varphi\) using \cref{eq:weierstrassfactorization_finitezeros}. It follows that \(\varphi f\) has can be analytically continued (on its removable singularities) to an entire function \(\phi(z)\) with the same zeros as \(f(z)\). Hence, \[f(z)\varphi(z)=\phi(z)\Longleftrightarrow f(z)=\frac{\phi(z)}{\varphi(z)},\]
    which is an explicit construction.
\end{proof}
Therefore, any meromorphic function on \(\mathbb{C}\) can be expressed as the quotient of two infinite products. Hence, any meromorphic function on \(\mathbb{C}\) can be explicitly written in terms of its zeros and poles.

We will now study the construction of meromorphic functions from their poles and the principal parts of their Laurent expansions at each pole.

Suppose \(n\in\mathbb{N}\) and \(\cbraces{a_k}_{k=1}^n\subset\mathbb{C}\) is a sequence of distinct values. Let \(\cbraces{\psi_k(z)}_{k=1}^n\) be a collection of functions in the form of
\begin{equation}
    \psi_k(z)=\sum_{j=m_k}^{p_k}\frac{c_{k,j}}{{\qty(z-a_k)}^j},\label{eq:meromorphicfunctionconstructionprincipalparts}
\end{equation} where \(p_k\geq m_k\) are finite integer constants and \(\cbraces{c_{k,j}}\) are complex constants.

Suppose that \(f(z)\) is meromorphic on \(\mathbb{C}\) such that \(f\) has finitely many poles. Therefore, \(f\) has an isolated singularity at \(\infty\). We have two cases:
\begin{enumerate}
    \item If \(z=\infty\) is a removable singularity or a pole, by the given proof of \cref{thm:rationalmeromorphicfunctions}, we may construct \(f(z)\) to have poles at each of \(\cbraces{a_k}_{k=1}^n\) such that the principal parts of \(f\) at each of \(\cbraces{a_k}_{k=1}^n\) are \(\cbraces{\psi_k(z)}_{k=1}^n\). It can be explicitly written as \[f(z)=p(z)+\sum_{k=1}^n\psi_k(z)\] (we can absorb the constant \(c\) into the polynomial \(\psi_\infty\) as used in the proof).
    \item In the case that \(f(z)\) is a transcendental meromorphic function with an isolated essential singularity at \(z=\infty\), notice that the function defined by \(\varphi(z)=f(z)-\sum_{k=1}^n\psi_k(z)\) has removable singularities at each of \(\cbraces{a_k}_{k=1}^n\). Indeed, since the singularities are isolated, for a fixed \(k\), \(\exists\varepsilon_k>0\) such that for any \(j\neq k\), \(a_j\notin D\qty(a_k,\varepsilon_k)\). It follows that \(\psi_j\) is holomorphic on \(D\qty(a_k,\varepsilon_k)\). Notice that \(f(z)-\psi_k\) is the holomorphic part of the Laurent expansion at \(a_k\) and is also holomorphic on the disk. Suppose \(f\) has \(\psi_k\) as the principal part of its Laurent expansion at \(a_k\). Then \(\varphi(z)\) is holomorphic on \(D\qty(a_k,\varepsilon_k)\). Since \(k\) was arbitrarily chosen, \(\varphi\) is entire and transcendental.

          Therefore, \(f\) can be constructed by \[f(z)=\varphi(z)+\sum_{k=1}^\infty\psi_k(z),\]
          for a transcendental entire function \(\varphi(z)\).
    \item The existence of a transcendental meromorphic function \(f\) whose poles have an accumulation point at \(z=\infty\) is the concern of the following theorem:
          \begin{theorem}[\textsc{Mittag--Leffler}]\label{thm:mittagleffler}
              Let \(\cbraces{a_n}_{n\in\mathbb{N}}\subset\mathbb{C}\) be a sequence of distinct complex numbers such that \(\forall n\in\mathbb{N}\), \(\abs{a_n}\leq\abs{a_{n+1}}\) and \(\lim_{n\to\infty}a_n=\infty\). Let \(\cbraces{\psi_n}_{n\in\mathbb{N}}\) be a function sequence, each in the form of \cref{eq:meromorphicfunctionconstructionprincipalparts}. Then,
              \begin{enumerate}[label=(\alph*)]
                  \item A meromorphic function \(f(z)\) on \(\mathbb{C}\) can be constructed such that \(\forall n\in\mathbb{N}\), \(f\) has a pole at \(a_n\) with a principal part of \(\psi_n\) at \(a_n\).\label{itm:mittagleffler_existence}
                  \item The function \(f(z)\) satisfying the criteria above can be explicitly represented as
                        \begin{equation}
                            f(z)=\varphi(z)+\sum_{n=1}^\infty\qty[\psi_n(z)-p_n(z)]\label{eq:mittagleffler_construction_statement}
                        \end{equation} for some sequence of polynomials \(\cbraces{p_n(z)}\) and an arbitrary entire function \(\varphi(z)\).\label{itm:mittagleffler_construction}
              \end{enumerate}
          \end{theorem}
          \begin{proof}
              The classical proof for this theorem allows for a more explicit construction, as in \cref{eq:mittagleffler_construction_statement}. As for the existence, we can prove \cref{itm:mittagleffler_existence} by use of the \(\overline{\partial}\)-problem.
              \begin{enumerate}[label=(\alph*)]
                  \item Fix \(n\in\mathbb{N}\), let \(U_n\) be an open neighborhood of \(a_n\) such that \(\forall i,j\in\mathbb{N}\) where \(i\neq j\), \(\overline{U_i}\cap \overline{U_j}=\varnothing\). Let \(V_n\) be a neighborhood of \(a_n\) that is relatively compact in \(U_n\). By \cref{thm:bumpfunctionexistence}, for each \(n\), there is a \(C^\infty\) function \(\varphi_n\) satisfying \[
                            \begin{dcases}
                                \varphi_n(z)\equiv 1 & \qif* z\in\overline{V_n},          \\
                                \varphi_n(z)\equiv 0 & \qif* z\in\mathbb{C}\setminus U_n.
                            \end{dcases}\]
                        Let \(u(z)=\sum_{k=1}^\infty\varphi_k(z)\psi_k(z)\), which is an element of \(C^\infty\qty(\mathbb{C}\setminus\cbraces{a_k}_{k\in\mathbb{N}})\). For a fixed \(n\in\mathbb{N}\), it is true that \(u\equiv\psi_n\) on \(\overline{V_n}\setminus\cbraces{a_n}\). Hence, although \(u\) is not meromorphic, it does have the required principal part near each \(a_k\). Let \[\phi(z)=
                            \begin{dcases}
                                \pdv{u}{\overline{z}} & \qif* z\in\mathbb{C}\setminus\cbraces{a_k}_{k\in\mathbb{N}}, \\
                                0                     & \qif* z\in\cbraces{a_k}_{k\in\mathbb{N}}.
                            \end{dcases}\]
                        Since \(\pdv{u}{\overline{z}}\equiv\pdv{\psi_n}{\overline{z}}\equiv 0\) and is \(C^\infty\) on \(\overline{V_n}\setminus\cbraces{a_n}\) and \(\phi\) vanishes on \(\cbraces{a_k}_{k\in\mathbb{N}}\), \(\phi\in C^\infty(\mathbb{C})\). By the discussion proceeding \cref{thm:onedimensionalpartialconjugatesolution}, there exists a \(C^\infty\) function \(v(z)\) such that \(\pdv{v}{\overline{z}}=\phi(z)\) on \(\mathbb{C}\). Since \(\phi\) is \(C^\infty\), it follows that \(v\) is also \(C^\infty\). Define \(f(z)=u(z)-v(z)\). Then
                        \[\pdv{f}{\overline{z}}=\pdv{u}{\overline{z}}-\pdv{v}{\overline{z}}=\phi(z)-\phi(z)=0,\]
                        which implies that \(f\) is holomorphic on \(\mathbb{C} \setminus \cbraces{a_k}_{k\in\mathbb{N}}\). Since \(u\) has the desired principal part \(\psi_n\) at each \(a_n\) and \(v\) is \(C^\infty\) (and hence removable at each singularity), it follows that \(f\) is meromorphic on \(\mathbb{C}\) with principal parts \(\psi_n\) at each \(a_n\), as desired.
                  \item Let \(\cbraces{\varepsilon_n}_{n\in\mathbb{N}}\) be a positive sequence such that \(\sum_{n=1}^\infty\varepsilon_n\) is convergent. Without loss of generality, let \(a_1=0\) (if \(a_1\) is not a pole, set \(\psi_1=0\)). Choose \(p_1(z)=0\) (this can actually be any arbitrary polynomial). Fix \(n\geq 2\), \(\psi_n\) is a polynomial in terms of \(\frac{1}{z-a_n}\) and has its only pole at \(z=a_n\). Therefore, \(\psi_n(z)\) is holomorphic on \(D\qty(0,\abs{a_n})\) and can be written as \[\psi_n(z)=\sum_{k=0}^\infty \frac{\psi_n^{(k)}(0)}{k!}z^k.\]

                        By \cref{thm:abelradius}, This series is uniformly convergent on \({D\qty(0,\abs{\frac{a_n}{2}})}\). Hence, \(\exists \lambda_n\in\mathbb{N}\) such that \[\abs{\psi_n(z)-\sum_{k=0}^{\lambda_n} \frac{\psi_n^{(k)}(0)}{k!}z^k}< \varepsilon_n.\] Let \(p_n(z)=\sum_{k=0}^{\lambda_n}\frac{\psi_n^{(k)}(0)}{k!}z^k\). Fix \(R>0\) and let \(N\in\mathbb{N}\) depend on \(R\) such that \(\abs{a_n}>2R\) for all \(n>N\) and \(\abs{a_n}\le 2R\) for all \(n\le N\). Therefore, \(\forall n>N\), \(R<\abs{\frac{a_n}{2}}\). Then \(\forall z\in D(0,R)\), we have \[\abs{\psi_n(z)-p_n(z)}<\varepsilon_n.\]

                        By the convergence of \(\sum_{n=N+1}^\infty\varepsilon_n\), by the Weierstrass \(M\)--Test (\cref{thm:weierstrassmtest}), the series
                        \begin{equation}
                            \Phi_N(z)=\sum_{n=N+1}^\infty\brackets{\psi_n(z)-p_n(z)}\label{eq:mittagleffler_construction_uniformlyconvergentseries}
                        \end{equation} converges uniformly on \(D(0,R)\). Since \(z<R<\abs{\frac{a_n}{2}}<\abs{a_n}\) when \(n>N\), the pole of \(\psi_n(z)\), \(z=a_n\), is not in \(D(0,R)\) (when \(n>N\)). By \cref{thm:weierstrassconvergence}, \cref{eq:mittagleffler_construction_uniformlyconvergentseries} is holomorphic on \(D(0,R)\). Let \[\Psi(z)=\sum_{n=1}^N\qty[\psi_n(z)-p_n(z)]+\Phi_N(z).\]
                        The poles of \(\Psi(z)\) in \(D(0,R)\) are all of \(a_n\) with corresponding principal parts \(\psi_n(z)\), where \(n\in\mathbb{N}\) and satisfies \(a_n\in D(0,R)\). Since \(R\) was arbitrarily chosen, \(\Psi\) has poles at each \(a_n\) with the corresponding principal part \(\psi_n(z)\) on \(\mathbb{C}\). Let \(\varphi(z)=f(z)-\Psi(z)\) be analytically continued onto each of \(\cbraces{a_n}_{n\in\mathbb{N}}\). Then \(\varphi(z)\) is an entire function (since the Laurent expansions of \(\varphi\) at each of \(\cbraces{a_n}_{n\in\mathbb{N}}\) vanish). By rearrangement, we obtain our desired result.\qedhere
              \end{enumerate}
          \end{proof}
\end{enumerate}
The Mittag--Leffler Theorem (\cref{thm:mittagleffler}) can also be generalized as follows:
\begin{theorem}\label{thm:mittaglefflerboundary}
    Let \(U\subset\mathbb{C}\) be an open set with a simple closed boundary \(\partial U\) and let \(E=\cbraces{a_n}_{n\in\mathbb{N}}\subset U\) be a sequence of distinct complex numbers whose accumulation points lie on \(\partial U\). Let \(\cbraces{\psi_n}_{n\in\mathbb{N}}\) be a sequence of functions in the form of \cref{eq:meromorphicfunctionconstructionprincipalparts}. Then there exists a meromorphic function \(f:U\to\mathbb{C}\) with poles at each \(a_n\) with principal parts \(\psi_n\) at each \(a_n\).
\end{theorem}
Indeed, since \(\partial U\cap U=\varnothing\), each \(a_n\) is not an accumulation of \(E\). In other words, for each \(n\in\mathbb{N}\), there exist neighborhoods \(U_n\) of \(a_n\) that are relatively compact in \(U\) with disjoint closures. The proceeding proof is analogous to that of \cref{itm:mittagleffler_existence} in \cref{thm:mittagleffler}.

Finally, we will examine the construction of entire functions interpolating prescribed values and derivatives at given points.

Let \(\cbraces{z_j}_{j=1}^n\subset\mathbb{C}\) be a sequence of distinct complex numbers and let \(\cbraces{w_j}_{j=1}^n\subset\mathbb{C}\) be a sequence of complex numbers. We can then construct a polynomial \(f(z)\) such that \(\forall j\in\cbraces{1,\ldots,n}\), \(f\qty(z_j)=w_j\). One such explicit formula is given by the \textit{Lagrange interpolation formula}:
\[f(z)=\sum_{j=1}^n\qty[w_j\prod_{\substack{k=1\\k\neq j}}^n\frac{z-z_k}{z_j-z_k}].\]
Then, following the assumption that \(\cbraces{z_j}_{j=1}^n\subset\mathbb{C}\) is a sequence of distinct complex numbers, let \(\cbraces{w_{j,k}}_{\substack{j\in\cbraces{1,\ldots,n}\\ k\in \cbraces{0,\ldots,n_j}}}\) be a sequence where \(\cbraces{n_j}_{j=1}^n\subset\mathbb{N}\). Then we can find a polynomial \(f(z)\) such that \(\forall j\in\cbraces{1,\ldots,n}\), \(\forall k\in\cbraces{0,\ldots,n_j}\), \(f^{(k)}\qty(z_j)=k!w_{j,k}\) (for clarity's sake, \(j\) selects the pair and \(k\) selects the order of the derivative, whose upper bound varies for each \(j\)). Oftentimes, the factorial coefficient is absorbed into \(\cbraces{w_{j,k}}\).

As it turns out, an entire function can in fact be constructed for infinitely many interpolation points, or when \(n\to\infty\).
\begin{theorem}\label{thm:generalinterpolationexistence}
    Let \(\cbraces{z_k}_{k\in\mathbb{N}}\subset\mathbb{C}\) be a discrete set and let \(\cbraces{w_{k,n}}_{\substack{k\in\mathbb{N}\\n\in\cbraces{0,\ldots,n_k}}}\) be a sequence where \(\cbraces{n_k}_{k\in\mathbb{N}}\subset\mathbb{N}\). Then there exists an entire function such that \(\forall k\in\mathbb{N}\), \(\forall n\in\cbraces{0,\ldots,n_k}\),
    \begin{equation}
        f^{(n)}\qty(z_k)=n!w_{k,n}.\label{eq:generalinterpolationexistence_statement}
    \end{equation}
    In other words, an entire function can be constructed by the given first \(n_k\) coefficients of the Taylor expansion at each each \(z_k\).
\end{theorem}
\begin{proof}
    According to the Weierstrass Factorization Theorem (\cref{thm:weierstrassfactorization}), we can construct an entire function \(\Phi(z)\) with zeros at each of \(\cbraces{z_k}_{k\in\mathbb{N}}\) with corresponding multiplicities of \(\cbraces{n_k}_{k\in\mathbb{N}}\). By the discreteness of \(\cbraces{z_k}_{k\in\mathbb{N}}\), there exists a corresponding sequence of radii \(\cbraces{\varepsilon_k}_{k\in\mathbb{N}}\) such that each \(\overline{D\qty(z_k,2\varepsilon_k)}\) are disjoint.

    Define a complex function sequence \(\cbraces{\phi_k(z)}_{k\in\mathbb{N}}\) with \[\phi_k(z)=\sum_{n=0}^{n_k-1}w_{k,n}\qty(z-z_k)^n,\] where \(k\in\mathbb{N}\). By \cref{thm:bumpfunctionexistence}, we can construct a \(C^\infty\) sequence of functions \(\cbraces{\varphi_k(z)}_{k\in\mathbb{N}}\) such that \(\forall k\in\mathbb{N}\), \(\supp\varphi_k\subset D\qty(z_k, 2\varepsilon_k)\), \(\varphi_k\equiv 1\) on \(\overline{D\qty(z_k,\varepsilon_k)}\), and \(0\le\varphi\le 1\) on \(\mathbb{C}\).

    Let \(\Psi\in C^\infty(\mathbb{C})\), and construct
    \begin{equation}
        f(z)=-\Phi(z)\Psi(z)+\sum_{k=1}^\infty\phi_k(z)\varphi_k(z).\label{eq:generalinterpolationexistence_constructionstatement}
    \end{equation}
    Under what conditions on \(\Psi\) will \(f\) be entire? Since the supports of each \(\varphi_k\) are disjoint, the summation \(\sum_{k=1}^\infty\phi_k(z)\varphi_k(z)\) contains at most one nonzero term and is convergent and well-defined. To construct \(f\) to be entire, we must have \(\pdv{f}{\overline{z}}=0\). In other words, we want for \[\pdv{\overline{z}}(\sum_{k=1}^\infty\phi_k\varphi_k)=\pdv{\overline{z}}[\Phi\Psi]\Longleftrightarrow\sum_{k=1}^\infty\phi_k\pdv{\varphi_k}{\overline{z}}=\Phi\pdv{\Psi}{\overline{z}}\] on all of \(\mathbb{C}\). Let \(g(z)=\sum_{k=1}^\infty\phi_k(z)\pdv{\varphi_k(z)}{\overline{z}}\). Since \(\varphi_k\equiv 1\) on \(\overline{D\qty(z_k,\varepsilon_k)}\), \(\pdv{\varphi_k}{\overline{z}}\equiv 0\) on \(\bigcup_{k=1}^\infty \overline{D\qty(z_k,\varepsilon_k)}\). Consequently, \(g(z)\equiv 0\) on \(\bigcup_{k=1}^\infty \overline{D\qty(z_k,\varepsilon_k)}\).

    From rearrangement, we have \(\frac{g(z)}{\Phi(z)}=\pdv{\Psi}{\overline{z}}\), which has removable singularities at each \(z_k\). Define \(\frac{g(z)}{\Phi(z)}=0\) at \(z=z_k\). Under this assertion, we have \(\frac{g(z)}{\Phi(z)}\in C^\infty(\mathbb{C})\). Since the support of \(\frac{g(z)}{\Phi(z)}\) is the union of disjoint compact sets, by \cref{thm:onedimensionalpartialconjugatesolution}, there exists a function \(\Psi\in C^\infty(\mathbb{C})\) satisfying \(\frac{g(z)}{\Phi(z)}=\pdv{\Psi}{\overline{z}}\). Since \(g\) is vanishing on \(\bigcup_{k=1}^\infty \overline{D\qty(z_k,\varepsilon_k)}\), it follows that \(\frac{g}{\Phi}\) is vanishing on \(\bigcup_{k=1}^\infty \overline{D\qty(z_k,\varepsilon_k)}\), and \(\Psi\) is holomorphic on \(\bigcup_{k=1}^\infty{D\qty(z_k,\varepsilon_k)}\).

    Fix \(k\in\mathbb{N}\) and let \(n\in\cbraces{0,\ldots,n_k-1}\). For \(z\in D\qty(z_k,\varepsilon_k)\), from \cref{eq:generalinterpolationexistence_constructionstatement}, we have \[f(z)=-\Phi(z)\Psi(z)+\phi_k(z).\] Since \(\Phi\) has a zero at \(z_k\) with multiplicity \(n_k\), we have \(\Phi(z)\Psi(z)\) vanishing at \(z_k\) with a multiplicity of at least \(n_k\). Therefore, we have that
    \begin{align*}
        f^{(n)}\qty(z_k) & =\left.\dv[n]{z}(\sum_{j=0}^{n_k-1}w_{k,j}\qty(z-z_k)^j)\right|_{z=z_k}-\left.\dv[n]{z}(\sum_{j=n_k}^\infty w'_{k,j}\qty(z-z_k)^j)\right|_{z=z_k} \\
                         & =\lim_{z\to z_k}\sum_{j=n}^{n_k-1}\qty(w_{k,j}\qty(z-z_k)^{j-n}\prod_{r=j-n+1}^{j}r)                                                              \\
                         & \quad-\sum_{j=n_k}^\infty\qty(w'_{k,j}\qty(z-z_k)^{j-n}\prod_{r=j-n+1}^{j}r)                                                                      \\
                         & =n!w_{k,n},
    \end{align*}
    as desired.
\end{proof}
\begin{remark}
    For a general power series, there is no assurance that it corresponds to the Taylor expansion of an entire function. However, for any polynomial of degree \(n\), there always exists an entire function whose Taylor expansion agrees with the polynomial up to the first \(n+1\) terms, which is the fundamental difference between a polynomial and a transcendental entire function.
\end{remark}
\begin{example}\label{ex:csc^2poleexpansion}
    Prove the pole expansion formula of \[\csc^2(\uppi z)=\frac{1}{\uppi^2}\sum_{n=-\infty}^\infty \frac{1}{\qty(z-n)^2}\] for \(z\in\mathbb{C}\setminus\mathbb{Z}\).
\end{example}
\begin{proof}
    It is evident that \(\csc^2(\uppi z)\) has double poles at each of \(z=n\in\mathbb{Z}\). Therefore, \((z-n)^2\csc^2(\uppi z)\) is holomorphic (or has an analytic continuation that is holomorphic) on \(D(n,1)\). Therefore, by repeatedly applying L'Hôpital's Rule, we can write its Taylor expansion that converges on \(D(n,1)\):
    \begin{align*}
        (z-n)^2\csc^2(\uppi z) & =\lim_{\zeta\to n}\frac{\qty(\zeta-n)^2}{\sin[2](\uppi\zeta)}+\lim_{\zeta\to n}\dv{\zeta}(\frac{\qty(\zeta-n)^2}{\sin[2](\uppi\zeta)})(z-n)+\order{(z-n)^2} \\
                               & =\lim_{\zeta\to n}\frac{2\qty(\zeta-n)}{\uppi\sin(2\uppi\zeta)}+\order{(z-n)^2}                                                                             \\
                               & =\lim_{\zeta\to n}\frac{2}{2\uppi^2\cos(2\uppi\zeta)}+\order{(z-n)^2}                                                                                       \\
                               & =\frac{1}{\uppi^2}+\order{(z-n)^2},
    \end{align*}
    where the linear term vanishes since \(\frac{\qty(\zeta-n)^2}{\sin[2](\uppi\zeta)}\) is even around \(\zeta=n\). Hence, the principal part of \(\csc^2(\uppi z)\) at \(z=n\) is \(\psi_n(z)=\frac{1}{\uppi^2(z-n)^2}\). Let \(z=x+\ii y\). Since
    \begin{align*}
        \abs{\sum_{n=-\infty}^\infty\psi_n(z)} & \leq\frac{1}{\uppi^2}\sum_{n=-\infty}^\infty\frac{1}{\qty|z-n|^2}\leq\frac{1}{\uppi^2}\sum_{n=-\infty}^\infty\frac{1}{\qty(x-n)^2}                                                \\
                                               & =\frac{1}{\uppi^2}\qty[\sum_{n=2}^\infty\frac{1}{\qty[n-\cbraces{x}]^2}+\sum_{n=1}^\infty\frac{1}{\qty[n+\cbraces{x}]^2}+\frac{1}{\cbraces{x}^2}+\frac{1}{\qty[1-\cbraces{x}]^2}] \\
                                               & \leq\frac{1}{\uppi^2}\qty[2\sum_{n=1}^\infty\frac{1}{n^2}+\frac{1}{\cbraces{x}^2}+\frac{1}{\qty[1-\cbraces{x}]^2}]
    \end{align*} (where \(\cbraces{x}\) is the fractional part of the real component of \(z\)), it converges uniformly on compact subsets of \(\mathbb{C}\setminus\mathbb{Z}\) by the Weierstrass \(M\)--Test (\cref{thm:weierstrassmtest}). It follows that this sum is holomorphic on \(\mathbb{C}\setminus\mathbb{Z}\) by \cref{thm:weierstrassconvergence}. The difference \[f(z)=\csc[2](\uppi z)-\frac{1}{\uppi^2}\sum_{n=-\infty}^\infty\frac{1}{(z-n)^2}\] is holomorphic on \(\mathbb{C}\setminus\mathbb{Z}\). Since the principal parts of \(f\) at each \(z=n\) are zero, it follows that \(f\) is entire. Therefore, it is bounded on every compact subset of the complex plane. Since \(f\) is bounded on the compact rectangle \(K=\cbraces{z\in\mathbb{C}}{0\leq\Re(z)\leq 1,-1\leq\Im(z)\leq 1}\), for some \(M>0\), we have \(\abs{f}\leq M\) on \(K\). For \(0\leq x=\Re(z)\leq 1\) satisfying \(\abs{y}=\abs{\Im(z)}>1\), by the reverse triangle inequality, we have
    \begin{align}
        \abs{f(z)} & \leq\abs{\csc[2](\uppi z)}+\frac{1}{\uppi^2}\qty[\sum_{n=2}^\infty\frac{1}{\qty[n-\cbraces{x}]^2}+\sum_{n=1}^\infty\frac{1}{\qty[n+\cbraces{x}]^2}+\frac{1}{\abs{z}^2}+\frac{1}{\abs{1-z}^2}]\nonumber \\
                   & \leq \frac{1}{\abs{\sin^2\qty[\uppi(x+\ii y)]}}+\frac{1}{\uppi^2}\qty[\frac{\uppi^2}{3}+\frac{2}{y^2}]\nonumber                                                                                        \\
                   & =\frac{1}{\abs{\sin(\uppi x)\cosh(\uppi y)+\ii\sinh(\uppi y)\cos(\uppi x)}^2}+\frac{1}{3}+\frac{2}{\uppi^2 y^2}\nonumber                                                                               \\
                   & \leq \frac{1}{\sin[2](\uppi x)\cosh[2](\uppi y)+\sinh[2](\uppi y)\qty[1-\sin[2](\uppi x)]}+\frac{1}{3}+\frac{2}{\uppi^2}\nonumber                                                                      \\
                   & \leq \frac{1}{\sin[2](\uppi x)+\sinh[2](\uppi y)}+\frac{1}{3}+\frac{2}{\uppi^2}\leq\csch[2](\uppi)+\frac{1}{3}+\frac{2}{\uppi^2}.\label{eq:csc^2poleexpansionuniformbound}
    \end{align}
    Since \[\abs{f}<\max\cbraces{M,\csch[2](\uppi)+\frac{1}{3}+\frac{2}{\uppi^2}}\] on \(\mathbb{C}\), Liouville's Theorem (\cref{thm:liouville}) gives that \(f\) is constant. At \(z=\frac{1}{2}\), we have
    \begin{align*}
        f\qty(\frac{1}{2}) & =\csc[2](\frac{\uppi}{2})-\frac{8}{\uppi^2}\sum_{n=1}^\infty\frac{1}{\qty[2n-1]^2}=1-\frac{8}{\uppi^2}\qty(\sum_{n=1}^\infty\frac{1}{n^2}-\sum_{n=1}^\infty\frac{1}{(2n)^2}) \\
                           & =1-\frac{8}{\uppi^2}\qty(\frac{\uppi^2}{6}-\frac{\uppi^2}{24})=0,
    \end{align*}
    which completes the proof.
\end{proof}
