\subsubsection{The Group of Meromorphic Automorphisms on \texorpdfstring{\(\extcomplex\)}{the Extended Complex Plane}}
It is generally common to consider a meromorphic function as a function in the form of \(f:U\to\extcomplex\). Let \(\Aut\qty(\extcomplex)\) denote the group of meromorphic automorphisms on \(\extcomplex\).

To make more profound conclusions on the structure of \(\Aut\qty(\extcomplex)\), we will introduce certain concepts from group theory.
\begin{definition}[Coset]\label{def:coset}
    Let \(G\) be a group, and let \(H\leq G\) be a subgroup (operation denoted by juxtaposition). Then the \textit{left coset} of \(H\) in \(G\) with respect to \(g\in G\) is defined as \[gH=\cbraces{gh}{h\in H}.\] The \textit{right coset} is defined as \[Hg=\cbraces{hg}{h\in H}.\] The subgroup \(H\) is \textit{normal} iff the left and right cosets are equal. The notation \(H\trianglelefteq G\) is used to represent a normal subgroup. Cosets, like groups and sets, are unordered.
\end{definition}
\begin{theorem}
    Let \(G\) be a group and \(N\leq G\) a subgroup. The set of left cosets \(G/N=\cbraces{gN}{g\in G}\) admits a group structure with operation
    \[(gN)(hN)=(gh)N\]
    if and only if \(N\) is a normal subgroup of \(G\).
\end{theorem}
\begin{proof}
    We prove the two implications separately.
    \begin{enumerate}
        \item \emph{If \(N\trianglelefteq G\) then \(G/N\) is a group.}

              Assume \(N\) is normal, \(N\trianglelefteq G\), so \(gNg^{-1}=N\) for every \(g\in G\) (equivalently \(g^{-1}Ng=N\)). Define a product on \(G/N\) by
              \[(gN)(hN)=(gh)N.\]
              We now verify that this product is well-defined: if \(gN=g'N\) and \(hN=h'N\) then we need \((gh)N=(g'h')N\). Since \(gN=g'N\), there exists \(n_1\in N\) with \(g'=gn_1\), and since \(hN=h'N\) there exists \(n_2\in N\) with \(h'=hn_2\). Then
              \[g'h'=\qty(gn_1)\qty(hn_2)=g\qty(n_1h)n_2=gh\qty(h^{-1}n_1h)n_2.\]
              Because \(N\) is normal we have \(h^{-1}n_1h\in N\), and \(n_2\in N\), so \(\qty(h^{-1}n_1h)n_2\in N\). Hence \(g'h'\in(gh)N\), meaning that \(\qty(g'h')N=(gh)N\). Thus the product is well-defined.

              Associativity follows from associativity in \(G\):
              \[\qty((gN)(hN))(kN)=(ghk)N=(gN)\qty((hN)(kN)).\]
              The identity is \(eN=N\), since \((eN)(gN)=(eg)N=gN\) and similarly on the other side. The inverse of \(gN\) is \(g^{-1}N\), because \((gN)\qty(g^{-1}N)=\qty(gg^{-1})N=N\). Thus \(G/N\) is a group.
        \item \emph{If \(G/N\) can be given a group structure via the coset multiplication, then \(N\trianglelefteq G\).}

              Fix \(g\in G\) and \(n\in N\) arbitrarily. By assumption, we have \[(gN)(nN)\qty(g^{-1}N)=\qty(gng^{-1})N.\]
              Because \(nN=(e)N\), we also have \[\qty(gN)(nN)\qty(g^{-1}N)=\qty(gN)(eN)\qty(g^{-1}N)=\qty(gg^{-1})N=N,\] implying that \(\qty(gng^{-1})N=N\), and hence \(gng^{-1}\in N\) for any \(g\in G\), \(n\in N\). Hence, \(gNg^{-1}\subseteq N\). Now replacing \(g\) with \(g^{-1}\) and rearranging yields \(n\in gNg^{-1}\), or that \(N\subseteq gNg^{-1}\). Therefore, \(N\) is normal.
    \end{enumerate}
\end{proof}
Under the normality of \(N\), the group \(G/N\) is known as the \textit{quotient group} of \(G\) by \(N\).
\begin{remark}
    Every subgroup of an abelian group is normal.
\end{remark}
\begin{definition}[Group Homomorphism]
    Let \((G,\cdot)\) and \((H,*)\) be groups. A function \(\varphi:G\to H\) is said to be a \textit{group homomorphism} if
    \[\varphi(g_1\cdot g_2)=\varphi(g_1)*\varphi(g_2)\qquad\forall g_1,g_2\in G.\]
\end{definition}
\begin{definition}[Group Isomorphism]
    A group homomorphism \(\varphi:G\to H\) is called an \textit{isomorphism} if it is
    bijective.
\end{definition}
If there exists an isomorphism between two groups \(G\) and \(H\), they are said to be
\textit{isomorphic}, denoted by \(G\cong H\). The utility of groups allows us to classify them according to their structure:
if two groups are isomorphic, they are essentially the same from a
group-theoretic perspective. This viewpoint lets us replace complicated groups
with simpler, isomorphic ones, and study their properties without loss of
generality.

Let us now examine \(\Aut\qty(\extcomplex)\). Let \(f(z)\in \Aut\qty(\extcomplex)\) such that \(f(\infty)=\infty\). It follows that \(f\) maps \(\mathbb{C}\) to \(\mathbb{C}\) bijectively and \(f\in\Aut(\mathbb{C})<\Aut\qty(\extcomplex)\). Therefore, \(f(z)\) has the form of \(az+b\), where \(a\in\mathbb{C}^*=\mathbb{C}\setminus\cbraces{0}\) and \(b\in\mathbb{C}\) are constants.

Let \(f(z)\in\Aut\qty(\extcomplex)\) such that \(f(\infty)\neq\infty\). Then, \(g(z)=\frac{1}{f(z)-f(\infty)}\in\Aut\qty(\extcomplex)\) and \(g(\infty)=\infty\). By the property above, \(g(z)=cz+d\) for some complex \(d\) and nonzero \(c\). Hence, \(f(z)=\frac{f(\infty)(cz+d)+1}{cz+d}\). Let \(a=cf(\infty)\), \(b=df(\infty)+1\), \(f(z)=\frac{az+b}{cz+d}\). Then in this specific construction, \(ad-bc=-c\neq0\). Let the matrix \(\mqty(a&b\\c&d)\) correspond to this transformation, where for any nonzero scalar \(k\), \(k\mqty(a&b\\c&d)\) corresponds to \(\mqty(a&b\\c&d)\). Therefore, we can arbitrarily pick \(ad-bc\) to be \(1\).

Therefore, there exists a one-to-one correspondence between \(\Aut\qty(\extcomplex)\) and the group (under matrix multiplication) of \[\left.\cbraces{\mqty(a&b\\c&d)}{\det\mqty(a&b\\c&d)=1}\middle/\cbraces{\pm\symbf{I}},\right.\]
where \(\symbf{I}=\mqty(\imat{2})\). The quotient group (a group of elements in the form \(\cbraces{\symbf{A},-\symbf{A}}\)) is taken because the matrix \(\mqty(a&b\\c&d)\) corresponds to the same transformation as \(\mqty(-a&-b\\-c&-d)\). This group, denoted by \(\mathrm{PSL}\qty(2,\mathbb{C})=\left.\mathrm{SL}(2,\mathbb{C})\middle/\cbraces{\pm \symbf{I}}\right.\cong\Aut\qty(\extcomplex)\), is known as the \textit{projective special linear group} of order 2, and is isomorphic to the \textit{group of Möbius transformations}, consisting of all complex linear fractional transformations.

Therefore, any meromorphic automorphism on \(\extcomplex\) is a composition of rotations, dilations, translations, and inversions. We will now state this formally:
\begin{theorem}[The Meromorphic Automorphism Group on \(\extcomplex\)]\label{thm:meromorphicautomorphismgrouponextendedcomplexplane}
    \(\forall f\in \Aut\qty(\extcomplex)\), \(f\) is a Möbius transformation. In other words, \(\exists a,b,c,d\in\mathbb{C}\) satisfying \(ad-bc\neq 0\) such that \[f(z)=\frac{az+b}{cz+d}.\] Moreover, every such Möbius transformation is in \(\Aut\qty(\extcomplex)\).
\end{theorem}
The group of holomorphic automorphisms on \(\mathbb{D}\), or \(\Aut(\mathbb{D})\), is also a subgroup of \(\Aut\qty(\extcomplex)\).
\begin{proposition}\label{prop:mobiustransformationcompositionmatrixmultiplication}
    Suppose we have two Möbius transformations represented by the matrices \(\mqty(a&b\\c&d)\) and \(\mqty(e&f\\g&h)\). Then their composition is a Möbius transformation represented by \(\mqty(a&b\\c&d)\mqty(e&f\\g&h)\).
\end{proposition}
\begin{proof}
    From simple substitution, we have
    \[\frac{a\frac{ez+f}{gz+h}+b}{c\frac{ez+f}{gz+h}+d}=\frac{aez+af+bgz+bh}{cez+cf+dgz+dh}=\frac{\qty(ae+bg)z+(af+bh)}{\qty(ce+dg)z+(cf+dh)},\]
    which corresponds to the product \(\mqty(a&b\\c&d)\mqty(e&f\\g&h)\).
\end{proof}
We have now introduced three of the most important regions in complex analysis: \(\mathbb{D}\), \(\mathbb{C}\), and \(\extcomplex\). Their importance will be later explained by the Uniformization Theorem (\cref{thm:uniformization}).
