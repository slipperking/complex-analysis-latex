\subsubsection{Classifying Growth and the Hadamard Factorization Theorem}
\begin{lemma}\label{lem:nonvanishingholomorphiclogarithmabsolutemeanvalueproperty}
    Let \(f:\overline{D(0,R)}\to\mathbb{C}^*\) (where \(R>0\)) be a nowhere-vanishing holomorphic function. It follows that \[\log\qty|f(0)|=\frac{1}{2\uppi}\int_0^{2\uppi}\log\abs{f\qty(R\ee^{\ii\theta})}\dd{\theta}.\]
\end{lemma}
\begin{proof}
    Without loss of generality, assume \(R=1\). Since \(f\) is non-vanishing and \(\overline{\mathbb{D}}\) is simply connected, we may define the \textit{holomorphic logarithm} as \[\log(f(z))=\int_\gamma \frac{f'(z)}{f(z)}\ddz+\log\qty(f\qty(z_0))\] for any fixed \(z_0\in \overline{\mathbb{D}}\) and all \(z\in\overline{\mathbb{D}}\), where \(\gamma\subset\overline{\mathbb{D}}\) is any piecewise smooth curve from \(z_0\) to \(z\).

    Hence, \(\log\abs{f(z)}=\Re\qty[\log\qty(f(z))]\) and is therefore harmonic. The assertion then follows from the mean-value property.
\end{proof}
\begin{theorem}[Jensen's Formula]\label{thm:jensensformula}
    Let \(f:D(0,R)\to\mathbb{C}\) be holomorphic such that \(f(0)\neq 0\). Suppose that \(f\) is continuous on \(\overline{D(0,R)}\). If \(a_1,\ldots,a_n\) are the zeros of \(f\) in \(\overline{D(0,R)}\), counted with multiplicities, then \[\log\abs{f(0)}=\frac{1}{2\uppi}\int_{0}^{2\uppi}\log\abs{f\qty(R\ee^{\ii\theta})}\dd{\theta}+\sum_{k=1}^n\log\abs{\frac{a_k}{R}}.\]
\end{theorem}
\begin{proof}
    First assume that no zeros lie on \(\partial D(0,R)\). It follows that \(\forall 1\leq k\leq n\), \[\varphi_{\frac{a_k}{R}}\qty(\frac{z}{R})=\frac{R\qty(z-a_k)}{R^2-\overline{a_k}z}\] is holomorphic on \(\overline{D(0,R)}\), and has a simple zero at \(z=a_j\). Additionally, the transformation maps \(\partial D(0,R)\) to \(\partial\mathbb{D}\). The function \[g(z)=\frac{f(z)}{\prod_{k=1}^n\varphi_{\frac{a_k}{R}}\qty(\frac{z}{R})}\] is then holomorphic and non-vanishing in \(\overline{D(0,R)}\). Hence, by \cref{lem:nonvanishingholomorphiclogarithmabsolutemeanvalueproperty}, \[\log\qty|g(0)|=\frac{1}{2\uppi}\int_0^{2\uppi}\log\abs{g\qty(R\ee^{\ii\theta})}\dd{\theta},\]
    implying that
    \begin{align}
        \log\abs{f(0)}-\sum_{k=1}^n\log\abs{\varphi_{\frac{a_k}{R}}(0)} & =\frac{1}{2\uppi}\int_0^{2\uppi}\log\abs{f\qty(R\ee^{\ii\theta})}\dd{\theta}\nonumber                                                                     \\
                                                                        & \quad-\frac{1}{2\uppi}\sum_{k=1}^n\int_0^{2\uppi}\log\abs{\varphi_{\frac{a_k}{R}}\qty(\ee^{\ii\theta})}\dd{\theta}.\label{eq:jensensformula_intermediate}
    \end{align}
    Since for fixed \(k\), \[\log\abs{\varphi_{\frac{a_k}{R}}(0)}=\log\abs{\frac{a_k}{R}}\qand\log\abs{\varphi_{\frac{a_k}{R}}\qty(\ee^{\ii\theta})}=\log(1)=0,\]
    \cref{eq:jensensformula_intermediate} becomes \[\log\abs{f(0)}=\frac{1}{2\uppi}\int_0^{2\uppi}\log\abs{f\qty(R\ee^{\ii\theta})}\dd{\theta}+\sum_{k=1}^n\log\abs{\frac{a_k}{R}}.\]
    Suppose \(f\) has additional zeros at each \(\cbraces{b_j=R\ee^{\ii\vartheta_j}}_{1\leq j\leq m}\subset\partial D(0,R)\), each with multiplicity \(k_j\). Since the zeros here are discrete, we can cover them with disks \(D\qty(b_j,\varepsilon_j)\) within the domain of holomorphy of \(f\) such that their closures are disjoint and do not intersect any \(a_k\). Notice that near each \(b_j\), \(f(z)=\qty(z-b_j)^{k_j}h_j(z)\), where \(h_j\) is nonzero on \(D\qty(b_j,\varepsilon_j)\). Thus, \(\log\abs{f\qty(R\ee^{\ii\theta})}=k_j\log\abs{z-b_j}+\log\abs{h_j(z)}\).

    Let \(\rho\) satisfy \(\max_{1\leq j\leq m}\qty(R-\varepsilon_j)<\rho<R\) and for simplicity, define \(\tau_j=2\arcsin(\frac{\varepsilon_j}{2R})\) to be half the angle subtended by the arc \(K_j=\partial D(0,R)\cap\overline{D\qty(b_j,\varepsilon_j)}\). By the previous result, we have \[\log\abs{f(0)}=\frac{1}{2\uppi}\qty(\int_{R\ee^{\ii\theta}\notin\bigcup \interior{K_j}}+\sum_{j=1}^m\int_{\vartheta_j-\tau_j}^{\vartheta_j+\tau_j})\log\abs{f\qty(\rho\ee^{\ii\theta})}\dd{\theta}+\sum_{k=1}^n\abs{\frac{a_k}{\rho}}.\] TO BE CONTINUED
\end{proof}
Since the summation on the right is negative, we have the following consequence:
\begin{corollary}[Jensen's Inequality]\label{cor:jensensinequality}
    Let \(f\) be holomorphic on \(\overline{D(0,R)}\) such that \(f\not\equiv 0\) on \(D(0,R)\). It follows that \[\log\abs{f(0)}\leq\frac{1}{2\uppi}\int_0^{2\uppi}\log\abs{f\qty(R\ee^{\ii\theta})}\dd{\theta}.\]
\end{corollary}
\begin{theorem}[\textsc{Poisson--Jensen Formula}]\label{thm:poissonjensenformula}
    Suppose \(f\) is a meromorphic function on \(\overline{D(0,R)}\) such that \(f\not\equiv 0\) on \(D(0,R)\) and is non-vanishing on \(\partial D(0,R)\). Let \(a_1,\ldots,a_m\) and \(b_1,\ldots,b_n\) be the zeros and poles of \(f\) in \(D(0,R)\), counted with multiplicity and order, respectively (multiplicities and orders count as multiple zeros or poles). Then it follows that
    \begin{align}
        \log\abs{f(z)} & =\int_0^{2\uppi}\log\abs{f(\zeta)}P\qty(\zeta,z)\dd{\theta}\nonumber                                                                                                          \\
                       & \quad+\sum_{j=1}^m\log\abs{\frac{R\qty(z-a_j)}{R^2-\overline{a_j}z}}-\sum_{k=1}^n\log\abs{\frac{R\qty(z-b_k)}{R^2-\overline{b_k}z}},\label{eq:poissonjensenformula_statement}
    \end{align}
    where \(\zeta=R\ee^{\ii\theta}\), \(z\in D(0,R)\setminus\qty(\cbraces{a_j}_{j=1}^m\cup\cbraces{b_k}_{j=1}^n)\), and \(P(\zeta,z)\) is the Poisson kernel in \cref{eq:poissonkernelgeneralform}.
\end{theorem}
\begin{proof}
    Let \(\varphi_{\frac{a_j}{R}}\qty(\frac{z}{R})=\frac{R\qty(z-a_j)}{R^2-\overline{a_j}z}\) and let \(\varphi_{\frac{b_k}{R}}\qty(\frac{z}{R})=\frac{R\qty(z-b_k)}{R^2-\overline{b_k}z}\) for all \(1\leq j\leq m\) and \(1\leq k\leq n\). Consider the function \[g(z)=f(z)\frac{\prod_{k=1}^n\varphi_{\frac{b_k}{R}}\qty(\frac{z}{R})}{\prod_{j=1}^m\varphi_{\frac{a_j}{R}}\qty(\frac{z}{R})},\]
    which is holomorphic and nonzero on \(\overline{D(0,R)}\). Fix \(z\in D(0,R)\setminus\qty(\cbraces{a_j}_{j=1}^m\cup\cbraces{b_k}_{j=1}^n)\) and let \(\Phi(\xi)=g\circ\qty(R\varphi_{-\frac{z}{R}})\qty(\xi)\), where \(R\varphi_{-\frac{z}{R}}\qty(\xi)=\frac{R^2\xi+Rz}{R+\overline{z}\xi}\) is holomorphic on \(\overline{\mathbb{D}}\) with \(\Phi(0)=z\). Since \(g\) is non-vanishing in \(\overline{D(0,R)}\), \(\Phi\) is non-vanishing in \(\overline{\mathbb{D}}\). By \cref{lem:nonvanishingholomorphiclogarithmabsolutemeanvalueproperty} on \(\Phi\),
    \begin{equation}
        \log\abs{f(z)}+\sum_{k=1}^n\log\abs{\varphi_{\frac{b_k}{R}}\qty(\frac{z}{R})}-\sum_{j=1}^m\log\abs{\varphi_{\frac{a_j}{R}}\qty(\frac{z}{R})}=\frac{1}{2\uppi}\int_0^{2\uppi}\log\abs{\Phi\qty(\ee^{\ii\theta})}\dd{\theta}.\label{eq:poissonjensenformula_intermediate1}
    \end{equation}
    Observe that
    \begin{equation}
        \log\abs{\Phi\qty(\ee^{\ii\theta})}=\log\abs{f\qty(R\varphi_{-\frac{z}{R}}\qty(\ee^{\ii\theta}))\frac{\prod_{k=1}^n\varphi_{\frac{b_k}{R}}\qty(\varphi_{-\frac{z}{R}}\qty(\ee^{\ii\theta}))}{\prod_{j=1}^m\varphi_{\frac{a_j}{R}}\qty(\varphi_{-\frac{z}{R}}\qty(\ee^{\ii\theta}))}}=\log\abs{f\qty(R\varphi_{-\frac{z}{R}}\qty(\ee^{\ii\theta}))}.\label{eq:poissonjensenformula_intermediate2}
    \end{equation}
    By letting \(\ee^{\ii\psi}=\varphi_{-\frac{z}{R}}\qty(\ee^{\ii\theta})\), it follows that \(\ee^{\ii\theta}=\varphi_{\frac{z}{R}}\qty(\ee^{\ii\psi})\), and we obtain that
    \begin{equation}
        \dd{\theta}=\frac{R^2-\abs{z}^2}{\abs{R\ee^{\ii\psi}-z}^2}\dd{\psi}=2\uppi P(\zeta,z)\dd{\psi},\qquad\zeta=R\ee^{\ii\psi}\label{eq:poissonjensenformula_intermediate3}
    \end{equation} after much simplification (the full computation of which can be found in \cref{eq:poissonintegralformula2_differentialcomputation} under different variables).

    Hence, by substituting \cref{eq:poissonjensenformula_intermediate2,eq:poissonjensenformula_intermediate3} into \cref{eq:poissonjensenformula_intermediate1}, we have
    \begin{align*}
        \log\abs{f(z)}+\sum_{k=1}^n\log\abs{\varphi_{\frac{b_k}{R}}\qty(\frac{z}{R})}-\sum_{j=1}^m\log\abs{\varphi_{\frac{a_j}{R}}\qty(\frac{z}{R})}=\int_0^{2\uppi}\log\abs{f\qty(R\ee^{\ii\psi})}P(\zeta,z)\dd{\psi},
    \end{align*}
    and the conclusion follows from simple rearrangement.
\end{proof}
\begin{remark}
    Jensen's Formula (\cref{thm:jensensformula}) can also be generalized for meromorphic functions; by letting \(z=0\) in \cref{eq:poissonjensenformula_statement}, we have: \[\log\abs{f(0)}=\frac{1}{2\uppi}\int_0^{2\uppi}\log\abs{f\qty(R\ee^{\ii\theta})}\dd{\theta}+\sum_{j=1}^m\log\abs{\frac{a_j}{R}}-\sum_{k=1}^n\log\abs{\frac{b_k}{R}}.\]
\end{remark}
\begin{lemma}\label{lem:boundedholomorphicfunctionblaschkecondition}
    Let \(f:\mathbb{D}\to\mathbb{C}\) be a non-constant bounded holomorphic function whose zeros are \(a_1,a_2,\ldots\), counted according to their multiplicities, ordered such that \(\abs{a_n}\leq \abs{a_{n+1}}\) for all \(n\in\mathbb{N}\). Then, \[\sum_{n=1}^\infty\qty(1-\abs{a_n})\] is convergent.
\end{lemma}
\begin{proof}
    First assume \(f(0)\neq 0\) and choose \(M\) such that \(f\leq M\) on \(\mathbb{D}\). The magnitudes \(\cbraces{\abs{a_n}}\) are countable in \(\mathbb{D}\), and therefore we can choose \(R\) to be arbitrarily near \(1\) without coinciding with \(\cbraces{\abs{a_n}}\). Let \(n(R)\) count the number of zeros, according to multiplicities, inside \(D(0,R)\). By Jensen's Formula (\cref{thm:jensensformula}), we have \[\log\abs{f(0)}=\frac{1}{2\uppi}\int_{0}^{2\uppi}\log\abs{f\qty(R\ee^{\ii\theta})}\dd{\theta}+\sum_{k=1}^{n(R)}\log\abs{\frac{a_k}{R}}\leq \log(M)+\sum_{k=1}^{n(R)}\log\abs{\frac{a_k}{R}}.\]
    Let \(R\to 1^{-}\). It follows that \[\log\abs{f(0)}\leq\log\abs{M}+\sum_{k=1}^\infty\log\abs{a_k}.\]
    For any \(0<a<1\), we have \(-\log(a)=1-a+\sum_{n=2}^\infty\qty(1-a)^n a^n>1-a\). Hence, \[0\leq\sum_{k=1}^\infty\qty(1-\abs{a_k})<-\sum_{k=1}^\infty\log\abs{a_k}\leq\log\abs{M}-\log\abs{f(0)}.\]
    If \(f\) has a zero of multiplicity \(m\) at \(0\), then the argument applies to \(z\mapsto \frac{f(z)}{z^m}\).
\end{proof}
\begin{theorem}[\textsc{Blaschke Product}]\label{thm:blaschkeproduct}
    Let \(\cbraces{a_k}_{k\in\mathbb{N}}\subset \mathbb{D}^*=\mathbb{D}\setminus\cbraces{0}\) be a sequence such that the series \(\sum_{k=1}^\infty \qty(1-\abs{a_k})\) is convergent (known as the \textit{Blaschke condition}). Then the \textit{Blaschke product}, defined by
    \begin{equation}
        B(z)=\prod_{k=1}^\infty \qty[-\frac{\abs{a_k}}{a_k}\varphi_{a_k}(z)],\label{eq:blaschkeproduct_statement}
    \end{equation} where \(\varphi_a(z
    )\) is a Möbius transformation in the form of \cref{eq:mobiustransformationgroupofholomorphicautomorphismsunitdisk_statement}, is uniformly convergent on compact subsets of \(\mathbb{D}\) to an analytic function on \(\mathbb{D}\) such that \(\abs{B}\leq 1\) on \(\mathbb{D}\), and its only zeros are precisely at each of \(\cbraces{a_k}\).
\end{theorem}
\begin{proof}
    If it can be shown that \[\sum_{k=1}^\infty\abs{\frac{\abs{a_k}}{a_k}\frac{a_k-z}{1-\overline{a_k}z}-1}\] locally uniformly converges, we can use \cref{lem:infiniteproductlocallyuniformconvergencecriterion2} to show that the infinite product converges uniformly on compact subsets of \(\mathbb{D}\). Let \(\overline{D(0,R)}\subset\mathbb{D}\) be a compact subset. The summand can be bounded with
    \begin{align*}
        \abs{\frac{\abs{a_k}}{a_k}\frac{a_k-z}{1-\overline{a_k}z}-1} & =\abs{\frac{\overline{a_k}}{\abs{a_k}}\frac{a_k-z}{1-\overline{a_k}z}-1}=\abs{\frac{\abs{a_k}^2-\overline{a_k}z}{\abs{a_k}\qty(1-\overline{a_k}z)}-1} \\
                                                                     & =\abs{\frac{\abs{a_k}^2-\overline{a_k}z-\abs{a_k}+\abs{a_k}\overline{a_k}z}{\abs{a_k}\qty(1-\overline{a_k}z)}}                                        \\
                                                                     & =\abs{\frac{\overline{a_k}z\qty(\abs{a_k}-1)+\abs{a_k}\qty(\abs{a_k}-1)}{\abs{a_k}\qty(1-\overline{a_k}z)}}                                           \\
                                                                     & =\abs{\frac{\qty(\overline{a_k}z+\abs{a_k})\qty(1-\abs{a_k})}{\abs{a_k}\qty(1-\overline{a_k}z)}}                                                      \\
                                                                     & \leq\qty(1-\abs{a_k})\frac{\abs{\overline{a_k}}(1+R)}{\abs{a_k}\qty(1-\abs{a_k}R)}<\qty(1-\abs{a_k})\frac{1+R}{1-R}.
    \end{align*}
    Since \[\sum_{k=1}^\infty\abs{\frac{\abs{a_k}}{a_k}\frac{a_k-z}{1-\overline{a_k}z}-1}<\frac{1+R}{1-R}\sum_{k=1}^\infty\qty(1-\abs{a_k})\] is convergent (Blaschke condition), by the Weierstrass \(M\)--Test (\cref{thm:weierstrassmtest}), \(\sum_{k=1}^\infty\abs{\frac{\abs{a_k}}{a_k}\frac{a_k-z}{1-\overline{a_k}z}-1}\) converges uniformly on \(\overline{D(0,R)}\). By \cref{lem:infiniteproductlocallyuniformconvergencecriterion2}, the infinite product in \cref{eq:blaschkeproduct_statement} converges uniformly on compact subsets of \(\mathbb{D}\). The properties of its zeros follow from the lemma.

    Lastly, since \(\abs{\varphi_{a_k}}\leq 1\) and each partial product is bounded by 1, it follows that \(\abs{B(z)}\leq 1\) on \(\mathbb{D}\).
\end{proof}
\begin{remark}
    A more general Blaschke product has an additional factor of \(z^m\) to account for a zero at the origin, similar to the case of the Weierstrass product.
\end{remark}
From the results above, a recurring theme in complex analysis is hinted at; the rate of growth of functions provides insight towards the distribution of its zeros.

The subjects to be discussed here are relevant and preliminary to Nevanlinna theory, or the study of holomorphic value distribution.
\begin{definition}[Growth Order of Entire Functions]
    An entire function \(f\) is said to be of \textit{finite order} if there exists \(\alpha, R_\alpha\in\mathbb{R}\) such that \[\abs{f(z)}\leq\exp\qty(\abs{z}^\alpha),\qquad\forall\abs{z}>R_\alpha,\] or in loose terms, \(f\) is of finite order if it grows at most exponentially for large \(z\). The \textit{order} of \(f\), or \(\lambda\qty(f)\) is defined to be the infimum of all \(\alpha\) satisfying the previous condition.
\end{definition}
\begin{example}
    The function \(\sin\) is of order 1, while \(\exp\circ\exp\) is not of finite order.
\end{example}
The goal of the classification is to find the implications of \(\lambda\) on the rate of which zeros tend to \(\infty\).
\begin{example}\label{ex:sinproductformula}
    Prove that \[\sin(\uppi z)=\uppi z\prod_{n=1}^\infty\qty(1-\frac{z^2}{n^2})\]
    is convergent and defines an entire function that uniformly converges on any compact disk \(\overline{D(0,R)}\).
\end{example}
\begin{proof}
    The zeros of \(\sin(\uppi z)\) are simple at each of \(\mathbb{Z}\). Aside from the simple zero at \(z=0\), let \[a_n=
        \begin{dcases}
            -\frac{n}{2}  & \qif* n\in 2\mathbb{N},                     \\
            \frac{n+1}{2} & \qif* n\in \mathbb{N}\setminus 2\mathbb{N}.
        \end{dcases}\]
    For each \(R>0\), the series \(\sum_{n=1}^\infty\abs{\frac{R}{a_n}}^2=\frac{R^2\uppi^2}{3}\) converges. Therefore, \[\sin(\uppi z)=\ee^{\varphi(z)}z\prod_{n=1}^\infty\qty(1-\frac{z}{a_n})\exp\qty[\frac{z}{a_n}]\] for some entire function \(\varphi(z)\).
    \begin{equation*}
        \sin(\uppi z)=\ee^{\varphi(z)}z\prod_{n=1}^\infty\qty[1-\frac{z^2}{n^2}].
    \end{equation*}
    Notice that
    \begin{equation}
        \ee^{\varphi(z)}=\frac{\sin(\uppi z )}{z\prod_{n=1}^\infty\qty(1-\frac{z^2}{n^2})}.\label{eq:sinproductformula_intermediate}
    \end{equation} As \(z\to 0\), we have \[\lim_{z\to 0}\ee^{\varphi(0)}=\lim_{z\to 0}\frac{\uppi\cos(\uppi z)}{\prod_{n=1}^\infty\qty(1-\frac{z^2}{n^2})+z\qty[\prod_{n=1}^\infty\qty(1-\frac{z^2}{n^2})]'}=\frac{\uppi}{1+0}=\uppi,\] since \(\prod_{n=1}^\infty\qty(1-\frac{z^2}{n^2})\) is even and therefore has a vanishing derivative at \(z=0\). TO BE CONTINUED
\end{proof}
