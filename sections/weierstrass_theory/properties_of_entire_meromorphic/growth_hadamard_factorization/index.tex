\subsubsection{Classifying Growth of Entire Functions}\label{sec:classifyinggrowthofentirefunctions}
\begin{lemma}\label{lem:nonvanishingholomorphiclogarithmabsolutemeanvalueproperty}
    Let \(f:\overline{D(0,r)}\to\mathbb{C}^*\) (where \(r>0\)) be a nowhere-vanishing holomorphic function. It follows that \[\log\qty|f(0)|=\frac{1}{2\uppi}\int_0^{2\uppi}\log\abs{f\qty(r\ee^{\ii\theta})}\dd{\theta}.\]
\end{lemma}
\begin{proof}
    Without loss of generality, assume \(r=1\). Since \(f\) is non-vanishing and \(\overline{\mathbb{D}}\) is simply connected, we may define the \textit{holomorphic logarithm} as \[\log(f(z))=\int_\gamma \frac{f'(z)}{f(z)}\ddz+\log\qty(f\qty(z_0))\] for any fixed \(z_0\in \overline{\mathbb{D}}\) and all \(z\in\overline{\mathbb{D}}\), where \(\gamma\subset\overline{\mathbb{D}}\) is any piecewise smooth curve from \(z_0\) to \(z\).

    Hence, \(\log\abs{f(z)}=\Re\qty[\log\qty(f(z))]\) and is therefore harmonic. The assertion then follows from the mean-value property.
\end{proof}
\begin{theorem}[Jensen's Formula]\label{thm:jensensformula}
    Let \(f:\overline{D(0,r)}\to\mathbb{C}\) be meromorphic such that \(f(0)\neq 0\). If \(a_1,\ldots,a_m\), \(b_1,\dots,b_n\) are the zeros and poles of \(f\) in \(\overline{D(0,r)}\), counted with multiplicities and orders, respectively, then \[\log\abs{f(0)}=\frac{1}{2\uppi}\int_{0}^{2\uppi}\log\abs{f\qty(r\ee^{\ii\theta})}\dd{\theta}+\sum_{k=1}^m\log\abs{\frac{a_k}{r}}-\sum_{k=1}^n\log\abs{\frac{b_k}{r}}.\]
\end{theorem}
\begin{proof}
    For simplicity, assume \(a_1,\ldots,a_{m_0}\) are the zeros in \(D(0,r)\) and \(a_{m_0+1},\ldots,a_m\) are the zeros on \(\partial D(0,r)\). Similarly, let \(b_1,\ldots,b_{n_0}\) be the poles in \(D(0,r)\) and \(b_{n_0+1},\ldots,b_n\) be the poles on \(\partial D(0,r)\). Let \[f(z)=g(z)\frac{\prod_{j=1}^m\qty(z-a_j)}{\prod_{k=1}^n\qty(z-b_k)},\]
    where \(g\) is holomorphic and non-vanishing on \(\overline{D(0,r)}\). Since \[\log\abs{f(0)}=\log\abs{g(0)}+\sum_{j=1}^{m}\log\abs{a_j}-\sum_{k=1}^{n}\log\abs{b_k},\] by \cref{lem:nonvanishingholomorphiclogarithmabsolutemeanvalueproperty} on \(g\),
    \begin{align*}
        \log\abs{f(0)} & =\frac1{2\uppi}\int_0^{2\uppi}\log\abs{f\qty(r\ee^{\ii\theta})\frac{\prod_{k=1}^{n}\qty(r\ee^{\ii\theta}-b_k)}{\prod_{j=1}^{m}\qty(r\ee^{\ii\theta}-a_j)}}\dd{\theta}+\sum_{j=1}^{m}\log\abs{a_j}-\sum_{k=1}^{n}\log\abs{b_k} \\
        & \!\!\!\!=\frac1{2\uppi}\int_0^{2\uppi}\log\abs{h\qty(r\ee^{\ii\theta})}\dd{\theta}+\sum_{j=1}^{m}\log\abs{a_j}-\sum_{k=1}^{n}\log\abs{b_k}                                                                                    \\
        & \!\!\!\!\quad+\frac1{2\uppi}\int_0^{2\uppi}\qty[\sum_{k=1}^{n}\qty(\log{r}+\log\abs{1-\frac{b_k}{r\ee^{\ii\theta}}})-\sum_{j=1}^{m}\qty(\log r+\log\abs{1-\frac{a_j}{r\ee^{\ii\theta}}})]\dd{\theta}                          \\
        & \!\!\!\!=\frac1{2\uppi}\int_0^{2\uppi}\log\abs{h\qty(r\ee^{\ii\theta})}\dd{\theta}+\sum_{j=1}^{m}\log\abs{\frac{a_j}r}-\sum_{k=1}^{n}\log\abs{\frac{b_k}r}                                                                    \\
        & \!\!\!\!\quad+\Re\frac1{2\uppi\ii}\qty[\qty(\sum_{j=1}^{m}\varointclockwise_{\partial D\qty(0,\abs{\frac{a_j}{r}})}-\sum_{k=1}^{n}\varointclockwise_{\partial D\qty(0,\abs{\frac{b_k}{r}})})\frac{\Log\qty(1-z)\ddz}{z}]      \\
        & \!\!\!\!=\frac1{2\uppi}\int_0^{2\uppi}\log\abs{h\qty(r\ee^{\ii\theta})}\dd{\theta}+\sum_{j=1}^{m}\log\abs{\frac{a_j}r}-\sum_{k=1}^{n}\log\abs{\frac{b_k}r}                                                                    \\
        & \!\!\!\!\quad+\Re\frac1{2\uppi\ii}\qty(\sum_{k=n_0+1}^{n}-\sum_{j=m_0+1}^{m})\ointctrclockwise_{\partial\mathbb{D}}\frac{\Log\qty(1-z)\ddz}{z}
    \end{align*}
    where \(z=\flatfrac{b_k}{r\ee^{\ii\theta}}\), \(\dd{\theta}=\flatfrac{\ii\ddz}{z}\) and the leftover integrals for interior points vanish by Cauchy--Goursat (\cref{thm:cauchygoursattheorem}), since \(\flatfrac{\Log(1-z)}z\) has a removable singularity at \(z=0\).

    We now are left to prove that the remaining integral \(I\) vanishes as well, which is not as immediate since the integrand does not extend continuously to the boundary. Let \(z=\ee^{\ii\theta}\), \(\ddz=\ii\ee^{\ii\theta}\), then (by \(\psi=\frac{\theta}{2}\))
    \begin{align*}
        I & =\mathop{\mathmakebox[\widthof{\(\ointctrclockwise\)}][l]{\ointctrclockwise_{\partial\mathbb{D}}}}\frac{\Log\qty(1-z)\ddz}{z}=\int_0^{2\uppi}\log\abs{1-\ee^{\ii\theta}}\dd{\theta}=2\int_0^{\uppi}\log\abs{\ee^{-\ii\psi}-\ee^{\ii\psi}}\dd{\psi} \\
        & =2\uppi\log 2+2\mathop{\mathmakebox[\widthof{\(\int\)}][l]{\int_0^{\uppi}}}\log\abs{\sin\psi}\dd{\psi}=2\uppi\log 2+4\mathop{\mathmakebox[\widthof{\(\int\)}][l]{\int_0^{\frac{\uppi}2}}}\log\abs{\sin\psi}\dd{\psi}=2\uppi\log 2+4J.
    \end{align*}
    Splitting at \(\frac{\uppi}4\) and using \(\cos\) with a substitution for the second integral then yields \[J=\int_0^{\frac{\uppi}4}\log\abs{\sin\psi}\dd{\psi}+\int_{0}^{\frac{\uppi}4}\log\abs{\cos\psi}\dd{\psi}=\int_0^{\frac{\uppi}4}\log\abs{\frac12\sin{2\psi}}\dd{\psi}.\]
    Changing back to \(\theta=2\psi\), we have \[J=\frac J2-\frac{\uppi}4\log2\implies 4J=-2\uppi\log 2\implies I=0.\qedhere\]
\end{proof}
As an immediate consequence, we have:
\begin{corollary}[Jensen's Inequality]\label{cor:jensensinequality}
    Let \(f\) be holomorphic on \(\overline{D(0,r)}\) such that \(f\not\equiv 0\) and \(f(0)\neq 0\). It follows that \[\log\abs{f(0)}\leq\frac{1}{2\uppi}\int_0^{2\uppi}\log\abs{f\qty(r\ee^{\ii\theta})}\dd{\theta}.\]
\end{corollary}
\begin{theorem}[\textsc{Poisson--Jensen Formula}]\label{thm:poissonjensenformula}
    Suppose \(f\) is a meromorphic function on \(\overline{D(0,r)}\) such that \(f\not\equiv 0\) on \(D(0,r)\) and is non-vanishing and non-infinity on \(\partial D(0,r)\). Let \(a_1,\ldots,a_m\) and \(b_1,\ldots,b_n\) be the zeros and poles of \(f\) in \(D(0,r)\), counted with multiplicity and order, respectively (multiplicities and orders count as multiple zeros or poles). Then it follows that
    \begin{align}
        \log\abs{f(z)} & =\int_0^{2\uppi}\log\abs{f(\zeta)}P\qty(\zeta,z)\dd{\theta}\nonumber                                                                                                          \\
        & \quad+\sum_{j=1}^m\log\abs{\frac{r\qty(z-a_j)}{r^2-\overline{a_j}z}}-\sum_{k=1}^n\log\abs{\frac{r\qty(z-b_k)}{r^2-\overline{b_k}z}},\label{eq:poissonjensenformula_statement}
    \end{align}
    where \(\zeta=r\ee^{\ii\theta}\), \(z\in D(0,r)\setminus\qty(\cbraces{a_j}_{j=1}^m\cup\cbraces{b_k}_{j=1}^n)\), and \(P(\zeta,z)\) is the Poisson kernel in \cref{eq:poissonkernelgeneralform}.
\end{theorem}
\begin{proof}
    For fixed \(z\in D(0,r)\) not at zeros or poles, let \[g_z(\zeta)=f\qty(r\varphi_{-\frac zr}\qty(\flatfrac\zeta r))\] where \(\varphi_{-a}=\qty(\varphi_a)^{-1}\) is the unit disk automorphism sending 0 to \(a\). Then \(g_z\) maps 0 to \(f\qty(z)\), and has zeros at \(r\varphi_{-\frac zr}\qty(\flatfrac\zeta r)=a_k\) or \(\zeta=r\varphi_{\frac zr}\qty(\flatfrac{a_k}r)\) and poles at \(r\varphi_{-\frac zr}\qty(\flatfrac\zeta r)=b_k\) or \(\zeta=r\varphi_{\frac zr}\qty(\flatfrac{b_k}r)\). By Jensen's formula (\cref{thm:jensensformula}), \[\log\abs{g(0)}=\log\abs{f(z)}=\frac1{2\uppi}\int_0^{2\uppi}\log\abs{g\qty(r\ee^{\ii\theta})}\dd{\theta}+\sum_{k=1}^m\abs{\varphi_{\frac zr}\qty(\frac{a_k}r)}-\sum_{k=1}^n\abs{\varphi_{\frac zr}\qty(\frac{b_k}r)}.\qedhere\]
\end{proof}
\begin{lemma}\label{lem:boundedholomorphicfunctionblaschkecondition}
    Let \(f:\mathbb{D}\to\mathbb{C}\) be a non-constant bounded holomorphic function whose zeros are \(a_1,a_2,\ldots\), counted according to their multiplicities, ordered such that \(\abs{a_n}\leq \abs{a_{n+1}}\) for all \(n\in\mathbb{N}\). Then, \[\sum_{n=1}^\infty\qty(1-\abs{a_n})\] is convergent.
\end{lemma}
\begin{proof}
    First assume \(f(0)\neq 0\) and choose \(M\) such that \(\abs{f}\leq M\) on \(\mathbb{D}\). Let \(n\qty(r,0,f)\) count the number of zeros of \(f\), according to multiplicities, inside \(\overline{D(0,r)}\). By Jensen's Formula (\cref{thm:jensensformula}), we have \[\log\abs{f(0)}=\frac{1}{2\uppi}\int_{0}^{2\uppi}\log\abs{f\qty(r\ee^{\ii\theta})}\dd{\theta}+\mathmakebox[\widthof{\(\sum\)}][c]{\sum_{k=1}^{n\qty(r,0,f)}}\log\abs{\frac{a_k}{r}}\leq\log(M)+\mathmakebox[\widthof{\(\sum\)}][c]{\sum_{k=1}^{n\qty(r,0,f)}}\log\abs{\frac{a_k}{r}}.\]
    For any fixed positive integer \(k\), choose \(r\) such that \(\abs{a_k}<r<1\). Then \(n\qty(r,0,f)\geq k\) and \[\mathmakebox[\widthof{\(\sum\)}][c]{\sum_{k=1}^{n\qty(r,0,f)}}\log\abs{\frac{a_j}{r}}\leq\sum_{j=1}^{k}\log\abs{\frac{a_j}{r}},\]
    since each \(\log\abs{\frac{a_j}{r}}<0\) for \(j=k+1,\dots,n\qty(r,0,f)\). Therefore, \[\log\abs{f(0)}\leq\log{M}+\sum_{j=1}^{k}\log\abs{\frac{a_j}{r}}=\log{M}+\sum_{j=1}^{k}\log\abs{a_j}-k\log{r}.\]
    Rearranging, \[\sum_{j=1}^{k}\log\abs{a_j}\geq\log\abs{f(0)}-\log{M}+k\log{r}.\] Now let \(r\to1^-\) with \(r>\abs{a_k}\). Since \(k\log r\to0\), it follows that \[\sum_{j=1}^{k}\log\abs{a_j}\geq\log\abs{f(0)}-\log{M}.\] This holds for every \(k\). Since \(\log\abs{a_j}<0\) for all \(j\), the partial sums \(\sum_{j=1}^{k}\log\abs{a_j}\) are decreasing and bounded below by \(\log\abs{f(0)}-\log{M}\), hence converge to some finite limit, and \[\sum_{j=1}^{\infty}\log\abs{a_j}\geq\log\abs{f(0)}-\log{M},\] or equivalently, \[\log\abs{f(0)}\leq\log\abs{M}+\sum_{k=1}^\infty\log\abs{a_k}.\]
    For any \(0<a<1\), we have \(-\log(a)=1-a+\sum_{n=2}^\infty\qty(1-a)^n a^n>1-a\). Hence, \[0\leq\sum_{k=1}^\infty\qty(1-\abs{a_k})<-\sum_{k=1}^\infty\log\abs{a_k}\leq\log\abs{M}-\log\abs{f(0)}.\]
    If \(f\) has a zero of multiplicity \(m\) at \(0\), then the argument applies to \(z\mapsto \frac{f(z)}{z^m}\).
\end{proof}
\begin{theorem}[\textsc{Blaschke Product}]\label{thm:blaschkeproduct}
    Let \(\cbraces{a_k}_{k\in\mathbb{N}}\subset \mathbb{D}^*=\mathbb{D}\setminus\cbraces{0}\) be a sequence such that the series \(\sum_{k=1}^\infty \qty(1-\abs{a_k})\) is convergent (known as the \textit{Blaschke condition}). Then the \textit{Blaschke product}, defined by
    \begin{equation}
        B(z)=\prod_{k=1}^\infty \qty[-\frac{\abs{a_k}}{a_k}\varphi_{a_k}(z)],\label{eq:blaschkeproduct_statement}
    \end{equation} (where \(\varphi_a(z
    )\) is a Möbius transformation in the form of \cref{eq:mobiustransformationgroupofholomorphicautomorphismsunitdisk_statement}), locally uniformly converges to an analytic function on \(\mathbb{D}\) such that \(\abs{B}\leq 1\) on \(\mathbb{D}\), and its only zeros are precisely at each of \(\cbraces{a_k}_{k\in\mathbb{N}}\), counted according to multiplicities.
\end{theorem}
\begin{proof}
    If it can be shown that \[\sum_{k=1}^\infty\abs{\frac{\abs{a_k}}{a_k}\frac{a_k-z}{1-\overline{a_k}z}-1}\] locally uniformly converges, we can use \cref{lem:infiniteproductlocallyuniformconvergencecriterion2} to show that the infinite product converges uniformly on compact subsets of \(\mathbb{D}\). Let \(\overline{D(0,r)}\subset\mathbb{D}\) be a compact subset. The summand can be bounded with
    \begin{align*}
        \abs{\frac{\abs{a_k}}{a_k}\frac{a_k-z}{1-\overline{a_k}z}-1} & =\abs{\frac{\overline{a_k}}{\abs{a_k}}\frac{a_k-z}{1-\overline{a_k}z}-1}=\abs{\frac{\abs{a_k}^2-\overline{a_k}z}{\abs{a_k}\qty(1-\overline{a_k}z)}-1} \\
        & =\abs{\frac{\abs{a_k}^2-\overline{a_k}z-\abs{a_k}+\abs{a_k}\overline{a_k}z}{\abs{a_k}\qty(1-\overline{a_k}z)}}                                        \\
        & =\abs{\frac{\overline{a_k}z\qty(\abs{a_k}-1)+\abs{a_k}\qty(\abs{a_k}-1)}{\abs{a_k}\qty(1-\overline{a_k}z)}}                                           \\
        & =\abs{\frac{\qty(\overline{a_k}z+\abs{a_k})\qty(1-\abs{a_k})}{\abs{a_k}\qty(1-\overline{a_k}z)}}                                                      \\
        & \leq\qty(1-\abs{a_k})\frac{\abs{\overline{a_k}}(1+r)}{\abs{a_k}\qty(1-\abs{a_k}r)}<\qty(1-\abs{a_k})\frac{1+r}{1-r}.
    \end{align*}
    Since \[\sum_{k=1}^\infty\abs{\frac{\abs{a_k}}{a_k}\frac{a_k-z}{1-\overline{a_k}z}-1}<\frac{1+r}{1-r}\sum_{k=1}^\infty\qty(1-\abs{a_k})\] is convergent (Blaschke condition), by the Weierstrass \(M\)--Test (\cref{thm:weierstrassmtest}), \(\sum_{k=1}^\infty\abs{\frac{\abs{a_k}}{a_k}\frac{a_k-z}{1-\overline{a_k}z}-1}\) converges uniformly on \(\overline{D(0,r)}\). By \cref{lem:infiniteproductlocallyuniformconvergencecriterion2}, the infinite product in \cref{eq:blaschkeproduct_statement} converges uniformly on compact subsets of \(\mathbb{D}\). The properties of its zeros follow from the lemma.

    Lastly, since \(\abs{\varphi_{a_k}}\leq 1\) and each partial product is bounded by 1, it follows that \(\abs{B(z)}\leq 1\) on \(\mathbb{D}\).
\end{proof}
\begin{remark}
    A more general Blaschke product has an additional factor of \(z^m\) to account for a zero at the origin, similar to the case of the Weierstrass product.
\end{remark}
\begin{corollary}\label{cor:blaschkeproductfactorization}
    Let \(f:\mathbb{D}\to\mathbb{C}\) be bounded and holomorphic whose multiplicity of the zero at 0 is \(m\) (if \(f\) does not vanish at 0, then \(m=0\)). If \(\cbraces{a_n}_{n\in\mathbb{N}}\) are its zeros in \(\mathbb{D}^*\), counting multiplicities, then \[f(z)=F(z)z^m\prod_{n=1}^{\infty}\qty[-\frac{\abs{a_n}}{a_n}\varphi_{a_n}(z)],\] where \(F\) is bounded, holomorphic, and non-vanishing on \(\mathbb{D}\). Moreover, \[\sup_{z\in\mathbb{D}}\abs{f(z)}=\sup_{z\in\mathbb{D}}\abs{F(z)}.\]
\end{corollary}
\begin{proof}
    Let \[F(z)=\frac{f(z)}{z^m\prod_{n=1}^\infty\qty[-\frac{\abs{a_k}}{a_k}\varphi_{a_n}(z)]}.\] By construction, \(F\) extends to its removable singularities to a holomorphic function that does not vanish. Because \[\sup_{z\in\mathbb{D}}\abs{z^m\prod_{n=1}^\infty\qty[-\frac{\abs{a_n}}{a_n}\varphi_{a_n}(z)]}\leq 1,\] it follows that
    \begin{equation}
        \sup_{z\in\mathbb{D}}\abs{F(z)}\geq\sup_{z\in\mathbb{D}}\abs{f(z)}.\label{eq:blaschkeproductfactorization_supremuminequalities}
    \end{equation}
    The partial products \[B_n(z)=\prod_{k=1}^n\qty[-\frac{\abs{a_k}}{a_k}\varphi_{a_k}(z)]\] give for fixed \(\theta\in\mathbb{R},\varepsilon>0\), the existence of \(0<r'<1\) such that \(r'<r<1\) implies \[\abs{B_n\qty(r\ee^{\ii\theta})}>1-\varepsilon.\]
    Then \[\sup_{z\in\mathbb{D}}\abs{\frac{f\qty(z)}{z^m B_n(z)}}=\sup_{z\in\mathbb{D}\setminus\overline{D(0,r)}}\abs{\frac{f\qty(z)}{z^mB_n(z)}}\leq\frac1{r^m\qty(1-\varepsilon)}\sup_{z\in\mathbb{D}}\abs{f\qty(z)}\to\frac1{1-\varepsilon}\sup_{z\in\mathbb{D}}\abs{f(z)}\] as \(r\to 1^-\) by the Maximum Modulus Principle (\cref{thm:maximummodulus}). Letting \(\varepsilon\to 0^+\), \(n\to\infty\) gives \[\sup_{z\in\mathbb{D}}\abs{F(z)}\leq\sup_{z\in\mathbb{D}}\abs{f(z)},\] which in conjunction with \cref{eq:blaschkeproductfactorization_supremuminequalities}, completes the final assertion.
\end{proof}
From the results above, a recurring theme in complex analysis is hinted at; the rate of growth of functions provides insight towards the distribution of its zeros.

The subjects to be discussed here are relevant and preliminary to Nevanlinna theory, or the study of holomorphic value distribution.

For an entire function \(f\), let \(M(r,f)=\sup_{\abs{z}=r}\abs{f(z)}=\sup_{\abs{z}\leq r}\abs{f(z)}\) (by the Maximum Modulus Principle in \cref{thm:maximummodulus}).
\begin{definition}[Growth Order of Entire Functions]
    An entire function \(f\) is said to be of \textit{finite order} if there exists \(\alpha, r_\alpha\in\mathbb{R}\) such that \[M\qty(r,f)\leq\exp\qty(r^\alpha),\qquad\forall r>r_\alpha,\] or in loose terms, \(f\) is of finite order if it grows at most exponentially for large \(z\). The \textit{order} of \(f\), or \(\rho\qty(f)\) is defined to be the infimum of all \(\alpha\) satisfying the previous condition.
\end{definition}
\begin{proposition}
    Let \(f\) be entire; then if there exist \(a,b,\alpha,r_{\alpha,\beta}>0\) such that \[M(r,f)\leq\exp\qty(ar^\alpha+b),\qquad\forall r>r_{\alpha,\beta},\] then \(\rho(f)\leq \alpha\).
\end{proposition}
\begin{proof}
    For \(\varepsilon>0\), since \(r^\varepsilon\to\infty\) as \(r\to\infty\), for any \(\varepsilon>0\), there exists \(r_\varepsilon\) such that \[r^{\varepsilon}\geq 2a\implies\frac12 r^{\alpha+\varepsilon}\geq ar^\alpha\] for \(r>r_{\varepsilon}\). There exists \(r'_\varepsilon>0\) such that \[r>r'_\varepsilon\implies\frac12r^{\alpha+\varepsilon}\geq b.\]
    For simplicity, let the value \(\max\qty{r_\varepsilon,r'_\varepsilon}\) be denoted by \(r_\varepsilon\). Then \[r>r_\varepsilon\implies ar^\alpha+b\leq \frac12r^{\alpha+\varepsilon}+\frac12r^{\alpha+\varepsilon}=r^{\alpha+\varepsilon}.\] By assumption, we have \[M(r,f)\leq\exp\qty(ar^\alpha+b)\leq\exp\qty(r^{\alpha+\varepsilon})\implies\alpha+\varepsilon\geq\rho(f).\]
    Letting \(\varepsilon\to 0^+\), the assertion follows.
\end{proof}
\begin{theorem}
    The order of an entire \(f\) may be explicitly given by \[\rho(f)=\limsup_{r\to\infty}\frac{\log\qty(\log M(r,f))}{\log{r}}.\]
\end{theorem}
\begin{proof}
    By assumption, we have \(\forall\varepsilon'>0\), \(\exists 0<\varepsilon<\varepsilon'\) (or simply just \(\forall\varepsilon>0\) by the nature of the exponential) such that
    \[M(r,f)\leq\exp\qty(r^{\rho(f)+\varepsilon})\] for some \(r'\) and any \(r>r'\). Taking logarithms twice we have \[\frac{\log\qty(\log M(r,f))}{\log{r}}\leq\limsup_{r\to\infty}\frac{\log\qty(\log M(r,f))}{\log{r}}\leq\rho(f)+\varepsilon\to\rho(f)\]
    as \(\varepsilon'\to 0\). Moreover, for any \(\varepsilon>0, r'>0\), \(\exists r>r'\) such that \[M(r,f)>\exp\qty(r^{\rho(f)-\varepsilon})\implies\varlimsup_{r\to\infty}\frac{\log\qty(\log M(r,f))}{\log r}\geq\rho(f)-\varepsilon\to\rho(f)\] as \(\varepsilon\to 0\). Therefore, \[\rho(f)\leq\limsup_{r\to\infty}\frac{\log\qty(\log M(r,f))}{\log r}\leq\rho(f).\qedhere\]
\end{proof}
\begin{example}\label{ex:entirefunctionfiniteordersinexpexp}
    The function \(\sin\) is of order 1, while \(\exp\circ\exp\) is not of finite order.
\end{example}
\begin{proof}
    We consider the two examples separately:
    \begin{enumerate}
        \item Observe that \[\sup_{\abs{z}=r}\abs{\sin(z)}\leq\sup_{\abs{z}=r}\frac{\abs{\ee^{\ii z}}+\abs{\ee^{-\ii z}}}{2}=\sup_{\abs{z}=r}\frac{\ee^{\abs{y}}+\ee^{-\abs{y}}}{2}\leq\sup_{\abs{z}=r}\ee^{\abs{y}}=\ee^r.\]
            For \(r>1\), we have \(\ee^{-r}<1<\frac12\ee^r\), and hence for \(z=\ii r\), we have \[\abs{\sin(z)}=\frac{\ee^{r}-\ee^{-r}}{2}>\frac14\ee^r.\]
            Therefore, \[\frac14\ee^r<\sup_{\abs{z}=r}\abs{\sin(z)}\leq\ee^r\implies\rho(f)=\limsup_{r\to\infty}\frac{\log\qty(r+\order{1})}{\log{r}}=1.\]
        \item Let \(z=r\), then
            \begin{align*}
                \sup_{\abs{z}=r}\abs{\exp\circ\exp}\geq\exp\circ\exp(r) & \implies\log\circ\log\sup_{\abs{z}=r}\abs{f(z)}\geq r                      \\
                & \implies\rho(f)\geq\limsup_{r\to\infty}\frac{r}{\log r}=\infty.\qedhere
            \end{align*}
    \end{enumerate}
\end{proof}
The utility of \(\rho\) is that it gives implications on the rate of which the zeros of an entire function tend to \(\infty\). The order for meromorphic functions is more general and is pertinent in Nevanlinna Theory (\cref{sec:nevanlinnatheory}). This is quantified technically by the convergence range of the sum given by \[\sum_{n=1}^\infty\frac{1}{\abs{a_n}^{k+1}},\] where each \(a_n\) is a zero. Specifically, the infimum of all such \(k\) under which the prescribed sum converges correlates to this right. For example, let \(a_n=n\) for each \(n\). Then for any \(k>0\), the integral test gives the convergence of the series, while if \(a_n=\sqrt{a_n}\) (corresponding to a slower approach to \(\infty\)), the series converges for \(k>1\).

For the following discussions, let \(n\qty(r,0,f)\) count the zeros of \(f\) in \(D(0,r)\) according to multiplicity.
\begin{lemma}\label{lem:maximummoduluszerocountingdoubleradius}
    If \(f\) is entire with \(f(0)=1\), then \[\log{2}\cdot n(r,0,f)\leq\log M(2r,f).\]
\end{lemma}
\begin{proof}
    By Jensen's formula (\cref{thm:jensensformula}), for \(r>0\), we have
    \[\mathmakebox[\widthof{\(\sum\)}][c]{\sum_{k=1}^{n\qty(2r,0,f)}}\log\abs{\frac{2r}{a_k}}=\frac1{2\uppi}\int_0^{2\uppi}\log\abs{f\qty(2r\ee^{\ii\theta})}\dd{\theta},\] where \(a_1,\dots,a_{n(2r,0,f)}\) are the zeros of \(f\) in \(D(0,2r)\), ordered such that each \(\abs{a_k}\leq\abs{a_{k+1}}\). Then \[\mathmakebox[\widthof{\(\sum\)}][c]{\sum_{k=1}^{\mathmakebox[\widthof{\(\sum\)}][l]{n(r,0,f)}}}\log2\leq\mathmakebox[\widthof{\(\sum\)}][c]{\sum_{k=1}^{n(r,0,f)}}\log\abs{\frac{2r}{a_k}}\leq\mathmakebox[\widthof{\(\sum\)}][c]{\sum_{k=1}^{n(2r,0,f)}}\log\abs{\frac{2r}{a_k}}=\frac1{2\uppi}\mathmakebox[\widthof{\(\int\)}][l]{\int_0^{2\uppi}}\log\abs{f\qty(2r\ee^{\ii\theta})}\dd{\theta}\leq\log M(2r,f).\qedhere\]
\end{proof}
\begin{theorem}\label{thm:nonzerosequencepowersummationconvergence}
    For a nonzero complex sequence \(\qty{a_k}_{k\in\mathbb{N}}\) counting multiplicities (such that \(\abs{a_1}\leq\abs{a_2}\), etc.), the sum \[\sum_{k=1}^\infty\frac1{\abs{a_k}^{\sigma}}\] converges for any \[\sigma>\varlimsup_{r\to\infty}\frac{\log n(r)}{\log r}\] where \(n(r)\) counts \(a_k\) in the closed disk of radius \(r\).
\end{theorem}
\begin{proof}
    Choose \(\sigma'\) such that \[\sigma>\sigma'>\varlimsup_{r\to\infty}\frac{\log n(r)}{\log r}.\]
    For sufficiently large \(r\), \[\frac{\log n(r)}{\log r}<\sigma'\implies n(r)\leq r^{\sigma'}.\]
    For sufficiently large \(k\in\mathbb{N}\), by the ordering of zeros, it follows that \[k\leq n\qty(\abs{a_k}+\delta)\leq\qty(\abs{a_k}+\delta)^{\sigma'}\] for sufficiently small \(\delta\). As \(\delta\to 0^+\), we have \[k\leq\abs{a_k}^{\sigma'}\implies\frac1k\geq\frac{1}{\abs{a_k}^{\sigma'}}\implies \frac1{k^{\frac{\sigma}{\sigma'}}}\geq\frac1{\abs{a_k}^{\sigma}}.\]
    By the comparison test, we then have the convergence of \[\sum_{k=1}^\infty\frac1{\abs{a_k}^{\sigma}}.\qedhere\]
\end{proof}
\begin{theorem}\label{thm:entirefunctionfiniteorderzerossummationconvergence}
    For an entire function \(f\) (\(f(0)=1\)) of finite order \(\rho(f)\) whose zeros are at \(\qty{a_k}_{k\in\mathbb{N}}\) counting multiplicities (such that \(\abs{a_1}\leq\abs{a_2}\), etc.), the sum \[\sum_{k=1}^\infty\frac1{\abs{a_k}^{\rho(f)+\eta}}\] converges for any \(\eta>0\).
\end{theorem}
\begin{proof}
    By trivial definition, we have \[M(2r,f)\leq\exp\qty(\qty(2r)^{\rho+\varepsilon})\] for all \(\varepsilon'>0\) and some \(0<\varepsilon<\varepsilon'\). \Cref{lem:maximummoduluszerocountingdoubleradius} gives that for any \(r>0\), \[\log 2\cdot n(r,0,f)\leq\log M(2r,f).\]
    Hence, \[\log 2\cdot n(r,0,f)\leq(2r)^{\rho(f)+\varepsilon}\Longleftrightarrow\frac{n(r,0,f)}{r^{\rho(f)+2\varepsilon}}\leq\frac1{\log 2}2^{\rho(f)+\varepsilon}r^{-\varepsilon}\to 0^+\]
    as \(r\to\infty\). Then for sufficiently large \(r\), we have \[n\leq\frac{n(r,0,f)}{r^{\rho(f)+2\varepsilon}}\leq1\implies n(r,0,f)\leq r^{\rho(f)+2\varepsilon}.\]
    For sufficiently large \(k\in\mathbb{N}\), by the ordering of zeros, it follows that \[k\leq n\qty(\abs{a_k}+\delta,0,f)\leq\qty(\abs{a_k}+\delta)^{\rho(f)+2\varepsilon}\] for sufficiently small \(\delta\). As \(\delta\to 0^+\), we have \[k\leq\abs{a_k}^{\rho(f)+2\varepsilon}\implies\frac1k\geq\frac{1}{\abs{a_k}^{\rho(f)+2\varepsilon}}\implies \frac1{k^{\frac{\rho(f)+\eta}{\rho(f)+2\varepsilon}}}\geq\frac1{\abs{a_k}^{\rho(f)+\eta}}.\]
    The left-hand side as a summation is convergent for \(2\varepsilon<\eta\) or lower, and hence we have the convergence of \[\sum_{k=1}^\infty\frac1{\abs{a_k}^{\rho(f)+\eta}}.\qedhere\]
\end{proof}
Therefore, for any \(r>0\), the series \[\sum_{k=1}^\infty\abs{\frac{r}{a_k}}^{\rho(f)+\eta}\leq\sum_{k=1}^\infty\abs{\frac{r}{a_k}}^{\floor{\rho}+1}\quad\text{for sufficiently small }\eta\]
converges. Then by the Weierstrass Factorization Theorem (\cref{thm:weierstrassfactorization}), \[f(z)=z^m\ee^{\varphi(z)}\prod_{k=1}^\infty E_{\floor{\rho}}\qty(\frac z{a_k})\]
locally uniformly converges on \(\mathbb{C}\), where \(\varphi\) is entire.
\begin{definition}
    The \textit{rank} of an entire function is the smallest \(p\in\mathbb{Z}_{\geq0}\) for which the associated sum \[\sum_{k=1}^\infty\frac1{\abs{a_k}^{p+1}}\] converges, where \(\qty{a_k}_k\) are its zeros in \(\mathbb{C}^*\).
\end{definition}
The conclusion of \cref{thm:entirefunctionfiniteorderzerossummationconvergence} is that the rank of an entire function with finite order is finite. Moreover, the rank \(\leq\floor{\rho}\).
\begin{definition}
    Let \(f\) be entire of finite rank \(p\). By the Weierstrass Factorization theorem (\cref{thm:weierstrassfactorization}), \[f(z)=z^m\ee^{\varphi(z)}\prod_{k=1}^\infty E_p\qty(\frac z{a_k}).\]
    If \(\varphi\) is a polynomial of degree \(q\), then \(f\) is said to be of finite \textit{genus} \(\mu=\max\qty{p,q}\).
\end{definition}
This particular Weierstrass factorization is the \textit{Weierstrass canonical factorization} of \(f\) (the portion corresponding to the product of elementary factors itself is the \textit{Weierstrass canonical product}). Now that we have indulged in the implications of \(\rho(f)\) to its zero distribution, we now turn to the function \(\varphi\) in the exponential.
\begin{lemma}\label{lem:entirefunctionfiniteorderserieslimitzero}
    Let \(f\) be entire with finite order such that \(f(0)=1\). Let \(\cbraces{a_k}_{k\in\mathbb{N}}\) be the zeros of \(f\), listed with multiplicities, such that \(\abs{a_1}\leq\abs{a_2}\leq\abs{a_3}\leq\cdots\). Suppose \(p>\rho(f)-1\); then for any \(z\in\mathbb{C}\),
    \[\lim_{r\to\infty}\mathop{\mathmakebox[\widthof{\(\sum\)}][c]{\sum_{k=1}^{n(r,0,f)}}}\overline{a_k}^{p+1}\qty(r^2-\overline{a_k}z)^{-p-1}=0.\]
\end{lemma}
\begin{proof}
    For fixed \(z\), let \(r>2\abs{z}\) such that \(a_1,\dots,a_{n(r,0,f)}\) lie in \(D(0,r)\). For each \(k\) we obtain \[\abs{r^2-\overline{a_k}z}\geq r^2-\abs{a_k}\abs{z}>r^2-r\frac r2=\frac{r^2}2\implies\abs{a_k}^{p+1}\abs{r^2-\overline{a_k}z}^{-p-1}<\qty(\frac2r)^{p+1}\] since \(\rho(f)\geq 0\) by the logarithm formula. Now by definition of \(\rho(f)\), \cref{lem:maximummoduluszerocountingdoubleradius} gives the estimate for sufficiently large \(r\) and arbitrarily small \(\varepsilon>0\): \[n(r,0,f)r^{-p-1}\leq\frac{\log M(2r,0,f)}{\log 2}r^{-p-1}\leq\frac{(2r)^{\rho(f)+\varepsilon}r^{-p-1}}{\log 2}.\]
    \newlength{\sumlength}%
    \newlength{\longsumlength}%
    \setlength{\sumlength}{\widthof{\(\sum\)}}%
    \setlength{\longsumlength}{\widthof{\(\sum_{k=1}^{n(r,0,f)}\)}}%
    Thus,
    \[\abs{\mathop{\mathmakebox[\dimexpr 0.5\sumlength+0.5\longsumlength\relax][l]{\sum_{k=1}^{n(r,0,f)}}}\overline{a_k}^{p+1}\qty(r^2-\overline{a_k}z)^{-p-1}}\leq n(r,0,f)\qty(\frac2r)^{p+1}\leq\frac{r^{\rho(f)+\varepsilon-p-1}2^{\rho(f)+\varepsilon+p+1}}{\log 2}.\]
    Letting \(\varepsilon=\frac{p+1-\rho(f)}2\) (positive by theorem assumption), we obtain
    \[\abs{\mathop{\mathmakebox[\dimexpr 0.5\sumlength+0.5\longsumlength\relax][l]{\sum_{k=1}^{n(r,0,f)}}}\overline{a_k}^{p+1}\qty(r^2-\overline{a_k}z)^{-p-1}}\leq\frac{2^{\frac{3p+3-\rho(f)}2}r^{-\varepsilon}}{\log 2}\to 0\qq{as}r\to\infty.\qedhere\]
\end{proof}
\begin{theorem}\label{thm:poissonjensenlogdiffintegralterm}
    Let \(f\) be entire with \(f(0)=1\). Then for \(p>\rho(f)-1\) (\(p\) integer) and \(z\in\mathbb{C}\),
    \[\lim_{r\to\infty}\int_0^{2\uppi}\frac{r\ee^{\ii\theta}\log\abs{f\qty(r\ee^{\ii\theta})}}{\qty(r\ee^{\ii\theta}-z)^{p+2}}\dd{\theta}=0.\]
\end{theorem}
\begin{proof}
    For fixed \(z\), \(r>2\abs{z}\), we have \[\int_0^{2\uppi}\frac{\ii r\ee^{\ii\theta}\dd{\theta}}{\qty(r\ee^{\ii\theta}-z)^{p+2}}=\ointctrclockwise_{\partial D(0,r)}\frac{\dd{w}}{(w-z)^{p+2}}=2\uppi\ii\residue_{w=z}\frac{1}{\qty(w-z)^{p+2}}\]
    by the Residue Theorem (\cref{thm:residuethm}). Since \(p+2>\rho(f)+1\geq 1\) where \(p\) is an integer, we must have \(p+2\geq 2\) and thus
    \[\int_0^{2\uppi}\frac{r\ee^{\ii\theta}\dd{\theta}}{\qty(r\ee^{\ii\theta}-z)^{p+2}}=0.\]
    Therefore,
    \begin{align*}
        \abs{\int_0^{2\uppi}\frac{r\ee^{\ii\theta}\log\abs{f\qty(r\ee^{\ii\theta})}}{\qty(r\ee^{\ii\theta}-z)^{p+2}}\dd{\theta}} & =\abs{\int_0^{2\uppi}\frac{r\ee^{\ii\theta}}{\qty(r\ee^{\ii\theta}-z)^{p+2}}\qty[\log\abs{f\qty(r\ee^{\ii\theta})}-\log M(r,f)]\dd{\theta}} \\
        & \leq\int_0^{2\uppi}\frac{r}{\qty(r/2)^{p+2}}\qty[\log M(r,f)-\log\abs{f\qty(r\ee^{\ii\theta})}]\dd{\theta}                                  \\
        & =2^{p+3}r^{-p-1}\qty[2\uppi\log M(r,f)-\int_0^{2\uppi}\log\abs{f\qty(r\ee^{\ii\theta})}\dd{\theta}]                                         \\
        & \leq 2^{p+4}r^{-p-1}\uppi\log M(r,f),
    \end{align*}
    where the last expression uses the inequality derived from Jensen's formula (\cref{cor:jensensinequality}) on the remaining integral.

    Now by assumption, we have \[\log M(r,f)\leq r^{\rho+\varepsilon}\] for any \(\varepsilon>0\) and sufficiently large \(r\). Hence, \[\abs{\int_0^{2\uppi}\frac{r\ee^{\ii\theta}\log\abs{f\qty(r\ee^{\ii\theta})}}{\qty(r\ee^{\ii\theta}-z)^{p+2}}\dd{\theta}}\leq 2^{p+4}r^{\rho(f)+\varepsilon-p-1}\uppi=2^{p+4}r^{\frac{\rho(f)-p-1}{2}}\uppi\] at \(\varepsilon=\frac{p+1-\rho(f)}{2}\). Then since \(\frac{\rho(f)-p-1}{2}<0\), the expression vanishes as \(R\to\infty\).
\end{proof}
\begin{proposition}\label{prop:entirefunctionfiniteorderlogdiffderivatives}
    Let \(f\) be entire, non-constant, and of finite order such that \(f(0)=1\). Let \(\qty{a_k}_{k\in\mathbb{N}}\) be the zeros of \(f\) counted according to multiplicities such that \(\abs{a_1}\leq\abs{a_2}\leq\abs{a_3}\leq\cdots\). If \(p>\rho(f)-1\) is an integer, then \[\dv[p]{}{z} \qty(\frac{f'(z)}{f(z)})\equiv-\sum_{k=1}^\infty\frac{p!}{\qty(a_k-z)^{p+1}}\] for all \(z\in\mathbb{C}\).
\end{proposition}
\begin{proof}
    Let \(r>2\abs{z}\). By the Poisson--Jensen Formula (\cref{thm:poissonjensenformula}), at each non-singular point, we have (the kernel representation derived in \cref{eq:poissonkernelgeneralform})
    \[\Re\log{f(z)}=\frac1{2\uppi}\int_0^{2\uppi}\log\abs{f\qty(r\ee^{\ii\theta})}\Re\qty(\frac{r\ee^{\ii\theta}+z}{r\ee^{\ii\theta}-z})\dd{\theta}+\mathmakebox[\widthof{\(\sum\)}][c]{\sum_{k=1}^{n(r,0,f)}}\Re\log\paren{\frac{r(z-a_k)}{r^2-\overline{a_k}z}}.\]
    For any holomorphic \(g=u+\ii v\), we have \[\pdv{\qty(\Re{g(z)})}{z}=\frac12\qty(\pdv{u(z)}{x}-\ii\pdv{u(z)}{y})=\frac12\qty(\pdv{u(z)}{x}+\ii\pdv{v}{x})=\frac12\pdv{g(z)}{x}=\frac{g'(z)}{2}.\]
    Therefore, by differentiation under the integral sign,
    \begin{align*}
        \frac{f'(z)}{f(z)} & =\frac1{\uppi}\int_0^{2\uppi}\log\abs{f\qty(r\ee^{\ii\theta})}\frac{r\ee^{\ii\theta}\dd{\theta}}{(r\ee^{\ii\theta}-z)^2}+\sum_{k=1}^{n\qty(r,0,f)}\frac{r^2-\abs{a_k}^2}{\qty(r^2-\overline{a_k}z)\qty(z-a_k)}                                                                                                                 \\
        & =\frac1{\uppi}\int_0^{2\uppi}\log\abs{f\qty(r\ee^{\ii\theta})}\frac{r\ee^{\ii\theta}\dd{\theta}}{\qty(r\ee^{\ii\theta}-z)^2}+\mathop{\mathmakebox[\widthof{\(\sum\)}][c]{\sum_{k=1}^{n(r,0,f)}}}\frac{\overline{a_k}}{r^2-\overline{a_k}z}+\mathop{\mathmakebox[\widthof{\(\sum\)}][c]{\sum_{k=1}^{n(r,0,f)}}}\frac{1}{z-a_k}.
    \end{align*}
    Differentiating \(p\) times from here gives
    \begin{align*}
        \dv[p]{}{z}\qty(\frac{f'(z)}{f(z)}) & =\frac1{\uppi}\mathop{\mathmakebox[\widthof{\(int\)}][l]{\int_0^{2\uppi}}}\log\abs{f\qty(r\ee^{\ii\theta})}\frac{r\ee^{\ii\theta}(p+1)!\dd{\theta}}{\qty(r\ee^{\ii\theta}-z)^{p+2}}                                                      \\
        & \quad+\mathop{\mathmakebox[\widthof{\(\sum\)}][c]{\sum_{k=1}^{n(r,0,f)}}}\frac{\overline{a_k}^{p+1}p!}{\qty(r^2-\overline{a_k}z)^{p+1}}-\mathop{\mathmakebox[\widthof{\(\sum\)}][c]{\sum_{k=1}^{n(r,0,f)}}}\frac{p!}{\qty(a_k-z)^{p+1}}.
    \end{align*}
    The first two terms vanish as \(r\to\infty\) by \cref{thm:poissonjensenlogdiffintegralterm,lem:entirefunctionfiniteorderserieslimitzero}.
\end{proof}
\begin{lemma}[Logarithmic Factorization]\label{lem:entirefunctionweierstrassproductfiniteorderlogdiffderivatives}
    Let \(f\) be entire, non-constant, and of finite order \(\rho\) such that \(f(0)=1\). Let \[P(z)=\prod_{k=1}^\infty E_{\operatorname{rank}f}\qty(\frac z{a_k})\] be the associated product. If \(p>\rho(f)-1\) is an integer, then \[\dv[p]{}{z} \qty(\frac{P'(z)}{P(z)})\equiv-\sum_{k=1}^\infty\frac{p!}{\qty(a_k-z)^{p+1}}\] for all \(z\in\mathbb{C}\).
\end{lemma}
\begin{proof}
    Let \(P_n\) be the \(n\)-th partial product of \(P\). Then \[\frac{P'_n(z)}{P_n(z)}=\sum_{k=1}^n\frac{\dv{}{z}E_{\operatorname{rank}f}\qty(\frac{z}{a_k})}{E_{\operatorname{rank}f}\qty(\frac{z}{a_k})}=\sum_{k=1}^{n}\qty[\frac{1}{z-a_k}+\sum_{j=1}^{\operatorname{rank}f}\frac{z^{j-1}}{a_k^j}],\] implying that \[\dv[p]{}{z} \qty(\frac{P_n'(z)}{P_n(z)})=-\sum_{k=1}^n\frac{p!}{\qty(a_k-z)^{p+1}}+\dv[p]{}{z}\sum_{j=1}^{\operatorname{rank}f}\sum_{k=1}^n\frac{z^{j-1}}{a_k^j}.\]
    Since the polynomial in the rightmost term has degree at most \(\max j-1=\operatorname{rank}f-1\), and because \(p>\floor{\rho}-1\geq\operatorname{rank}f-1\), after \(p\) derivatives each term of the expression vanishes. For an arbitrarily chosen compact \(K\subset\mathbb{C}\) avoiding \(a_k\), some \(N\in\mathbb{N}\) such that \(\abs{a_k}\geq\max_{z\in K}\abs{z}\) for all \(k>N\), we have \(\forall z\in K\), \(\abs{a_k-z}\leq\abs{a_k}+\abs{z}\leq2\abs{a_k}\). Then the convergence of \[\sum_{k=1}^\infty\frac1{2\abs{a_k}^{p+1}}\] from \cref{thm:entirefunctionfiniteorderzerossummationconvergence} implies the absolute convergence of \[\sum_{k=1}^\infty\frac1{\qty(a_k-z)^{p+1}}\] in \(K\). Moreover, it can be shown that the uniform convergence of \(P_n\to P\) (from the Weierstrass factorization) and \(P'_n\to P'_n\) (by the Weierstrass Convergence Theorem, \cref{thm:weierstrassconvergence}) in \(K\) implies that of \(\flatfrac{P'(z)}{P(z)}\). Hence, the Weierstrass Convergence Theeorem implies that
    \[\lim_{n\to\infty}\dv[p]{}{z} \qty(\frac{P_n'(z)}{P_n(z)})=\frac{P'(z)}{P(z)}=-\lim_{n\to\infty}\sum_{k=1}^n\frac{p!}{\qty(a_k-z)^{p+1}}=-\sum_{k=1}^\infty\frac{p!}{\qty(a_k-z)^{p+1}}.\qedhere\]
\end{proof}
The two preceding results are similar in conclusion, but \cref{lem:entirefunctionweierstrassproductfiniteorderlogdiffderivatives} is not a special case of \cref{prop:entirefunctionfiniteorderlogdiffderivatives} since we have not asserted that the canonical product is of finite order.
\subsubsection{Hadamard Factorization Theorem}
\begin{theorem}\label{thm:entirefunctionfiniteordercanonicalweierstrassfactorizationpolynomialdegree}
    Let \(f(z)=\ee^{\varphi(z)}P(z)\) be the Weierstrass canonical factorization of \(f\), where \(f\) is entire with finite order \(\rho=\rho(f)\) and \(\rho(0)=1\). Then \(\varphi\) is a polynomial of degree \(\leq\rho\).
\end{theorem}
\begin{proof}
    By logarithmic differentiation and by taking \(p>\rho-1\) subsequent derivatives, we have
    \[\frac{f'(z)}{f(z)}=\varphi'(z)+\frac{P'(z)}{P(z)}\implies\dv[p]{}{z} \qty(\frac{f'(z)}{f(z)})=\varphi^{\qty(p+1)}(z)+\dv[p]{}{z} \qty(\frac{P'(z)}{P(z)}).\]
    By applying \cref{prop:entirefunctionfiniteorderlogdiffderivatives,lem:entirefunctionweierstrassproductfiniteorderlogdiffderivatives}, we have \[-\sum_{k=1}^\infty\frac{p!}{\qty(a_k-z)^{p+1}}=\varphi^{\qty(p+1)}(z)-\sum_{k=1}^\infty\frac{p!}{\qty(a_k-z)^{p+1}}\implies \varphi^{(p+1)}\equiv 0.\]
    Hence, \(\varphi\) is a polynomial of degree \(\leq p\). Choosing \(p=1+\floor{\rho-1}>\rho-1\) so that \(p\leq\rho\), the assertion follows.
\end{proof}
\begin{corollary}\label{cor:hadamardfactorizationpolynomial}
    Let \(f(z)=z^m\ee^{\varphi(z)}P(z)\) be the Weierstrass canonical
    factorization of \(f\), where \(f\) is entire with finite order
    \(\rho=\rho(f)\). Then \(\varphi\) is a
    polynomial of degree \(\leq\rho\).
\end{corollary}
\begin{proof}
    Let \(f(z)=z^m g(z)\), where \(f\) and \(g\) are entire, \(g(0)\neq 0\), and \(f\) has finite order \(\rho(f)\) and \(M(r,f)=r^mM(r,g)\) for \(r>0\). For \(\varepsilon>0\), \(\exists r'>0\) such that \(r>r'\) implies \[M(r,f)=r^m M(r,g)\leq\ee^{r^{\rho(f)+\varepsilon}}\implies M(r,g)\leq\ee^{r^{\rho(f)+\varepsilon}}.\]
    Thus, \(\rho(g)\leq\rho(f)\) by letting \(\varepsilon\to 0^+\). Additionally, for any \(\varepsilon>0\), \(\forall r'>0\), \(\exists r>r'\) such that \[M(r,f)\geq\ee^{r^{\rho(f)-\varepsilon}}\implies M(r,g)\geq\exp\paren{r^{\rho(f)-2\varepsilon}\qty(r^\varepsilon-\frac{m\log{r}}{r^{\rho(f)-2\varepsilon}})}\geq\ee^{r^{\rho(f)-2\varepsilon}}\]
    because for sufficiently large \(r\),
    \[r^\varepsilon-\frac{m\log{r}}{r^{\rho(f)-2\varepsilon}}>1.\]
    Hence, \(\rho(g)\geq\rho(f)-2\varepsilon\). Letting \(\varepsilon\to 0^+\) implies \(\rho(f)=\rho(g)\). Let \(g(z)=ch(z)\) where \(c\) is a constant, so that \(h(0)=1\). It is also trivial that \(\rho(g)=\rho(h)\). Explicitly, we have \(h(z)=\ee^{\varphi-\Log c}P(z)\).

    By \cref{thm:entirefunctionfiniteordercanonicalweierstrassfactorizationpolynomialdegree} on \(h\), \(\varphi-\Log c\) is a polynomial of degree \(\leq\rho\), and so is \(\varphi\).
\end{proof}
%TODO
Then the results of \cref{cor:hadamardfactorizationpolynomial,thm:entirefunctionfiniteorderzerossummationconvergence} may be consolidated into a single statement:
\begin{theorem}[Hadamard Factorization Theorem]\label{thm:hadamardfactorization}
    Let \(\mu\) be the genus of \(f\) and let \(\rho\) be the order of \(f\), where \(f\) is entire with finite order. Then \(\mu\leq\rho\).
\end{theorem}
\begin{theorem}\label{thm:sinproductformula}
    The factorization \[\sin z=z\prod_{k=1}^\infty\qty(1-\frac{z^2}{\uppi^2 k^2})\]
    defines an entire function and uniformly converges on any compact disk \(\overline{D(0,r)}\).
\end{theorem}
\begin{proof}
    The zeros of \(\sin\) are simple at each of \(\mathbb{Z}\). Aside from the simple zero at \(z=0\), let \[a_k=
        \begin{dcases}
            -\flatfrac{\uppi k}{2}       & \qif* k\in 2\mathbb{N},                    \\
            \flatfrac{\uppi\qty(k+1)}{2} & \qif* k\in\mathbb{N}\setminus 2\mathbb{N},
    \end{dcases}\]
    enumerate the zeros of \(\sin\). By \cref{ex:entirefunctionfiniteordersinexpexp}, and the Hadamard Factorization Theorem (\cref{thm:hadamardfactorization}), the order of \(\sin\) is 1, the genus does not exceed 1, and \[\sin{z}=z\ee^{\varphi(z)}\prod_{k=1}^\infty E_{1}\qty(\frac{z}{a_k})=z\ee^{\varphi(z)}\prod_{k=1}^\infty\qty(1-\frac z{a_k})\exp\paren{\frac{z}{a_k}},\]
    where \(\varphi(z)=az+b\) is a polynomial (and where the product locally uniformly converges in \(\mathbb{C}\)). Since the partial products \(\qty{P_n}_{n\in\mathbb{N}}\), where \[P_n=\prod_{k=1}^n\qty(1-\frac z{a_k})\exp\paren{\frac{z}{a_k}},\]
    have a single accumulation point, the subsequence \(\qty{P_{2n}}_{n\in\mathbb{N}}\) converges to the same point. Since \[P_{2n}=\prod_{k=1}^{2n}\qty(1-\frac z{a_k})\exp\paren{\frac{z}{a_k}}=\prod_{k=1}^n\qty[\qty(1-\frac z{\uppi k})\exp\paren{\frac{z}{\uppi k}}\qty(1+\frac{z}{\uppi k})\exp\paren{-\frac{z}{\uppi k}}],\] we have \[\sin z=z\ee^{az+b}\prod_{k=1}^\infty\qty(1-\frac{z^2}{\uppi^2k^2}).\]
    Then from \(\sin z=-\sin(-z)\) we have \[z\ee^{az+b}\prod_{k=1}^\infty\qty(1-\frac{z^2}{\uppi^2k^2})\equiv z\ee^{-az+b}\prod_{k=1}^\infty\qty(1-\frac{z^2}{\uppi^2k^2})\implies\ee^{2az}\equiv 1\implies a=0.\]
    Since \(\lim_{\zeta\to 0}\frac{\sin\zeta}\zeta=1\), we have \[\lim_{z\to0}\ee^b\prod_{k=1}^\infty\qty(1-\frac{z^2}{\uppi^2 k^2})=1\implies b=0.\qedhere\]
\end{proof}
