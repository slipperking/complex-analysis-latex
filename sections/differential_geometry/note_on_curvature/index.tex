\subsection{A Note on Curvature}
We will give a brief introduction to the curvature of a surface for convenience.

Suppose \(U\subseteq\mathbb{R}^2\) is a region, and let \(\qty(u,v)\in U\). Consider a surface parameterized via \[\va{r}(u,v)=\qty(x(u,v),y(u,v),z(u,v))\in\mathbb{R}^3,\] where \(x,y,z\in C^2\qty(U)\). If \(\va{r}'_u\times\va{r}'_v\) never vanishes for \(\qty(u,v)\in U\), then \(\va{r}(U)\) defines a smooth surface \(\Sigma\). For a fixed \(\qty(u,v)\in U\), the vectors \(\va{r}'_u\) and \(\va{r}'_v\) form the basis of the tangent space (a plane) of \(\Sigma\) at \(P=\va{r}\qty(u,v)\), denoted by \(T_P\Sigma=\mathrm{span}\qty(\va{r}'_u(P),\va{r}'_v(P))\).

The square of the length of the vector infinitesimal \(\dd{\va{r}}=\va{r}'_u\dd{u}+\va{r}'_v\dd{v}\), or
\begin{equation}
    \mathrm{I}(u,v)=\dd{s}^2=E\dd{u}^2+2F\dd{u}\dd{v}+G\dd{v}^2,\label{eq:firstfundamentalform}
\end{equation} is known as the \textit{first fundamental form} of \(\Sigma\), where \(E=\va{r}'_u\cdot\va{r}'_u\), \(F=\va{r}'_u\cdot\va{r}'_v\), and \(G=\va{r}'_v\cdot\va{r}'_v\).

Let \(Q=\va{r}\qty(u+\Delta u,v+\Delta v)\) be near \(P\). It follows that \(\overrightarrow{PQ}=\va{r}\qty(u+\Delta u,v+\Delta v)-\va{r}\qty(u,v)\). The distance between \(Q\) and \(T_P\Sigma\) is \(\overrightarrow{PQ}\cdot\vu{n}\), where \(\vu{n}=\frac{\va{r}'_{u}\times\va{r}'_v}{\norm{\va{r}'_{u}\times\va{r}'_v}}\). By application of the multivariate Taylor's Theorem, we have
\begin{align*}
    \overrightarrow{PQ} & =\va{r}'_u\Delta u+\va{r}'_v\Delta v+\frac{1}{2}\qty(\va{r}''_{uu}\Delta u^2+2\va{r}''_{uv}\Delta u\Delta v+\va{r}''_{vv}\Delta v^2)+\order{\Delta u^3+\Delta v^3},
\end{align*}
and therefore, \[\overrightarrow{PQ}\cdot\vu{n}=\frac{1}{2}\qty(\va{r}''_{uu}\cdot\vu{n}\Delta u^2+2\va{r}''_{uv}\cdot\vu{n}\Delta u\Delta v+\va{r}''_{vv}\cdot\vu{n}\Delta v^2)+\order{3}\cdot\vu{n}.\] The first two linear terms vanish by properties of the triple scalar product. The \textit{second fundamental form} of \(\Sigma\) is defined as
\begin{equation}
    \mathrm{I\!I}(u,v)=L\dd{u}^2+2M\dd{u}\dd{v}+N\dd{v}^2,\label{eq:secondfundamentalform}
\end{equation} where \(L=\va{r}''_{uu}\cdot\vu{n}\), \(M=\va{r}''_{uv}\cdot\vu{n}\), and \(N=\va{r}''_{vv}\cdot\vu{n}\). Since \(\va{r}'_u\cdot\vu{n}=0\) and \(\va{r}'_v\cdot\vu{n}=0\), by differentiation, we have
\begin{align*}
    \va{r}''_{uu}\cdot\vu{n}+\va{r}'_u\cdot\vu{n}'_u & =0, & \va{r}''_{uv}\cdot\vu{n}+\va{r}'_u\cdot\vu{n}'_v & =0, \\
    \va{r}''_{uv}\cdot\vu{n}+\va{r}'_v\cdot\vu{n}'_u & =0, & \va{r}''_{vv}\cdot\vu{n}+\va{r}'_v\cdot\vu{n}'_v & =0.
\end{align*}
It follows that \(L=-\va{r}'_u\cdot\vu{n}'_u\), \(M=-\va{r}'_u\cdot\vu{n}'_v=-\va{r}'_v\cdot\vu{n}'_u\), and \(N=-\va{r}'_v\cdot\vu{n}'_v\). Because \(\dd{\vu{n}}=\vu{n}'_u\dd{u}+\vu{n}'_v\dd{v}\), \[\mathrm{I\!I}=-\dd{\va{r}}\cdot\dd{\vu{n}}.\]
\begin{figure}
    \centerline{\includesvg[width=0.75\linewidth]{secondfundamentalform.svg}}
    \caption{\(Q\) has a greater heuristic distance to \(T_P\Sigma\) for a more curved surface.}\label{fig:secondfundamentalform}
\end{figure}The second fundamental form, in a rough sense, measures the curvature of the surface \(\Sigma\) at \(P\) (refer to \cref{fig:secondfundamentalform}). Both the first and second fundamental forms are geometric invariants; they are independent of the parameterization \(\va{r}\) of \(\Sigma\). The first fundamental form is also referred to as the \textit{intrinsic metric} (we will not delve into the metric tensor here) of \(\Sigma\), and the second fundamental form is an \textit{extrinsic} property of \(\Sigma\) as it is invariant up to the orientation of the surface (consequent direction of the normal vector).

Let \(\gamma\subset\Sigma\) be a curve parameterized by arc length, \(\va{r}(s)=\va{r}\qty(u(s),v(s))\). Then the unit tangent vector at \(P=\va{r}(s)\) is \[\va{T}(s)=\dv{\va{r}}{s}=\va{r}'_u\dv{u}{s}+\va{r}'_v\dv{v}{s}.\]
Consequently, \[\va{T}'(s)=\va{r}''_{uu}\qty(\dv{u}{s})^2+2\va{r}''_{uv}\qty(\dv{u}{s})\qty(\dv{v}{s})+\va{r}''_{vv}\qty(\dv{v}{s})^2+\va{r}'_u\dv[2]{u}{s}+\va{r}'_v\dv[2]{v}{s},\]
where the last two terms are in \(T_P\Sigma\). Because \(\norm{\va{T}(s)}=1\) for all \(s\) by the arc-length parameterization, we have \[0=\dv{s}(\norm{\va{T}(s)}^2)=\dv{s}(\va{T}(s)\cdot\va{T}(s))=2\va{T}(s)\cdot\va{T}'(s).\]
Hence, \(\va{T}(s)\) and \(\va{T}'(s)\) are orthogonal and \(\va{T}'(s)\) is a normal to the curve \(\gamma\). Let \(\vu{n}=\frac{\va{r}'_{u}\times\va{r}'_v}{\norm{\va{r}'_{u}\times\va{r}'_v}}\) be the unit normal to \(\Sigma\) at \(P\). The \textit{normal curvature} of \(\gamma\) at \(P\) in \(\Sigma\) is defined as \[\kappa_n=\va{T}'(s)\cdot\vu{n}=\qty[\va{r}''_{uu}\qty(\dv{u}{s})^2+2\va{r}''_{uv}\qty(\dv{u}{s})\qty(\dv{v}{s})+\va{r}''_{vv}\qty(\dv{v}{s})^2]\cdot\vu{n}.\]
The quotient \[\kappa_n=\frac{\mathrm{I\!I}}{\mathrm{I}}=\frac{L\dd{u}^2+2M\dd{u}\dd{v}+N\dd{v}^2}{E\dd{u}^2+2F\dd{u}\dd{v}+G\dd{v}^2},\] varies depending on the curve traversing \(\Sigma\) (and ultimately, depending on the direction induced by \(\dd{u}\) and \(\dd{v}\)). On \(\gamma\), the two representations are equivalent since \(\mathrm{I}=\dd{s}^2\). The maximum and minimum values of \(\kappa_n\) are known as the \textit{principal curvatures} \(\kappa_1\) and \(\kappa_2\) of \(\Sigma\) at \(P\), achieved along the \textit{principle directions} of the (unit) tangent vectors at \(P\).

The \textit{mean curvature} of \(\Sigma\) at \(P\) is defined to be \(H=\frac{\kappa_1+\kappa_2}{2}\). Let \(r_1,r_2\) be the radii of curvature corresponding to \(\kappa_1\) and \(\kappa_2\). Hence, we have \[H=\frac{\frac{1}{r_1}+\frac{1}{r_2}}{2}=\frac{r_1+r_2}{2r_1r_2}=\kappa_1\kappa_2\frac{r_1+r_2}{2}.\] The product of the two principle curvatures is known as the \textit{Gaussian curvature} of \(\Sigma\) at \(P\), denoted by \(K=\kappa_1\kappa_2\). We will now derive the explicit formulas for \(H\) and \(K\) in terms of \(E,F,G,L,M,N\).

Obviously, the matrix \(\mqty(E&F\\F&G)\) is symmetric and positive definite. Indeed, by Sylvester's Criterion, the assertion follows from the fact that \[\det\mqty(E&F\\F&G)=EG-F^2=\norm{\va{r}'_u}^2\norm{\va{r}'_v}^2-\qty(\va{r}'_u\cdot\va{r}'_v)^2=\norm{\va{r}'_u\times\va{r}'_v}^2>0\] and \(E=\norm{\va{r}_u}^2>0\).
