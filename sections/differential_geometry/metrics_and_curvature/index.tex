\subsection{Metrics and Curvature}
Let \(\Omega\subseteq\mathbb{C}\) be a region and let \(\rho\in C^0(\Omega)\) be a positive function. The \textit{conformal metric} (or just \textit{metric}) induced by \(\rho\) is given by \[\dd{s}=\rho(z)\abs{\dd{z}}\qor\dd{s}^2=\rho^2(z)\abs{\dd{z}}^2.\] Note here that the term ``conformal'' here refers not to the holomorphy of \(\rho\), but rather, the geometric emphasis that only lengths are not preserved, and the scaling factor by \(\rho\) vanishes in the calculation of angles. The distance between two points \(z_1,z_2\in\Omega\) is defined as \[d\qty(z_1,z_2)=\inf_{\gamma\subset\Omega}\int_\gamma\rho(z)\abs{\dd{z}},\] where the infimum is taken over all piecewise smooth curves \(\gamma\) in \(\Omega\) joining \(z_1\) and \(z_2\).

A \(C^2\) metric is said to be \textit{regular}. The (Gaussian) \textit{curvature} of the regular metric \(\rho\) at \(z\in\Omega\) is defined as
\begin{equation}
    K_\rho(z)=-\frac{\laplacian(\log\rho)(z)}{\rho(z)^2},\label{eq:curvatureofmetric}
\end{equation} where \(\laplacian=\pdv[2]{}{x}+\pdv[2]{}{y}=4\pdv[2]{}{\overline{z}}{z}\) is the Laplacian operator. This is the same definition as the Gaussian curvature in TO BE CONTINUED.

The three following metrics are of particular interest in complex differential geometry:
\begin{enumerate}
    \item Perhaps the most trivial metric is the \textit{Euclidean metric} (also known
        as the \textit{parabolic metric}) on \(\mathbb{C}\), and is given by \[\rho=1,\qquad\dd{s}^2=\abs{\ddz}^2.\] The \textit{Euclidean distance} between two points \(z_1,z_2\in\mathbb{C}\)
        is \[\inf_{\gamma}\int_\gamma\abs{\ddz}=\abs{z_2-z_1}\] is the length of the straight line segment connecting \(z_1\) and \(z_2\). The
        group formed by all transformations in the form of
        \(z\mapsto\ee^{\ii\theta}z+a\) (where \(a\in\mathbb{C}\) and
        \(\theta\in\mathbb{R}\)) is known as \textit{the group of rigid motions}, or
        more abstractly, the \textit{special Euclidean group} of order 2, denoted by
        \(\mathrm{SE}(2)<\Aut(\mathbb{C})\), intuitively consists of all rotations and
        translations and their compositions, while the \textit{Euclidean} group
        \(\mathrm{E}(2)>\mathrm{SE}(2)\) consists of reflections in the form of
        \(z\mapsto\ee^{\ii\theta}\overline{z}+a\). Obviously, the Euclidean metric is
        invariant under both groups.

        From \cref{eq:curvatureofmetric}, we find that Euclidean metric has curvature
        \(K=0\).
    \item The \textit{Poincaré metric} (also referred to as the \textit{hyperbolic
        metric}) on \(\mathbb{D}\) is given by
        \begin{equation}
            \rho=\lambda(z)=\frac{2}{1-\abs{z}^2},\qquad\dd{s}_{\lambda}^2=\frac{4\abs{\ddz}^2}{\qty(1-\abs{z}^2)^2}\label{eq:poincaremetricdefinition}
        \end{equation}
        In \cref{lem:schwarzpick}, it was shown that the metric is invariant under \(\Aut(\mathbb{D})\).

        We will now calculate the Poincaré distance between two points
        \(z_1,z_2\in\mathbb{C}\). First assume the case where \(z_1=0\) and \(z_2=R\in
        (0,1)\). Consider a piecewise smooth curve \(\gamma\subset\mathbb{D}\)
        parameterized by \(z(t)\) connecting \(z_1\) and \(z_2\); or in other words \[z(t)=x(t)+\ii y(t),\qquad z(0)=z_1=0,\quad z(1)=z_2=R,\] where \(x\in C^1([0,1])\) and \(y\in C^1([0,1])\) are real-valued functions.
        Then
        \begin{align*}
            \int_\gamma\dd{s} & =\int_0^1\frac{2\sqrt{x'(t)^2+y'(t)^2}}{1-x(t)^2-y(t)^2}\dd{t}\geq\int_0^1\frac{2\abs{x'(t)}}{1-x(t)^2}\ddt\geq\abs{\int_0^1\frac{2x'(t)}{1-x(t)^2}\dd{t}} \\
            & =\abs{\int_0^R\frac{2}{1-x^2}\ddx}=\log(\frac{1+R}{1-R}).
        \end{align*} Assuming that \(\gamma\) is in the form of \(z(t)=Rt, z'(t)=R\) where \(t\in[0,1]\), we have \[\int_\gamma\dd{s}=\int_0^1\frac{2R\ddt}{1-R^2t^2}=\log(\frac{1+R}{1-R}).\] Hence, the Poincaré distance between \(0\) and \(R\) is given by \[d\qty(0,R)=\log(\frac{1+R}{1-R})\] and the straight line segment connecting the two points is a
        \textit{geodesic}. For fixed \(\theta\in\mathbb{R}\) since \(z\mapsto
        z\ee^{\ii\theta}\in\Aut(\mathbb{D})\), by the Schwarz--Pick Lemma
        (\cref{lem:schwarzpick}), we have \[d\qty(0,R)=d\qty(0,R\ee^{\ii\theta})=\log(\frac{1+R}{1-R})\] by the invariance under \(\Aut(\mathbb{D})\). Now let \(z_1\) and \(z_2\) be
        arbitrary points in \(\mathbb{D}\). The Möbius transformation \[\varphi_{z_1}(z)=\frac{z-z_1}{1-\overline{z_1}z}\]
        maps \(z_1\) to \(0\) and maps \(z_2\) to
        \(\frac{z_2-z_1}{1-\overline{z_1}z_2}\). Hence, we have \[d\qty(z_1,z_2)=d\qty(0,\frac{z_2-z_1}{1-\overline{z_1}z_2})=\log\qty[\frac{1+\abs{\frac{z_2-z_1}{1-\overline{z_1}z_2}}}{1-\abs{\frac{z_2-z_1}{1-\overline{z_1}z_2}}}]=\inf_{\gamma}\int_\gamma\dd{s},\] which is the Poincaré distance (or \textit{hyperbolic distance}) between
        \(z_1\) and \(z_2\). The infimum is attained along the geodesic curve
        \(\gamma\) parameterized by
        \[z(t)=\qty(\varphi_{z_1})^{-1}\qty(\frac{z_2-z_1}{1-\overline{z_1}z_2}t)\]
        for \(t\in[0,1]\). By
        \cref{thm:linearfractionaltransformationmapscirclestocircles}, the geodesic is
        either an arc or a straight line segment passing through \(z_1\) and \(z_2\).
        Since \(\partial\mathbb{D}\) is orthogonal to the straight line passing through
        \(0\) and \(\frac{z_2-z_1}{1-\overline{z_1}z_2}\), by the conformality of
        \(\varphi_{z_1}^{-1}\),
        \(\varphi_{z_1}^{-1}\qty(\partial\mathbb{D})=\partial\mathbb{D}\) is orthogonal
        to the circular (or straight line) extension of the geodesic curve.

        As a consequence of the Schwarz--Pick Lemma (\cref{lem:schwarzpick}), for any
        \(f:\mathbb{D}\to\mathbb{D}\) is holomorphic, we have \[d\qty(f(z_1),f(z_2))\leq d\qty(z_1,z_2),\] where equality is attained iff \(f\in\Aut(\mathbb{D})\). The Poincaré metric
        has constant negative curvature \(-1\) since
        \begin{align*}
            K_\lambda & =-\frac{4}{\lambda^2}\pdv[2]{}{\overline{z}}{z}\qty(\log\circ\lambda)=-\frac{4}{\lambda^2}\pdv{\overline{z}}(\frac{\lambda'_z}{\lambda})=-\frac{2}{\lambda^2}\pdv{\overline{z}}(\frac{2\overline{z}}{1-\abs{z}^2}) \\
            & =-\frac{2}{\lambda^2}\qty(\lambda+\overline{z}\lambda'_{\overline{z}})=-\frac{\qty(1-\abs{z}^2)^2}{2}\qty(\lambda+\overline{z}\qty(\frac{2z}{\qty(1-\abs{z}^2)^2}))                                                \\
            & =-\qty(1-\abs{z}^2)-\abs{z}^2=-1,
        \end{align*} where \(\lambda'_z=\pdv{\lambda}{z}\) and \(\lambda'_{\overline{z}}=\pdv{\lambda}{\overline{z}}\).
    \item The \textit{spherical metric} (also referred to as the \textit{elliptic
        metric}) on \(\extcomplex\) is given by
        \begin{equation}
            \rho=\sigma(z)=\frac{2}{1+\abs{z}^2},\qquad\dd{s}^2_\sigma=\frac{4\abs{\ddz}^2}{\qty(1+\abs{z}^2)^2}.\label{eq:sphericalmetricdefinition}
        \end{equation}
        Under the inverse stereographic projection of \(S^2\to\extcomplex\), for a given \(z\in\extcomplex\), the corresponding point in \(S^2\) is \[\qty(x_1,x_2,x_3)=\qty(\frac{z+\overline{z}}{\abs{z}^2+1},\frac{z-\overline{z}}{\ii\abs{z}^2+\ii},\frac{\abs{z}^2-1}{\abs{z}^2+1}).\] If we let \(P=\qty(x_1,x_2,x_3)\) and
        \(Q=\qty(\widetilde{x_1},\widetilde{x_2},\widetilde{x_3})\) be two points in
        \(S^2\), the distance between the two points is the length of the shortest arc
        \(\widearc{PQ}\) (a subset of great circle passing the two points). By
        considering \(P\) and \(Q\) as vectors from \(\qty(0,0,0)\), this distance is
        equal to
        \begin{align*}
            \arccos(P\cdot Q) & =2\arctan\sqrt{\frac{1-x_1\widetilde{x_1}-x_2\widetilde{x_2}-x_3\widetilde{x_3}}{1+x_1\widetilde{x_1}+x_2\widetilde{x_2}+x_3\widetilde{x_3}}}                                                                                                                                                                                                                                                                                                                                                                                                                                                                                                                                                                                           \\
            & =2\arctan\sqrt{\frac{1-\frac{\qty(z+\overline{z})\qty(\widetilde{z}+\overline{\widetilde{z}})}{\qty(\abs{z}^2+1)\qty(\abs{\widetilde{z}}^2+1)}+\frac{\qty(z-\overline{z})\qty(\widetilde{z}-\overline{\widetilde{z}})}{\qty(\abs{z}^2+1)\qty(\abs{\widetilde{z}}^2+1)}-\frac{\qty(\abs{z}^2-1)\qty(\abs{\widetilde{z}}^2-1)}{\qty(\abs{z}^2+1)\qty(\abs{\widetilde{z}}^2+1)}}{1+\frac{\qty(z+\overline{z})\qty(\widetilde{z}+\overline{\widetilde{z}})}{\qty(\abs{z}^2+1)\qty(\abs{\widetilde{z}}^2+1)}-\frac{\qty(z-\overline{z})\qty(\widetilde{z}-\overline{\widetilde{z}})}{\qty(\abs{z}^2+1)\qty(\abs{\widetilde{z}}^2+1)}+\frac{\qty(\abs{z}^2-1)\qty(\abs{\widetilde{z}}^2-1)}{\qty(\abs{z}^2+1)\qty(\abs{\widetilde{z}}^2+1)}}} \\
            & =2\arctan\sqrt{\frac{-z\overline{\widetilde{z}}-\overline{z}\widetilde{z}+\abs{z}^2+\abs{\widetilde{z}}^2}{z\overline{\widetilde{z}}+\overline{z}\widetilde{z}+\abs{z}^2\abs{\widetilde{z}}^2+1}}=2\arctan\sqrt{\frac{\qty(z-\widetilde{z})\qty(\overline{z}-\overline{\widetilde{z}})}{\qty(z\overline{\widetilde{z}}+1)\qty(\overline{z}\widetilde{z}+1)}}.
        \end{align*}
        Notice that the fraction within the square root is a product between a complex number and its conjugate. Thus, this distance is equal to \[d\qty(z,\widetilde{z})=2\arctan\abs{\frac{z-\widetilde{z}}{z\overline{\widetilde{z}}+1}}\] in the extended complex plane. Let \(\widetilde{z}=z+\Delta z\). It follows
        that
        \begin{align*}
            d(z,z+\Delta z) & =2\arctan\abs{\frac{\Delta z}{\abs{z}^2+z\overline{\Delta z}+1}}=2\arctan\abs{\frac{\Delta z}{\abs{z}^2+1}\frac{1}{1+\frac{z\overline{\Delta z}}{\abs{z}^2+1}}} \\
            & =2\arctan\abs{\frac{\Delta z}{\abs{z}^2+1}\qty(1+\order{\Delta z})}=2\arctan\abs{\frac{\Delta z}{\abs{z}^2+1}+\order{\Delta z^2}}                               \\
            & =2\qty[\abs{\frac{\Delta z}{\abs{z}^2+1}+\order{\Delta z^2}}+\order{\Delta z^3\qty[\frac{1}{\abs{z}^2+1}+\order{\Delta z}]^3}]                                  \\
            & =2\qty[\abs{\frac{\Delta z}{\abs{z}^2+1}+o\qty(\Delta z^2)}],
        \end{align*}
        where we have taken the liberty to coalesce orders for simplification. Since \[\lim_{\Delta z\to 0}\abs{\frac{d\qty(z,z+\Delta z)}{\Delta z}}=\frac{2}{\abs{z}^2+1},\] the metric as defined in \cref{eq:sphericalmetricdefinition} has a clear
        geometric meaning: the distance between two points \(z\) and \(\widetilde{z}\)
        under the metric in \cref{eq:sphericalmetricdefinition} is the shortest
        distance between the corresponding points in \(S^2\), or their spherical
        distance.

        Thus, if curve \(\gamma\) joins \(z\) and \(\widetilde{z}\), we have \[d\qty(z,\widetilde{z})=\inf_{\gamma}\int_\gamma\sigma(z)\abs{\ddz},\] which attains its infimum when the inverse stereographic projection of
        \(\gamma\) is a great circle of \(S^2\). Thus, \(\sigma\) is known as the
        spherical metric.

        The corresponding curvature is given by
        \begin{align*}
            K_\sigma & =-\frac{4}{\sigma^2}\pdv[2]{}{\overline{z}}{z}\qty(\log(\sigma))=-\frac{4}{\sigma^2}\pdv{\overline{z}}(\frac{\sigma'_z}{\sigma})=\frac{2}{\sigma^2}\pdv{\overline{z}}(\frac{2\overline{z}}{1+\abs{z}^2}) \\
            & =\frac{2}{\sigma^2}\qty(\sigma+\overline{z}\sigma'_{\overline{z}})=\frac{\qty(1+\abs{z}^2)^2}{2}\qty(\frac{2}{1+\abs{z}^2}-\frac{2\abs{z}^2}{\qty(1+\abs{z}^2)^2})                                       \\
            & =\qty(1+\abs{z}^2)-\abs{z}^2=1,
        \end{align*} where \(\sigma'_z=\pdv{\sigma}{z}\) and \(\sigma'_{\overline{z}}=\pdv{\sigma}{\overline{z}}\).
\end{enumerate}
The importance of the selected regions lies in the uniformization as mentioned in \cref{sec:riemannsurfaces}.

Let \(\Omega_1\) and \(\Omega_2\) be two open regions in \(\mathbb{C}\) such
that \(f:\Omega_1\to\Omega_2\) is univalent (implying that \(f'\neq 0\) by
\cref{lem:univalentnonvanishingderivative}). If \(\rho\) is a metric on
\(\Omega_2\), then
\begin{equation}
    f^*\rho=(\rho\circ f)\abs{f'}\label{eq:pullbackmetric}
\end{equation} defines a metric on \(\Omega_1\), referred to as the \textit{metric pullback of} \(\rho\) \textit{by} \(f\).

Curvature as defined in \cref{eq:curvatureofmetric} is invariant under
pullbacks of conformal mappings, or in the case above, we now aim to show that (under assumptions of regularity)
\begin{equation}
    K_\rho(f(z))=K_{f^*\rho}(z).\label{eq:curvatureinvarianceunderholomorphicpullback}
\end{equation}
By explicit definition, \[K_{f^*\rho}(z)=-\frac{\laplacian(\log\circ f^*\rho)(z)}{(f^*\rho)(z)^2}=-\frac{\qty(\laplacian\log\circ\rho\qty(f))(z)+\laplacian\log\abs{f'(z)}}{\qty(f^*\rho)^2(z)}.\] Since \(f'(z)\neq 0\), \(\log\circ\abs{f'}=\Re\log(f')\) is harmonic on
\(\Omega_1\) with a vanishing Laplacian. Hence,
\begin{align*}
    K_{f^*\rho}(z) & =-\frac{\qty(\laplacian\log\circ\rho\qty(f))(z)}{\qty(\rho\circ f)^2\abs{f'}^2}=-\frac{4}{\qty(\rho\circ f)^2\abs{f'}^2}\pdv{\overline{z}}(\pdv{z}(\log\circ\rho\circ f(z))) \\
    & =-\frac{4}{\qty(\rho\circ f)^2\abs{f'}^2}\pdv{\overline{z}}(\pdv{\log\circ\rho}{f}\pdv{f}{z}+\pdv{\log\circ\rho}{\overline{f}}\overline{\qty(\pdv{f}{\overline{z}})})        \\
    & =-\frac{4}{\qty(\rho\circ f)^2\abs{f'}^2}\pdv{f}{z}\pdv{\overline{z}}(\pdv{\log\circ\rho}{f})                                                                                \\
    & =-\frac{4}{\qty(\rho\circ f)^2\abs{f'}^2}\pdv{f}{z}\qty(\pdv[2]{\log\circ\rho}{f}\pdv{f}{\overline{z}}+\pdv[2]{\log\circ\rho}{\overline{f}}{f}\overline{\qty(\pdv{f}{z})})   \\
    & =-\frac{4}{\qty(\rho\circ f)^2}\pdv[2]{}{\overline{f}}{f}\qty(\log\circ\rho)=-\frac{\laplacian_f\qty(\log\circ\rho)}{\qty(\rho\circ f)^2}=K_\rho\qty(f(z)).
\end{align*}