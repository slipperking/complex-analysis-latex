\subsection{The Great Picard, Bloch, Landau, and Schottky Theorems}
Recall the Casorati--Weierstrass Theorem, one of the earliest results on the value distribution near essential singularities: \getkeytheorem{thm:casoratiweierstrass} We will now prove a more advanced characterization of this distribution by methods of differential geometry.
\begin{theorem}[name=\textsc{Picard's Great Theorem},store=thm:greatpicard]\label{thm:greatpicard}
    Suppose \(f\) is holomorphic on a punctured neighborhood \(D^*\qty(z_0,\delta)\) of \(z_0\). If \(z_0\) is an essential singularity of \(f\), then \(f\qty(D^*\qty(z_0,\delta))\) omits at most one value of \(\mathbb{C}\).
\end{theorem}
\begin{proof}
    Without loss of generality, assume \(z_0=0\) and that \(f\) omits the values \(0\) and \(1\) (otherwise, consider \(z\mapsto\frac1{\beta-\alpha}(f\qty(z+z_0)-\alpha)\), where \(\alpha\) and \(\beta\) are the omitted values). Define the family
    \[\mathcal{F}=\cbraces{z\mapsto f\qty(\frac zn)}{n\in\mathbb{N}}\]
    of holomorphic functions on \(D^*\qty(0,\delta)\). Since \(f\) omits \(0\) and \(1\), each element of \(\mathcal{F}\) does as well. By the Fundamental Normality Test (\cref{thm:fundamentalnormalitytest}), \(\mathcal{F}\) is spherically normal. Thus, there exists a subsequence \(\cbraces{f_{n_k}}_{k\in\mathbb{N}}\subseteq\mathcal{F}\) that converges locally uniformly on \(D^*\qty(0,\delta)\) in the spherical metric. By \cref{prop:locallyuniformlysphericallyconvergentholomorphicsequenceuniformlimit}, this subsequence converges locally uniformly either to a holomorphic function on \(D^*\qty(0,\delta)\) or to \(\infty\) thereon.
    \begin{enumerate}
        \item Suppose \(\cbraces{f_{n_k}}_{k\in\mathbb{N}}\) converges locally uniformly to a holomorphic function on \(D^*\qty(0,\delta)\). Then \(\cbraces{f_{n_k}}_{k\in\mathbb{N}}\) is uniformly bounded on \(\partial D\qty(0,\frac\delta2)\). Hence, there exists \(M>0\) such that
              \[\abs{f\qty(\frac z{n_k})}=\abs{f_{n_k}(z)}<M\qquad\forall z\in\partial D\qty(0,\frac\delta2),\ k\in\mathbb{N}.\]
              In other words, \(f\) is bounded by \(M\) on every circle \(\partial D\qty(0,\frac{\delta}{2n_k})\) for \(k\in\mathbb{N}\). By the Maximum Modulus Principle (\cref{thm:maximummodulus}), \(f\) is then bounded by \(M\) on each annulus \(\overline{D\qty(0,\frac{\delta}{2n_k})}\setminus D\qty(0,\frac{\delta}{2n_{k+1}})\) for \(k\in\mathbb{N}\). As
              \[\bigcup_{k\in\mathbb{N}}\overline{D\qty(0,\frac{\delta}{2n_k})}\setminus D\qty(0,\frac{\delta}{2n_{k+1}})=\overline{D\qty(0,\frac\delta{2n_1})}\setminus\cbraces{0},\]
              it follows that \(f\) is bounded on \(D^*\qty(0,\frac\delta2)\). By Riemann's Removable Singularity Theorem (\cref{thm:riemannremovablesingularities}), \(f\) therefore extends holomorphically to \(0\).

        \item Suppose \(\cbraces{f_{n_k}}_{k\in\mathbb{N}}\) converges locally uniformly to \(\infty\) on \(D^*\qty(0,\delta)\). Then, for every \(\varepsilon>0\), there exists \(N\in\mathbb{N}\) such that, for all \(k>N\),
              \[\abs{\frac{1}{f\qty(\frac{z}{n_k})}}=\abs{\frac1{f_{n_k}(z)}}<\varepsilon\qquad\forall z\in\partial D\qty(0,\frac\delta2).\]
              By the same reasoning as in the previous case, \(\abs{\frac1f}<\varepsilon\) on
              \[\bigcup_{k>N}\overline{D\qty(0,\frac{\delta}{2n_k})}\setminus D\qty(0,\frac{\delta}{2n_{k+1}})\supset D^*\qty(0,\frac{\delta}{2n_{N+1}}).\]
              Thus, by the definition of the limit, \(\lim_{z\to 0}1/f(z)=0\), so \(f\) has a pole at \(0\).
    \end{enumerate}
    In either case, we have derived a meromorphic continuation of \(f\) to \(0\), contradicting the assumption that \(0\) is an essential singularity of \(f\).
\end{proof}
\begin{corollary}[store=cor:greatpicardmeromorphic]\label{cor:greatpicardmeromorphic}
    Suppose that \(f\) is meromorphic on a punctured neighborhood \(D^*\qty(z_0,\delta)\) of \(z_0\). If \(f\qty(D^*\qty(z_0,\delta))\) omits at least three different values of \(\extcomplex\), then \(f\) has a meromorphic continuation to \(z_0\).
\end{corollary}
\begin{proof}
    A linear fractional transformation maps the omitted values to \(0,1,\infty\), mapping \(f\) so that it exhibits holomorphy. Similar to \cref{cor:montelcaratheodory}, the preceding result is preserved under the inverse linear fractional transformation.
\end{proof}
\begin{remark}
    An accumulation point of poles is an essential singularity on the Riemann sphere.
\end{remark}
Picard's Great Theorem is also a generalization of Picard's Little Theorem (\cref{thm:littlepicard}):
\getkeytheorem{thm:littlepicard}
\begin{proof}
    Let \(g(z)=f\qty(\frac 1z)\) with an isolated singularity at 0 and a removable singularity at \(\infty\). By Picard's Great Theorem (\cref{thm:greatpicard}), \(g(z)\) has a meromorphic extension to \(z=0\). If \(z=0\) is removable, by virtue of \cref{prop:removablesingularityatinftyentireconstant,thm:liouville}, the constancy of \(g\) and \(f\) follows.

    If instead \(z=0\) is a pole of \(g\), then \(z=\infty\) is a pole of \(f\), and hence \(f\) is a polynomial. Assume, for the sake of contradiction that \(f\) is constant. Then \(\forall w\in\mathbb{C}\), the Fundamental Theorem of Algebra (\cref{thm:fundamentaltheoremofalgebra}) gives the existence of some \(z\in\mathbb{C}\) such that \(f(z)=w\). Hence, \(f\) attains every value \(w\in\mathbb{C}\). This contradicts the statement and hence \(f\) is constant.
\end{proof}
The efforts of many mathematicians resulted in several alternative proofs following that of Picard; the geometric realization of Ahlfors (\cref{thm:schwarzahlforspick}) was followed by results discovered by R. M. Robinson. Other approaches from Nevanlinna theory appeared later in the 20th century.

Picard's original proof, providing an advanced characterization of the value distribution at essential singularities, relied primarily on the properties of the elliptic modular function (as a ``covering map''). From this, Picard deduced that the function would necessarily extend holomorphically across the singularity, contradicting its essential nature. Thus, his proof established that near an essential singularity, a holomorphic function attains every complex value, with at most one exception, infinitely often.

More importantly, we have shown the utility of even seemingly fundamental differential geometry, which can also be used in the proof of many other important results.
\begin{theorem}[name=\textsc{Bloch}]\label{thm:bloch}
    Let \(f:\mathbb{D}\to\mathbb{C}\) be holomorphic such that \(\abs{f'(0)}=1\). Then there is a region \(S\subseteq\mathbb{D}\) on which \(f\) is univalent such that \(f(S)\) contains a disk with a radius of at least \(\frac{\sqrt{3}}{4}\) (known as a schlicht disk).
\end{theorem}
\begin{proof}
    For \(w\in f(S)\), let \(\phi(w)\) denote the radius of the largest schlicht disk in \(f(S)\) centered at \(w\) (it is mapped to univalently by \(f\) on some subdomain). Trivially, \(\phi\) is \(C^0\).
    
    Define the metric \[\rho(w)=\]
\end{proof}
\begin{remark}
    \textit{Bloch's constant} \(B\) is defined as the supremum of the radii of such disks that can be contained in \(f(\mathbb{D})\) for any holomorphic function \(f:\mathbb{D}\to\mathbb{C}\) satisfying \(f'(0)=1\).

    The precise value of \(B\) remains unknown to this day. In 1937, H. Grunsky and L. Ahlfors established the bound
    \[B\leq\frac{\Gamma\qty(\frac{1}{3})\Gamma\qty(\frac{11}{12})}{\Gamma\qty(\frac{1}{4})}\sqrt{\frac{\sqrt{3}-1}{2}},\]
    where \(\Gamma\) denotes the Gamma function (as in \cref{eq:gammafunction}). Later the lower bound of \(\frac{\sqrt{3}}{4}\) was given, then to be refined to \(B\geq\frac{\sqrt{3}}{4}+\frac{10^{-12}}{13}\) by M. Bonk, which was further improved to \(B\geq\frac{\sqrt{3}}{4}+\frac{1}{5000}\) in 1996 by H. Chen and P. M. Gauthier.

    Grunsky and Ahlfors actually conjectured that the upper bound in their inequality is exact---that is, \(B=\frac{\Gamma\qty(\frac{1}{3})\Gamma\qty(\frac{11}{12})}{\Gamma\qty(\frac{1}{4})}\sqrt{\frac{\sqrt{3}-1}{2}}\).
\end{remark}
\begin{theorem}[\textsc{Landau}]
    The image of any holomorphic function \(f\) in \(\mathbb{D}\) satisfying \(f(0)=0\) and \(f'(0)=1\) contains a disk with radius of at least \(\frac12\).
\end{theorem}
\begin{proof}
    
\end{proof}
\begin{remark}
    Similarly, the estimate \(\frac12\) is not optimal. It was established that the corresponding \textit{Landau's constant} lies between \(\frac12\) and \(\frac{\Gamma\qty(\frac13)\Gamma\qty(\frac56)}{\Gamma\qty(\frac16)}\).
\end{remark}
\begin{theorem}[\textsc{Landau--Carathéodory}]
    Let \(f(z)=\sum_{n=0}^\infty a_n z^n\) such that \(a_1\neq0\) and \(f\) is holomorphic on \(D(0,r)\). If \(f\) omits 0 and 1, then \(\exists R\) dependent only on \(a_0\) and \(a_1\) such that \(r\leq R\).
\end{theorem}
\begin{theorem}[\textsc{Schottky}]
    Suppose that \(f(z)=\sum_{n=0}^\infty a_n z^n\) defines an analytic function on \(D(0,r)\) omitting 0 and 1.
\end{theorem}
