\subsection{From Schwarz--Pick to Ahlfors and Value Distribution of Entire Functions}
While Schwarz Lemma in \cref{lem:schwarz} concerns self-maps of \(\mathbb{D}\) with a fixed point at the origin, the Schwarz--Pick Lemma in \cref{lem:schwarzpick} generalizes this to arbitrary points in \(\mathbb{D}\) as well as the hyperbolic contraction property of holomorphic maps.

In 1938, Lars Ahlfors provided a further generalization by curvature, prompting the study of complex functions from a differential-geometric approach.

The hyperbolic metric \(\lambda\) in \cref{eq:poincaremetricdefinition} does not increase under any holomorphic \(f:\mathbb{D}\to\mathbb{D}\). It was realized that this was a consequence of the constant negative curvature \(-1\) of \(\lambda\).
\begin{theorem}[\textsc{Schwarz--Ahlfors--Pick}]\label{thm:schwarzahlforspick}
    Let \(f\) be holomorphic on \(\mathbb{D}\). Suppose that \(\rho\) is a conformal metric defined on an open neighborhood \(U\supseteq f(\mathbb{D})\), where \(\dd{s}^2_\rho=\rho^2(z)\abs{\ddz}^2\) and \(K_\rho(z)\leq -1\) for all \(z\in U\). Then \[f^*\rho(z)\leq\lambda(z)\quad\forall z\in\mathbb{D},\] where \(\lambda\) is the Poincaré metric, and equivalently, \[\dd{s}^2_{f^*\rho}\leq\dd{s}^2_\lambda,\] or that the metric \(\rho\) does not exceed the hyperbolic metric under the map \(f\).
\end{theorem}
\begin{proof}
    Define
    \begin{equation}
        \lambda_r(z)=\qty(z\mapsto\frac{z}{r})^*\lambda(z)=\frac{2r}{r^2-\abs{z}^2},\qquad 0<r<1\label{eq:poincaremetriconscaleddisks}
    \end{equation} to generalize the Poincaré metric to \(D(0,r)\). \Cref{eq:curvatureinvarianceunderholomorphicpullback} gives that \(K_{\lambda_r}(z)=K_\lambda\qty(\frac{z}{r})=-1\) for any \(z\in D(0,r)\). Define the real-valued function \[u_r(z)=\frac{f^*\rho\qty(z)}{\lambda_r(z)}\qfor z\in D(0,r),\] which is nonnegative and continuous on \(D(0,r)\). The pullback metric \(f^*\rho=\qty(\rho\circ f)\abs{f'}\) is continuous on \(\mathbb{D}\) and thus bounded on \(\overline{D(0,r)}\) (as a consequence of \cref{thm:continuousfunctionboundedoncompact}). As \(\abs{z}\to r^-\), \(\lambda_r(z)\to\infty\), and hence \(\lim_{\abs{z}\to r^{-}}u_r(z)=0\). Thus, \[M_r=\max_{z\in\overline{D(0,r)}}u_r(z)\] must be attained at some \(z=\tau_r\in D(0,r)\) (within the interior).

    If \(M_r=0\), then \(\forall z\in D(0,r)\),
    \(\frac{f^*\rho(z)}{\lambda_r(z)}=0\Rightarrow M_r=0\leq 1\) by maximality. On
    the contrary, if \(M_r>0\), \(f^*\rho\) has well-defined Gaussian curvature at
    \(\tau_r\). Since
    \begin{align*}
        \qty(\laplacian{\log(u_r)})\qty(\tau_r) & =\qty(\laplacian{\log\qty(f^*\rho)})\qty(\tau_r)-\qty(\laplacian{\log\qty(\lambda_r)})\qty(\tau_r) \\
        & =-K_{f^*\rho}\qty(\tau_r)f^*\rho\qty(\tau_r)^2+K_{\lambda_r}\qty(\tau_r)\lambda_r\qty(\tau_r)^2    \\
        & =-K_{f^*\rho}(\tau_r)f^*\rho\qty(\tau_r)^2-\lambda_r(\tau_r)^2.
    \end{align*}
    By assumption, we have \(-K_{f^*\rho}\qty(\tau_r)\geq 1\). Hence, \(\qty(\laplacian{\log(u_r)})\qty(\tau_r)\geq f^*\rho\qty(\tau_r)^2-\lambda_r\qty(\tau_r)^2\). Since \(\log\) is increasing in \(\mathbb{R}\), \(\tau_r\) is a local maximum of \(\log\circ u_r\) and hence \(\qty(\laplacian{\log(u_r)})\qty(\tau_r)\leq 0\). Thus, we have \[f^*\rho\qty(\tau_r)^2-\lambda_r\qty(\tau_r)^2\leq 0\Longleftrightarrow M_r\leq 1.\] Now let \(r\to 1^-\), and it follows that
    \(M_r\to\sup_{z\in\mathbb{D}}\frac{f^*\rho(z)}{\lambda(z)}\leq 1\).
\end{proof}
The result above indeed generalizes the Schwarz--Pick Theorem when \(\rho\) is chosen to be \(\lambda\) and \(f\) is chosen such that \(f(\mathbb{D})\subseteq\mathbb{D}\).

For the purpose of the proceeding generalization, we define the conformal
metric
\begin{equation}
    \lambda_r^\alpha(z)=\frac{1}{\sqrt{\alpha}}\qty(z\mapsto\frac{z}{r})^*\lambda(z)=\frac{2r}{\sqrt{\alpha}\qty(r^2-\abs{z}^2)},\qquad r>0,z\in D(0,r).\label{eq:poincaremetricscaledcurvature}
\end{equation}
Its Gaussian curvature is
\begin{align*}
    K_{\lambda_r^\alpha}(z) & =-4\pdv[2]{}{\overline{z}}{z}\qty[\log(\frac{2r}{\sqrt{\alpha}\qty(r^2-\abs{z}^2)})]\qty(\frac{\sqrt{\alpha}\qty(r^2-\abs{z}^2)}{2r})^2  \\
    & =-4\alpha\pdv[2]{}{\overline{z}}{z}\qty[\log(\frac{2r}{r^2-\abs{z}^2})]\qty(\frac{r^2-\abs{z}^2}{2r})^2=\alpha K_{\lambda_r}(z)=-\alpha,
\end{align*} via the results and definitions in \cref{eq:poincaremetriconscaleddisks}.
\begin{corollary}\label{cor:generalizedahlfors}
    Let \(r>0\) and suppose \(f:D(0,r)\to U\) is holomorphic, where \(U\subseteq\mathbb{C}\) is a region. For any \(\beta>0\), define \(\rho\) to be a conformal metric on \(U\) with \(\dd{s}^2_\rho=\rho^2(z)\abs{\ddz}^2\) such that \[K_\rho\qty(z)\leq-\beta,\qquad\forall z\in U.\] Then \(\forall\alpha>0\), \[f^*\rho(z)\leq\sqrt{\frac{\alpha}{\beta}}\lambda_r^\alpha(z)\] for any \(z\in D(0,r)\), where \(f^*\rho(z)=(\rho\circ f)\abs{f'}\) is the
    metric pullback.
\end{corollary}
\begin{proof}
    Consider the \(\qty(z\mapsto zr)^*f^*\qty(\rho\sqrt{\beta})\), a conformal metric pullback of \(\rho\sqrt{\beta}\) to \(\mathbb{D}\), which satisfies \[K_{\qty(z\mapsto zr)^*f^*\qty(\rho\sqrt{\beta})}\leq -1.\] By Schwarz--Ahlfors--Pick (\cref{thm:schwarzahlforspick}), we have \[\qty(z\mapsto zr)^*f^*\qty(\rho\sqrt{\beta})(z)\leq\lambda(z)=\sqrt{\alpha}\qty(z\mapsto rz)^*\lambda_r^\alpha(z)\qfor z\in\mathbb{D}.\]
    Since \(r\neq 0\), this implies that \[f^*\qty(\rho\sqrt{\beta})(z)\leq\sqrt{\alpha}\lambda_r^\alpha(z)\qfor z\in D(0,r).\] Since \(\sqrt{\beta}\) is a constant, \[\sqrt{\beta}f^*\rho(z)\leq\sqrt{\alpha}\lambda_r^\alpha(z),\qquad\forall z\in D(0,r).\qedhere\]
\end{proof}
\begin{corollary}[\textsc{Generalized Liouville}]\label{cor:generalizedliouville}
    If \(f:\mathbb{C}\to U\) is entire and \(U\) admits a conformal metric of curvature bounded above by a negative constant, then \(f\) must be constant.
\end{corollary}
\begin{proof}
    By assumption, \(\exists\beta>0\) such that \(\sup_{z\in U}\qty(K_\rho(z))\leq-\beta\). Then \cref{cor:generalizedahlfors} gives that \[f^*\rho(z)\leq\frac{1}{\sqrt{\beta}}\lambda_r(z)\qquad\forall z\in D(0,r)\] for any \(r>0\). As \(r\to\infty\), \(\lambda_r\to 0\). Hence,
    \(f^*\rho(z)=0\), implying that \(\qty(\rho\circ f)(z)\abs{f'}=0\). Hence,
    \(f\) is constant.
\end{proof}
\begin{remark}
    \Cref{cor:generalizedliouville} implies Liouville's Theorem (\cref{thm:liouville}). To justify this differential-geometric generalization, suppose \(f:\mathbb{C}\to U\) is entire such that \(U\) is bounded. There then exists some \(R>0\) such that \(U\subseteq D(0,R)\). The metric \(\lambda_R\) has constant negative curvature \(K=-1\) on \(D(0,R)\), and hence, under \(\beta=1\), \cref{cor:generalizedliouville} implies that \(f\) is constant.
\end{remark}
It is understood that a entire function is guaranteed to be constant if it is bounded. This is a statement of sufficiency, but it begs the question of the capacity for possible generalization of boundedness under which constancy is still always satisfied.

Consider an entire function \(f:\mathbb{C}\to U\), where \(U\) is an unbounded
region such that \(\mathbb{C}\setminus U\) has positive area. Fix
\(\zeta\in\operatorname{int}\qty(\mathbb{C}\setminus U)\). Then the map
\(z\mapsto\frac{1}{z-\zeta}\) maps \(U\) to a bounded region and hence
\(z\mapsto\frac{1}{f(z)-\zeta}\) is constant by Liouville's Theorem
(\cref{thm:liouville}), implying the constancy of \(f\) (the essential proof of
\cref{thm:casoratiweierstrassentire}).

In contrast, if \(f:\mathbb{C}\to U\) is entire and \(\mathbb{C}\setminus U\) has zero area (one readily considers sets consisting of curves or isolated points), we must be more specific in determining sufficient conditions that still imply constancy of \(f\).

Similar to in the proof of the Riemann Mapping Theorem (\cref{thm:riemannmapping}), one may use holomorphic square roots or other transformations to reduce to the bounded setting.
\begin{example}
    If \(f:\mathbb{C}\to\mathbb{C}\setminus\cbraces{x\in\mathbb{R}}{0\leq x\leq 1}\) is entire, then \(f\) must be constant.
\end{example}
\begin{proof}
    Consider the biholomorphism \(\varphi(z)=\frac{1}{z}\), mapping \(\mathbb{C}\setminus\cbraces{x\in\mathbb{R}}{0\leq x\leq 1}\) to \(\mathbb{C}^*\setminus\mathbb{R}_{\geq 1}\). By simple connectivity of \(\mathbb{C}\setminus\mathbb{R}_{\geq 1}\), there exists a univalent branch \(\psi\) of \(z\mapsto\sqrt{z-1}\) on \(\mathbb{C}\setminus\mathbb{R}_{\geq 1}\). Now omitting the origin, it is trivially realized that \(\psi\qty(\mathbb{C}^*\setminus\mathbb{R}_{\geq 1})\cap-\psi\qty(\mathbb{C}^*\setminus\mathbb{R}_{\geq 1})=\varnothing\). If otherwise, then \(\exists\xi\in\psi\qty(\mathbb{C}^*\setminus\mathbb{R}_{\geq 1})\) such that \(-\xi\in\psi\qty(\mathbb{C}^*\setminus\mathbb{R}_{\geq 1})\), implying that \(\exists z_1,z_2\in\mathbb{C}^*\setminus\mathbb{R}_{\geq 1}\) such that \(\phi\qty(z_1)=\xi\) and \(\phi\qty(z_2)=-\xi\), implying that \(z_1=z_2\) and \(\xi=0\Rightarrow z_1=z_2=1\), which does not lie in \(\psi\qty(\mathbb{C}^*\setminus\mathbb{R}_{\geq 1})\).

    Now fix \(\xi\in\psi\qty(\mathbb{C}^*\setminus\mathbb{R}_{\geq 1})\). By the Open Mapping Theorem (\cref{thm:openmapping}), \(\exists\varepsilon>0\) such that \(D(\xi,\varepsilon)\subseteq\psi\qty(\mathbb{C}^*\setminus\mathbb{R}_{\geq 1})\). Consequently, \(D(-\xi,\varepsilon)\cap\psi\qty(\mathbb{C}^*\setminus\mathbb{R}_{\geq 1})=\varnothing\). Lastly, the function \(\phi(z)=\frac{\varepsilon}{z+\xi}\) maps \(\psi\qty(\mathbb{C}^*\setminus\mathbb{R}_{\geq 1})\) to \(\mathbb{D}\).
    By Liouville (\cref{thm:liouville}), \(\phi\circ\psi\circ\varphi\circ f\) is constant, which implies \(f\) is constant by the injectivity of \(\phi\),
    \(\psi\), and \(\varphi\).
\end{proof}
The preceding examples show that if the omitted set is sufficiently ``large'' (in the sense of having positive area or disconnecting the plane in certain ways), then any entire function avoiding it must reduce to a constant. However, there are natural limits to the smallness of the omitted set. For instance, the exponential function \(\exp\) is an entire nonconstant function whose image is \(\mathbb{C}^*\), omitting only a single point. Thus, the property that \emph{an entire function omits a set} is not by itself sufficient to guarantee constancy unless that set is suitably substantial. This observation is formalized by Picard's Little Theorem (\cref{thm:littlepicard}), which as preluded to before, asserts that any nonconstant entire function can omit at most one complex value.
\begin{proposition}\label{prop:conformalmetricnegativecurvatureexistencewhenomits2points}
    Let \(U\subset\mathbb{C}\) be an open set such that \(\mathbb{C}\setminus U\) contains at least two points. Then \(U\) admits a conformal metric \(\rho\in C^2(U)\), \(\dd{s}_\rho^2=\rho^2(z)\abs{\ddz}^2\) such that \[K_\rho(z)\leq -\beta<0\qquad\forall z\in U\] for some \(\beta>0\).
\end{proposition}
\begin{proof}
    Without loss of generality, we may assume that \(\cbraces{0,1}\subseteq\mathbb{C}\setminus U\) (if not, a linear transformation \(z\mapsto\frac{z-\xi_1}{\xi_2-\xi_1}\) where \(\xi_1,\xi_2\in\mathbb{C}\setminus U\) are distinct will suffice to transform \(U\) to such a region).

    Define a conformal metric with
    \begin{equation}
        \rho(z)=\frac{\sqrt{1+\abs{z}^{\frac{1}{3}}}\sqrt{1+\abs{z-1}^{\frac{1}{3}}}}{\abs{z}^{\frac{5}{6}}\abs{z-1}^{\frac{5}{6}}},\qquad\dd{s}_{\rho}^2=\rho^2(z)\abs{\dd{z}}^2\label{eq:conformalmetricnegativecurvatureexistencewhenomits2points_metric}
    \end{equation} on \(\mathbb{C}\setminus\cbraces{0,1}\).

    Since
    \(\laplacian(\log\abs{z}^{\frac{5}{6}})=\frac{5}{6}\laplacian(\log\abs{z})=\frac5{6}\laplacian(\Re\log(z))\),
    \begin{align*}
        \laplacian{\log(\frac{\sqrt{1+\abs{z}^{\frac13}}}{\abs{z}^{\frac56}})} & =2\pdv{\overline{z}}\pdv{z}(\log\qty(1+\abs{z}^{\frac13}))=\frac{z^{-\frac{5}{6}}}{3}\pdv{\overline{z}}(\frac{\overline{z}^{\frac{1}{6}}}{1+\abs{z}^{\frac{1}{3}}})                                                                                                                   \\
        & =\frac{z^{-\frac{5}{6}}}{3}\pdv{\overline{z}}(\frac{\overline{z}^{\frac{1}{6}}}{1+\abs{z}^{\frac{1}{3}}})=\frac{z^{-\frac{5}{6}}\overline{z}^{-\frac56}\qty(1+\abs{z}^{\frac13})-z^{-\frac56}\overline{z}^{\frac16}z^{\frac16}\overline{z}^{-\frac56}}{18\qty(1+\abs{z}^{\frac13})^2} \\
        & =\frac{1}{18\abs{z}^{\frac53}\qty(1+\abs{z}^{\frac13})^2},
    \end{align*} and a similar calculation yields \[\laplacian{\log(\frac{\sqrt{1+\abs{z-1}^{\frac13}}}{\abs{z-1}^{\frac56}})}=\frac{1}{18\abs{z-1}^{\frac53}\qty(1+\abs{z-1}^{\frac13})^2}.\]
    Hence, \[K_\rho(z)=-\frac{1}{18}\qty[\frac{\abs{z-1}^\frac53}{\qty(1+\abs{z}^{\frac13})^3\qty(1+\abs{z-1}^{\frac13})}+\frac{\abs{z}^{\frac53}}{\qty(1+\abs{z-1}^{\frac13})^3\qty(1+\abs{z}^{\frac13})}],\]
    and that
    \begin{enumerate}
        \item \(K_\rho\in C^0\qty(\mathbb{C}\setminus\cbraces{0,1})\).
        \item \(\forall z\in\mathbb{C}\setminus\cbraces{0,1}\), \(K_\rho(z)<0\).
        \item \(\lim_{z\to 0}K_\rho(z)=-\frac1{36}\).
        \item \(\lim_{z\to 1}K_\rho(z)=-\frac1{36}\).
        \item \(\lim_{z\to\infty}K_\rho(z)=-\infty\) in any direction (as in the one-point compactification).
    \end{enumerate}
    Hence, \(\exists\delta>0\) such that \(\abs{K_\rho(z)+\frac{1}{36}}<\frac{1}{72}\) for any \(z\in D^*(0,\delta)\cup D^*(1,\delta)\) and \(\exists R>0\) such that \(K_\rho(z)<-1\) for any \(z\) satisfying \(\abs{z}>R\). By compactness of \(\overline{D(0,R)}\setminus\qty(D(0,\delta)\cup D(1,\delta))\) and continuity, it attains its supremum of some value \(-M<0\) by \cref{thm:extremevalue}. Let \(-\beta=\max\cbraces{-\frac{1}{72},-M}<0\).
    \[\therefore\quad K_\rho(z)\leq-\beta<0\quad\forall z\in\mathbb{C}\setminus\cbraces{0,1}.\qedhere\]
\end{proof}
And we have the final implication:
\begin{theorem}[name=\textsc{Picard's Little Theorem},store=thm:littlepicard]\label{thm:littlepicard}
    Let \(f:\mathbb{C}\to U\) be entire such that \(\mathbb{C}\setminus U\) contains two or more points. Then \(f\) is constant.
\end{theorem}
\begin{proof}
    By the result of \cref{prop:conformalmetricnegativecurvatureexistencewhenomits2points}, we may find a conformal metric \(\rho\) on \(U\) such that \(\exists\beta>0\) satisfying \(K_\rho(U)\subseteq\mathbb{R}_{\leq-\beta}\). Then by the aforementioned generalization of Liouville (\cref{cor:generalizedliouville}), \(f\) exhibits constancy on \(\mathbb{C}\) and the assertion follows.
\end{proof}
\begin{remark}
    This is commonly stated in its contrapositive: the image of any non-constant entire function omits at most one value.
\end{remark}