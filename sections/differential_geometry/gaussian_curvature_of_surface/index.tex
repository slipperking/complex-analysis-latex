\subsection{Gaussian Curvature of a Surface}\label{sec:gaussiancurvatureofsurface}
We will give a brief introduction to the curvature of a surface for heuristic intuition.

Suppose \(U\subseteq\mathbb{R}^2\) is a region, and let \(\qty(u,v)\in U\). Consider a surface parameterized via \[\va{r}(u,v)=\qty(x(u,v),y(u,v),z(u,v))\in\mathbb{R}^3,\] where \(x,y,z\in C^2\qty(U)\). If \(\va{r}'_u\times\va{r}'_v\) never vanishes for \(\qty(u,v)\in U\), then \(\va{r}(U)\) defines a smooth surface \(\Sigma\). For a fixed \(\qty(u,v)\in U\), the vectors \(\va{r}'_u\) and \(\va{r}'_v\) form the basis of the tangent space (a plane) of \(\Sigma\) at \(P=\va{r}\qty(u,v)\), denoted by \(T_P\Sigma=\mathrm{span}\qty(\va{r}'_u(P),\va{r}'_v(P))\).

The square of the length of the vector infinitesimal \(\dd{\va{r}}=\va{r}'_u\dd{u}+\va{r}'_v\dd{v}\), or
\begin{equation}
    \mathrm{I}(u,v)=\dd{s}^2=E\dd{u}^2+2F\dd{u}\dd{v}+G\dd{v}^2,\label{eq:firstfundamentalform}
\end{equation} is known as the \textit{first fundamental form} of \(\Sigma\), where \(E=\va{r}'_u\cdot\va{r}'_u\), \(F=\va{r}'_u\cdot\va{r}'_v\), and \(G=\va{r}'_v\cdot\va{r}'_v\).

Let \(Q=\va{r}\qty(u+\Delta u,v+\Delta v)\) be near \(P\). It follows that \(\overrightarrow{PQ}=\va{r}\qty(u+\Delta u,v+\Delta v)-\va{r}\qty(u,v)\). The distance between \(Q\) and \(T_P\Sigma\) is \(\overrightarrow{PQ}\cdot\vu{n}\), where \(\vu{n}=\frac{\va{r}'_{u}\times\va{r}'_v}{\norm{\va{r}'_{u}\times\va{r}'_v}}\). By application of the multivariate Taylor's Theorem, we have
\begin{align*}
    \overrightarrow{PQ} & =\va{r}'_u\Delta u+\va{r}'_v\Delta v+\frac{1}{2}\qty(\va{r}''_{uu}\Delta u^2+2\va{r}''_{uv}\Delta u\Delta v+\va{r}''_{vv}\Delta v^2)+\order{\Delta u^3+\Delta v^3},
\end{align*}
and therefore, \[\overrightarrow{PQ}\cdot\vu{n}=\frac{1}{2}\qty(\va{r}''_{uu}\cdot\vu{n}\Delta u^2+2\va{r}''_{uv}\cdot\vu{n}\Delta u\Delta v+\va{r}''_{vv}\cdot\vu{n}\Delta v^2)+\order{3}\cdot\vu{n}.\] The first two linear terms vanish by properties of the triple scalar product. The \textit{second fundamental form} of \(\Sigma\) is defined as
\begin{equation}
    \mathrm{I\!I}(u,v)=L\dd{u}^2+2M\dd{u}\dd{v}+N\dd{v}^2,\label{eq:secondfundamentalform}
\end{equation} where \(L=\va{r}''_{uu}\cdot\vu{n}\), \(M=\va{r}''_{uv}\cdot\vu{n}\), and \(N=\va{r}''_{vv}\cdot\vu{n}\). Since \(\va{r}'_u\cdot\vu{n}=0\) and \(\va{r}'_v\cdot\vu{n}=0\), by differentiation, we have
\begin{align*}
    \va{r}''_{uu}\cdot\vu{n}+\va{r}'_u\cdot\vu{n}'_u & =0, & \va{r}''_{uv}\cdot\vu{n}+\va{r}'_u\cdot\vu{n}'_v & =0, \\
    \va{r}''_{uv}\cdot\vu{n}+\va{r}'_v\cdot\vu{n}'_u & =0, & \va{r}''_{vv}\cdot\vu{n}+\va{r}'_v\cdot\vu{n}'_v & =0.
\end{align*}
It follows that \(L=-\va{r}'_u\cdot\vu{n}'_u\), \(M=-\va{r}'_u\cdot\vu{n}'_v=-\va{r}'_v\cdot\vu{n}'_u\), and \(N=-\va{r}'_v\cdot\vu{n}'_v\). Because \(\dd{\vu{n}}=\vu{n}'_u\dd{u}+\vu{n}'_v\dd{v}\), \[\mathrm{I\!I}=-\dd{\va{r}}\cdot\dd{\vu{n}}.\]
\begin{figure}
    \centerline{\includesvg[width=0.75\linewidth]{build/svg/second_fundamental_form.svg}}
    \caption{\(Q\) has a greater heuristic distance to \(T_P\Sigma\) for a more curved surface.}\label{fig:secondfundamentalform}
\end{figure}The second fundamental form, in a rough sense, measures the curvature of the surface \(\Sigma\) at \(P\) (refer to \cref{fig:secondfundamentalform}). Both the first and second fundamental forms are geometric invariants; they are independent of the parameterization \(\va{r}\) of \(\Sigma\). The first fundamental form is also referred to as the \textit{intrinsic metric} (we will not delve into the metric tensor here) of \(\Sigma\), and the second fundamental form is an \textit{extrinsic} property of \(\Sigma\) as it is invariant up to the orientation of the surface (consequent direction of the normal vector).

Let \(\gamma\subset\Sigma\) be a curve parameterized by arc length, \(\va{r}(s)=\va{r}\qty(u(s),v(s))\). Then the unit tangent vector at \(P=\va{r}(s)\) is \[\va{T}(s)=\dv{\va{r}}{s}=\va{r}'_u\dv{u}{s}+\va{r}'_v\dv{v}{s}.\]
Consequently, \[\va{T}'(s)=\va{r}''_{uu}\qty(\dv{u}{s})^2+2\va{r}''_{uv}\qty(\dv{u}{s})\qty(\dv{v}{s})+\va{r}''_{vv}\qty(\dv{v}{s})^2+\va{r}'_u\dv[2]{u}{s}+\va{r}'_v\dv[2]{v}{s},\]
where the last two terms are in \(T_P\Sigma\). Because \(\norm{\va{T}(s)}=1\) for all \(s\) by the arc-length parameterization, we have \[0=\dv{s}(\norm{\va{T}(s)}^2)=\dv{s}(\va{T}(s)\cdot\va{T}(s))=2\va{T}(s)\cdot\va{T}'(s).\]
Hence, \(\va{T}(s)\) and \(\va{T}'(s)\) are orthogonal and \(\va{T}'(s)\) is a normal to the curve \(\gamma\). Let \(\vu{n}=\tfrac{\va{r}'_{u}\times\va{r}'_v}{\norm{\va{r}'_{u}\times\va{r}'_v}}\) be the unit normal to \(\Sigma\) at \(P\). The \textit{normal curvature} of \(\gamma\) at \(P\) in \(\Sigma\) is defined as \[\kappa_n=\va{T}'(s)\cdot\vu{n}=\qty[\va{r}''_{uu}\qty(\dv{u}{s})^2+2\va{r}''_{uv}\qty(\dv{u}{s})\qty(\dv{v}{s})+\va{r}''_{vv}\qty(\dv{v}{s})^2]\cdot\vu{n}.\]
The quotient \[\kappa_n=\frac{\mathrm{I\!I}}{\mathrm{I}}=\frac{L\dd{u}^2+2M\dd{u}\dd{v}+N\dd{v}^2}{E\dd{u}^2+2F\dd{u}\dd{v}+G\dd{v}^2},\] varies depending on the curve traversing \(\Sigma\) (and ultimately, depending on the direction induced by \(\dd{u}\) and \(\dd{v}\)). On \(\gamma\), the two representations are equivalent since \(\mathrm{I}=\dd{s}^2\). The maximum and minimum values of \(\kappa_n\) are known as the \textit{principal curvatures} \(\kappa_1\) and \(\kappa_2\) of \(\Sigma\) at \(P\), achieved along the \textit{principal directions} of the (unit) tangent vectors at \(P\).

The \textit{mean curvature} of \(\Sigma\) at \(P\) is defined to be \(H=\tfrac{\kappa_1+\kappa_2}{2}\). Let \(r_1,r_2\) be the radii of curvature corresponding to \(\kappa_1\) and \(\kappa_2\). The product of the two principal curvatures is known as the \textit{Gaussian curvature} of \(\Sigma\) at \(P\), denoted by \(K=\kappa_1\kappa_2\). We will now heuristically derive the explicit formulas for \(H\) and \(K\) in terms of \(E,F,G,L,M,N\).

Suppose \(p\in\Sigma\). Adopt the matrix notation of \(\vb{I}\), \(\vb{I\!I}\) as in \[\vb{I}=\mqty(E&F\\F&G),\qquad\vb{I\!I}=\mqty(L&M\\M&N),\] to reduce to the optimization problem of \[\kappa_n=\frac{\va{v}^\top\vb{I\!I}\va{v}}{\va{v}^\top\vb{I}\va{v}},\qquad\va{v}\in T_p\Sigma.\]
We may restrict \(\va{v}=\qty(v_1,v_2)\) so that the denominator is always \(1\), aiming to optimize the numerator. By the method of Lagrange multipliers, we write \[\mathcal{L}\qty(\va{v},\lambda)=\va{v}^\top\vb{I\!I}\va{v}-\lambda\qty(\va{v}^\top\vb{I}\va{v}-1).\]
The equation \(\grad\mathcal{L}=\vb{0}\) for \(\textstyle\grad{}=\qty(\pdv{v_1},\pdv{v_2},\pdv{\lambda})\) can then be decomposed into (where \(\va{v}=\qty(v_1,v_2)\)):
\begin{gather*}
    2Lv_1+2Mv_2-\lambda\qty(2Ev_1+2Fv_2)=0,\\
    2Mv_1+2Nv_2-\lambda\qty(2Fv_1+2Gv_2)=0,\\
    \qty(\va{v}^\top\vb{I}\va{v}=1).
\end{gather*}
The first two equations can be written as
\begin{equation}
    \mqty(L-\lambda E&M-\lambda F\\M-\lambda F&N-\lambda G)\va{v}=\vb{0}.\label{eq:gaussiancurvaturelambdarootsmatrixvectorvanish}
\end{equation}
Let the matrix on the left be denoted by \(\vb{M}\). In order for non-trivial (\(\vb{v}\neq\vb{0}\)) to exist, we must have \(\det\vb{M}=0\). That is,
\[(L-\lambda E)(N-\lambda G)-(M-\lambda F)^2=\lambda^2\qty(EG-F^2)+\lambda(2MF-EN-GL)+LN-M^2=0.\]
This is a quadratic giving two solutions for \(\lambda\). From \[\grad\qty(\va{v}^\top\vb{I\!I}\va{v})=\lambda\grad\qty(\va{v}^\top\vb{I}\va{v})\] it is apparent that the roots \(\lambda_1,\lambda_2\in\mathbb{R}\). Moreover, from \cref{eq:gaussiancurvaturelambdarootsmatrixvectorvanish} we have \[\vb{I\!I}\va{v}=\lambda\vb{I}\va{v}\implies\lambda=\frac{\va{v}^\top\vb{I\!I}\va{v}}{\va{v}^\top\vb{I}\va{v}}.\]
Hence, the two roots \(\lambda_1,\lambda_2\) are precisely the principal curvatures. Vieta's formulas give that \[K=\lambda_1\lambda_2=\frac{LN-M^2}{EG-F^2},\quad H=\frac{\lambda_1+\lambda_2}2=\frac{EN+GL-2MF}{2EG-2F^2}.\]
Now, assume a parameterization of \(\Sigma\) by \(\va{r}\qty(u,v)\) (thrice continuously differentiable) such that \[\mathrm{I}(u,v)=\rho^2\dd{u}^2+\rho^2\dd{v}^2=\rho^2\qty(\dd{u}^2+\dd{v}^2)\] (which we will later formalize as a conformal metric). Then there is an alternate representation of the Gaussian curvature in terms of \(\rho\).

By definition, \(E\equiv G\equiv\rho^2\) while \(F\equiv 0\). Moreover,
\begin{align*}
    LN&=\qty(\va{r}''_{uu}\cdot\frac{\va{r}'_u\times\va{r}'_v}{\norm{\va{r}'_u\times\va{r}'_v}})\qty(\va{r}''_{vv}\cdot\frac{\va{r}'_u\times\va{r}'_v}{\norm{\va{r}'_u\times\va{r}'_v}})=\frac{\det\mqty(\va{r}''_{uu}&\va{r}'_u&\va{r}'_v)\det\mqty(\va{r}''_{vv}&\va{r}'_u&\va{r}'_v)}{\norm{\va{r}'_u}^2\norm{\va{r}'_v}^2-\qty(\va{r}'_u\cdot\va{r}'_v)^2}\\
    &=\frac{\det\mqty(\va{r}''_{uu}&\va{r}'_u&\va{r}'_v)\det\mqty(\va{r}''_{vv}&\va{r}'_u&\va{r}'_v)}{EG-F^2}=\frac1{\rho^4}\det\mqty(\va{r}''_{vv}\cdot\va{r}''_{uu}&\va{r}''_{vv}\cdot\va{r}'_u&\va{r}''_{vv}\cdot\va{r}'_v\\\va{r}'_u\cdot\va{r}''_{uu}&\va{r}'_u\cdot\va{r}'_u&\va{r}'_u\cdot\va{r}'_v\\\va{r}'_v\cdot\va{r}''_{uu}&\va{r}'_v\cdot\va{r}'_u&\va{r}'_v\cdot\va{r}'_v).
\end{align*}
Similarly,
\begin{align*}
    M^2=\frac{1}{\rho^4}\det\mqty(\va{r}''_{uv}\cdot\va{r}''_{uv}&\va{r}''_{uv}\cdot\va{r}'_u&\va{r}''_{uv}\cdot\va{r}'_v\\\va{r}'_{u}\cdot\va{r}''_{uv}&\va{r}'_{u}\cdot\va{r}'_u&\va{r}'_{u}\cdot\va{r}'_v\\\va{r}'_{v}\cdot\va{r}''_{uv}&\va{r}'_{v}\cdot\va{r}'_u&\va{r}'_{v}\cdot\va{r}'_v).
\end{align*}
By differentiation of the equations \[\va{r}'_u\cdot\va{r}'_v\equiv F\equiv0,\qquad\va{r}'_u\cdot\va{r}'_u\equiv E\equiv G\equiv \va{r}'_v\cdot\va{r}'_v\equiv \rho^2,\]
we have
\begin{equation}
    \va{r}''_{uu}\cdot\va{r}'_v+\va{r}'_u\cdot\va{r}''_{uv}\equiv0,\qquad\va{r}''_{uv}\cdot\va{r}'_v+\va{r}'_u\cdot\va{r}''_{vv}\equiv0,\label{eq:gaussiancurvatureofsurface_conformalzerodifferentiation}
\end{equation}
and
\begin{equation}
    2\va{r}''_{uu}\cdot\va{r}'_u\equiv2\rho\rho'_u\equiv 2\va{r}''_{uv}\cdot\va{r}'_v,\qquad2\va{r}''_{uv}\cdot\va{r}'_u\equiv2\rho\rho'_v\equiv 2\va{r}''_{vv}\cdot\va{r}'_v.\label{eq:gaussiancurvatureofsurface_conformalnonzerodifferentiation}
\end{equation}
Substituting \cref{eq:gaussiancurvatureofsurface_conformalnonzerodifferentiation} into \cref{eq:gaussiancurvatureofsurface_conformalzerodifferentiation} then gives \[\va{r}''_{uu}\cdot\va{r}'_v=-\rho\rho'_v,\qquad\qty(\va{r}''_{vv}\cdot\va{r}'_u=-\rho\rho'_u).\]
Differentiating these give \[\va{r}'''_{uuv}\cdot\va{r}'_v+\va{r}''_{uu}\cdot\va{r}_{vv}=-{\rho'_v}^2-\rho\rho''_{vv},\qquad\qty(\va{r}'''_{vvu}\cdot\va{r}'_u+\va{r}''_{vv}\cdot\va{r}_{uu}=-{\rho'_u}^2-\rho\rho''_{uu}).\]
Differentiating the inner two expressions of \cref{eq:gaussiancurvatureofsurface_conformalnonzerodifferentiation}, we have \[\va{r}'''_{uuv}\cdot\va{r}'_v+\va{r}''_{uv}\cdot\va{r}_{vu}={\rho'_u}^2+\rho\rho''_{uu},\qquad\qty(\va{r}'''_{vvu}\cdot\va{r}'_u+\va{r}''_{uv}\cdot\va{r}_{uv}={\rho'_v}^2+\rho\rho''_{vv}).\]
It follows that \[\va{r}''_{uv}\cdot\va{r}_{uv}-\va{r}''_{uu}\cdot\va{r}''_{vv}={\rho'_u}^2+{\rho'_v}^2+\rho\laplacian\rho,\] where \(\laplacian\) here is \(\pdv*[2]{}{u}+\pdv*[2]{}{v}\). Then
\begin{align*}
    LN&=\frac1{\rho^4}\det\mqty(\va{r}''_{vv}\cdot\va{r}''_{uu}&-\rho\rho'_u&\rho\rho'_v\\\rho\rho'_u&\rho^2&0\\-\rho\rho'_v&0&\rho^2)=\frac1{\rho^4}\qty[\va{r}''_{vv}\cdot\va{r}''_{uu}\rho^4+\rho^4{\rho'_u}^2+\rho^4{\rho'_v}^2]\\
    &=\va{r}''_{vv}\cdot\va{r}''_{uu}+{\rho'_u}^2+{\rho'_v}^2,
\end{align*}
and
\begin{align*}
    M^2&=\frac1{\rho^4}\det\mqty(\va{r}''_{uv}\cdot\va{r}''_{uv}&\rho\rho'_v&\rho\rho'_u\\\rho\rho'_v&\rho^2&0\\\rho\rho'_u&0&\rho^2)=\frac1{\rho^4}\qty[\va{r}''_{uv}\cdot\va{r}''_{uv}\rho^4-\rho^4{\rho'_v}^2-\rho^4{\rho'_u}^2]\\
    &=\va{r}''_{uv}\cdot\va{r}''_{uv}-{\rho'_v}^2-{\rho'_u}^2.
\end{align*}
Combining the two expressions, we have
\begin{align}
    K&=\frac{LN-M^2}{EG-F^2}=\frac{\va{r}''_{vv}\cdot\va{r}''_{uu}+2{\rho'_u}^2+2{\rho'_v}^2-\va{r}''_{uv}\cdot\va{r}''_{uv}}{\rho^4}\nonumber\\
    &=\frac{{\rho'_u}^2+{\rho'_v}^2-\rho\laplacian\rho}{\rho^4}=-\frac{1}{\rho^2}\laplacian(\log\rho).\label{eq:gaussiancurvatureofsurface_gaussiancurvatureconformalmetricformula}
\end{align}
