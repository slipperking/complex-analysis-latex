\subsection{A Spherical Generalization of Normal Families}\label{sec:sphericalgeneralizationofnormalfamilies}
Picard's Great Theorem requires a more profound concept by generalizing normal families in the one-point compactification of \(\mathbb{C}\).
\begin{definition}
    Let \(\cbraces{f_n(z)}\) be a (not necessarily analytic) complex function sequence on a connected set \(\Omega\subseteq\mathbb{C}\). If \(\forall K\subset\Omega\) compact, \(\forall R>0\), \(\exists N\in\mathbb{N}\) such that \(\forall n>N\), \(\forall z\in K\), \(\abs{f_n(z)}>R\), then \textit{\(f_n\to\infty\) locally uniformly spherically on \(\Omega\)}.
\end{definition}
When the ``locally uniform limit'' is taken to be \(\infty\), the condition of \(\varepsilon\)-closeness is instead replaced by the requirement that the values eventually leave every fixed compact subset of \(\mathbb{C}\) (the given definition is equivalent to: \(\forall K\subset\Omega\) compact, \(\forall L\subset\mathbb{C}\) compact, \(\exists N\in\mathbb{N}\) such that \(\forall n>N\), \(\forall z\in K\), \(f_n(z)\notin L\)). In this way, convergence to infinity is treated symmetrically with convergence to finite values by working in the Riemann sphere \(\extcomplex\), where \(\infty\) is simply another accumulation point.

By equipping the extended complex plane \(\extcomplex\) with the spherical metric instead of the Euclidean metric, convergence to \(\infty\) can be treated like convergence to any finite point. In this setting, \(\infty\) is simply another accumulation point, so there is no need to handle it differently from other values.

Let \(\cbraces{a_n}_{n\in\mathbb{N}}\subset\extcomplex\) be a sequence. Then we say \(a_n\to a_\infty\) \emph{spherically} iff \(\forall\varepsilon>0\), \(\exists N\in\mathbb{N}\) such that \(\forall n>N\), \(d_\sigma\qty(a_n,a_\infty)<\varepsilon\), where \(d_\sigma\) is the spherical distance.
\begin{definition}
    A family of meromorphic functions \(\mathcal{F}\) on some \(\Omega\subseteq\mathbb{C}\) is said to be \textit{spherically normal} iff every sequence has a locally uniformly spherically convergent subsequence on \(\Omega\).
\end{definition}
Montel's Theorem for holomorphically normal families in \cref{thm:montel} can be generalized via the spherical metric by the statement of Marty's Criterion (\cref{thm:marty}).
\begin{definition}[Spherical Derivative]
    Let \(\Omega\subseteq\mathbb{C}\) be an open region or domain. Suppose \(f:\Omega\to\extcomplex\) is meromorphic. Then the \textit{spherical derivative} of \(f\) is given by \[f^\sharp(z)=f^*\sigma(z)=\frac{2\abs{f'(z)}}{1+\abs{f(z)}^2}\] for \(f(z)\neq\infty\) and \[f^\sharp(z)=\lim_{\zeta\to z}f^\sharp(\zeta)\] otherwise.
\end{definition}
\begin{proposition}\label{prop:linearfractionaltransformationuniformlysphericallycontinuous}
    Any linear fractional transformation is spherically uniformly continuous on \(\mathbb{C}\).
\end{proposition}
\begin{proof}
    Let \(\psi(z)=\frac{az+b}{cz+d}\), where \(ad-bc\neq 0\). Then, \[\psi'(z)=\frac{ad-bc}{(cz+d)^2}.\]
    The spherical distance between two points \(w_1=\psi\qty(z_1),w_2=\psi\qty(z_2)\) is given by \[d_\sigma\qty(w_1,w_2)=\inf_{\gamma}\int_\gamma\psi^\sharp(z)\abs{\ddz}=\inf_{\gamma}\int_\gamma\frac{2\abs{\frac{ad-bc}{(cz+d)^2}}}{1+\abs{\frac{az+b}{cz+d}}^2}\abs{\ddz}\] where \(\gamma\) joins \(z_1\) and \(z_2\). The spherical distance is bounded by the integral over the Euclidean straight line \(\gamma'\) joining \(z_1\) and \(z_2\):
    \[d_\sigma\qty(w_1,w_2)\leq\int_{\gamma'}\frac{2\abs{ad-bc}}{\abs{cz+d}^2+\abs{az+b}^2}\abs{\ddz}.\] Since \(\frac{2\abs{ad-bc}}{\abs{cz+d}^2+\abs{az+b}^2}\to 0\) as \(z\to\infty\) and \(z\mapsto\frac{2\abs{ad-bc}}{\abs{cz+d}^2+\abs{az+b}^2}\in C^0(\mathbb{C})\), it is bounded by some constant \(M\) on \(\mathbb{C}\). Hence, we have \[d_\sigma\qty(w_1,w_2)\leq M\abs{z_1-z_2}.\] Hence, \(\forall\varepsilon>0\), \(\forall\abs{z_1-z_2}<\frac{\varepsilon}{M}\), \[d_\sigma\qty(\psi\qty(z_1),\psi\qty(z_2))<\varepsilon.\qedhere\]
\end{proof}
\begin{proposition}\label{prop:locallyuniformlysphericallyconvergentholomorphicsequenceuniformlimit}
    Let \(\cbraces{f_n}_{n\in\mathbb{N}}\) be a sequence of holomorphic functions on a domain \(\Omega\subseteq\mathbb{C}\). If \(f_n\to f\) locally uniformly spherically, then \(f\) is either holomorphic on \(\Omega\) or identically \(\infty\).
\end{proposition}
\begin{proof}
    A result analogous to \cref{thm:uniformlimit} can be used to show that \(f\) is spherically continuous. Let \(z\in\Omega\) be arbitrary.
    \begin{enumerate}
        \item If \(f(z)\neq\infty\), then by spherical continuity, \(\exists\delta>0\) such
            that \(\forall\zeta\in D(z,\delta)\), \[d_\sigma\qty(f(\zeta),f(z))<\frac12d_\sigma\qty(\infty, f(z)).\] Similarly, \(\exists N\in\mathbb{N}\) such that \(\forall n>N\), \[d_\sigma\qty(f(\zeta),f_{n}(\zeta))<\frac12d_\sigma\qty(\infty, f(z)).\]
            Hence, we have \[d_\sigma\qty(\infty,f(z))-d_\sigma\qty(f(z),f_{n}(\zeta))>0.\] By the reverse triangle inequality, we have \[d_\sigma\qty(\infty,f_{n}(\zeta))>0.\]
            By Weierstrass (\cref{thm:weierstrassconvergence}), \(f\) is holomorphic on
            \(D(z,\delta)\).
        \item Consider \(f(z)=\infty\). Assume, for the sake of contradiction, \(z\) is an
            isolated pole of \(f\). Hence, \(\exists\delta\) such that \(f\) is holomorphic
            on \(D^*(z,\delta)\).

            Because each \(f_n\) is holomorphic on \(D(z,\delta)\), by the Maximum Modulus
            Principle (\cref{thm:maximummodulus}), \(\forall n\in\mathbb{N}\), \[\abs{f_n(\zeta)}\leq\sup_{\xi\in\partial D(z,\delta)}\abs{f_n(\xi)}\qquad\forall\zeta\in D(z,\delta).\]
            By letting \(n\to\infty\), we have \[\abs{f(\zeta)}\leq\sup_{\xi\in\partial D(z,\delta)}\abs{f(\xi)}<\infty\qquad\forall\zeta\in D(z,\delta),\] contradicting the assumption that \(f(z)=\infty\) is an isolated pole. Hence,
            \(z\) must be an accumulation of values evaluating to \(\infty\). By spherical
            continuity, \(\exists\delta>0\) such that \[d_\sigma\qty(f(\zeta),\infty)<\frac\muppi2\qquad\forall\zeta\in D(z,\delta).\] Similarly, \(\exists N\in\mathbb{N}\) such that \(\forall n>N\), \[d_\sigma\qty(f(\zeta),f_n(\zeta))<\frac\muppi2.\]
            Hence, we have \[\muppi-d_\sigma\qty(\infty,f_n(\zeta))=d_\sigma(\infty,0)-d_\sigma\qty(\infty,f_n(\zeta))>0.\] By the reverse triangle inequality, we have \[d_\sigma\qty(0,f_n(\zeta))>0.\]
            Hence each \(\frac{1}{f_n}\) is holomorphic on \(D(z,\delta)\) and converges
            locally uniformly spherically to \(\frac{1}{f}\) on \(D(z,\delta)\). By
            Weierstrass (\cref{thm:weierstrassconvergence}), \(\frac1f\) is holomorphic on
            \(D(z,\delta)\) and has zeros that accumulate at \(z\). By the Identity
            Theorem, \(\frac1f\equiv0\Rightarrow f\equiv\infty\) on \(D(z,\delta)\).
    \end{enumerate}
    Let \(S\) be the set of all \(z\in\Omega\) such that \(f(z)\) is finite. By the argument above, \(S\) is open. The complement \(\Omega\setminus S\) then consists of all points where \(f(z)=\infty\). By the argument above, \(\Omega\setminus S\) is also open. Since \(\Omega\) is connected, by \cref{thm:connectedtopologicalspaceclopensets}, either \(S=\varnothing\) or \(S=\Omega\). In the former case, \(f\equiv\infty\) on \(\Omega\), and in the latter case, \(f\) is holomorphic on \(\Omega\).
\end{proof}
\begin{theorem}[\textsc{Marty's Criterion}]\label{thm:marty}
    A family of meromorphic functions \(\mathcal{F}\) on some \(\Omega\subseteq\mathbb{C}\) is spherically normal iff \[\cbraces{f^\sharp}{f\in\mathcal{F}},\] or the family of spherical derivatives, is locally uniformly bounded in
    \(\Omega\).
\end{theorem}
\begin{proof}
    The condition is equivalent to that of \[\frac{2\abs{f'(z)}}{1+\abs{f(z)}^2}\leq M\qquad\forall f\in\mathcal{F}\] for all compact \(K\subset\Omega\), \(\forall z\in K\), where \(M\) depends
    only on \(K\). Under the assumption that this holds, then \[d_\sigma\qty(f\qty(z_1),f\qty(z_2))=\inf_\gamma\int_{\gamma}\dd{s}_{\sigma}\leq M\abs{z_2-z_1}\qquad\forall f\in\mathcal{F}\] where \(\gamma\) joins \(f\qty(z_1)\) and \(f\qty(z_2)\) where \(z_1,z_2\in
    K\). Hence, \(\forall\varepsilon>0\), \(\forall z_1,z_2\in K\) such that
    \(\abs{z_1-z_2}<\frac{\varepsilon}{M}\),
    \(d_\sigma\qty(f\qty(z_1),f\qty(z_2))<\varepsilon\), and hence \(\mathcal{F}\)
    is \textit{uniformly spherically equicontinuous}. Since \(d_\sigma\leq\muppi\)
    for any two points by geometry of \(S^2\), \(\mathcal{F}\) is also
    \emph{uniformly spherically bounded} (the compactness of \(S^2\)). Then the
    Arzelà--Ascoli Theorem (\cref{thm:arzelaascoli}) under the spherical metric
    gives that \(\mathcal{F}\) is a normal family.

    Conversely, assume for the sake of contradiction that \(\mathcal{F}\) is a
    normal family such that conclusion is not satisfied. Then, \(\exists
    K\subset\Omega\) compact and a sequence
    \(\cbraces{f_n}_{n\in\mathbb{N}}\subseteq\mathcal{F}\) such that the sequence \[\cbraces{\sup_{z\in K}f^\sharp_n(z)}_{n\in\mathbb{N}}\] tends to \(\infty\) (specifically, suppose that \(\forall n\in\mathbb{N}\),
    \(\sup_{z\in K}f_n^\sharp(z)>n\)). By normality, we may extract a locally
    uniformly spherically convergent subsequence
    \(\cbraces{f_{n_k}}_{k\in\mathbb{N}}\subseteq\cbraces{f_n}_{n\in\mathbb{N}}\).
    By \cref{thm:uniformlimit} under the spherical metric, the uniform spherical
    limit of \(\cbraces{f_{n_k}}_{k\in\mathbb{N}}\), \(f\), is spherically
    continuous on \(\Omega\). For every point \(z\in\Omega\), there are two
    possibilities:
    \begin{enumerate}
        \item If \(f(z)\neq\infty\), then by continuity, \(\exists\delta>0\) such that
            \(\forall\zeta\in D(z,\delta)\), \[d_\sigma\qty(f(\zeta),f(z))<\frac12d_\sigma\qty(\infty, f(z)).\] Similarly, \(\exists N\in\mathbb{N}\) such that \(\forall k>N\), \[d_\sigma\qty(f(\zeta),f_{n_k}(\zeta))<\frac12d_\sigma\qty(\infty, f(z)).\]
            Hence, we have \[d_\sigma\qty(\infty,f(z))-d_\sigma\qty(f(z),f_{n_k}(\zeta))>0.\] By the reverse triangle inequality, we have \[d_\sigma\qty(\infty,f_{n_k}(\zeta))>0.\]
            Hence, the meromorphy of each \(f_{n_k}\) is actually holomorphy. By
            continuity, \(f\) is locally uniformly bounded on \(D(z,\delta)\). Hence,
            \(\cbraces{f_{n_k}}_{k>N}\) locally uniformly converges on \(D(z,\delta)\). By
            a result of Weierstrass (\cref{thm:weierstrassconvergence}), \(f\) is
            holomorphic on \(D(z,\delta)\) and the sequence \(\cbraces{f'_{n_k}}_{k>N}\)
            locally uniformly converges to \(f'\) on \(D(z,\delta)\).

            By holomorphy of \(f'\) on \(\overline{D\qty(z,\frac\delta2)}\), \(\exists
            M'>0\) such that
            \(\sup_{\zeta\in\overline{D\qty(0,\frac\delta2)}}\abs{f'(\zeta)}<M'\). Uniform
            convergence of \(\cbraces{f'_{n_k}}_{k>N}\) gives the existence of some
            \(N'>N\) such that \(\forall k>N'\), \[\abs{f'_{n_k}(\zeta)-f'(\zeta)}<1\implies\abs{f'_{n_k}(\zeta)}\leq M'+1\qquad\forall\zeta\in\overline{D\qty(z,\frac\delta2)}.\] Therefore, \(\cbraces{f'_{n_k}}_{k>N}\) is uniformly bounded by \[M=\max\qty(\cbraces{M'+1}\cup\cbraces{\sup_{\zeta\in\overline{D\qty(0,\frac\delta2)}}\abs{f_{n_k}'(\zeta)}}_{N<k\leq N'})\] on this compact disk. Hence, \(\forall k>N\), \[f^\sharp_{n_k}(\zeta)=\frac{2\abs{f_{n_k}'(\zeta)}}{1+\abs{f_{n_k}(\zeta)}^2}\leq 2\abs{f'_{n_k}(\zeta)}\leq 2M\qquad\forall\zeta\in D\qty(z,\frac\delta2)\subset\overline{D\qty(z,\frac\delta2)}.\]
        \item \(f\qty(z)=\infty\), then by continuity, \(\exists\delta>0\) such that \(\forall\zeta\in D(z,\delta)\), \[d_\sigma\qty(f(\zeta),\infty)<\frac\muppi2.\] Similarly, \(\exists N\in\mathbb{N}\) such that \(\forall k>N\), \[d_\sigma\qty(f(\zeta),f_{n_k}(\zeta))<\frac\muppi2.\]
            Hence, we have \[\muppi-d_\sigma\qty(\infty,f_{n_k}(\zeta))=d_\sigma(\infty,0)-d_\sigma\qty(\infty,f_{n_k}(\zeta))>0.\] By the reverse triangle inequality, we have \[d_\sigma\qty(0,f_{n_k}(\zeta))>0.\]
            Hence, each \(g_{n_k}=\frac1{f_{n_k}}\) is holomorphic on \(D(z,\delta)\). By
            continuity, \(g=\frac1f\) is locally uniformly bounded on \(D(z,\delta)\). It
            can also be realized that \(\cbraces{g_{n_k}}_{k>N}\) locally uniformly
            converges on \(D(z,\delta)\). By a result of Weierstrass
            (\cref{thm:weierstrassconvergence}), \(g\) is holomorphic on \(D(z,\delta)\)
            and the sequence \(\cbraces{g_{n_k}'}_{k>N}\) locally uniformly converges to
            \(g'\) on \(D(z,\delta)\).

            By holomorphy of \(g'\) on \(\overline{D\qty(z,\frac\delta2)}\), \(\exists
            M'>0\) such that
            \(\sup_{\zeta\in\overline{D\qty(0,\frac\delta2)}}\abs{g'(\zeta)}<M'\). Uniform
            convergence of \(\cbraces{g'_{n_k}}_{k>N}\) gives the existence of some
            \(N'>N\) such that \(\forall k>N'\), \[\abs{g'_{n_k}(\zeta)-g'(\zeta)}<1\implies\abs{g'_{n_k}(\zeta)}\leq M'+1\qquad\forall\zeta\in\overline{D\qty(z,\frac\delta2)}.\] Therefore, \(\cbraces{g'_{n_k}}_{k>N}\) is uniformly bounded by \[M=\max\qty(\cbraces{M'+1}\cup\cbraces{\sup_{\zeta\in\overline{D\qty(0,\frac\delta2)}}\abs{g_{n_k}'(\zeta)}}_{N<k\leq N'})\] on this compact disk. Hence, \(\forall k>N\), \[f^\sharp_{n_k}(\zeta)=\frac{2\abs{-\frac{g_{n_k}'(\zeta)}{g_{n_k}(\zeta)^2}}}{1+\abs{g_{n_k}(\zeta)}^{-2}}=\frac{2\abs{g_{n_k}'(\zeta)}}{\abs{g_{n_k}(\zeta)}^2+1}\leq 2\abs{g'_{n_k}(\zeta)}\leq 2M\qquad\forall\zeta\in D\qty(z,\frac\delta2).\]
    \end{enumerate}
    In essence, for any point \(z\), there exists an open disk \(D_z\) centered at \(z\) on which the spherical derivatives \(f^\sharp_{n_k}\) are bounded by some constant \(M_z\) for \(k>N_z\). By Heine--Borel (\cref{thm:heineborel}), there exists a finite collection of disks \(\cbraces{D_{z_j}}_{1\leq j\leq n}\) that cover \(K\). Thus, \(\cbraces{f_{n_k}^\sharp(z)}_{k>N}\) is uniformly bounded on \(K\) by \(\max_{1\leq j\leq n}M_{z_j}\), where \(N=\max_{1\leq j\leq n} N_{z_j}\), contradicting the assumption that \(\sup_{z\in K}f^\sharp_n(z)>n\) for all \(n\in\mathbb{N}\).
\end{proof}
\begin{theorem}[\textsc{Fundamental Normality Test}]\label{thm:fundamentalnormalitytest}
    Let \(\Omega\subseteq\mathbb{C}\) be a region and suppose that \(\mathcal{F}\) is a family of holomorphic functions on \(\Omega\). If there exist two different points \(\alpha,\beta\in\mathbb{C}\) such that \(\cbraces{\alpha,\beta}\cap\bigcup_{f\in\mathcal{F}}f(\Omega)=\varnothing\), then \(\mathcal{F}\) must be a spherically normal family.
\end{theorem}
\begin{proof}
    Map \(\alpha\) and \(\beta\) to \(0,1\) by a linear function \(\varphi(z)=\frac{z-\alpha}{\beta-\alpha}\). Then the family of holomorphic functions \[\widetilde{\mathcal{F}}=\cbraces{\varphi\circ f}{f\in\mathcal{F}}\] omits \(0\) and \(1\) for all \(z\in\Omega\).

    By \cref{prop:conformalmetricnegativecurvatureexistencewhenomits2points},
    \(\exists\beta>0\) such that for \[\rho(z)=\frac{\sqrt{1+\abs{z}^{\frac13}}\sqrt{1+\abs{z-1}^{\frac13}}}{\abs{z}^{\frac56}\abs{z-1}^{\frac56}},\qquad\dd{s}^2_{\rho}=\rho(z)^2\abs{\ddz}^2\] as in
    \cref{eq:conformalmetricnegativecurvatureexistencewhenomits2points_metric},
    \[K_\rho(z)\leq-\beta\qquad\forall z\in\mathbb{C}\setminus\cbraces{0,1}.\]
    Therefore, if we let \(\mu=\rho\sqrt{\beta}\), then
    \begin{equation}
        K_\mu=-\frac{\laplacian(\log\circ\mu)}{\mu^2}=-\frac{\laplacian(\log\circ\rho)}{\rho^2\beta}=\frac{K_\rho}{\beta}\leq-1\qq{on}\mathbb{C}\setminus\cbraces{0,1}.\label{eq:fundamentalnormalitytest_f_mu_pullback_inequality}
    \end{equation}
    Let \(\zeta\in\Omega\) be arbitrary and let \(r_\zeta>0\) satisfy \(D\qty(\zeta,r_\zeta)\subseteq\Omega\). By \cref{cor:generalizedahlfors}, the pullback of \(\mu\) from \(\mathbb{C}\setminus\cbraces{0,1}\) to \(D\qty(\zeta,r_\zeta)\subseteq\Omega\) satisfies \[f^*\mu(z)\leq\lambda_{r_\zeta}(z-\zeta)\implies\mu\qty(f(z))\abs{f'(z)}\leq\frac{2r_\zeta}{r_\zeta^2-\abs{z-\zeta}^2}\qquad\forall z\in D\qty(\zeta,r_\zeta),f\in\widetilde{\mathcal{F}}.\]
    Since \(\forall w\in\mathbb{C}\setminus\cbraces{0,1}\), \[\frac{\sigma(w)}{\mu(w)}=\frac{\frac{2}{1+\abs{w}^2}}{\frac{\sqrt{1+\abs{w}^{\frac13}}\sqrt{1+\abs{w-1}^{\frac13}}}{\abs{w}^{\frac56}\abs{w-1}^{\frac56}}}\to
        \begin{dcases}
            0                                            & \qq*{as}w\to 0,     \\
            0                                            & \qq*{as}w\to 1,     \\
            \frac{2\abs{w}^{-2}}{\abs{w}^{-\frac43}}\to0 & \qq*{as}w\to\infty.
    \end{dcases}\]
    Hence, there exist open neighborhoods \(U_0,U_1,U_\infty\) of \(0,1,\infty\)
    respectively on which \(\frac\sigma\mu<1\). Since \(\frac{\sigma}{\mu}\in
    C^0(\mathbb{C})\), by \cref{thm:continuousfunctionboundedoncompact}, \(\exists
    M'>0\) such that \(\frac{\sigma}{\mu}<M'\) on \(\mathbb{C}\setminus\qty(U_0\cup
    U_1\cup U_\infty)\). Let \(M=\max\cbraces{M',1}\), and \[\therefore\sigma\leq M\mu\qq{on}\mathbb{C}\setminus\cbraces{0,1}.\]
    Hence, \(\forall f\in\widetilde{\mathcal{F}}\), we have by virtue of
    \cref{eq:fundamentalnormalitytest_f_mu_pullback_inequality}, \[f^\sharp(z)=\sigma\circ f(z)\abs{f'(z)}\leq M\mu\circ f(z)\abs{f'(z)}\leq\frac{2rM}{r_\zeta^2-\abs{z-\zeta}^2}\] for any \(z\in D\qty(\zeta,r_\zeta)\). Now restricting \(z\) to
    \(D\qty(\zeta,\frac{r_\zeta}{2})\), we have \[\abs{z-\zeta}^2<\frac{r_\zeta^2}4\implies r_\zeta^2-\abs{z-\zeta}^2>\frac{3r_\zeta^2}4\implies\abs{f^\sharp(z)}<\frac{8r_\zeta M}{3r_\zeta^2}=\frac{8M}{3r_\zeta}.\]
    For any compact \(K\subset\Omega\), the collection of open disks \[\cbraces{D\qty(\zeta,\frac{r_\zeta}{2})}{\zeta\in K}\] forms an open cover of \(K\). Hence, by Heine--Borel (\cref{thm:heineborel}),
    it admits a finite subcover \[\cbraces{D\qty(\zeta_k,\frac{r_{\zeta_k}}{2})}{1\leq k\leq n}\] for some \(n\in\mathbb{N}\). Then
    \(\cbraces{f^\sharp}{f\in\widetilde{\mathcal{F}}}\) is uniformly bounded on
    \(K\) by \[M_K=\max\cbraces{\frac{8M}{3r_{\zeta_k}}}{1\leq k\leq n}\] and is thus locally uniformly bounded on \(\Omega\). Marty's Criterion
    (\cref{thm:marty}) gives the normality of \(\widetilde{\mathcal{F}}\); since
    \(\varphi\) is linear, it follows that \(\mathcal{F}\) is also normal on
    \(\Omega\).
\end{proof}
\begin{corollary}[name=\textsc{Montel--Carathéodory}]\label{cor:montelcaratheodory}
    Let \(\Omega\subseteq\mathbb{C}\) be a region and suppose that \(\mathcal{F}\) is a family of meromorphic functions on \(\Omega\). If there exist three different points \(\alpha,\beta,\gamma\in\extcomplex\) such that \(\cbraces{\alpha,\beta,\gamma}\cap\bigcup_{f\in\mathcal{F}}f(\Omega)=\varnothing\), then \(\mathcal{F}\) must be a spherically normal family.
\end{corollary}
\begin{proof}
    Let \(\varphi(z)=\frac{(z-\alpha)(\beta-\gamma)}{(z-\gamma)(\beta-\alpha)}\) be a Möbius transformation mapping \(\alpha,\beta,\gamma\) to \(0,1,\infty\), respectively.
    Hence, the family of meromorphic functions \[\widetilde{\mathcal{F}}=\cbraces{\varphi\circ f}{f\in\mathcal{F}}\] omits \(0\), \(1\), and \(\infty\) (and hence each function is holomorphic). By
    the Fundamental Holomorphic Normality Test
    (\cref{thm:fundamentalnormalitytest}), \(\widetilde{\mathcal{F}}\) is normal.

    By \cref{prop:linearfractionaltransformationuniformlysphericallycontinuous},
    \(\forall\varepsilon>0\), \(\exists\delta>0\) such that
    \(\forall\abs{w_1-w_2}<\delta\) in \(\mathbb{C}\), \[d_\sigma\qty(\varphi^{-1}\qty(w_1),\varphi^{-1}\qty(w_2))<\varepsilon.\]
    Let \(\cbraces{\widetilde{f}_n}_{n\in\mathbb{N}}\) be any function sequence in
    \(\mathcal{F}\) and let \(\cbraces{\widetilde{f}_{n_k}}_{k\in\mathbb{N}}\) be
    locally uniformly convergent to \(\widetilde{f}\) on a compact set
    \(K\subset\Omega\). Then \(\exists N\in\mathbb{N}\) such that \(\forall k>N\), \[\abs{\widetilde{f}_{n_k}(z)-\widetilde{f}(z)}<\delta\qquad\forall z\in K.\]
    Therefore, \(\forall z\in K\), \(k>N\), we have \[d_\sigma\qty(\varphi^{-1}\circ\widetilde{f}_{n_k}(z),\varphi^{-1}\circ\widetilde{f}(z))=d_\sigma\qty(f_{n_k}(z),f(z))<\varepsilon.\] Hence, every sequence \(f_n\) has a locally uniformly spherically convergent
    subsequence, and the normality of \(\mathcal{F}\) follows.
\end{proof}