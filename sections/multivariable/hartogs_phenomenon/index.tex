\subsection{Hartogs' Phenomenon}
One of the most prominent results of complex function theory is \textit{Hartogs' phenomenon}, which reveals a fundamental difference between holomorphic functions of one variable and those of several variables. First
discovered by Friedrich Hartogs in 1906, it essentially states that holomorphic
functions in several complex variables exhibit a surprising rigidity.

To understand the sense of Hartogs' phenomenon, consider the simple example of
\(z\mapsto\frac1z\) in one complex variable. This function is holomorphic on
\(\mathbb{C}^*\) with a singularity at 0 that cannot be removed. However, in
the case of several complex variables, Hartogs' phenomenon asserts that if a
function is holomorphic in a domain that excludes a compact subset, it can be
extended to a holomorphic function on the entire domain, including the excluded
subset. One immediate conclusion from this is that isolated singularities are
\emph{always} removable in higher dimensions.
\begin{theorem}[\textsc{Hartogs' Extension Theorem}]\label{thm:hartogsextensiontheorem}
    Let \(\Omega\subseteq\mathbb{C}^n\) (\(n\ge2\)) be a domain and \(K\subseteq\Omega\) be a compact subset such that \(\Omega\setminus K\) is connected. Then any holomorphic function \(f:\Omega\setminus K\to\mathbb{C}\) has a unique extension to a holomorphic function \(\widetilde{f}:\Omega\to\mathbb{C}\) such that \(\widetilde{f}\equiv f\) on \(\Omega\setminus K\).
\end{theorem}
Hartogs, in 1906, first proved his extension theorem using an integral formula, but was considered to be incomplete with gaps. In 1939, Fueter gave a proof for the case \(n=2\), and later Bochner and Martinelli developed more general integral kernel methods for higher dimensions. Finally, Ehrenpreis, in 1961, provided a succinct and analytic proof using the \(\overline{\partial}\) operator: by multiplying the function by a
smooth cutoff or bump function and then solving a \(\overline{\partial}\)-problem to correct the non-holomorphy introduced by the cutoff, one can construct the desired extension.

The general study of complex function theory in multiple variables is made
practical with the definition of differential forms in multiple variables and
in particular the \(\overline\partial\)-problem, whose utility in a single
variable has already been preluded to many times before.

Only certain cases of \cref{thm:hartogsextensiontheorem} are proved here.
\begin{proposition}\label{prop:reinhardtlaurentexpansion}
    Let \(\Omega\subseteq\mathbb{C}^n\) be a Reinhardt domain centered at \(\symbf{0}\) and let \(f:\Omega\to\mathbb{C}\) be holomorphic. Then \(\symbf{f}\) admits the unique Laurent expansion
    \begin{equation}
        \symbf{f}(\symbf{z})=\sum_{\symbf{k}\in\mathbb{Z}^n}\symbf{a}_{\symbf{k}}\symbf{z}^{\symbf{k}},\qquad\symbf{a}_{\symbf{k}}\in\mathbb{C},\label{eq:reinhardtlaurentexpansion}
    \end{equation} converging absolutely and uniformly on compact subsets of \(\Omega\).
\end{proposition}
\begin{proof}
    Let \(\symbf{w}=\qty(w_1,\dots,w_n)\in\Omega\) with each \(w_j\neq0\). Since \(\Omega\) is a Reinhardt domain, for every \(\symbf{\theta}=\qty(\theta_1,\dots,\theta_n)\in[-\uppi,\uppi]^n\) the point
    \[z_j=w_j\ee^{\ii\theta_j},\qquad j\in\mathbb{N}_{\leq n}\]
    lies in \(\Omega\); moreover, the set of all such points, \(K\), is compact in
    \(\Omega\). Assuming that a Laurent expansion exists and converges uniformly on
    this compact set, termwise integration gives
    \[a_{\mathbf{k}}=\frac{1}{\qty(2\uppi\ii)^n}\int_K\frac{f(\symbf{z})}{\symbf{z}^{\symbf{k}+\qty(1,\dots,1)}}\dd{\symbf{z}}=\frac{\symbf{w}^{-\symbf{k}}}{\qty(2\uppi)^n}\mathmakebox[\widthof{\(\int_{-\uppi}\)}][l]{\int_{[-\uppi,\uppi]^n}}
    f\qty(w_1\ee^{\ii\theta_1},\dots,w_n\ee^{\ii\theta_n})\ee^{-\ii(k_1\theta_1+\cdots+k_n\theta_n)}\dd{\symbf{\theta}},\]
    which shows that the coefficients \(a_{\mathbf{k}}\) are uniquely determined.

    To prove the existence of such an expansion, again fix \(w\in\Omega\). Because
    \(\Omega\) is Reinhardt, there exists \(\varepsilon>0\) such that the
    polyannulus
    \[\Omega\qty(\symbf{w},\varepsilon)=\cbraces{\symbf{z}=\qty(z_1,\dots,z_n)\in\mathbb{C}^n}{\abs{w_j}-\varepsilon<\abs{z_j}<\abs{w_j}+\varepsilon,j\in\mathbb{N}_{\leq n}}\] is contained in \(\Omega\). On this set one may perform one-variable Laurent
    expansions (\cref{thm:laurentexpansionofholomorphicfunction}) successively in
    each coordinate, holding the others fixed. Since \(f\) is holomorphic on
    \(\Omega\), this procedure yields
    \[f(\symbf{z})=\sum_{\mathbf{k}\in\mathbb{Z}^n}a_{\mathbf{k}}(\symbf{w})\symbf{z}^{\mathbf{k}},\]
    a Laurent series converging uniformly on a neighborhood of \(w\). If
    \(\symbf{w}'\in\Omega(\symbf{w},\varepsilon)\), then the same construction gives
    \[f(z)=\sum_{\mathbf{k}\in\mathbb{Z}^n}a_{\mathbf{k}}(\symbf{w}')z^{\mathbf{k}}.\]
    By uniqueness of Laurent coefficients (in the one-variable expansions), it
    follows that \(a_{\mathbf{k}}(\symbf{w}')=a_{\mathbf{k}}(\symbf{w})\). Thus the
    coefficient functions \(a_{\mathbf{k}}(\symbf{w})\) are locally constant on
    \(\Omega\), and since \(\Omega\) is connected they are constant throughout
    \(\Omega\); we therefore write them simply as \(a_{\mathbf{k}}\).

    Consequently,
    \[f(\symbf{z})=\sum_{\mathbf{k}\in\mathbb{Z}^n}a_{\mathbf{k}}\symbf{z}^{\mathbf{k}}\]
    holds in a neighborhood of every point in \(\Omega\). If \(K\subset\Omega\) is
    compact, then it lies inside some polyannulus
    \(\Omega(\symbf{w},\varepsilon)\subset\Omega\), and on this polyannulus the series
    converges uniformly; therefore it converges uniformly on \(K\). The absolute
    convergence follow from the absolute convergence of the one-variable Laurent
    expansions used in the construction.
\end{proof}
\begin{proposition}\label{prop:hartogslaurentexpansionnonnegative}
    Let \(\Omega\subset\mathbb{C}^n\) be a Reinhardt domain centered at \(\symbf{0}\) such that \(\Omega\)
    satisfies the condition that for every \(j\in\mathbb{N}_{\leq n}\), there
    exists a point in \(\Omega\) of the form \(w_j\symbf{e}_j\) (\(\symbf{e}_j\) is the \(j\)-th unit vector and \(w_j\notin\mathbb{C}^*\)). Then any holomorphic function \(f:\Omega\to\mathbb{C}\) has the expansion \[f(\symbf{z})=\sum_{\symbf{k}\in\mathbb{Z}^n_{\geq 0}}a_{\symbf{k}}\symbf{z}^{\symbf{k}},\] which converges locally uniformly and absolutely on \(\Omega\).
\end{proposition}
\begin{proof}
    By \cref{prop:reinhardtlaurentexpansion}, \(f\) has an expansion matching the form of \cref{eq:reinhardtlaurentexpansion}. Now for a fixed \(j\in\mathbb{N}_{\leq n}\), if not all \(a_{\symbf{k}}\)'s with \(k_j<0\) are zero, then fixing all variables except \(z_j\) gives a Laurent expansion that does not uniformly converge on all compact neighborhoods of \(w_j\symbf{e}_j\in\Omega\) (either a pole or an essential singularity with respect to \(z_j\)). Thus, we are left only with \(\symbf{k}\) containing nonnegative components, and the conclusion holds.
\end{proof}
\begin{theorem}[Hartogs' Extension Theorem for Reinhardt Domains]\label{thm:hartogsextensiontheoremforreinhardtdomains}
    Let \(\Omega\subset\mathbb{C}^n\) be a Reinhardt domain centered at \(\symbf{0}\) such that \(\Omega\)
    satisfies the condition that for every \(j\in\mathbb{N}_{\leq n}\), there
    exists a point in \(\Omega\) of the form \(a_j\symbf{e}_j\) (\(\symbf{e}_j\) is the \(j\)-th unit vector and \(a_j\notin\mathbb{C}^*\)). Let \(f:\Omega\to\mathbb{C}\) be
    holomorphic. Then \(f(\symbf{z})\) can be
    analytically continued to the complete Reinhardt domain \(\widetilde{\Omega}\) defined by \[\widetilde{\Omega}=\cbraces{\qty(\rho_1z_1,\rho_2z_2,\dots,\rho_nz_n)\in\mathbb{C}^n}{0\leq\rho_j\leq 1,j\in\mathbb{N}_{\leq n},\qty(z_1,\dots,z_n)\in\Omega}\]
    In other words, \(\exists\widetilde{f}:\widetilde{\Omega}\to\mathbb{C}\)
    holomorphic such that \(\widetilde{f}\equiv f\) on \(\Omega\).
\end{theorem}
\begin{proof}
    By \cref{prop:hartogslaurentexpansionnonnegative}, \(f\) has the expansion
    \[f(\symbf{z})=\sum_{\symbf{k}\in\mathbb{Z}^n_{\geq 0}}a_{\symbf{k}}\symbf{z}^{\symbf{k}},\] which absolutely converges on \(\Omega\). Substituting
    \(\qty(\rho_1z_1,\dots,\rho_nz_n)\) for \(\symbf{z}\in\Omega\) with
    \(0\leq\rho_j\leq1\) for each \(j\in\mathbb{N}_{\leq n}\) gives
    \[\abs{\widetilde{f}(\rho_1z_1,\dots,\rho_nz_n)}=\abs{\sum_{\symbf{k}\in\mathbb{Z}^n_{\geq 0}}a_{\symbf{k}}\qty(\rho_1z_1)^{k_1}\cdots\qty(\rho_nz_n)^{k_n}}\leq\sum_{\symbf{k}\in\mathbb{Z}^n_{\geq0}}\abs{a_{\symbf{k}}\symbf{z}^{\symbf{k}}},\] which converges. The function \(\widetilde{f}:\widetilde{\Omega}\to\mathbb{C}\)
    is then holomorphic as it is given by a power series expansion, and it
    satisfies \(\widetilde{f}\equiv f\) on \(\Omega\).
\end{proof}
\begin{example}
    Set \(r>0\) and let \[\Omega\qty(r)=B^n\setminus r\overline{B^n}=\cbraces{\symbf{z}}{r<\norm{\symbf{z}}<1}\] define a Reinhardt domain. Then any holomorphic function \(f:\Omega(r)\to\mathbb{C}\) can be analytically continued to the entire unit ball \(B^n\) by \cref{thm:hartogsextensiontheoremforreinhardtdomains}.
\end{example}