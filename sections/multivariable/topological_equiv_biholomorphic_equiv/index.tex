\subsection{Topological Equivalence and Biholomorphic Equivalence}
\begin{theorem}[Poincaré]\label{thm:poincarepolydiskandunitball}
    For any \(n\geq2\), the \(n\)-dimensional unit ball \(B^n\) and the \(n\)-dimensional polydisk \(\mathbb{D}^n\) are not biholomorphically equivalent.
\end{theorem}
\begin{proof}
    Suppose, for the sake of contradiction, that there exists a biholomorphism \(\vb*{\varphi}:\mathbb{D}^n\to B^n\). Let \(\vb*{\alpha}=\vb*{\varphi}(\vb{0})\in B^n\), and define \(\vb*{\Phi}=\vb*{\varphi}_{\abs{\vb*{\alpha}}}\circ\vb{U}\vb*{\varphi}\), where \(\vb{U}\) is a unitary matrix such that \(\vb{U}\vb*{\alpha}=\qty(\abs{\vb*{\alpha}},0,\ldots,0)\) and \(\vb*{\varphi}_{\abs{\vb*{\alpha}}}\) is as in \cref{prop:unitballsimpleautomorphism}.

    The definition of \(\vb*{\Phi}\) ensures that \(\vb*{\Phi}:\mathbb{D}^n\to
    B^n\) and \(\vb*{\Phi}(\vb{0})=\vb{0}\). Then
    \(\vb*{\Phi}^{-1}\circ\Aut\qty(B^n)\circ\vb*{\Phi}\) consists of functions
    mapping \(\mathbb{D}^n\) to \(\mathbb{D}^n\), or that \[\vb*{\Phi}^{-1}\circ\Aut\qty(B^n)\circ\vb*{\Phi}\subseteq\Aut\qty(\mathbb{D}^n)\implies\Aut\qty(B^n)\subseteq\vb*{\Phi}\circ\Aut\qty(\mathbb{D}^n)\circ\vb*{\Phi}^{-1}.\]
    Similarly,
    \(\vb*{\Phi}\circ\Aut\qty(\mathbb{D}^n)\circ\vb*{\Phi}^{-1}\subseteq\Aut\qty(B^n)\).
    Hence,
    \(\Aut\qty(B^n)=\vb*{\Phi}\circ\Aut\qty(\mathbb{D}^n)\circ\vb*{\Phi}^{-1}\),
    and
    \begin{equation}
        \vb*{\psi}\mapsto\vb*{\Phi}\circ\vb*{\psi}\circ\vb*{\Phi}^{-1}\label{eq:poincarepolydiskandunitball_isomorphism}
    \end{equation} defines a group isomorphism between \(\Aut\qty(\mathbb{D}^n)\) and \(\Aut\qty(B^n)\). Let \(\Aut'\qty(\mathbb{D}^n)<\Aut\qty(\mathbb{D}^n)\) and \(\Aut'\qty(B^n)<\Aut\qty(B^n)\) be subgroups fixing \(\vb{0}\). Therefore, \cref{eq:poincarepolydiskandunitball_isomorphism} induces a group isomorphism between \(\Aut'\qty(\mathbb{D}^n)\) and \(\Aut'\qty(B^n)\) as well.

    By \cref{thm:holomorphicautomorphismgrouponpolydisk}, every element of \(\Aut'\qty(\mathbb{D}^n)\) may be uniquely identified with a matrix in the form of \[\vb{P}\operatorname{diag}\qty(\ee^{\ii\theta_1},\ldots,\ee^{\ii\theta_n}),\] where \(\vb{P}\) is a permutation matrix and \(\qty(\theta_1,\ldots,\theta_n)\in[0,2\piup)^n\). Hence \(\Aut'\qty(\mathbb{D}^n)\) is isomorphic to the group of unitary monomial matrices. The structure of \(\Aut'\qty(B^n)\) is given by
    \cref{prop:unitballautomorphismfixedpointatzero}, and each element corresponds
    uniquely to a unitary matrix. Thus there is a natural isomorphism
    \(\Aut'\qty(B^n)\cong\mathrm{U}(n)\), the \(n\times n\) \textscsl{unitary
    group}.

    For \(\vb{U}\in\mathrm{U}(n)\), the spectral theorem allows it to be expressed
    in the form of
    \(\vb{V}\operatorname{diag}\qty(\ee^{\ii\theta_1},\ldots,\ee^{\ii\theta_n})\vb{V}^\dagger\).
    Hence, for any positive integer \(m\), we have \[\vb{V}\operatorname{diag}\qty(\ee^{\ii\frac{\theta_1}m},\ldots,\ee^{\ii\frac{\theta_n}m})\vb{V}^\dagger\in\mathrm{U}(n)\] and \[\qty(\vb{V}\operatorname{diag}\qty(\ee^{\ii\frac{\theta_1}m},\ldots,\ee^{\ii\frac{\theta_n}m})\vb{V}^\dagger)^m=\vb{V}\operatorname{diag}\qty(\ee^{\ii\frac{\theta_1}m},\ldots,\ee^{\ii\frac{\theta_n}m})\vb{V}^\dagger\cdots\vb{V}\operatorname{diag}\qty(\ee^{\ii\frac{\theta_1}m},\ldots,\ee^{\ii\frac{\theta_n}m})\vb{V}^\dagger.\] The adjacent products of \(\vb{V}^\dagger\vb{V}\) simplify to the identity and
    the entire expression then simplifies to \(\vb{U}\). Hence the unitary group is
    divisible.

    Consider the unitary monomial matrix \[\vb{P}_\tau=\mqty(0&1&0&0&\cdots&0\\1&0&0&0&\cdots&0\\0&0&1&0&\cdots&0\\0&0&0&1&\cdots&0\\\vdots&\vdots&\vdots&\vdots&\ddots&\vdots\\0&0&0&0&\cdots&1).\] inducing the permutation \(\tau\), swapping the first and second entries.
    Assume that there exists some unitary monomial matrix
    \(\vb{Q}=\vb{P}_\sigma\vb{D}\) (where \(\vb{D}\) is diagonal and
        \(\vb{P}_\sigma\) is a permutation matrix corresponding to the permutation
    \(\sigma\)) such that \(\vb{Q}^2=\vb{P}_\tau\). This is equivalent to \[\vb{P}_\sigma\vb{D}\vb{P}_\sigma\vb{D}=\vb{P}_\tau\implies\vb{P}^2_\sigma\vb{P}^{-1}_\sigma\vb{D}\vb{P}_\sigma\vb{D}=\vb{P}_\tau\implies\vb{P}^2_\sigma\vb{D}^2=\vb{P}_\tau\]
    because \(\vb{P}^{-1}_\sigma\vb{D}\vb{P}_\sigma=\vb{D}\). Thus,
    \(\vb{P}^2_\sigma=\vb{P}_\tau\) (and \(\vb{D}^2=\vb{I}\)) since their
    permutation parts must match. This is an impossibility since
    \(\vb{P}^2_\sigma\) corresponds to an even permutation, while \(\vb{P}_\tau\)
    corresponds to an odd permutation. Thus, the unitary monomial group is not
    divisible.

    By \cref{prop:groupdivisibilitypreservedunderisomorphisms}, the two groups
    cannot be isomorphic to each other.

    This contradicts the existence of
    \cref{eq:poincarepolydiskandunitball_isomorphism}, and therefore, no such
    biholomorphism \(\vb*{\varphi}\) exists.
\end{proof}
\begin{remark}
    A more succinct proof of the nonexistence of an isomorphism in the proof of \cref{thm:poincarepolydiskandunitball} can be briefly described by means of topology:

    \begin{quote}
        Let \(M_n\) denote the subgroup of all monomial matrices in \(\mathrm{U}(n)\), or the subgroup of unitary matrices with exactly one nonzero entry in each row and each column, and those nonzero entries lying in \(\mathrm{U}(1)\).

        For each permutation \(\sigma\in S_n\) (the \textscsl{symmetric group} of
        permutations) let \(\vb{P}_\sigma\) be the corresponding permutation matrix and
        define
        \[T_\sigma=\cbraces{\vb{D}\vb{P}_\sigma}{\vb{D}=\operatorname{diag}\qty(\ee^{\ii\theta_1},\dots,\ee^{\ii\theta_n})\in\mathrm{U}(1)^n}.\]
        Each \(T_\sigma\) is homeomorphic to the torus \(\mathrm{U}(1)^n\), and every
        element of \(M_n\) lies in exactly one \(T_\sigma\). Hence
        \[M_n=\bigsqcup_{\sigma\in S_n} T_\sigma,\]
        a disjoint union of \(\abs{S_n}=n!\) copies of \(\mathrm{U}(1)^n\).

        Each \(T_\sigma\) is clopen in \(M_n\) by their pairwise disjointness, the
        topology of the torus, and the fact that their union is \(M_n\). Therefore each
        \(T_\sigma\) is a connected component of \(M_n\). Because each \(T_\sigma\) is
        connected, \(M_n\) has \(n!\) connected components.

        The elements of \(\mathrm{U}(n)\) may be unitarily diagonalized (by the
        spectral theorem) into \(\vb{VD}\vb{V}^\dagger\), where \[\vb{D}=\operatorname{diag}\qty(\ee^{\ii\theta_1},\ldots,\ee^{\ii\theta_n})\] is a diagonal unitary matrix and \(\vb{V}\in\mathrm{U}(n)\). Then there exists
        a connected path connecting \(\qty(\theta_1,\dots,\theta_n)\) to
        \(\qty(0,\ldots,0)\), which corresponds to the matrix
        \(\vb{VI}\vb{V}^\dagger=\vb{I}\). Because every matrix is path-connected to the
        identity, \(\mathrm{U}(n)\) is connected.

        Consequently for \(n\ge2\), the subgroup \(M_n\) (which has more than one
        connected component) cannot be isomorphic to \(\mathrm{U}(n)\) as a topological
        group.
    \end{quote}
    Although we do not justify these topological claims in detail here, it is worth noting, heuristically, why such topological considerations naturally arise.

    A biholomorphism between two domains induces a homeomorphism with respect to
    their natural topologies (the \textscsl{compact-open topology}) by
    \(\vb*{\psi}\mapsto\vb*{\Phi}\circ\vb*{\psi}\circ\vb*{\Phi}^{-1}\). Hence any
    induced topological invariant of an automorphism group---such as connectivity,
    is to be preserved under this equivalence. Of course, we are yet to verify the
    rigor and intuition used within the topology, but the intuitive picture already
    hints to the validity of the connectivity argument.
\end{remark}