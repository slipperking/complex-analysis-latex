\subsection{Consequences of Holomorphy}
Obviously, we will first formally define the concept of holomorphy in higher
dimensions.
\begin{definition}
    A function \(f:\Omega\subseteq\mathbb{C}^n\to\mathbb{C}\) is \textit{holomorphic} if it is holomorphic in each variable when the others are held constant.
\end{definition}
If we consider \(f\) to be a function of \(z_1,\overline{z_1},z_2,\overline{z_2},\ldots,z_n,\overline{z_n}\), then \(f\) is holomorphic iff \(\pdv{f}{\overline{z_k}}\equiv 0\) for all \(1\leq k\leq n\) and \(f\) has all continuous partial derivatives.
\begin{theorem}[Cauchy's Integral Formula on Polydisks]\label{thm:cauchyintegralformulapolydisks}
    Fix \(\symbf{a}=\qty(a_1,\ldots,a_n)\in\mathbb{C}^n\) arbitrarily and suppose \(r_1, r_2,\ldots, r_n>0\) are the radii of the polydisk defined by \(\Omega=\prod_{k=1}^n D\qty(a_k, r_k)\) (where the product here is the Cartesian product).
    Suppose \(f:\overline{\Omega}\to\mathbb{C}\) is holomorphic. For fixed \(k_1, k_2,\ldots, k_n\in\mathbb{Z}_{\geq 0}\), we have that \[\qty(\prod_{j=1}^n\frac{\partial^{k_j}}{\partial z_j^{k_j}})f(z)=\frac{\prod_{j=1}^n k_j!}{(2\uppi\ii)^n}\oint_{\partial D\qty(a_1,r_1)}\cdots\oint_{\partial D\qty(a_n,r_n)}\frac{f\qty(\zeta_1,\ldots,\zeta_n)}{\prod_{j=1}^n\qty(\zeta_j-z_j)^{k_j+1}}\dd{\zeta_n}\cdots\dd{\zeta_1}\] for any \(z=\qty(z_1,z_2,\ldots, z_n)\in\Omega\).
\end{theorem}
\begin{proof}
    By Cauchy--Goursat (\cref{thm:cauchydifferentiationformula}), we have \[\frac{\partial^{k_1}}{\partial z_1^{k_1}}f(z)=\frac{k_1!}{2\uppi\ii}\oint_{\partial D\qty(a_1,r_1)}\frac{f\qty(\zeta_1,z_2\ldots,z_n)}{\qty(\zeta_1-z_1)^{k_1+1}}\dd{\zeta_1}\] for \(z\in\Omega\), which is holomorphic. Thus, by the same application on
    \(\frac{\partial^{k_1}}{\partial z_1^{k_1}}f(z)\), we have
    \[\frac{\partial^{k_{2}}}{\partial z_{2}^{k_{2}}}\frac{\partial^{k_1}}{\partial z_1^{k_1}}f(z)=\frac{k_{2}!k_1!}{\qty(2\uppi\ii)^2}\oint_{\partial D\qty(a_{2},r_{2})}\oint_{\partial D\qty(a_1,r_1)}\frac{f\qty(\zeta_1,\zeta_2,z_3,\ldots,z_n)}{\qty(\zeta_1-z_1)^{k_1+1}\qty(\zeta_{2}-z_{2})^{k_{2}+1}}\dd{\zeta_1}\dd{\zeta_2}.\]
    By reiterating \(n\) times and reversing the order of differentiation and
    integration, the conclusion follows.
\end{proof}
By the boundedness assumption for \(f\), we have:
\begin{corollary}[Cauchy's Estimate on Polydisks]\label{cor:cauchysestimatepolydisks}
    Let \(\symbf{a}=\qty(a_1,a_2,\ldots,a_n)\in\mathbb{C}^n\) be fixed and suppose \(r_1, r_2,\ldots, r_n>0\) are the radii of the polydisk defined by \(\Omega=\prod_{k=1}^n D\qty(a_k, r_k)\) (where the product here is the Cartesian product).
    Suppose \(f:\overline{\Omega}\to\mathbb{C}\) is holomorphic. For fixed \(k_1, k_2,\ldots, k_n\in\mathbb{Z}_{\geq 0}\), we have that \[\abs{\qty(\prod_{j=1}^n\frac{\partial^{k_j}}{\partial z_j^{k_j}})f(\symbf{z})}\leq\prod_{j=1}^n\qty(\frac{k_j!}{r_j^{k_j}})\sup_{\zeta\in\prod_{j=1}^n\partial D\qty(a_j,r_j)}\abs{f(\zeta)}\] for any \(\symbf{z}=\qty(z_1,z_2,\ldots, z_n)\in\Omega\).
\end{corollary}
\begin{proof}
    For each \(j\), let \(\varepsilon_j\) satisfy \(D\qty(z_j,\varepsilon_j)\subseteq D\qty(a_j,r_j)\). By Cauchy's Integral Formula (\cref{thm:cauchyintegralformulapolydisks}), we have \[\abs{\qty(\prod_{j=1}^n\frac{\partial^{k_j}}{\partial z_j^{k_j}})f(\symbf{z})}\leq\frac{\prod_{j=1}^n k_j!}{(2\uppi)^n}\oint_{\partial D\qty(a_1,r_1)}\cdots\oint_{\partial D\qty(a_n,r_n)}\abs{\frac{f(\zeta_1,\ldots,\zeta_n)}{\prod_{j=1}^n\qty(\zeta_j-z_j)^{k_j+1}}}\abs{\dd{\zeta_n}}\cdots\abs{\dd{\zeta_1}}.\]
    For each \(j\), let \(\zeta_j=a_j+r_j\ee^{\ii t_j}\), and it follows that
    \(\dd{\zeta_j}=\ii r_j\ee^{\ii t_j}\dd{t_j}\). Because
    \(\abs{\zeta_j-z_j}>\varepsilon_j\), we have, after substition,
    \begin{align*}
        \abs{\qty(\prod_{j=1}^n\frac{\partial^{k_j}}{\partial z_j^{k_j}})f(\symbf{z})} & \leq\frac{\prod_{j=1}^n\qty(r_j k_j!)}{(2\uppi)^n}\int_0^{2\uppi}\cdots\int_0^{2\uppi}\abs{\frac{f(\zeta_1,\ldots,\zeta_n)}{\prod_{j=1}^n\varepsilon_j^{k_j+1}}}\dd{t_n}\cdots\dd{t_1} \\
        & \leq\prod_{j=1}^n\qty(\frac{r_j k_j!}{2\uppi\varepsilon_j^{k_j+1}})\sup_{\zeta\in\prod_{j=1}^n\partial D\qty(a_j,r_j)}\abs{f(\zeta)}\idotsint_{[0,2\uppi]^n}\dd{t_n}\cdots\dd{t_1}.    \\
        & \leq\prod_{j=1}^n\qty(\frac{k_j!}{r_j^{k_j}})\sup_{\zeta\in\prod_{j=1}^n\partial D\qty(a_j,r_j)}\abs{f(\zeta)},
    \end{align*} since \(\varepsilon_j\leq r_j\) for all \(j\).
\end{proof}
Similar to the univariate case, there are Taylor expansions of holomorphic functions in several complex variables.
\begin{theorem}\label{thm:taylorexpansionmultivariable}
    Let \(f:\mathbb{C}^n\to\mathbb{C}\) be holomorphic on (a neighborhood of) the closure \(\overline{\Omega}\) of a polydisk \(\Omega=\prod_{k=1}^n D\qty(a_k,r_k)\) centered at \(\symbf{a}=(a_1,a_2,\ldots,a_n)\in\mathbb{C}^n\). Then, for any \(\symbf{z}=(z_1,z_2,\ldots,z_n)\in\Omega\), we have the expansion
    \begin{equation}
        f(\symbf{z})=\sum_{k_1=0}^\infty\cdots\sum_{k_n=0}^\infty a_{k_1,\ldots,k_n}\qty(z_1-a_1)^{k_1}\cdots\qty(z_n-a_n)^{k_n},\label{eq:taylorexpansionmultivariable_series}
    \end{equation}
    where \(\forall k_1,\ldots,k_n\in\mathbb{Z}_{\ge0}\),
    \[a_{k_1,\ldots,k_n}=\frac{1}{\prod_{j=1}^n k_j!}\qty(\prod_{j=1}^n\pdv[k_j]{z_j})f\qty(\symbf{a}).\]
    The series converges absolutely and uniformly on \(\Omega\).
\end{theorem}
\begin{proof}
    By \cref{thm:cauchyintegralformulapolydisks} we have
    \[f(\symbf{z})=\frac{1}{(2\uppi\ii)^n}\oint_{\partial D\qty(a_1,r_1)}\cdots\oint_{\partial D\qty(a_n,r_n)}\frac{f(\zeta)}{(\zeta_1-z_1)\cdots(\zeta_n-z_n)}\dd{\zeta_n}\cdots\dd{\zeta_1}.\]

    For each \(j\), since \(\abs{z_j-a_j}<r_j=\abs{\zeta_j-a_j}\) on \(\partial
    D\qty(a_j,r_j)\), the geometric series expansion holds:
    \[\frac{1}{\zeta_j-z_j}=\frac{1}{\zeta_j-a_j}\cdot\frac{1}{1-\frac{z_j-a_j}{\zeta_j-a_j}}=\sum_{k_j=0}^\infty\frac{(z_j-a_j)^{k_j}}{(\zeta_j-a_j)^{k_j+1}},\]
    which converges uniformly in \(\zeta_j\) on \(\partial D\qty(a_j,r_j)\). Hence,
    we have \[f(\symbf{z})=\frac{1}{\qty(2\uppi\ii)^n}\sum_{k_1=0}^\infty\oint_{\partial D\qty(a_1,r_1)}\cdots\oint_{\partial D\qty(a_n,r_n)}\frac{f(\zeta)\qty(z_1-a_1)^{k_1}\dd{\zeta_n}\cdots\dd{\zeta_1}}{\qty(\zeta_1-a_1)^{k_1+1}\qty(\zeta_2-z_2)\cdots\qty(\zeta_n-z_n)}\] where uniform convergence has allowed the interchange of summation and
    integration. Reiteration of this process gives \[f(\symbf{z})=\sum_{k_1=0}^\infty\cdots\sum_{k_n=0}^\infty\frac{\prod_{j=1}^n\qty(z_j-a_j)^{k_j}}{\qty(2\uppi\ii)^n}\oint_{\partial D\qty(a_1,r_1)}\cdots\oint_{\partial D\qty(a_n,r_n)}\frac{f(\zeta)\dd{\zeta_n}\cdots\dd{\zeta_1}}{\prod_{j=1}^n\qty(\zeta_j-a_j)^{k_j+1}}.\]
    By the Cauchy Integral Formula (\cref{thm:cauchyintegralformulapolydisks}), we
    have \[\qty(\prod_{j=1}^n\pdv[k_j]{z_j})f\qty(\symbf{a})=\frac{\prod_{j=1}^nk_j!}{\qty(2\uppi\ii)^n}\oint_{\partial D\qty(a_1,r_1)}\cdots\oint_{\partial D\qty(a_n,r_n)}\frac{f(\zeta)}{\prod_{j=1}^n\qty(\zeta_j-a_j)^{k_j+1}}\dd{\zeta_n}\cdots\dd{\zeta_1},\] and hence if we let \[a_{k_1,\ldots,k_n}=\frac{1}{\prod_{j=1}^n k_j!}\qty(\prod_{j=1}^n\pdv[k_j]{z_j})f\qty(\symbf{a}),\] then \cref{eq:taylorexpansionmultivariable_series} follows.

    Cauchy's Estimate (\cref{cor:cauchysestimatepolydisks}) gives that \[\abs{a_{k_1,\ldots,k_n}}\leq M\prod_{j=1}^n\qty(\frac{1}{\rho_j^{k_j}}),\] where \(M=\sup_{\zeta\in\prod_{j=1}^n\partial D\qty(a_j,r_j)}\abs{f(\zeta)}\)
    for some \(\rho_j>r_j\) for all \(j\). Hence,
    \begin{align*}
        \abs{\sum_{k_1=0}^\infty\cdots\sum_{k_n=0}^\infty a_{k_1,\ldots,k_n}\prod_{j=1}^n\qty(z_j-a_j)^{k_j}} & \leq\sum_{k_1=0}^\infty\cdots\sum_{k_n=0}^\infty\abs{a_{k_1,\ldots,k_n}}\prod_{j=1}^n\abs{z_j-a_j}^{k_j} \\
        & \leq M\sum_{k_1=0}^\infty\cdots\sum_{k_n=0}^\infty\prod_{j=1}^n\abs{\frac{r_j}{\rho_j}}^{k_j}            \\
        & =M\prod_{j=1}^n\sum_{k_j=0}^\infty\abs{\frac{r_j}{\rho_j}}^{k_j}<\infty.
    \end{align*}
    By the Weierstrass \(M\)--Test (\cref{thm:weierstrassmtest}), the series converges absolutely and uniformly on \(\Omega\).
\end{proof}
\begin{theorem}[\textsc{Identity}]\label{thm:identitymultivar}
    Let \(f\) be a holomorphic function on \(\Omega\subseteq\mathbb{C}^n\). If the set \(\cbraces{z\in\Omega}{f(z)=0}\) has an accumulation point in \(\Omega\), then \(f\equiv 0\) on \(\Omega\).
\end{theorem}
\begin{theorem}[\textsc{Maximum Modulus Principle}]
    Let \(\Omega\subset\mathbb{C}^n\) be a open bounded region, and suppose that \(f:\Omega\to\mathbb{C}\) is holomorphic. If \[M=\sup_{\zeta\in\partial\Omega}\lim_{\substack{z\to\zeta\\z\in\Omega}}\abs{f(z)},\] then \(\abs{f(z)}<M\) for all \(z\in\Omega\), unless \(f\) is constant.
\end{theorem}
\begin{theorem}[Weierstrass]\label{thm:weierstrassconvergencemultivar}
    Suppose that \(\Omega\subseteq\mathbb{C}^n\) is a region and that \(\cbraces{f_k}_{k\in\mathbb{N}}\) is a sequence of holomorphic functions on \(\Omega\). If \(\cbraces{f_k}_{k\in\mathbb{N}}\) converges locally uniformly to \(f\) on \(\Omega\), then \(f\) is holomorphic on \(\Omega\). Moreover, \(\forall k_1,\ldots,k_n\in\mathbb{Z}_{\ge0}\),
    \[\qty(\prod_{j=1}^n\pdv[k_j]{}{z_j})f_k\rightrightarrows\qty(\prod_{j=1}^n\pdv[k_j]{}{z_j})f\] on compact subsets of \(\Omega\).
\end{theorem}
\begin{theorem}[Montel]\label{thm:montelmultivar}
    A family \(\mathcal{F}\) of holomorphic functions on some region \(\Omega\subseteq\mathbb{C}^n\) is normal iff it is locally uniformly bounded on \(\Omega\).
\end{theorem}