\subsection{The Group of Holomorphic Automorphisms on \texorpdfstring{\(\mathbb{D}^n\)}{the Unit Polydisk} and \texorpdfstring{\(B^n\)}{the Unit Ball}}
A function \[\symbf{f}:\Omega\subseteq\mathbb{C}^m\to\mathbb{C}^n\] is called \textit{holomorphic} iff each of its component functions is holomorphic. It is important to allow for vector-valued outputs, since we are interested in automorphisms on complex domains in higher dimensions.

For the aforesaid purpose, we require a generalization of the Schwarz Lemma (\cref{lem:schwarz}), which is equivalent to several results of Cartan.

In preparation, we will introduce several relevant concepts.
\begin{definition}[Multi-Index Notation]\label{def:multiindex}
    A \textit{multi-index} is an \(n\)-tuple of nonnegative integers \(\symbf{k}=(k_1,\ldots,k_n)\in\mathbb{N}_{\geq0}^n\).
    We define
    \[\abs{\symbf{k}}=\sum_{j=1}^n k_j,\qquad\symbf{z}^{\symbf{k}}=\prod_{j=1}^n z_j^{k_j},\qquad\partial^{\symbf{k}}=\pdv[\symbf{k}]{}{\symbf{z}}=\prod_{j=1}^n\pdv[k_j]{z_j},\qquad\symbf{z}=\qty(z_1,\dots,z_n)\in\mathbb{C}^n.\]
\end{definition}
\begin{definition}\label{def:homogeneouspolynomial}
    A polynomial \(\symbf{\psi}:\mathbb{C}^n\to\mathbb{C}^m\) of several variables is said to be \textit{homogeneous of degree \(d\)} iff
    \[\symbf{\psi}(\lambda\symbf{z})=\lambda^d\symbf{\psi}(\symbf{z})\qquad\forall\lambda\in\mathbb{C},\symbf{z}\in\mathbb{C}^n,\]
    or equivalently, iff \(\symbf{\psi}\) can be written as
    \[\symbf{\psi}(\symbf{z})=\sum_{\abs{\symbf{k}}=d}\symbf{a}_{\symbf{k}}\symbf{z}^{\symbf{k}}\] where \(\symbf{k}\in\mathbb{N}_{\ge0}^n\) is a multi-index.
\end{definition}
\begin{proposition}\label{prop:homogeneouspolynomialderivatives}
    Let \(\symbf{\psi}:\mathbb{C}^n\to\mathbb{C}^m\) be a homogeneous polynomial of degree \(d\).
    \begin{enumerate}
        \item For any multi-index \(\symbf{\alpha}=(\alpha_1,\ldots,\alpha_n)\) with \(\abs{\symbf{\alpha}}=r\leq d\), \[\partial^{\symbf{\alpha}}\symbf{\psi}(\symbf{z})=\frac{\partial^r\symbf{\psi}}{\partial z_1^{\alpha_1}\cdots\partial z_n^{\alpha_n}}(\symbf{z})\] is a homogeneous polynomial of degree \(d-r\).\label{itm:homogeneouspolynomialderivatives_less}
        \item If \(r=d\neq 0\), then \(\partial^{\symbf{\alpha}}\symbf{\psi}\) is constant (and there exists a multi-index \(\symbf{\alpha}\) with \(\abs{\symbf{\alpha}}=d\) such that \(\partial^{\symbf{\alpha}}\symbf{\psi}\) is nonzero).\label{itm:homogeneouspolynomialderivatives_equality}
        \item If \(r>d\), then \(\partial^{\symbf{\alpha}}\symbf{\psi}\equiv 0\).\label{itm:homogeneouspolynomialderivatives_greater}
    \end{enumerate}
\end{proposition}
\begin{proof}
    Writing \(\symbf{\psi}(\symbf{z})=\sum_{\abs{\symbf{k}}=d}\symbf{a}_{\symbf{k}}\symbf{z}^{\symbf{k}}\) with coefficients \(\symbf{a}_{\symbf{k}}\in\mathbb{C}^m\), we compute
    \[\partial^{\symbf{\alpha}}\symbf{\psi}(\symbf{z})=\sum_{\abs{\symbf{k}}=d}\symbf{a}_{\symbf{k}}\prod_{j=1}^n\frac{k_j!}{(k_j-\alpha_j)!}z_j^{k_j-\alpha_j},\qquad\symbf{k}=\qty(k_1,\dots,k_n),\]
    where terms with \(k_j<\alpha_j\) vanish. For each remaining term, the total degree is
    \[(k_1-\alpha_1)+\cdots+(k_n-\alpha_n)=d-\abs{\symbf{\alpha}}.\]
    Hence, \(\partial^{\symbf{\alpha}}\symbf{\psi}\) is a homogeneous polynomial of degree \(d-\abs{\symbf{\alpha}}\), establishing \cref{itm:homogeneouspolynomialderivatives_less}.

    If \(r=d\), every surviving monomial has degree \(0\), so \(\partial^{\symbf{\alpha}}\symbf{\psi}\) is constant. Moreover, since \(\symbf{\psi}\) has degree exactly \(d\), there exists some multi-index \(\symbf{k}\) with \(\abs{\symbf{k}}=d\) and \(\symbf{a}_{\symbf{k}}\neq\symbf{0}\); choosing \(\symbf{\alpha}=\symbf{k}\) yields a nonzero constant derivative. This proves \cref{itm:homogeneouspolynomialderivatives_equality}.

    Finally, if \(r>d\), then for every term in the expansion, at least one \(k_j<\alpha_j\), so all summands vanish identically. Thus \(\partial^{\symbf{\alpha}}\symbf{\psi}\equiv 0\), verifying \cref{itm:homogeneouspolynomialderivatives_greater}.
\end{proof}
\begin{lemma}[Cartan]\label{lem:multivarcartan1}
    Let \(\Omega\subset\mathbb{C}^n\) be a bounded region, and suppose that \(\symbf{f}=\qty(f_1,\ldots,f_n):\Omega\to\Omega\) is holomorphic. If \(\exists\symbf{a}\in\Omega\) such that \(\symbf{f}\qty(\symbf{a})=\symbf{a}\) and the complex Jacobian at \(\symbf{a}\) is the identity matrix, or equivalently, if
    \begin{equation}
        \symbf{J}_{\symbf{f}}\qty(\symbf{a})=\mqty(\pdv{f_1}{z_1}&\cdots&\pdv{f_1}{z_n}\\\vdots&\ddots&\vdots\\\pdv{f_n}{z_1}&\cdots&\pdv{f_n}{z_n})(\symbf{a})=\symbf{I}=\mqty(1&\cdots&0\\\vdots&\ddots&\vdots\\0&\cdots&1),\label{eq:multivarcartan1_jacobian}
    \end{equation} then \(\symbf{f}(\symbf{z})\equiv\symbf{z}\) is the identity map.
\end{lemma}
\begin{proof}
    By \cref{thm:taylorexpansionmultivariable}, we have the expansion
    \begin{equation}
        \symbf{f}(\symbf{z})=\sum_{\abs{\symbf{k}}=0}^\infty\symbf{a}_{\symbf{k}}(\symbf{z}-\symbf{a})^{\symbf{k}}=\sum_{j=0}^\infty\symbf{\psi}_j(\symbf{z}-\symbf{a})=\symbf{a}+\sum_{j=1}^\infty\sum_{\abs{\symbf{k}}=j}\symbf{a}_{\symbf{k}}\qty(\symbf{z}-\symbf{a})^{\symbf{k}},\label{eq:multivarcartan1_taylorseries}
    \end{equation} which is absolutely convergent on some polydisk centered at \(\symbf{a}\), where \(\symbf{a}_{\symbf{k}}=\frac{\partial^{\symbf{k}}\symbf{f}(\symbf{a})}{\prod_{j=1}^n k_j!}\) and \(\symbf{k}=\qty(k_1,\dots,k_n)\). The terms have been rearranged (from absolute convergence) so that the inner summation is a homogeneous polynomial \(\symbf{\psi}_j\) with a zero at \(\symbf{z}=\symbf{a}\) and degree \(j\).

    Trivially, \(\symbf{a}_{1,0,\dots,0}=\pdv{\symbf{f}}{z_1}(\symbf{a})=\qty(1,0,\ldots,0)\) by \cref{eq:multivarcartan1_jacobian}. Similarly, \(\symbf{a}_{0,1,0,\ldots,0}=\qty(0,1,0,\ldots,0),\ldots,\symbf{a}_{0,\ldots,0,1}=\qty(0,\ldots,0,1)\). Hence, the linear homogeneous polynomial of \cref{eq:multivarcartan1_taylorseries} equals \[\qty(z_1-a_1,\dots,z_n-a_n)=\symbf{z}-\symbf{a},\] and the entire expansion is thus equal to \[\symbf{f}(\symbf{z})=\symbf{z}+\sum_{j=2}^\infty\sum_{\abs{\symbf{k}}=j}\symbf{a}_{\symbf{k}}\qty(\symbf{z}-\symbf{a})^{\symbf{k}}.\]
    Define a sequence of holomorphic functions \(\cbraces{\symbf{f}_k(\symbf{z})}_{k\in\mathbb{N}}\) by \[\symbf{f}_1=\symbf{f},\qquad\symbf{f}_{k+1}=\symbf{f}_k\circ\symbf{f}\qquad\forall k\in\mathbb{N}.\]
    Assume the existence of some \(m\in\mathbb{N}\), the smallest \(j\geq 2\) such that \(\symbf{\psi}\) is not identically zero. Because \[\symbf{f}_1(z)=\symbf{z}+\symbf{\psi}_m\qty(\symbf{z}-\symbf{a})+\sum_{j>m}\symbf{\psi}_j\qty(\symbf{z}-\symbf{a}),\] it then follows that
    \begin{align*}
        \symbf{f}_2(\symbf{z}) & =\symbf{z}+\symbf{\psi}_m(\symbf{z}-\symbf{a})+\sum_{j>m}\symbf{\psi}_j(\symbf{z}-\symbf{a})+\symbf{\psi}_m\qty(\symbf{z}-\symbf{a}+\sum_{j\geq m}\symbf{\psi}_j(\symbf{z}-\symbf{a}))+\sum_{j>m}\symbf{\psi}_j(\symbf{f}(\symbf{z})-\symbf{a}) \\
                         & =\symbf{z}+2\symbf{\psi}_m(\symbf{z}-\symbf{a})+(\text{homogeneous polynomials of degree \(>m\)})(\symbf{z}-\symbf{a}).
    \end{align*}
    Assume, for induction, that \[\symbf{f}_k(\symbf{z})=\symbf{z}+k\symbf{\psi}_m(\symbf{z}-\symbf{a})+\qty(\text{homogeneous polynomials of degree \(>m\)})(\symbf{z}-\symbf{a}).\]
    Then we have
    \begin{align*}
        \symbf{f}_{k+1}(\symbf{z}) & =\symbf{z}+\sum_{j\geq m}\symbf{\psi}_j(\symbf{z}-\symbf{a})+k\symbf{\psi}_m\qty(\symbf{z}-\symbf{a}+\sum_{j\geq m}\qty(\text{degree \(j\) hom. polynomial})(\symbf{z}-\symbf{a})) \\
                             & \quad+\sum_{j>m}\qty(\text{homogeneous polynomial of degree \(j\)})\qty(\symbf{f}(\symbf{z})-\symbf{a})                                                            \\
                             & =\symbf{z}+(k+1)\symbf{\psi}_m(\symbf{z}-\symbf{a})+\qty(\text{degree \(>m\) homogeneous polynomials})(\symbf{z}-\symbf{a}).
    \end{align*}
    Since \(\symbf{f}_k(\Omega)\subseteq\Omega\) for any \(k\), the sequence \(\cbraces{\symbf{f}_k}_{k\in\mathbb{N}}\) is uniformly bounded on \(\Omega\). By Montel's Theorem (\cref{thm:montelmultivar}), there exists a subsequence \(\cbraces{\symbf{f}_{k_l}}_{l\in\mathbb{N}}\) that converges locally uniformly to some holomorphic function \(\widetilde{\symbf{f}}\) by virtue of Weierstrass (\cref{thm:weierstrassconvergencemultivar}).

    Since \(\symbf{\psi}_m\not\equiv 0\), there exists \(\symbf{\alpha}\) satisfying \(\abs{\symbf{\alpha}}=m\) such that \[\partial^{\symbf{\alpha}}\symbf{\psi}_m\equiv\symbf{c}\neq\symbf{0}\] is a nonzero constant by \cref{prop:homogeneouspolynomialderivatives}. Consequently, \[\partial^{\symbf{\alpha}}\qty(\text{homogeneous polynomials of degree \(>m\)})(\symbf{z}-\symbf{a})\] is a homogeneous polynomial with degree \(\geq 1\) and thus vanishes as \(\symbf{z}\to\symbf{a}\). Similarly, \(\symbf{z}\mapsto\symbf{z}\) is homogeneous with degree \(1<m\) and thus \(\partial^{\symbf{\alpha}}z\) vanishes. Therefore, \[\partial^{\symbf{\alpha}}\symbf{f}_k(\symbf{a})=k\symbf{c},\] which diverges as \(k\to\infty\). Weierstrass' Convergence Theorem (\cref{thm:weierstrassconvergencemultivar}) gives that \(\partial^{\symbf{\alpha}}\symbf{f}_{k_l}(\symbf{a})\to\partial^{\symbf{\alpha}}\widetilde{\symbf{f}}(\symbf{a})\) which must be finite by holomorphy, contradicting the divergence. Hence, the assumed value for \(m\) cannot exist and hence \(\symbf{\psi}_j\equiv 0\) for all \(j\geq 2\). Thus, \(\symbf{f}(\symbf{z})\equiv\symbf{z}\) on some polydisk centered at \(\symbf{a}\). By the Identity Theorem (\cref{thm:identitymultivar}), \(\symbf{f}(\symbf{z})\equiv\symbf{z}\) on \(\Omega\).
\end{proof}
\begin{definition}[Reinhardt Domain]\label{def:reinhardtdomain}
    An open domain \(\Omega\subseteq\mathbb{C}^n\) is a \textit{Reinhardt domain} centered at \(\symbf{a}=\qty(a_1,\ldots,a_n)\in\mathbb{C}^n\) iff \(\forall\symbf{\zeta}=\qty(\zeta_1,\dots,\zeta_n)\in\Omega\), \[\cbraces{\qty(z_1,\ldots,z_n)\in\mathbb{C}^n}{\abs{z_k-a_k}=\abs{\zeta_k-a_k},1\leq k\leq n}\] is fully contained in \(\Omega\). In other words, \(\Omega\) is invariant under all rotations about the center \(\symbf{a}\) in each coordinate.
\end{definition}
\begin{definition}\label{def:completereinhardtdomain}
    A Reinhardt domain \(\Omega\subseteq\mathbb{C}^n\) centered at \(\symbf{a}=\qty(a_1,\ldots,a_n)\) is said to be \textit{complete} iff \(\forall\symbf{\zeta}=\qty(\zeta_1,\dots,\zeta_n)\in\Omega\), the polydisk \[\cbraces{\qty(z_1,\dots,z_n)\in\mathbb{C}^n}{\abs{z_k-a_k}\leq\abs{\zeta_k-a_k},1\leq k\leq n}\] is contained in \(\Omega\).
\end{definition}
\begin{definition}[Circular Domain]\label{def:circulardomain}
    An open domain \(\Omega\subseteq\mathbb{C}^n\) is a \textit{circular domain} centered at \(\symbf{a}\in\mathbb{C}^n\) iff \(\forall\symbf{\zeta}\in\Omega\), \[\cbraces{\symbf{a}+\ee^{\ii\theta}\qty(\symbf{\zeta}-\symbf{a})}{0\leq\theta<2\uppi}\] is fully contained in \(\Omega\).
\end{definition}
\begin{definition}\label{def:completecirculardomain}
    A circular domain \(\Omega\subseteq\mathbb{C}^n\) centered at \(\symbf{a}=\qty(a_1,\ldots,a_n)\) is said to be \textit{complete} iff \(\forall\symbf{\zeta}\in\Omega\), \[\cbraces{\symbf{a}+\mu\qty(\symbf{\zeta}-\symbf{a})}{\forall\mu\in\overline{\mathbb{D}}}\] is contained in \(\Omega\).
\end{definition}
\begin{proposition}\label{prop:jacobianchainrule}
    Let \(U_0\subseteq\mathbb{C}^{n_0}, U_1\subseteq\mathbb{C}^{n_1}, U_2\subseteq\mathbb{C}^{n_2}\) be open domains with \(n_i\geq 1\) for each \(i\), and let \(\symbf{f}:U_1\to U_2\) and \(\symbf{g}:U_0\to U_1\) be holomorphic maps. Define the composition \(\symbf{h}:U_0\to U_2\) by \(\symbf{h}(\symbf{z})=\symbf{f}(\symbf{g}(\symbf{z}))\). Then for every \(\symbf{z}\in U_0\), the complex Jacobian matrix of \(\symbf{h}\) at \(\symbf{z}\) is
    \[\symbf{J}_{\symbf{h}}(\symbf{z})=\symbf{J}_{\symbf{f}}(\symbf{g}(\symbf{z}))\cdot\symbf{J}_{\symbf{g}}(\symbf{z}).\]
\end{proposition}
\begin{proof}
    Fix \(\symbf{z}\in U_0\) and let \(\symbf{w}=\symbf{g}(\symbf{z})\in U_1\). Write \[\symbf{h}(\symbf{z})=(h_1(\symbf{z}),\dots,h_{n_2}(\symbf{z})),\] where each \(h_l:U_0\to\mathbb{C}\) is holomorphic for \(l=1,\dots,n_2\). Similarly, write \[\symbf{g}(\symbf{z})=(g_1(\symbf{z}),\dots,g_{n_1}(\symbf{z})),\qquad\symbf{f}(\symbf{z})=(f_1(\symbf{z}),\dots,f_{n_2}(\symbf{z})),\] where each \(g_p:U_0\to\mathbb{C}\) and each \(f_l:U_1\to\mathbb{C}\) is holomorphic for \(p=1,\dots,n_1\) and \(l=1,\dots,n_2\). Then \(h_l(\symbf{z})=f_l(\symbf{g}(\symbf{z}))\) for each \(l\). By the chain multivariable rule, the complex Jacobian of \(\symbf{h}\) at \(\symbf{z}\) is the \(n_2\times n_0\) matrix
    \begin{align*}
        \symbf{J}_{\symbf{h}}      & =\mqty(\pdv{h_1}{z_1}             & \cdots             & \pdv{h_1}{z_{n_0}}             \\\vdots&\ddots&\vdots\\\pdv{h_{n_2}}{z_1}&\cdots&\pdv{h_{n_2}}{z_{n_0}})=\mqty(\sum_{p=1}^{n_1}\pdv{f_1}{g_p}\qty(\symbf{g})\pdv{g_p}{z_1}&\cdots&\sum_{p=1}^{n_1}\pdv{f_1}{g_p}\qty(\symbf{g})\pdv{g_p}{z_{n_0}}\\\vdots&\ddots&\vdots\\\sum_{p=1}^{n_1}\pdv{f_{n_2}}{g_p}\qty(\symbf{g})\pdv{g_p}{z_1}&\cdots&\sum_{p=1}^{n_1}\pdv{f_{n_2}}{g_p}\qty(\symbf{g})\pdv{g_p}{z_{n_0}})\\
                             & =\mqty(\pdv{f_1}{g_1}\qty(\symbf{g}) & \cdots             & \pdv{f_1}{g_{n_1}}\qty(\symbf{g}) \\\vdots&\ddots&\vdots\\\pdv{f_{n_2}}{g_1}\qty(\symbf{g}) & \cdots & \pdv{f_{n_2}}{g_{n_1}}\qty(\symbf{g}))
        \mqty(\pdv{g_1}{z_1} & \cdots                            & \pdv{g_1}{z_{n_0}}                                  \\\vdots&\ddots&\vdots\\\pdv{g_{n_1}}{z_1}&\cdots&\pdv{g_{n_1}}{z_{n_0}})=\symbf{J}_{\symbf{f}}(\symbf{g})\cdot\symbf{J}_{\symbf{g}}.\qedhere
    \end{align*}
\end{proof}
\begin{lemma}[Cartan]\label{lem:multivarcartan2}
    Let \(\Omega\subset\mathbb{C}^n\) be a bounded complete circular domain centered at \(\symbf{0}\), and suppose that \(\symbf{f}=\qty(f_1,\ldots,f_n):\Omega\to\Omega\) is a biholomorphism. If \(\symbf{f}(\symbf{0})=\symbf{0}\), then \(\symbf{f}\) is linear.
\end{lemma}
\begin{proof}
    Let \(\symbf{\rho}_\theta(\symbf{z})=\ee^{\ii\theta}\symbf{z}\) for all \(\theta\in\mathbb{R}\) and suppose that \(\symbf{\varphi}=\symbf{\rho}_{-\theta}\circ\symbf{f}^{-1}\circ\symbf{\rho}_\theta\circ\symbf{f}\). By \cref{prop:jacobianchainrule}, we must have that
    \begin{align*}
        \symbf{J}_{\symbf{\varphi}}(\symbf{z}) & =\symbf{J}_{\symbf{\rho}_{-\theta}}\qty(\symbf{f}^{-1}\circ\symbf{\rho}_\theta\circ\symbf{f}(\symbf{z}))\cdot\symbf{J}_{\symbf{\rho}_{-\theta}\circ\symbf{f}^{-1}}\qty(\symbf{\rho}_\theta\circ\symbf{f}(\symbf{z}))\cdot\symbf{J}_{\symbf{\rho}_{-\theta}\circ\symbf{f}^{-1}\circ\symbf{\rho}_\theta}\qty(\symbf{f}(\symbf{z}))\cdot\symbf{J}_{\symbf{\rho}_{-\theta}\circ\symbf{f}^{-1}\circ\symbf{\rho}_\theta\circ\symbf{f}}\qty(\symbf{z})              \\
        \symbf{J}_{\symbf{\varphi}}(\symbf{0}) & =\mqty(\ee^{-\ii\theta}                                                                                                                                                                                                                                                                                                                                                         & \cdots & 0 \\\vdots&\ddots&\vdots\\0&\cdots&\ee^{-\ii\theta})\cdot\symbf{J}_{\symbf{f}^{-1}}(\symbf{0})\cdot\mqty(\ee^{\ii\theta}&\cdots&0\\\vdots&\ddots&\vdots\\0&\cdots&\ee^{\ii\theta})\cdot\symbf{J}_{\symbf{f}}(\symbf{0})=\ee^{-\ii\theta}\ee^{\ii\theta}\qty(\symbf{J}_{\symbf{f}^{-1}}\cdot\symbf{J}_{\symbf{f}})(\symbf{0})=\symbf{I}.
    \end{align*}
    By \cref{lem:multivarcartan1}, \(\symbf{\varphi}(\symbf{z})\equiv\symbf{z}\) on \(\Omega\). Hence, \(\symbf{f}\circ\symbf{\rho}_\theta=\symbf{\rho}_\theta\circ\symbf{f}\) for all \(\theta\in\mathbb{R}\). Together with \cref{thm:taylorexpansionmultivariable}, write
    \begin{equation}
        \symbf{f}(\symbf{z})=\sum_{\symbf{k}:\abs{\symbf{k}}=0}^\infty\symbf{a}_{\symbf{k}}\symbf{z}^{\symbf{k}}\label{eq:multivarcartan2_taylorseries}
    \end{equation} on a polydisk centered at \(\symbf{0}\). Thus,
    \begin{equation*}
        \symbf{f}\circ\symbf{\rho}(\symbf{z})=\sum_{\symbf{k}:\abs{\symbf{k}}=0}^\infty\symbf{a}_{\symbf{k}}\qty(\ee^{\ii\theta}\symbf{z})^{\symbf{k}}=\sum_{\symbf{k}:\abs{\symbf{k}}=0}^\infty\symbf{a}_{\symbf{k}}\ee^{\ii\theta\abs{\symbf{k}}}\symbf{z}^{\symbf{k}}.
    \end{equation*}
    On the other hand, composing with \(\symbf{\rho}_\theta\) with \cref{eq:multivarcartan2_taylorseries} gives
    \begin{equation*}
        \symbf{\rho}_\theta\circ\symbf{f}(\symbf{z})=\ee^{\ii\theta}\sum_{\symbf{k}:\abs{\symbf{k}}=0}^\infty\symbf{a}_{\symbf{k}}\symbf{z}^{\symbf{k}}=\sum_{\symbf{k}:\abs{\symbf{k}}=0}^\infty\symbf{a}_{\symbf{k}}\ee^{\ii\theta}\symbf{z}^{\symbf{k}}.
    \end{equation*}
    Hence, by the uniqueness of power series expansions, we must either have that \(\symbf{a}_{\symbf{k}}=\symbf{0}\), \(\ee^{\ii\theta}\equiv\ee^{\ii\theta\abs{\symbf{k}}}\), or equivalently, that \[\abs{\symbf{k}}\equiv 1\pmod{2\uppi}.\] This is only possible when \(\abs{\symbf{k}}=1\) by irrationality, and thus \(\symbf{a}_{\symbf{k}}=\symbf{0}\) for all \(\abs{\symbf{k}}\neq 1\). Therefore, \(\symbf{f}\) must be linear.
\end{proof}
\begin{remark}
    If \(n=1\), then \(\Omega=D(0,R)\) for some \(R>0\) and any automorphism \(f\) with a fixed point \(0\) is a rotation in the form of \(z\mapsto\ee^{\ii\theta}z\), hence linear, the effective statement of the Schwarz Lemma (\cref{lem:schwarz}).
\end{remark}
\begin{theorem}[The Holomorphic Automorphism Group on \(\mathbb{D}^n\)]\label{thm:holomorphicautomorphismgrouponpolydisk}
    The holomorphic automorphism group of the polydisk \(\mathbb{D}^n\) consists solely of biholomorphisms in the form of
    \begin{equation}
        \symbf{z}=\qty(z_1,\ldots,z_n)\mapsto\symbf{P}\qty(\ee^{\ii\theta_1}\frac{z_1-a_1}{1-\overline{a_1}z_1},\ldots,\ee^{\ii\theta_n}\frac{z_n-a_n}{1-\overline{a_n}z_n}),\label{eq:holomorphicautomorphismgrouponpolydisk_statement}
    \end{equation} where \(\symbf{P}\) is a \(n\times n\) permutation matrix (for coordinate permutations), \(\qty(\theta_1,\dots,\theta_n)\in\mathbb{R}^n\), and \(\qty(a_1,\ldots,a_n)\in\mathbb{D}^n\). Moreover, every such map is indeed an automorphism.
\end{theorem}
\begin{proof}
    Let \(\symbf{f}\in\Aut\qty(\mathbb{D}^n)\) be arbitrary, and set \(\symbf{\alpha}=\qty(\alpha_1,\ldots,\alpha_n)=\symbf{f}(\symbf{0})\). Define the Möbius transformation \(\symbf{\varphi}\qty(z_1,\ldots,z_n)=\qty(\frac{z_1-\alpha_1}{1-\overline{\alpha_1}z_1},\ldots,\frac{z_n-\alpha_n}{1-\overline{\alpha_n}z_n})\in\Aut\qty(\mathbb{D}^n)\). It follows that \(\symbf{\varphi}\circ\symbf{f}(\symbf{0})=\symbf{0}\) and \(\symbf{\varphi}\circ\symbf{f}\in\Aut\qty(\mathbb{D}^n)\).

    By \cref{lem:multivarcartan2}, the map \(\symbf{\varphi}\circ\symbf{f}\) is linear, so \(\symbf{\varphi}\circ\symbf{f}\qty(\symbf{z})=\symbf{A}\symbf{z}\) for some invertible constant matrix \(\symbf{A}=\mqty(\zeta_{1,1}&\cdots&\zeta_{1,n}\\\vdots&\ddots&\vdots\\\zeta_{n,1}&\cdots&\zeta_{n,n})\). Thus,
    \[\abs{\sum_{j=1}^n\zeta_{k,j}z_j}<1\qquad\forall\symbf{z}\in\mathbb{D}^n,\ \forall k\in\cbraces{1,\ldots,n},\]
    which implies \(\abs{\zeta_{k,j}}\leq1\) for all \(j,k\in\cbraces{1,\ldots,n}\) (for if \(\abs{\zeta_{k,j}}>1\), then choosing \(z_j=\frac1{\abs{\zeta_{k,j}}}+\varepsilon\) with \(0<\varepsilon<1-\frac1{\abs{\zeta_{k,j}}}\) and \(z_l=0\) for \(l\neq j\) yields a contradiction).

    For each \(j\in\cbraces{1,\ldots,n}\), define the sequence \(\cbraces{\symbf{z}_{j,k}}_{k\in\mathbb{N}}\) by
    \[\symbf{z}_{j,k}=\qty(z_{j,k,1},\dots,z_{j,k,n})=\qty(\qty(1-\frac{1}{k})\frac{\abs{\zeta_{j,1}}}{\zeta_{j,1}},\dots,\qty(1-\frac{1}{k})\frac{\abs{\zeta_{j,n}}}{\zeta_{j,n}})\in\mathbb{D}^n\qquad\forall k\in\mathbb{N},\]
    where we informally let \(\frac{\abs{\zeta_{j,i}}}{\zeta_{j,i}}=0\) if \(\zeta_{j,i}=0\). Then, for all \(j\in\cbraces{1,\ldots,n}\) and \(k\in\mathbb{N}\),
    \[\qty(\symbf{\varphi}\circ\symbf{f})\qty(\symbf{z}_{j,k})=\qty(1-\frac{1}{k})\qty(\sum_{i=1}^n\frac{\abs{\zeta_{j,i}}}{\zeta_{j,i}}\zeta_{1,i},\dots,\sum_{i=1}^n\frac{\abs{\zeta_{j,i}}}{\zeta_{j,i}}\zeta_{j,i},\dots,\sum_{i=1}^n\frac{\abs{\zeta_{j,i}}}{\zeta_{j,i}}\zeta_{n,i})\in\mathbb{D}^n.\]
    In particular, the \(j\)-th component is
    \[\qty(1-\frac{1}{k})\sum_{i=1}^n\abs{\zeta_{j,i}}\in\mathbb{D}.\]
    As \(k\to\infty\),
    \begin{equation}
        \sum_{i=1}^n\abs{\zeta_{j,i}}\in\overline{\mathbb{D}}\qquad\forall j\in\cbraces{1,\ldots,n}.\label{eq:holomorphicautomorphismgrouponpolydisk_absolutesumestimate}
    \end{equation}
    Now consider, for each \(j\in\cbraces{1,\ldots,n}\), the sequence \(\symbf{z}_{j,k}=\qty(0,\ldots,0,1-\frac{1}{k},0,\ldots,0)\), where \(1-\frac{1}{k}\) is in the \(j\)-th position. Then
    \[\symbf{\varphi}\circ\symbf{f}\qty(\symbf{z}_{j,k})=\qty(1-\frac{1}{k})\qty(\zeta_{1,j},\ldots,\zeta_{n,j}).\]
    As \(k\to\infty\), \(\symbf{z}_{j,k}\to\symbf{e}_j\in\partial\qty(\mathbb{D}^n)\) (the \(j\)-th unit basis vector), so the limit
    \[\qty(\zeta_{1,j},\ldots,\zeta_{n,j})\in\partial\qty(\mathbb{D}^n)\]
    holds by continuity (extending across the boundary in this direction). Thus, \(\max_{i\in\cbraces{1,\ldots,n}}\abs{\zeta_{i,j}}=1\). Combined with \cref{eq:holomorphicautomorphismgrouponpolydisk_absolutesumestimate}, this forces exactly one entry in the \(j\)-th column of \(\symbf{A}\) to have absolute value \(1\) (of the form \(\ee^{\ii\theta_j}\)), with all others zero.

    Invertibility of \(\symbf{A}\) ensures each column has at least one nonzero entry, so \(\symbf{A}\) is a monomial matrix, which factors to
    \[\symbf{A}=\symbf{P}\operatorname{diag}\qty(\ee^{\ii\theta_1},\ldots,\ee^{\ii\theta_n})\]
    for some permutation matrix \(\symbf{P}\). Therefore,
    \[\symbf{f}\qty(\symbf{z})=\symbf{\varphi}^{-1}\circ\qty(\symbf{P}\operatorname{diag}\qty(\ee^{\ii\theta_1},\ldots,\ee^{\ii\theta_n})\symbf{z}).\]
    Let \(\sigma:\mathbb{N}_{\leq n}\to\mathbb{N}_{\leq n}\) be the permutation induced by \(\symbf{P}\). The map \(\symbf{A}\) multiplies the \(m\)-th input coordinate \(z_m\) by \(\ee^{\ii\theta_m}\) and permutes to place it in the \(\sigma(m)\)-th output position, so the \(\sigma(m)\)-th coordinate of \(\symbf{A}\symbf{z}\) is \(\ee^{\ii\theta_m}z_m\). Applying \(\symbf{\varphi}^{-1}\) componentwise then gives, for the \(k\)-th output coordinate,
    \[\qty(\symbf{f}\qty(\symbf{z}))_k=\varphi_{\alpha_k}^{-1}\qty(\ee^{\ii\theta_{\sigma^{-1}(k)}}z_{\sigma^{-1}(k)})=\frac{\ee^{\ii\theta_{\sigma^{-1}(k)}}z_{\sigma^{-1}(k)}+\alpha_k}{1+\overline{\alpha_k}\ee^{\ii\theta_{\sigma^{-1}(k)}}z_{\sigma^{-1}(k)}}.\]
    Set \(a_{\sigma^{-1}(k)}=-\alpha_k\ee^{-\ii\theta_{\sigma^{-1}(k)}}\in\mathbb{D}\). Then
    \[\qty(\symbf{f}\qty(\symbf{z}))_k=\frac{\ee^{\ii\theta_{\sigma^{-1}(k)}}z_{\sigma^{-1}(k)}+\alpha_k}{1+\overline{\alpha_k}\ee^{\ii\theta_{\sigma^{-1}(k)}}z_{\sigma^{-1}(k)}}=\ee^{\ii\theta_{\sigma^{-1}(k)}}\frac{z_{\sigma^{-1}(k)}-a_{\sigma^{-1}(k)}}{1-\overline{a_{\sigma^{-1}(k)}}z_{\sigma^{-1}(k)}}.\] Hence, \[\qty(\symbf{f}\qty(\symbf{z}))_{\sigma(k)}=\ee^{\ii\theta_{k}}\frac{z_{k}-a_k}{1-\overline{a_k}z_{k}}\Longleftrightarrow\symbf{f}(\symbf{z})=\symbf{P}\qty(\ee^{\ii\theta_1}\frac{z_1-a_1}{1-\overline{a_1}z_1},\ldots,\ee^{\ii\theta_n}\frac{z_n-a_n}{1-\overline{a_n}z_n}),\]
    as in \cref{eq:holomorphicautomorphismgrouponpolydisk_statement}. Finally, each automorphism of this form lies in \(\Aut\qty(\mathbb{D}^n)\) trivially.
\end{proof}
\begin{definition}
    The \textit{conjugate transpose} or \textit{Hermitian transpose} of a complex matrix \(\symbf{U}\) is defined as \(\symbf{U}^\dagger=\overline{\symbf{U}}^\top\), or the transpose of the matrix with each element replaced with its complex conjugate.
\end{definition}
\begin{definition}
    A matrix \(\symbf{U}\) is said to be \textit{unitary} iff its inverse is its conjugate transpose, or iff \(\symbf{U}^\dagger\symbf{U}=\symbf{U}\symbf{U}^\dagger=\symbf{I}\).
\end{definition}
\begin{theorem}[\textsc{Spectral Theorem}]\label{thm:unitaryspectraltheorem}
    For any unitary matrix \(\symbf{U}\), there exists a unitary matrix \(\symbf{V}\) such that \(\symbf{U}=\symbf{VD}\symbf{V}^\dagger\), where \(\symbf{D}\) is a diagonal matrix whose diagonal entries are all of unit modulus.
\end{theorem}
\begin{proof}
    Because \(\norm{\symbf{U}\symbf{z}}^2=\symbf{z}^\dagger\symbf{U}^\dagger\symbf{U}\symbf{z}=\norm{\symbf{z}}^2\) for any \(\symbf{z}\in\mathbb{C}^n\), any eigenvalue \(\lambda_1\) (existence given by the Fundamental Theorem of Algebra in \cref{thm:fundamentaltheoremofalgebra} on the characteristic equation) of \(\symbf{U}\) must satisfy \[\symbf{U}\symbf{v}_1=\lambda_1\symbf{v}_1\implies\norm{\symbf{U}\symbf{v}_1}=\norm{\symbf{v}_1}=\abs{\lambda_1}\norm{\symbf{v}_1}\implies\abs{\lambda_1}=1,\]
    where \(\norm{\symbf{v}_1}=1\) is the corresponding eigenvector in \(\mathbb{C}^n\). Then \[\symbf{U}^{-1}\symbf{U}\symbf{v}_1=\symbf{U}^{-1}\lambda_1\symbf{v}_1=\lambda_1\symbf{U}^{-1}\symbf{v}_1\implies\frac1\lambda_1\symbf{v}_1=\symbf{U}^{-1}\symbf{v}_1\implies\overline{\lambda_1}\symbf{v}_1=\symbf{U}^\dagger\symbf{v}_1.\]
    Let \(\symbf{v}_1^\perp=\cbraces{\symbf{w}:\symbf{v}_1^\dagger\symbf{w}=\symbf{0}}\subset\mathbb{C}^n\) be an \((n-1)\)-dimensional subspace. For any \(\symbf{w}\in\symbf{v}_1^\perp\),
    \[\symbf{v}_1^\dagger\symbf{Uw}=\qty(\symbf{U}^\dagger\symbf{v}_1)^\dagger\symbf{w}=\qty(\overline{\lambda_1}\symbf{v}_1)^\dagger\symbf{w}=\lambda_1\symbf{v}_1^\dagger\symbf{w}=0,\]
    so \(\symbf{Uw}\in\symbf{v}_1^\perp\). Hence \(\symbf{v}_1^\perp\) is invariant under \(\symbf{U}\). The restriction of \(\symbf{U}\) to \(\symbf{v}_1^\perp\), \(\symbf{U}{\restriction_{\symbf{v}_1^\perp}}\), yields another eigenvalue \(\lambda_2\in\partial\mathbb{D}\) with eigenvector \(\symbf{v}_2\in\symbf{v}_1^\perp\) satisfying \(\abs{\lambda_2}=1\) and \(\norm{\symbf{v}_2}=1\). Similarly, we may define \(\symbf{v}_2^\perp\subset\symbf{v}_1^\perp\), which is an \((n-2)\)-dimensional subspace invariant under \(\symbf{U}\). Repeating this process inductively, we obtain an orthonormal basis \(\cbraces{\symbf{v}_1,\ldots,\symbf{v}_n}\) of eigenvectors of \(\symbf{U}\) with corresponding eigenvalues \(\lambda_1,\ldots,\lambda_n\in\partial\mathbb{D}\). Setting \[\symbf{V}=\mqty(\symbf{v}_1&\cdots&\symbf{v}_n),\qquad\symbf{D}=\operatorname{diag}\qty(\lambda_1,\ldots,\lambda_n)\] gives that \[\symbf{V}^\dagger\symbf{UV}=\symbf{V}^\dagger\mqty(\symbf{U}\symbf{v}_1&\cdots&\symbf{U}\symbf{v}_n)=\symbf{V}^\dagger\mqty(\lambda_1\symbf{v}_1&\cdots&\lambda_n\symbf{v}_n)=\symbf{V}^\dagger\symbf{VD}.\]
    The \(k\)-th diagonal entry of \(\symbf{V}^\dagger\symbf{V}\) is equal to \(\symbf{v}_k^\dagger\symbf{v}_k=\norm{\symbf{v}_k}^2=1\), while the non-diagonal entries correspond to \(\symbf{v}_k^\dagger\symbf{v}_l\) for some \(k\neq l\), which vanish by orthogonality in construction. Thus, \(\symbf{V}^\dagger\symbf{V}=\symbf{I}\) (unitary) and \(\symbf{VD}\symbf{V}^\dagger=\symbf{U}\).
\end{proof}
A \textit{unitary transformation} is a map in the form of \(\symbf{z}\mapsto\symbf{U}\symbf{z}\), where \(\symbf{U}\) is a unitary matrix.
\begin{proposition}\label{prop:unitballsimpleautomorphism}
    For any \(a\in\mathbb{D}\),
    \begin{equation}
        \symbf{w}=\qty(w_1,\dots,w_n)=\symbf{\varphi}_{a}(\symbf{z})=\qty(\frac{z_1-a}{1-\overline{a}z_1},z_2\frac{\sqrt{1-\abs{a}^2}}{1-\overline{a}z_1},z_3\frac{\sqrt{1-\abs{a}^2}}{1-\overline{a}z_1},\ldots,z_n\frac{\sqrt{1-\abs{a}^2}}{1-\overline{a}z_1})\label{eq:unitballsimpleautomorphism_statement}
    \end{equation} lies in \(\Aut\qty(B^n)\), where \(\symbf{z}=\qty(z_1,\dots,z_n)\). Moreover, \(\symbf{\varphi}_a^{-1}=\symbf{\varphi}_{-a}\).
\end{proposition}
\begin{proof}
    For \(\symbf{z}=\qty(z_1,\dots,z_n)\in B^n\), because \(\sum_{k=2}^n\abs{z_k}^2<1-\abs{z_1}^2\),
    \begin{align*}
        \norm{\symbf{\varphi}_a(\symbf{z})}^2 & =\frac{1}{\abs{1-\overline{a}z_1}^2}\qty[\abs{z_1-a}^2+\sum_{k=2}^n\qty(1-\abs{a}^2)\abs{z_k}^2]                                                 \\
                                         & <\frac{1}{\qty(1-\overline{a}z_1)\qty(1-a\overline{z_1})}\qty[\qty(z_1-a)\qty(\overline{z_1}-\overline{a})+\qty(1-\abs{z_1}^2)\qty(1-\abs{a}^2)] \\
                                         & =\frac{\abs{z_1}^2+\abs{a}^2-2\Re\qty(\overline{a}z_1)+1+\abs{az_1}^2-\abs{a}^2-\abs{z_1}^2}{1+\abs{az_1}^2-2\Re\qty(\overline{a}z_1)}=1.
    \end{align*}
    Hence, \(\symbf{\varphi}_a\) maps \(B^n\) to \(B^n\). A simple calculation shows that \[w_1=\frac{z_1-a}{1-\overline{a}z_1}\implies z_1=\frac{w_1+a}{1+\overline{a}w_1},\quad z_k=w_k\frac{1-\overline{a}z_1}{\sqrt{1-\abs{a}^2}}=w_k\frac{\sqrt{1-\abs{a}^2}}{1+\overline{a}w_1},\] and hence \(\symbf{\varphi}_a\) is bijective, admitting the inverse \(\symbf{\varphi}_{-a}\). Therefore, \(\symbf{\varphi}_a\in\Aut\qty(B^n)\).
\end{proof}
\begin{proposition}\label{prop:unitballautomorphismfixedpointatzero}
    A function \(\symbf{f}\) is a unitary transformation iff \(\symbf{f}\in\Aut\qty(B^n)\) and \(\symbf{f}(\symbf{0})=\symbf{0}\).
\end{proposition}
\begin{proof}
    Because \(B^n\) is a bounded complete circular domain centered at \(\symbf{0}\), from \cref{lem:multivarcartan2} we have that \(\symbf{f}\equiv\symbf{U}\) for some constant invertible matrix \[\symbf{U}=\mqty(\zeta_{1,1}&\cdots&\zeta_{1,n}\\\vdots&\ddots&\vdots\\\zeta_{n,1}&\cdots&\zeta_{n,n}).\]
    Similarly, we have \(\symbf{f}^{-1}=\symbf{U}^{-1}\), so \(\norm{\symbf{z}}=\norm{\symbf{U}^{-1}\symbf{U}\symbf{z}}\). Observe that \[\norm{\frac1{\norm{\symbf{z}}}\symbf{f}\qty(\symbf{z})}=\norm{\symbf{f}\qty(\frac{\symbf{z}}{\norm{\symbf{z}}})}=1\implies\norm{\symbf{Uz}}^2=\norm{\symbf{z}}^2.\]
    More explicitly, we have \[\symbf{U}\symbf{z}=\qty(\sum_{k=1}^n\zeta_{1,k}z_k,\ldots,\sum_{k=1}^n\zeta_{n,k}z_k)\implies\norm{\symbf{Uz}}^2=\sum_{j=1}^n\abs{\sum_{k=1}^n\zeta_{j,k}z_k}^2.\]
    Letting \(\symbf{z}=\symbf{e}_i\) (\(1\leq i\leq n\)) be the \(i\)-th unit basis vector, we obtain
    \begin{equation}
        \norm{\symbf{Uz}}=1=\norm{\qty(\zeta_{1,i},\ldots,\zeta_{n,i})}^2=\sum_{k=1}^n\abs{\zeta_{k,i}}^2=\sum_{k=1}^n\zeta_{k,i}\overline{\zeta_{k,i}}.\label{eq:unitballautomorphismfixedpointatzero_diagonalentries}
    \end{equation}
    Letting \(\symbf{z}=\frac{\sqrt{2}}{2}\qty(\symbf{e}_i+\symbf{e}_j)\) (\(i\neq j\)), we have
    \begin{align*}
        \norm{\symbf{Uz}}=1 & =\frac12\norm{\qty(\zeta_{1,i}+\zeta_{1,j},\dots,\zeta_{n,i}+\zeta_{n,j})}^2=\frac12\sum_{k=1}^n\abs{\zeta_{k,i}+\zeta_{k,j}}^2                                         \\
                         & =\frac12\sum_{k=1}^n\qty(\abs{\zeta_{k,i}^2}+\abs{\zeta_{k,j}^2}+2\Re\qty(\zeta_{k,i}\overline{\zeta_{k,j}}))=1+\sum_{k=1}^n\Re\qty(\zeta_{k,i}\overline{\zeta_{k,j}}),
    \end{align*}
    which implies that \(\sum_{k=1}^n\Re\qty(\zeta_{k,i}\overline{\zeta_{k,j}})=0\). Similarly, letting \(\symbf{z}=\frac{\sqrt{2}}{2}\qty(\symbf{e}_i+\ii\symbf{e}_j)\) gives
    \begin{align*}
        \norm{\symbf{Uz}}=1 & =\frac12\norm{\qty(\zeta_{1,i}+\ii\zeta_{1,j},\dots,\zeta_{n,i}+\ii\zeta_{n,j})}^2=\frac12\sum_{k=1}^n\abs{\zeta_{k,i}+\ii\zeta_{k,j}}^2                                \\
                         & =\frac12\sum_{k=1}^n\qty(\abs{\zeta_{k,i}^2}+\abs{\zeta_{k,j}^2}+2\Im\qty(\zeta_{k,i}\overline{\zeta_{k,j}}))=1+\sum_{k=1}^n\Im\qty(\zeta_{k,i}\overline{\zeta_{k,j}}),
    \end{align*}
    which implies that \(\sum_{k=1}^n\Im\qty(\zeta_{k,i}\overline{\zeta_{k,j}})=0\). Therefore, by \cref{eq:unitballautomorphismfixedpointatzero_diagonalentries}, for all \(i,j\in\cbraces{1,\ldots,n}\), observe that
    \[\qty(\symbf{U}^\dagger\symbf{U})_{j,i}=\sum_{k=1}^n\zeta_{k,i}\overline{\zeta_{k,j}}=\delta_{j,i},\] where \(\delta_{j,i}\) is the Kronecker delta. Hence, we have \(\symbf{U}^\dagger\symbf{U}=\symbf{I}\), and thus \(\symbf{U}\) is unitary.

    Conversely, if \(\symbf{f}(\symbf{z})=\symbf{U}\symbf{z}\) for some unitary matrix \(\symbf{U}\), then for any \(\symbf{z}\in B^n\), \[\norm{\symbf{f}(\symbf{z})}^2=\norm{\symbf{Uz}}^2=\symbf{z}^\dagger\symbf{U}^\dagger\symbf{U}\symbf{z}=\symbf{z}^\dagger\symbf{z}=\norm{z}^2,\]
    so \(\symbf{f}\) maps \(B^n\) to \(B^n\). Since \(\symbf{U}\) is invertible with unitary inverse \(\symbf{U}^\dagger\), the map \(\symbf{f}\) is bijective with inverse \(\symbf{f}^{-1}(\symbf{w})=\symbf{U}^\dagger\symbf{w}\), which also maps \(B^n\) to \(B^n\). Therefore, \(\symbf{f}\in\Aut\qty(B^n)\) and \(\symbf{f}(\symbf{0})=\symbf{0}\).
\end{proof}
\begin{definition}
    A group \(G\) (under juxtaposition) is said to be \textit{divisible} iff for every \(g\in G\) and every positive integer \(n\), there exists some \(h\in G\) such that \(h^n=g\).
\end{definition}
\begin{proposition}\label{prop:groupdivisibilitypreservedunderisomorphisms}
    The divisibility of a group is preserved under group isomorphisms.
\end{proposition}
\begin{proof}
    Let \(\varphi:G\to H\) be a group isomorphism between groups \(G\) and \(H\) with juxtaposition.

    Assume \(G\) is divisible. Fix \(y\in H\) and a positive integer \(n\). Since \(\varphi\) is bijective there is \(x\in G\) with \(\varphi(x)=y\). By divisibility of \(G\) there exists \(h\in G\) with \(h^n=x\). Applying \(\varphi\) and using the homomorphism property gives
    \[\varphi(h)^n=\varphi\qty(h^n)=\varphi(x)=y.\]
    Thus every element of \(H\) has an \(n\)-th root, so \(H\) is divisible.

    Conversely, if \(H\) is divisible then the same argument applied to \(\varphi^{-1}:H\to G\) shows \(G\) is divisible. Therefore divisibility is preserved under group isomorphisms.
\end{proof}
\begin{theorem}[The Holomorphic Automorphism Group on \(B^n\)]\label{thm:holomorphicautomorphismgrouponunitball}
    The holomorphic automorphism group \(\Aut\qty(B^n)\) consists solely of biholomorphisms in the form of
    \begin{equation}
        \symbf{z}\mapsto\symbf{U}^{-1}\symbf{\varphi}_{a}\circ\symbf{V}\symbf{z},\label{eq:holomorphicautomorphismgrouponunitball_statement}
    \end{equation} where \(\symbf{U},\symbf{V}\) are unitary matrices, \(a\in\mathbb{D}\), and \(\symbf{\varphi}_a\) is defined as in \cref{eq:unitballsimpleautomorphism_statement} (and every such function lies in \(\Aut\qty(B^n)\)).
\end{theorem}
\begin{proof}
    Let \(\symbf{f}\in\Aut\qty(B^n)\) be arbitrary, and set \(\symbf{\alpha}=\symbf{f}(\symbf{0})\). Then there exists a unitary matrix \(\symbf{U}\) such that \(\symbf{U}\symbf{\alpha}=\qty(\abs{\alpha},0,\ldots,0)\).

    Now let \(\symbf{\varphi}_{\abs{\symbf{\alpha}}}\) be as in \cref{prop:unitballsimpleautomorphism}, mapping \(\qty(\abs{\symbf{\alpha}},0,\ldots,0)\) to \(\symbf{0}\). Then, the map \(\symbf{\varphi}_{\abs{\symbf{\alpha}}}\circ\symbf{U}\symbf{f}\in\Aut\qty(B^n)\) fixes \(\symbf{0}\), so by \cref{prop:unitballautomorphismfixedpointatzero} it is a unitary transformation, say \(\symbf{V}\). Therefore, \[\symbf{\varphi}_{\abs{\symbf{\alpha}}}\circ\symbf{U}\symbf{f}\equiv\symbf{V}\implies\symbf{f}(\symbf{z})\equiv\symbf{U}^{-1}\symbf{\varphi}_{\abs{\symbf{\alpha}}}^{-1}\circ\symbf{Vz}.\] The converse is trivial.
\end{proof}