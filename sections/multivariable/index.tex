\section{A Glimpse into the Function Theory of Multiple Complex Variables}\label{sec:multivariatecomplexanalysis}
As in the single-variable case, there exist natural extensions of the
definitions of holomorphy and of derivatives, as well as direct analogs of
integral theorems.

Many other results, however, have to be separately derived.

Throughout mathematical history, many efforts were made to study the nature of
complex functions in a multivariate setting. In the 20th century, Poincaré
proved that the unit ball
\(B^n=\cbraces{\qty(z_1,\dots,z_n)\in\mathbb{C}^n}{\sum_{k=1}^n\abs{z_k}^2<1}\)
and the polydisk \(\mathbb{D}^n=\mathbb{D}\times\cdots\times\mathbb{D}\) are
not biholomorphically equivalent by comparison of their automorphism groups,
under certain assumptions of the automorphisms on their respective boundaries;
the proof was later formalized by Cartan, but is nonetheless largely attributed
to Poincaré. As one of the first of many deviations, the miracle of the Riemann
Mapping Theorem (\cref{thm:riemannmapping}) fails in higher dimensions. Hartogs
later also showed that poles and essential singularities cannot exist as
isolated singularities of multivariate holomorphic functions. Perhaps these are
the results of an unsatisfactory generalization.

Whereas the efforts of mathematicians of over two centuries give rise to the
development of function theory of one complex variable, the theory of
multivariate complex functions is still largely rudimentary. Many seemingly
fundamental problems still largely remain as conjecture.
\subimport{holomorphy_consequences/}{index.tex}
\subimport{ball_polydisk_holomorphic_automorphisms/}{index.tex}
\subimport{topological_equiv_biholomorphic_equiv/}{index.tex}
\subimport{hartogs_phenomenon/}{index.tex}