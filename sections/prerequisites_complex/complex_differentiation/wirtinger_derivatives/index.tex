\subsubsection{Wirtinger Derivatives}
We have previously introduced the concept of expressing a complex function as a function of \(x\) and \(y\). It can also be expressed in terms of \(z\) and \(\overline{z}\), where \(z=x+\ii y\) and \(\overline{z}=x-\ii y\). Then \(\abs{z}^2=z\overline{z}\), \(x=\frac{z+\overline{z}}{2}\), and \(y=\frac{z-\overline{z}}{2\ii}\). By the rules of the derivative, it is only natural that we define
\begin{equation}
    \pdv{z}=\pdv{x}\pdv{x}{z}+\pdv{y}\pdv{y}{z}=\frac{1}{2}\qty(\pdv{x}-\ii\pdv{y})\label{eq:wirtingerderivative1}
\end{equation} and
\begin{equation}
    \pdv{}{\overline{z}}=\pdv{}{x}\pdv{x}{\overline{z}}+\pdv{}{y}\pdv{y}{\overline{z}}=\frac{1}{2}\qty(\pdv{}{x}+\ii\pdv{}{y}).\label{eq:wirtingerderivative2}
\end{equation}
If \cref{eq:wirtingerderivative1} is set equal to 0, then it is the equivalent form of the homogeneous Cauchy--Riemann Equations. Then for a holomorphic function \(f(z)\), the Wirtinger derivative \(\pdv{f}{z}=\dv{f}{z}\).

In terms of \(u\) and \(v\), the two derivatives of a function \(f(z)\) are equal to:
\[\pdv{f}{z}=\frac{1}{2}\paren{\pdv{u}{x}+\ii\pdv{v}{x}-\ii\pdv{u}{y}+\pdv{v}{y}},\]
and
\[\pdv{f}{\overline{z}}=\frac{1}{2}\paren{\pdv{u}{x}+\ii\pdv{v}{x}+\ii\pdv{u}{y}-\pdv{v}{y}}.\] If \(f\) is holomorphic,
\begin{equation}\label{eq:holomorphicderivativedecomposition}
    \dv{f}{z}=\pdv{u}{x}+\ii\pdv{v}{x}=\pdv{v}{y}+\ii\pdv{v}{x}=\pdv{u}{x}-\ii\pdv{u}{y}=\pdv{v}{y}-\ii\pdv{u}{y}.
\end{equation}
On the contrary, by the rules of the derivative,
\begin{equation*}
    \pdv{x}=\pdv{z}\pdv{z}{x}+\pdv{}{\overline{z}}\pdv{\overline{z}}{x}=\pdv{z}+\pdv{\overline{z}}
\end{equation*} and
\begin{equation*}
    \pdv{}{y}=\pdv{}{z}\pdv{z}{y}+\pdv{}{\overline{z}}\pdv{\overline{z}}{y}=\ii\pdv{}{z}-\ii\pdv{}{\overline{z}}.
\end{equation*}
The Laplacian is equal to
\begin{align}
    \Delta=\pdv[2]{}{x}+\pdv[2]{}{y} & =\paren{\pdv{z}+\pdv{}{\overline{z}}}^2+\paren{\ii\pdv{z}-\ii\pdv{}{\overline{z}}}^2                                               \nonumber \\
    & =\pdv[2]{}{z}+\pdv[2]{}{\overline{z}}+2\pdv[2]{}{z}{\overline{z}}-\pdv[2]{}{z}-\pdv[2]{}{\overline{z}}+2\pdv[2]{}{z}{\overline{z}} \nonumber \\
    & =4\pdv[2]{}{z}{\overline{z}}.\label{eq:laplaciancomplexform}
\end{align}
Under this definition, we can derive the chain rule:
\begin{theorem}[Chain Rule]\label{thm:wirtingerchainrule}
    Let \(\Omega\subseteq\mathbb{C}\) is a region such that \(g\in C^1(\Omega)\) and \(f\in C^1(g(\Omega))\). It follows that
    \begin{align*}
        \pdv{z}(f\circ g)            & =\qty(\pdv{f}{z}\circ g)\pdv{g}{z}+\qty(\pdv{f}{\overline{z}}\circ g)\pdv{\overline{g}}{z}                        \\
        \pdv{\overline{z}}(f\circ g) & =\qty(\pdv{f}{z}\circ g)\pdv{g}{\overline{z}}+\qty(\pdv{f}{\overline{z}}\circ g)\pdv{\overline{g}}{\overline{z}}.
    \end{align*}
\end{theorem}
\begin{proof}
    Write \(z=x+\ii y\). Let
    \[g(z)=\xi(x,y)+\ii\eta(x,y),\qquad\zeta=\xi+\ii\eta\] so that \(\zeta=g(z)\) with \(\xi=\xi(x,y),\eta=\eta(x,y)\). Let \(f\) be regarded as a \(C^1\) function of the real variables \(\xi,\eta\); equivalently we may view \(f\) as \(f\qty(\zeta,\overline{\zeta})\) where \(\overline{\zeta}=\xi-\ii\eta\). The composition is \(h(z)=f\circ g(z)=f\qty(\xi(x,y),\eta(x,y))\).

    Using the real chain rule (provided by the continuous differentiability), we have
    \[\pdv{h}{x}=\pdv{f}{\xi}\pdv{\xi}{x}+\pdv{f}{\eta}\pdv{\eta}{x},\qquad\pdv{h}{y}=\pdv{f}{\xi}\pdv{\xi}{y}+\pdv{f}{\eta}\pdv{\eta}{y}.\]
    Hence, \[\pdv{h}{z}=\frac{1}{2}\qty[\pdv{f}{\xi}\qty(\pdv{\xi}{x}-\ii\pdv{\xi}{y})+\pdv{f}{\eta}\qty(\pdv{\eta}{x}-\ii\pdv{\eta}{y})].\]
    Now recall
    \[\pdv{f}{\zeta}=\frac{1}{2}\qty(\pdv{f}{\xi}-\ii\pdv{f}{\eta}),\qquad\pdv{f}{\overline\zeta}=\frac{1}{2}\qty(\pdv{f}{\xi}+\ii\pdv{f}{\eta}).\]
    Thus,
    \[\pdv{f}{\xi}=\pdv{f}{\zeta}+\pdv{f}{\overline\zeta},\qquad \pdv{f}{\eta}=\ii\qty(\pdv{f}{\zeta}-\pdv{f}{\overline\zeta}).\]
    Then by substitution,
    \begin{align*}
        \pdv{h}{z} & =\frac{1}{2}\qty[\qty(\pdv{f}{\zeta}+\pdv{f}{\overline\zeta})\qty(\pdv{\xi}{x}-\ii\pdv{\xi}{y})+\ii\qty(\pdv{f}{\zeta}-\pdv{f}{\overline\zeta})\qty(\pdv{\eta}{x}-\ii\pdv{\eta}{y})]                                           \\
        & =\pdv{f}{\zeta}\frac{1}{2}\qty[\qty(\pdv{\xi}{x}-\ii\pdv{\xi}{y})+\ii\qty(\pdv{\eta}{x}-\ii\pdv{\eta}{y})]+\pdv{f}{\overline\zeta}\frac{1}{2}\qty[\qty(\pdv{\xi}{x}-\ii\pdv{\xi}{y})-\ii\qty(\pdv{\eta}{x}-\ii\pdv{\eta}{y})].
    \end{align*}
    The terms in brackets equal \(\pdv{g}{z}\) and \(\pdv{\overline{g}}{z}\). Thus,
    \[\pdv{h}{z}=\qty(\pdv{f}{\zeta}\circ g)\pdv{g}{z}+\qty(\pdv{f}{\overline\zeta}\circ g)\pdv{\overline g}{z}.\]
    Renaming the variables yields
    \[\pdv{z}\qty(f\circ g)=\qty(\pdv{f}{z}\circ g)\pdv{g}{z}+\qty(\pdv{f}{\overline z}\circ g)\pdv{\overline{g}}{z}.\]
    A similar calculation using \cref{eq:wirtingerderivative2} gives
    \[\pdv{\overline{z}}(f\circ g)=\qty(\pdv{f}{z}\circ g)\pdv{g}{\overline{z}}+\qty(\pdv{f}{\overline z}\circ g)\pdv{\overline{g}}{\overline{z}}.\]
    These are exactly the proclaimed identities.
\end{proof}