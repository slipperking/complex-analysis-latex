\subsection{The Conformality of Holomorphic Maps}\label{sec:conformalityintroduction}
Let \(f:U\to\mathbb{C}\) be a holomorphic function defined on an open and connected subset \(U\subseteq\mathbb{C}\), and let \(z_0\in U\) be a point such that \(f'(z_0)\neq 0\). Consider a differentiable curve \(\gamma\in C^1([0,1])\) with \(\gamma(0)=z_0\). The direction of the curve at \(z_0\) is given by the argument of its derivative: \(\arg(\gamma'(0))\).

The image of \(\gamma\) under \(f\), defined by \(\sigma(t)=f(\gamma(t))\), is also a smooth curve passing through \(w_0=f(z_0)\). By the chain rule, the derivative of \(\sigma\) at \(t=0\) is given by
\[\sigma'(0)=f'(\gamma(0))\cdot\gamma'(0)=f'(z_0)\cdot\gamma'(0),\]
and hence,
\[\arg(\sigma'(0))=\arg(f'(z_0)\cdot\gamma'(0))=\arg(f'(z_0))+\arg(\gamma'(0)).\]
It follows that
\[\arg(\sigma'(0))-\arg(\gamma'(0))=\arg(f'(z_0)).\]
This shows that the change in direction between a curve and its image under \(f\) is independent of the curve itself, depending only on the value of \(f'(z_0)\).

Now consider two smooth curves \(\gamma_1,\gamma_2\in C^1([0,1])\) such that \(\gamma_1(0)=\gamma_2(0)=z_0\), with respective images \(\sigma_1(t)=f(\gamma_1(t))\) and \(\sigma_2(t)=f(\gamma_2(t))\). Then,
\[\arg(\sigma_1'(0))-\arg(\gamma_1'(0))=\arg(f'(z_0))=\arg(\sigma_2'(0))-\arg(\gamma_2'(0)),\]
and by rearrangement,
\[\arg(\sigma_1'(0))-\arg(\sigma_2'(0))=\arg(\gamma_1'(0))-\arg(\gamma_2'(0)).\]
This equality demonstrates that the angle between two smooth curves at \(z_0\) is preserved under \(f\), provided \(f'(z_0)\neq 0\). In other words, holomorphic functions with non-vanishing derivatives preserve angles and orientation locally---a property known as \textit{conformality}.

Furthermore, for any smooth curve \(\gamma\in C^1([0,1])\) passing through \(z_0\), the limit
\[\lim_{\substack{z\to z_0\\z\in\gamma}}\frac{\abs{f(z)-f(z_0)}}{\abs{z-z_0}}=\abs{f'(z_0)}\]
shows that infinitesimal lengths are locally scaled by a factor of \(\abs{f'(z_0)}\) under the mapping \(f\).

Therefore, at points where \(\abs{f'(z)}\neq 0\), the function \(f\) is conformal; it preserves angles but not necessarily lengths.