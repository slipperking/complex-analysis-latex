\subsection{The Extended Complex Plane and its Spherical Representation}
All complex numbers form a field that extends the real number field. A complex number \(\alpha+\ii\beta\) can be visualized on a rectangular plane as the point \((\alpha,\beta)\), with two axes: the real axis and the imaginary axis. It is well known that any complex number also has the polar form \(r\ee^{\ii\theta}=r\paren{\cos\theta+\ii\sin\theta}\).

The infinity point, \(\infty\), extends \(\mathbb{C}\) with \(\extcomplex=\mathbb{C}\cup\cbraces{\infty}\). The following arithmetic operations are defined: \(\forall a\in\mathbb{C}\), \(a+\infty=\infty+a=\infty\), and \(\forall b\in\mathbb{C}\setminus\cbraces{0}\), \(b\cdot\infty=\infty\cdot b=\infty\) and \(\frac{a}{\infty}=0\).

Let \(S^2=\cbraces{\qty(x_1,x_2,x_3)\in\mathbb{R}}{x_1^2+x_2^2+x_3^2=1}\). There exists a \textit{stereographic projection} of \(S^2\) onto \(\extcomplex\). For every point other than \((0,0,1)\), there is a corresponding complex number
\begin{equation}\label{eq:extcomplexformula1}
    z=\frac{x_1+\ii x_2}{1-x_3}.
\end{equation}
This correspondence between \(\mathbb{C}\) and \(S^2\setminus\cbraces{(0,0,1)}\) is injective. In fact, the inverse can be solved for:
\[\abs{z}^2=\frac{1-x_3^2}{\paren{1-x_3}^2}=\frac{1+x_3}{1-x_3},\]
which results in \[x_3=\frac{\abs{z}^2-1}{\abs{z}^2+1},\]
and consequently,
\[x_1=\Re\paren{z}\paren{1-x_3}=\frac{z+\overline{z}}{\abs{z}^2+1},\quad x_2=\Im\paren{z}\paren{1-x_3}=\frac{z-\overline{z}}{\ii\abs{z}^2+\ii}.\]
By letting \(\infty\) correspond to \(\paren{0,0,1}\), the bijection is complete. The sphere \(S^2\) is also called the \textit{Riemann sphere}. The region given by the disk \(\abs{z}<1\) is given by \(x_3<0\), and the region \(\abs{z}>1\) is given by \(x_3>0\).

We will now give a geometric visualization of this projection. Let \(z=x+\ii y\). Then we obtain that \(x=\frac{x_1}{1-x_3}\) and \(y=\frac{x_2}{1-x_3}\). Therefore, \[x:y:1=x_1:x_2:1-x_3.\]
It follows that the points \(0\), \(\qty(x,y,1)\), and \(\qty(x_1,x_2,1-x_3)\) are collinear in \(\mathbb{R}^3\). Under the linear map \(\va{v}\mapsto\va{v}\mqty(\dmat[0]{1,1,-1})+\qty(0,0,1)\), we get that \(\qty(0,0,1)\), \(\qty(x,y,0)\), and \(\qty(x_1,x_2,x_3)\) are collinear. In other words, this correspondence is a central projection with center \(\qty(0,0,1)\), projecting the points from \(S^2\setminus \qty(0,0,1)\) onto \(\mathbb{C}\). Let this center correspond to \(\infty\). In this representation, \(\infty\in\extcomplex\) is no longer considered to be special.

For the purpose of notation, we will define \(\mathbb{H}^+=\cbraces{z\in\mathbb{C}}{\Im(z)>0}\).
