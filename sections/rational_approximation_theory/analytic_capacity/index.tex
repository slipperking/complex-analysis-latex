\subsection{Analytic Capacity}
The theory of rational approximation is essentially built upon the concept of \textit{analytic capacity}, which was introduced in 1940 by Finnish mathematician Lars Ahlfors. Our purpose here is to give a brief and elementary introduction. Despite its importance, still many trivially simple results remain conjecture.
\begin{definition}[Analytic Capacity]\label{def:analyticcapacity}
    Let \(K\subset\mathbb{C}\) be compact. The \textit{analytic capacity} of \(K\) is defined as \[\gamma(K)=\sup\cbraces{\abs{f'(\infty)}:f\text{ is holomorphic on \(\extcomplex\setminus K\)}\wedge f(\infty)=0\wedge f(\extcomplex\setminus K)\subseteq\overline{\mathbb{D}}},\] where \(f'(\infty)\) is defined as in \cref{eq:derivativeatinfinity}.
\end{definition}
Intuitively,