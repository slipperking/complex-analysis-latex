\subsection{Analytic Capacity}
The theory of rational approximation is essentially built upon the concept of \textit{analytic capacity}, which was introduced in 1940 by Finnish mathematician Lars Ahlfors. Our purpose here is to give a brief and elementary introduction. Despite its importance, still many trivially simple results remain conjecture.

The uses of analytic capacity are present in many other topics of complex analysis. Analytic capacity serves as a natural framework for general rational approximation theory. Our purpose here is to hint at how analytic capacity theory relates to the proof of \cref{thm:mergelyan} and pertinent problems in general.
\begin{definition}[Analytic Capacity]\label{def:analyticcapacity}
    Let \(K\subset\mathbb{C}\) be compact. The \textit{analytic capacity} of \(K\) is defined as \[\gamma(K)=\sup\cbraces{\abs{f'(\infty)}:f\text{ is holomorphic on \(\extcomplex\setminus K\)}\wedge f(\infty)=0\wedge f\qty(\extcomplex\setminus K)\subseteq\overline{\mathbb{D}}},\] where \(f'(\infty)\) is defined as in \cref{eq:derivativeatinfinity}.
\end{definition}
Intuitively, \(\gamma\) measures the extent to which bounded analytic functions outside \(K\) can deviate from constancy. Generally, the ``larger'' \(K\) is, the greater the capacity is.
\begin{proposition}
    If \(K\subset\mathbb{C}\) is a compact set of discrete points, then \(\lambda(K)=0\).
\end{proposition}
\begin{proof}
    For any \(f:\extcomplex\setminus K\to\mathbb{C}\) holomorphic with \(f\qty(\extcomplex\setminus K)\subseteq\overline{\mathbb{D}}\), since \(f\) is bounded, the Riemann's Theorem for removable singularities (\cref{thm:riemannremovablesingularities}) allows for an analytic continuation onto all of \(\extcomplex\). Then Liouville's Theorem (\cref{thm:liouville}) implies that \(f\) is constant and \(f'(\infty)=0\). Hence \(\gamma(K)=0\).
\end{proof}
\begin{theorem}\label{thm:analyticcapacitymonotonicity}
    For \(K_1\subseteq K_2\) both compact in \(\mathbb{C}\), \(\gamma\qty(K_1)\leq\gamma\qty(K_2)\).
\end{theorem}
\begin{proof}
    This follows directly from the definition and the fact that any function holomorphic on \(\extcomplex\setminus K_1\) is also holomorphic on \(\extcomplex\setminus K_2\).
\end{proof}
The preceding results above hint at the monotonous behavior of capacity. However, currently it is not known whether a general \textit{subadditivity} property holds for analytic capacity, or that \[\gamma\qty(K_1\cup K_2)\mathrel{\overset{?}{\leq}}\gamma\qty(K_1)+\gamma\qty(K_2).\]
Recent results hint the affirmative, as many special cases of the relation have been proved; the question of subadditivity has been proved in the affirmative for disjoint compact continua, and recent findings by Xavier Tolsa show that capacity is (countably) semi-(sub)additive (the existence of an absolute constant \(C\) such that \(\gamma\qty(K_1\cup K_2)\leq C\qty[\gamma\qty(K_1)+\gamma\qty(K_2)]\)).

We now give some quantifying examples of how analytic capacity measures a type of ``largeness'' of compact sets, (rather much like area, which satisfies the subadditivity relation). First we define a specific classification of compact sets.

A compact set \(K\subset\mathbb{C}\) is a \textit{continuum} if it is connected, \(\mathbb{C}\setminus K\) is connected, and if it is not a singleton (\(K\) contains at least 2 distinct points).
\begin{proposition}\label{prop:analyticcapacitycontinuumbiholomorphism}
    Let \(K\subset\mathbb{C}\) be a continuum. Then \(\gamma(K)=\abs{f'(\infty)}\) where \(f:\extcomplex\setminus K\to\mathbb{D}\) is a biholomorphism satisfying \(f(\infty)=0\) (or that the maximal \(\abs{f'(\infty)}\) is attained when \(f\) is biholomorphic).
\end{proposition}
\begin{proof}
    Let \(f\) be the biholomorphism, \(g:\extcomplex\setminus K\to\mathbb{D}\) be holomorphic (not necessarily surjective) mapping \(\infty\) to 0. Since \(h\equiv g\circ f^{-1}:\mathbb{D}\to\mathbb{D}\) and maps 0 to 0, the Schwarz Lemma (\cref{lem:schwarz}) implies that \[\abs{h(z)}\leq\abs{z}\] for all \(z\in\mathbb{D}\). Thus, \(\abs{g(z)}\leq\abs{f(z)}\), and \[\abs{g'(\infty)}=\lim_{z\to\infty}\abs{zg(z)}\leq\lim_{z\to\infty}\abs{zf(z)}=\abs{f'(\infty)}.\qedhere\]
\end{proof}
\begin{proposition}\label{prop:analyticcapacitycloseddisk}
    The analytic capacity of any closed disk is the radius.
\end{proposition}
\begin{proof}
    Since \(\overline{D(a,r)}\) is a continuum, a biholomorphism \(f:\extcomplex\setminus\overline{D(a,r)}\to\mathbb{D}\) such that \(f(\infty)=0\). One such biholomorphism is given by \[f(z)=\frac{r}{z-a},\qquad f'(\infty)=\lim_{z\to 0}\dv{}{z}\qty(\frac{r}{\frac1z-a})=r.\]
    Hence, \cref{prop:analyticcapacitycontinuumbiholomorphism}, gives that \(\gamma\qty(\overline{D(a,r)})=r\).
\end{proof}
\begin{proposition}
    If \(K\subset\mathbb{C}\) is a continuum, then \[\frac{\diam K}4\leq\gamma(K)\leq\diam K.\]
\end{proposition}
\begin{proof}
    Assume \(f:\extcomplex\setminus K\to\mathbb{D}\) is a biholomorphism mapping \(\infty\) to 0. The lower bound follows directly from \cref{prop:complementbiholomorphismquarterestimate}. Let \(p\in K\) be arbitrary, then for any \(q\in K\), we obtain \(\abs{p-q}\leq\diam K\), implying that \(K\subseteq\overline{D\qty(p,\diam K)}\). By \cref{prop:analyticcapacitycloseddisk}, we have \(\gamma\qty(\overline{D\qty(p,\diam K)})=\diam K\), and \cref{thm:analyticcapacitymonotonicity} consequently gives the upper bound of \[\gamma(K)\leq\diam K.\qedhere\]
\end{proof}
