\subsection{Mergelyan's Theorem}
Although many mathematicians have since tried after the efforts of Weierstrass and Runge to approximation continuous functions holomorphic on the interior restriction, it was only 67 years later when Armenian mathematician provided the first widely accepted proof. The proof of Runge's Theorem (specifically in \cref{prop:rungesimplepolesandremovablesingularityatinfinity}) relied heavily on the assumption of holomorphy on a neighborhood, a rational function was created by placing poles in prescribed points of a contour that lay outside of \(K\) but within its domain of holomorphy. Obviously, these assumptions are null under the context of this new formulation.

The proof proposed by Mergelyan is almost trivial when compared with the results of many other mathematicians at the time. It even uses the concepts previously proposed by Runge. This begs the question: why was there such a prolonged time gap between the two similar formulations? Many mathematicians felt that the conclusion was too ``too good to be true''; during this elapsed time period there were many efforts of mathematicians that resulted in many technical partial results. Mergelyan's Theorem came as a surprise as it encapsulated many of those results with simplicity.

As we have previously seen, there is a prevalent notion in complex analysis that regards \(\infty\) intrinsically as essentially any other point of \(\extcomplex\). An appertaining question relates to the complex derivative at \(\infty\). Although \[f'(\infty)\overset{?}{=}\lim_{z\to\infty} f'(z)\] may seem to be a natural object to consider, it is quite impractical; there exist functions which decay quickly to 0, while \(f'(z)\) is unbounded as \(z\to\infty\) (take \(z\mapsto\frac{\sin\qty(z^2)}{\sqrt{z}}\) as an example). Even the assumption that \(\lim_{z\to\infty}f'(z)=0\) does not imply that \(f(z)\) has a removable singularity at \(\infty\) (consider \(z\mapsto\sqrt{z}\)).
\begin{definition}
    Let \(R>0\), \(f:\mathbb{C}\setminus\overline{D(0,R)}\to\mathbb{C}\) be holomorphic such that \(f\) has a removable singularity at \(\infty\). Then we define the derivatives of \(f\) at \(\infty\) to be \[f^{(n)}(\infty)=\eval{\dv[n]{z}f\qty(\frac1z)}_{z=0}.\]
    In the case that \(n=1\), we have
    \begin{equation}
        f'(\infty)=-\lim_{z\to\infty}z^2 f'\qty(z).\label{eq:derivativeatinfinity}
    \end{equation}
\end{definition} This is precisely the first singular term of the Laurent expansion of \(f\) at \(\infty\).
\begin{remark}
    This definition may feel unsatisfactory, but the underlying logic here is similar to the method used to generalize residues to \(\infty\).
\end{remark}
If \(f\) is bijective and meromorphic on some neighborhood of a point \(a\in\mathbb{C}\) such that \(f(a)=\infty\), then we informally define the derivative at the pole \(a\) to be
\begin{equation}
    f'(a)=\frac{1}{\qty(f^{-1})'\qty(\infty)}=-\lim_{w\to\infty}\frac{1}{w^2\qty(f^{-1})'(w)}=-\lim_{w\to\infty}\frac{f'\qty(f^{-1}(w))}{w^2}.\label{eq:derivativeatpole1}
\end{equation}
Let \(z=f^{-1}(w)\). Then we have
\begin{equation}
    f'(a)=-\lim_{z\to a}\frac{f'(z)}{f(z)^2}=\eval{\dv{z}(\frac{1}{f(z)})}_{z=a}.\label{eq:derivativeatpole2}
\end{equation}
\begin{proposition}\label{prop:complementbiholomorphismquarterestimate}
    For any connected compact set \(K\subset\mathbb{C}\) containing at least two distinct points such that \(\extcomplex\setminus K\) is connected, let \(\phi\) be an arbitrary biholomorphism mapping \(\extcomplex\setminus K\) to \(\mathbb{D}\) such that \(\phi(\infty)=\lim_{z\to\infty}\phi(z)=0\). It follows that \[\abs{\phi'(\infty)}\geq\frac14\diam(K),\] where \(\diam(K)=\sup_{z,\zeta\in K}\abs{\zeta-z}\).
\end{proposition}
\begin{proof}
    Denote the derivative of \(\phi\) at the infinity to be \(\alpha\). By \cref{eq:derivativeatpole1}, we have \[\qty(\phi^{-1})'(0)=\frac{1}{\phi'(\infty)}=\frac1\alpha=-\lim_{z\to 0}\frac{\qty(\phi^{-1})'(z)}{\phi^{-1}(z)^2}\Longleftrightarrow-\lim_{z\to0}\frac{\phi^{-1}(z)^2}{\alpha\qty(\phi^{-1})'(z)}=1.\]
    Fix \(\tau\in K\) and let \(\psi(z)=\frac{\alpha}{\phi^{-1}(z)-\tau}\), univalent on \(\mathbb{D}\). By direct calculation, we have \(\psi(0)=0\). Additionally,
    \begin{align*}
        \psi'(0)=-\lim_{z\to 0}\frac{\alpha\qty(\phi^{-1})'(z)}{\qty(\phi^{-1}(z)-\tau)^2}=\lim_{z\to 0}\frac{\alpha\qty(\phi^{-1})'(z)}{\qty(\phi^{-1}(z)-\tau)^2}\cdot\frac{\phi^{-1}(z)^2}{\alpha\qty(\phi^{-1})'(z)}=1.
    \end{align*}
    By the Koebe Quarter Theorem (\cref{thm:koebequarter}), whose proof is independent of results of this section, in accordance, \(D\qty(0,\frac14)\subseteq\psi(\mathbb{D})\). Let \(\mu\in K\setminus\cbraces{\tau}\). Obviously, \(\mu\notin\phi^{-1}(\mathbb{D})=\extcomplex\setminus K\).

    Let \(z\mapsto\frac{\alpha}{z-\tau}\) be injective on \(\extcomplex\). For the sake of contradiction, assume that \(\qty(z\mapsto\frac{\alpha}{z-\tau})(\mu)\in\psi(\mathbb{D})\). Then \(\exists\zeta\in\phi^{-1}(\mathbb{D})\) such that \(\frac{\alpha}{\zeta-\tau}=\frac{\alpha}{\mu-\tau}\). By injectivity, \(\zeta=\mu\), which contradicts \(\mu\in K\), and accordingly, \(\frac{\alpha}{\mu-\tau}\notin\psi(\mathbb{D})\supseteq D\qty(0,\frac14)\).

    Hence, \[\abs{\frac{\alpha}{\mu-\tau}}\geq\frac14\Longleftrightarrow\abs{\alpha}\geq\frac{\abs{\mu-\tau}}{4}.\] By taking the supremum for \(\mu,\tau\in K\), the proof is complete.
\end{proof}
\begin{remark}
    Such a biholomorphism will always exist; for arbitrary \(\zeta\in K\), the map \(z\mapsto\frac{1}{z-\zeta}\) maps \(\extcomplex\setminus K\) to a simply connected, proper subset of \(\mathbb{C}\), which is biholomorphic to \(\mathbb{D}\) by the Riemann Mapping Theorem (\cref{thm:riemannmapping}).
\end{remark}
\begin{proposition}\label{prop:complementbiholomorphism584r4767r2estimates}
    Let \(a\in\mathbb{C}\), \(r>0\), and suppose \(K\subseteq D(a,r)\) is compact such that \(\extcomplex\setminus K\) is connected and \(\diam(K)\geq\frac r2\). Then there is a family of holomorphic functions \(\mathcal{F}=\cbraces{\phi_\zeta}_{\zeta\in D(a,r)}\), where \(\forall\zeta\in D(a,r)\), \[\phi_\zeta:\extcomplex\setminus K\to\mathbb{C},\] and
    \begin{enumerate}
        \item \(\abs{\phi_\zeta(z)}\leq\frac{584}{r}\) for any \(z\in\extcomplex\setminus K\).\label{itm:complementbiholomorphism584r4767r2estimates_absolute584}
        \item \(\abs{\phi_\zeta(z)-\frac{1}{z-\zeta}}\leq\frac{4676r^2}{\abs{\zeta-z}^3}\) for any \(z\in\extcomplex\setminus\qty(K\cup\cbraces{\zeta})\).\label{itm:complementbiholomorphism584r4767r2estimates_absolutedifference4676}
        \item The function \(\phi(\zeta,z)\equiv\phi_\zeta(z)\) is jointly continuous in \(\zeta\) and \(z\).
    \end{enumerate}
\end{proposition}
\begin{proof}
    For brevity, assume \(a=0\).

    Let \(\widetilde{\varphi}\) be a conformal mapping from \(\extcomplex\setminus K\) to \(\mathbb{D}\), such that \(\widetilde{\varphi}(\infty)=0\) and \(\alpha=\widetilde{\varphi}'(\infty)\in\mathbb{R}_{>0}\). Let \(\varphi(z)=\frac{1}{\alpha}\widetilde{\varphi}(z)\). It follows that \(\varphi'(\infty)=1\), \(\varphi(\infty)=0\). By \cref{prop:complementbiholomorphismquarterestimate}, \[\abs{\alpha}\geq\frac{1}{4}\diam(K)\Longleftrightarrow\abs{\varphi(z)}\leq\frac{4\abs{\widetilde{\varphi}(z)}}{\diam(K)}.\] Consequently, we have the crucial estimate of \(\varphi\qty(\extcomplex\setminus K)\subseteq D\qty(0,\frac{4}{\diam(K)})\subseteq D\qty(0,\frac{8}{r})\). For fixed \(\zeta\in D(0,r)\), define \[\phi_\zeta(z)=\varphi(z)+(\zeta-\beta)\varphi^2(z),\qquad z\in\extcomplex\setminus K\]
    where \(\beta=\frac{\varphi''(\infty)}{2}\). The application of Cauchy's Estimate (\cref{thm:cauchysestimate}) on \(\qty(z\mapsto\frac{1}{z})\qty(\extcomplex\setminus D(0,r))=D\qty(0,\frac{1}{r})\) gives: \[\abs{\beta}=\frac{1}{2}\abs{\eval{\dv[2]{z}(\varphi\qty(\frac{1}{z}))}_{z=0}}\leq\frac{\sup_{D\qty(0,\frac1r)}\abs{\varphi\qty(\frac{1}{z})}}{\operatorname{dist}\qty(0,\partial D\qty(0,\frac{1}{r}))^2}=8r.\]
    Hence, \[\abs{\phi_\zeta(z)}\leq\abs{\varphi(z)}+\abs{\zeta-\beta}\abs{\varphi^2(z)}\leq\abs{\varphi(z)}+\abs{\zeta-\beta}\abs{\varphi^2(z)}\leq\frac{8}{r}+9r\frac{64}{r^2}=\frac{584}{r}.\] This is \cref{itm:complementbiholomorphism584r4767r2estimates_absolute584}. Suppose that \(\abs{z-\zeta}>2r\). It follows from \(\abs{\zeta}<r\) that \(\abs{z}>r\) (from the reverse triangle inequality) and hence disjoint from \(K\) and \(\zeta\). On this infinite annulus, we have the Laurent expansion (from \cref{thm:laurentexpansionofholomorphicfunction}) that \[\varphi(z)=\sum_{k=1}^\infty\frac{\mu_k}{\qty(z-\zeta)^k}=\frac{1}{z-\zeta}+\frac{\mu}{(z-\zeta)^2}+\order{\frac1{\qty(z-\zeta)^3}}\] where \(\mu_1=1\) because \(\varphi\sim\frac{1}{z}\) as \(z\to\infty\). Since \(\abs{z}>r\), we have the global Laurent expansion \[\varphi(z)=\frac1z+\frac\beta{z^2}+\order{\frac1{z^3}}.\]
    Hence,
    \begin{align*}
        \frac1{z-\zeta}+\frac\mu{\qty(z-\zeta)^2} & =\frac1z+\frac\beta{z^2}+\order{\frac1{z^3}}                                                                    \\
        z+\zeta+\mu                               & =z+\frac{\zeta^2}z+\beta+\frac{\beta\zeta^2}{z^2}-\frac{2\beta\zeta}{z}+\order{\frac1z}=z+\beta+\order{\frac1z} \\
        \mu                                       & =\beta-\zeta
    \end{align*} by letting \(z\to\infty\). Since \[\varphi^2(z)=\qty(\frac{1}{z-\zeta}+\order{\frac1{\qty(z-\zeta)^2}})^2=\frac{1}{\qty(z-\zeta)^2}+\order{\frac{1}{\qty(z-\zeta)^3}},\] from the definition of \(\phi_\zeta\), we have
    \begin{align*}
        \phi_\zeta(z)-\frac{1}{z-\zeta}=\varphi-\mu\varphi^2-\frac{1}{z-\zeta} & =\frac{\mu}{\qty(z-\zeta)^2}+\order{\frac1{(z-\zeta)^3}}-\frac{\mu}{\qty(z-\zeta)^2}-\order{\frac{\mu}{\qty(z-\zeta)^3}} \\
        & =\order{\frac1{\qty(z-\zeta)^3}}.
    \end{align*}
    Hence, there exists some \(M>0\) such that \[\abs{\phi_\zeta(z)-\frac{1}{z-\zeta}}<\frac{M}{\abs{z-\zeta}^3}\Longleftrightarrow\abs{\phi_\zeta(z)-\frac{1}{z-\zeta}}\abs{z-\zeta}^3<M\] for all \(z\) satisfying \(\abs{z-\zeta}>2r\). By \cref{thm:riemannremovablesingularities}, \(\qty(\phi_\zeta(z)-\frac{1}{z-\zeta})\qty(z-\zeta)^3\) has a removable singularity at \(\infty\). On the other hand, for \(\abs{z-\zeta}\leq 2r\) such that \(z\in\extcomplex\setminus\qty(K\cup\cbraces{\zeta})\), we have \[\abs{\phi_\zeta(z)-\frac{1}{z-\zeta}}\abs{z-\zeta}^3\leq\abs{\phi_\zeta(z)}\abs{z-\zeta}^3+\abs{z-\zeta}^2\leq\frac{584}{r}(2r)^3+(2r)^2=4676r^2\] from \cref{itm:complementbiholomorphism584r4767r2estimates_absolute584}. The Maximum Modulus Principle (\cref{thm:maximummodulus}) implies that \[\sup_{\abs{z-\zeta}>2r}\abs{\phi_\zeta(z)-\frac{1}{z-\zeta}}\abs{z-\zeta}^3\leq\sup_{\abs{z-\zeta}=2r}\abs{\phi_\zeta(z)-\frac{1}{z-\zeta}}\abs{z-\zeta}^3\leq 4676r^2\] and thus \cref{itm:complementbiholomorphism584r4767r2estimates_absolutedifference4676} follows. The joint continuity of \(\phi_\zeta\) is immediate from the definition.

    Lastly, if \(a\neq 0\), we may define \(\phi_\zeta(z)=\widetilde{\phi}_{\zeta-a}(z-a)\) where \(\cbraces{\widetilde{\phi}_{\zeta-a}}\) is the family constructed above for the set \(\cbraces{z-a}{z\in K}\subset D(0,r)\).
\end{proof}
\begin{proposition}\label{prop:diracdeltaapproximation}
    Suppose
    \begin{equation}
        \lambda(z)=
        \begin{dcases}
            \qty(1-\abs{z}^2)^2 & \qif*\abs{z}<1,   \\
            0                   & \qif*\abs{z}\ge1,
        \end{dcases}\qquad\lambda_r(z)=\frac{3}{\muppi r^2}\lambda\qty(\frac{z}{r})\quad\forall r>0\label{eq:diracdeltaapproximation_lambdadefinition}
    \end{equation}
    For fixed \(r\), the function \(\lambda_r\) satisfies:
    \begin{enumerate}
        \item \(\int_{\mathbb{C}}\lambda_r(\zeta)\dd{\xi}\wedge\dd{\eta}=1\), where \(\zeta=\xi+\ii\eta\).\label{itm:diracdeltaapproximation_integralto1}
        \item \(\lambda_r\in C^1(\mathbb{C})\) and is compactly supported.\label{itm:diracdeltaapproximation_compactsupportcontinuousdifferentiability}
        \item \(\int_{\mathbb{C}}\pdv{\lambda_r}{\overline{\zeta}}\dd{\xi}\wedge\dd{\eta}=0\).\label{itm:diracdeltaapproximation_antiholomorphicderivativeintegral}
        \item \(\int_{\mathbb{C}}\abs{\pdv{\lambda_r}{\overline{\zeta}}}\dd{\xi}\wedge\dd{\eta}\leq\frac{2\muppi}{r}\).\label{itm:diracdeltaapproximation_absoluteantiholomorphicderivativeintegral}
        \item \(\norm{\grad{\lambda_r(z)}}\leq\frac4{r^3}\) for all \(z\), where \(\grad=\qty(\pdv{}{x},\pdv{}{y})\) denotes the vector differential operator.\label{itm:diracdeltaapproximation_gradientstatement}
        \item\label{itm:diracdeltaapproximation_integralformula} For any \(z\in\mathbb{C}\) such that \(f\) is a holomorphic function on \(D(z,r)\), we have the integral formula
            \begin{equation}
                f(z)=\int_{D(0,r)}f(z-\zeta)\lambda_r(\zeta)\dd{\xi}\wedge\dd{\eta}.\label{eq:diracdeltaapproximation_integralformula}
            \end{equation}
    \end{enumerate}
\end{proposition}
\begin{proof}
    Let \(\zeta=\rho\ee^{\ii\theta}\). Then we have
    \begin{align*}
        \iint_{\mathbb{C}}\lambda_r(\zeta)\dd{\xi}\dd{\eta} & =\int_{0}^{2\muppi}\int_0^r\lambda_r\qty(\rho\ee^{\ii\theta})\rho\dd{\rho}\dd{\theta}=\int_{0}^{2\muppi}\int_0^r\frac{3\rho}{\muppi r^2}\qty(1-\qty(\frac{\rho}{r})^2)^2\dd{\rho}\dd{\theta} \\
        & =\frac{6}{r^2}\int_0^r\qty(\rho+\frac{\rho^5}{r^4}-2\frac{\rho^3}{r^2})\dd{\rho}=\frac{6}{r^2}\qty(\frac{r^2}{2}+\frac{r^6}{6r^4}-\frac{r^4}{2r^2})=1,
    \end{align*}
    which confirms \cref{itm:diracdeltaapproximation_integralto1}. Let \(z\in\mathbb{C}\) be arbitrary. The integral in \cref{eq:diracdeltaapproximation_integralformula} is equal to
    \begin{align*}
        \iint_{D(0,R)}f(z-\zeta)\lambda_r(\zeta)\dd{\xi}\dd{\eta} & =\int_0^r\frac{3\rho}{\muppi r^2}\qty(1-\qty(\frac{\rho}{r})^2)^2\int_0^{2\muppi}f\qty(z-\rho\ee^{\ii\theta})\dd{\theta}\dd{\rho} \\
        & =2\muppi f(z)\int_0^r\frac{3\rho}{\muppi r^2}\qty(1-\qty(\frac{\rho}{r})^2)^2\dd{\rho}=f(z)
    \end{align*} by the mean-value property (\cref{lem:holomorphicmeanvalueproperty}), proving \cref{itm:diracdeltaapproximation_integralformula}. For \(z\in\mathbb{D}\), we have \[\norm{\grad{\lambda(z)}}=
    2\qty(1-\abs{z}^2)\norm{\grad(\abs{z}^2)}=2\qty(1-\abs{z}^2)2\abs{z}\norm{\grad\sqrt{x^2+y^2}}=4\qty(1-\abs{z}^2)\abs{z}.\]
    Hence, \[\norm{\grad\lambda_r(z)}=\frac{3}{\muppi
        r^2}\norm{\grad\qty[\lambda\qty(\frac{z}{r})]}=\frac{3}{\muppi
        r^2}\norm{(\grad\lambda)\qty(\frac{z}{r})}\frac{1}{r}=\frac{12}{\muppi
    r^3}\qty(1-\abs{z}^2)\abs{z}<\frac{4}{r^3},\] which confirms \cref{itm:diracdeltaapproximation_gradientstatement}. Since \(\abs{\pdv{\lambda_r}{\overline{\zeta}}}=\abs{\frac{1}{2}\qty(\pdv{\lambda_r}{\xi}+\ii\pdv{\lambda_r}{\eta})}=\frac{1}{2}\norm{\grad\lambda_r(\zeta)}<\frac{2}{r^3}\), we have \[\iint_{\mathbb{C}}\abs{\pdv{\lambda_r}{\overline{\zeta}}}\dd{\xi}\dd{\eta}=\iint_{D(0,r)}\abs{\pdv{\lambda_r}{\overline{\zeta}}}\dd{\xi}\dd{\eta}<\muppi r^2\frac{2}{r^3}=\frac{2\muppi}{r}\] since \(\supp\qty(\lambda_r)=\overline{D(0,r)}\) which verifies the inequality in \cref{itm:diracdeltaapproximation_absoluteantiholomorphicderivativeintegral}.

    The \cref{itm:diracdeltaapproximation_antiholomorphicderivativeintegral} is also true since
    \begin{align*}
        \iint_{\mathbb{C}}\pdv{\lambda_r}{\overline{\zeta}}\dd{\xi}\dd{\eta} & =\frac{1}{2}\int_{-r}^r\int_{-r}^{r}\pdv{\lambda_r}{\xi}\dd{\xi}\dd{\eta}+\frac{\ii}{2}\int_{-r}^r\int_{-r}^{r}\pdv{\lambda_r}{\eta}\dd{\eta}\dd{\xi} \\
        & =\frac{1}{2}\int_{-r}^r\qty[\lambda_r\qty(r+\ii\eta)-\lambda\qty(-r+\ii\eta)]\dd{\eta}                                                                \\
        & \quad+\frac{\ii}{2}\int_{-r}^r\qty[\lambda_r(\xi+\ii r)-\lambda_r(\xi-\ii r)]\dd{\xi}=0.
    \end{align*}
    Trivially, \(\lambda_r\) is continuous on \(D\qty(0,r)\) and \(\mathbb{C}\setminus\overline{D(0,r)}\). Thus, we only need to prove the joint continuity of \(\lambda\) (the continuity of \(\lambda_r\) implies that of \(\lambda\)) on an open neighborhood of \(\partial D(0,r)\).

    Let \(\lambda(x,y)=\qty(1-x^2-y^2)^2\). By simple calculation, we have \[\pdv{\lambda}{x}=-4\qty(1-x^2-y^2)x,\qquad\pdv{\lambda}{y}=-4\qty(1-x^2-y^2)y.\] At \(x^2+y^2=1\), both partial derivatives vanish, and hence, they match the vanishing derivative on the complement of \(\supp(\lambda)\), completing the proof of \cref{itm:diracdeltaapproximation_compactsupportcontinuousdifferentiability}.
\end{proof}
\begin{theorem}[\textsc{Tietze--Urysohn--Brouwer}]\label{thm:tietzeextension}
    Let \(K\subset\mathbb{C}\) be compact and \(f:K\to\mathbb{R}\) be continuous. Then \(\exists g\in C^0(\mathbb{C})\) such that \(g\equiv f\) on \(K\).
\end{theorem}
\begin{proof}
    For any two disjoint closed \(A,B\subseteq\mathbb{C}\), consider the continuous separation function \[\eta_{A,B}(z)=\frac{\operatorname{dist}(z,A)-\operatorname{dist}(z,B)}{\operatorname{dist}(z,A)+\operatorname{dist}(z,B)}\] so that \(\eta_{A,B}(A)=\cbraces{-1}\) and \(\eta_{A,B}(B)=\cbraces{1}\).

    For simplicity, by the boundedness of \(f\), we may assume that \(f(K)=[-1,1]\) (by a scaling and shift). We now aim to construct a sequence \(\cbraces{g_n}_{n\in\mathbb{N}_{\geq 0}}\) inductively such that \[\abs{g_n}\leq\frac{2^n}{3^{n+1}}\text{ on }\mathbb{C},\quad\abs{f-\sum_{k=0}^n g_k}\leq\qty(\frac23)^{n+1}\text{ on }K\quad\forall n\in\mathbb{N}.\]
    In the case that \(n=0\), define the disjoint closed sets \[A_0=\cbraces{z\in K}{f(z)\leq-\frac13}\qand B_0=\cbraces{z\in K}{f(z)\geq\frac13}.\]
    Let \(g_0(z)=\frac13\eta_{A_0,B_0}(z)\). It is clear that \(\abs{g_0}\leq\frac13\) on \(\mathbb{C}\). If \(z\in A_0\), then \(-1\leq f(z)\leq-\frac13\), \(g_0(z)=-\frac13\), and hence \(\abs{f-g_0}\leq\frac23\). If \(z\in B_0\), then \(\frac13\leq f(z)\leq1\), \(g_0(z)=\frac13\), and thus \(\abs{f-g_0}\leq\frac23\). If \(z\notin A_0\cup B_0\), then \(-\frac13<f(z)<\frac13\) and \(\abs{f-g_0}\leq\abs{f}+\abs{g_0}<\frac13+\frac13=\frac23\). Hence \(\forall z\in K\), \[\abs{f(z)-g_0(z)}\leq\frac23.\]
    This proves the base case. For the inductive step, assume the claim holds for each \(g_0,g_1,\dots,g_{n-1}\). Define \[h_n(z)=f(z)-\sum_{k=0}^{n-1}g_k(z)\] for \(z\in K\). By the inductive hypothesis, we have \(\abs{h_n}\leq\qty(\frac23)^n\) on \(K\). Define the disjoint closed sets \[A_n=\cbraces{z\in K}{-\frac{2^n}{3^n}\leq h_n(z)\leq-\frac{2^n}{3^{n+1}}}\qand B_n=\cbraces{z\in K}{\frac{2^n}{3^n}\geq h_n(z)\geq\frac{2^n}{3^{n+1}}}.\]
    Let \(g_n(z)=\frac{2^n}{3^{n+1}}\eta_{A_n,B_n}(z)\), so that \(\abs{g_n}\leq\frac{2^n}{3^{n+1}}\) on \(\mathbb{C}\), and \[\abs{h_n(z)-g_n(z)}\leq\frac{2^{n+1}}{3^{n+1}}\] for all \(z\in K\) by the same argument as in the base case. Hence, \[\abs{f(z)-\sum_{k=0}^n g_k(z)}=\abs{h_n(z)-g_n(z)}\leq\qty(\frac23)^{n+1}\] for all \(z\in K\), completing the induction. Because \[\abs{g(z)}\leq\sum_{n=0}^\infty\abs{g_n(z)}\leq\frac13\sum_{n=0}^\infty\frac{2^n}{3^n}=1\qquad\forall z\in\mathbb{C},\] the Weierstrass \(M\)--Test (\cref{thm:weierstrassmtest}) implies that the series \(\sum_{n=0}^\infty g_n(z)\) converges uniformly on \(\mathbb{C}\) to \(g\). Since each \(g_n\) is continuous, \cref{thm:uniformlimit} gives the continuity of \(g\) on \(\mathbb{C}\). Finally, for any \(z\in K\), we have \[\abs{f(z)-g(z)}\leq\lim_{n\to\infty}\frac{2^{n+1}}{3^{n+1}}=0.\qedhere\]
\end{proof}
\begin{corollary}\label{cor:tietzeextensioncomplexcompactsupport}
    If \(K\subset\mathbb{C}\) is compact and \(f:K\to\mathbb{C}\) is continuous, then \(\exists g\in C^0(\mathbb{C})\) such that \(g\equiv f\) on \(K\) and has compact support.
\end{corollary}
\begin{proof}
    Let \(f=u+\ii v\) where \(u,v:K\to\mathbb{R}\) are continuous. By Tietze--Urysohn--Brouwer (\cref{thm:tietzeextension}), \(\exists\widetilde{u},\widetilde{v}\in C^0(\mathbb{C})\) such that \(\widetilde{u}\equiv u\) and \(\widetilde{v}\equiv v\) on \(K\). Let \(R>0\) be such that \(K\subset D(0,R)\), provided by compactness. Define the piecewise-linear function \[\psi(z)=
        \begin{dcases}
            1                   & \qif*\abs{z}\leq R,  \\
            2-\frac{\abs{z}}{R} & \qif*R<\abs{z}<2R,   \\
            0                   & \qif*\abs{z}\geq 2R,
    \end{dcases}\] such that \(\psi\in C^0(\mathbb{C})\) and is compactly supported. Let \(g(z)=\qty(\widetilde{u}(z)+\ii\widetilde{v}(z))\psi(z)\), and the assertion follows.
\end{proof}
Let \(f\in C^0(K)\) be holomorphic on \(\interior{K}\). Then \(f\) has a continuous extension to all of \(\mathbb{C}\) by virtue of \cref{cor:tietzeextensioncomplexcompactsupport}. Define the \textit{modulus of continuity} of \(f\) to be the function \(\omega_f:\mathbb{R}_{\geq0}\to\mathbb{R}_{\geq0}\) with \[\omega_f(\delta)=\sup_{\substack{z,\zeta\in\mathbb{C}\\\abs{z-\zeta}\leq\delta}}\abs{f(z)-f(\zeta)}.\]
Because \(f\) has compact support, it must be uniformly continuous; hence we have \(\lim_{\delta\to0^+}\omega_f (\delta)=0\).

For \(r>0\), define
\begin{equation}
    \Phi(z)=\iint_{\mathbb{C}}\lambda_r(z-\zeta)f(\zeta)\dd{\xi}\dd{\eta}\qq{where}\zeta=\xi+\ii\eta,\label{eq:integralofcontinuousextensionofholomorphic}
\end{equation}
where \(\lambda_r\) employs the same definition as in \cref{eq:diracdeltaapproximation_lambdadefinition}.
\begin{proposition}\label{prop:integralofcontinuousextensionofholomorphicproperties}
    The function \(\Phi\) as in \cref{eq:integralofcontinuousextensionofholomorphic} satisfies:
    \begin{enumerate}
        \item \(\Phi\in\mathbb{C}^1(\mathbb{C})\) and has compact support.\label{itm:integralofcontinuousextensionofholomorphicproperties_continuousdifferentiabilitycompactsupport}
        \item \(\Phi\equiv f\) on \(U=\cbraces{z\in K}{\operatorname{dist}\qty(z,\mathbb{C}\setminus K)>r}\).\label{itm:integralofcontinuousextensionofholomorphicproperties_equivalenceonU}
        \item \(\abs{f(z)-\Phi(z)}\leq\omega_f(r)\) for all \(z\in\mathbb{C}\).\label{itm:integralofcontinuousextensionofholomorphicproperties_differbymodulusofcontinuity}
        \item For all \(z\in\mathbb{C}\), \(\abs{\pdv{\Phi}{\overline{z}}(z)}\leq\frac{4\muppi\omega_f(r)}{r}\).\label{itm:integralofcontinuousextensionofholomorphicproperties_antiholomorphicderivativebound}
        \item \(\Phi(z)=-\frac1\muppi\iint_H\pdv{\Phi}{\overline{\zeta}}(\zeta)\frac{\dd{\xi}\dd{\eta}}{\zeta-z}\) for \(z\in\mathbb{C}\), where \(H=\supp(\Phi)\setminus U\).\label{itm:integralofcontinuousextensionofholomorphicproperties_integralformula}
    \end{enumerate}
\end{proposition}
\begin{proof}
    Because \(\supp\qty(\lambda_r(z-\zeta))=\overline{D(z,r)}\) and \(\supp f\) is compact, for sufficiently large \(z\), the two supports will be disjoint and hence the integrand vanishes for all \(\zeta\). We can explicitly find that \[\pdv{\Phi}{x}=\lim_{\substack{\Delta x\to 0\\\Delta x\in\mathbb{R}}}\frac{\Phi(z+\Delta x)-\Phi(\Delta x)}{\Delta x}=\lim_{\substack{\Delta x\to 0\\\Delta x\in\mathbb{R}}}\int_{\mathbb{C}}\frac{\lambda_r(z+\Delta x-\zeta)-\lambda_r(z-\zeta)}{\Delta x}f(\zeta)\dd{\xi}\wedge\dd{\eta}.\] Because \(f\) is continuous and vanishes on a compact set, it is bounded. Similarly, \cref{itm:diracdeltaapproximation_compactsupportcontinuousdifferentiability} of \cref{prop:diracdeltaapproximation} implies that \(\pdv{\lambda_r}{x}\) is bounded. Hence, by Lebesgue's Dominated Convergence Theorem, we have \[\pdv{\Phi}{x}=\int_{\mathbb{C}}{\pdv{\lambda_r}{x}}(z-\zeta)f(\zeta)\dd{\xi}\wedge\dd{\eta},\] and similarly, \[\pdv{\Phi}{y}=\int_{\mathbb{C}}{\pdv{\lambda_r}{y}}(z-\zeta)f(\zeta)\dd{\xi}\wedge\dd{\eta}.\] Hence, \(\Phi\in C^1(\mathbb{C})\) and this is \cref{itm:integralofcontinuousextensionofholomorphicproperties_continuousdifferentiabilitycompactsupport}. Because \[\Phi(z)=\iint_{\mathbb{C}}\lambda_r(z-\zeta)f(\zeta)\dd{\xi}\dd{\eta}=\iint_{\mathbb{C}}\lambda_r(\zeta)f(z-\zeta)\dd{\xi}\dd{\eta},\] by \cref{itm:diracdeltaapproximation_integralto1} of \cref{prop:diracdeltaapproximation}, we have
    \begin{align}
        \abs{f(z)-\Phi(z)} & \leq\abs{\int_{\mathbb{C}}f(z)\lambda_r(\zeta)\dd{\xi}\wedge\dd{\eta}-\int_{\mathbb{C}}f(z-\zeta)\lambda_r(\zeta)\dd{\xi}\wedge\dd{\eta}}\nonumber                  \\
        & =\abs{\int_{\mathbb{C}}\lambda_r(\zeta)\qty(f(z)-f(z-\zeta))\dd{\xi}\wedge\dd{\eta}}\label{eq:integralofcontinuousextensionofholomorphicproperties_differencebound} \\
        & \leq\int_{D(0,r)}\lambda_r(\zeta)\abs{f(z)-f(z-\zeta)}\dd{\xi}\wedge\dd{\eta}\leq\omega_f(r),\nonumber
    \end{align}
    which implies \cref{itm:integralofcontinuousextensionofholomorphicproperties_differbymodulusofcontinuity}. For \(z\in U\), \(\zeta\in D(0,r)\) now implies that \(z-\zeta\in\interior{K}\) and hence \(f(z)-f(z-\zeta)\) is holomorphic in \(\zeta\) on \(D(z,r)\). By \cref{itm:diracdeltaapproximation_integralformula} of \cref{prop:diracdeltaapproximation}, \cref{eq:integralofcontinuousextensionofholomorphicproperties_differencebound} becomes \[\abs{\int_{\mathbb{C}}\lambda_r(\zeta)\qty(f(z)-f(z-\zeta))\dd{\xi}\wedge\dd{\eta}}=\abs{f(z)-f(z-0)}=0,\] which proves \cref{itm:integralofcontinuousextensionofholomorphicproperties_equivalenceonU}. Because \(\forall z\in\mathbb{C}\),
    \begin{align*}
        \pdv{\Phi}{\overline{z}}\qty(z) & =\frac12\qty(\pdv{\Phi}{x}+\ii\pdv{\Phi}{y})=\int_{\mathbb{C}}\pdv{\lambda_r}{\overline{z}}\qty(z-\zeta)f(\zeta)\dd{\xi}\wedge\dd{\eta}                                       \\
        & =\int_{\mathbb{C}}\pdv{\lambda_r}{\overline{\zeta}}\qty(\zeta)f(z-\zeta)\dd{\xi}\wedge\dd{\eta}                                                                               \\
        & =\int_{\mathbb{C}}\pdv{\lambda_r}{\overline{\zeta}}\qty(\zeta)f(z-\zeta)\dd{\xi}\wedge\dd{\eta}-f(z)\int_{\mathbb{C}}\pdv{\lambda_r}{\overline{\zeta}}\dd{\xi}\wedge\dd{\eta} \\
        & =\int_{\mathbb{C}}\pdv{\lambda_r}{\overline{\zeta}}\qty(\zeta)\qty(f(z-\zeta)-f(z))\dd{\xi}\wedge\dd{\eta}
    \end{align*}
    by \cref{itm:diracdeltaapproximation_antiholomorphicderivativeintegral} of \cref{prop:diracdeltaapproximation}. Hence,
    \begin{align*}
        \abs{\pdv{\Phi}{\overline{z}}} & \leq\iint_{D(0,r)}\abs{\pdv{\lambda_r}{\overline{\zeta}}}\abs{f(z-\zeta)-f(z)}\dd{\xi}\dd{\eta}\leq\omega_f(r)\iint_{D(0,r)}\norm{\grad{\lambda_r}}\dd{\xi}\dd{\eta} \\
        & \leq\frac{4\omega_f(r)}{r^3}\iint_{D(0,r)}\dd{\xi}\dd{\eta}\leq\frac{4\omega_f(r)}{r^3}\cdot\muppi r^2=\frac{4\muppi\omega_f(r)}{r},
    \end{align*}
    by \cref{itm:diracdeltaapproximation_gradientstatement} of \cref{prop:diracdeltaapproximation}, confirming \cref{itm:integralofcontinuousextensionofholomorphicproperties_antiholomorphicderivativebound}. Finally, \cref{itm:integralofcontinuousextensionofholomorphicproperties_integralformula} follows from \cref{cor:pompeiuwithoutcauchyterm} (since outside the support the integral trivially vanishes and within \(U\), \(\pdv{\Phi}{\overline{\zeta}}\) vanishes as a consequence of holomorphy).
\end{proof}
\begin{theorem}[Mergelyan]\label{thm:mergelyan}
    Let \(K\subset\mathbb{C}\) be compact such that \(\extcomplex\setminus K\) has finitely many connected components. Let \(E\subset\extcomplex\setminus K\) contain exactly one point from each of the connected components of \(\extcomplex\setminus K\). Suppose \(f\in C^0(K)\) is holomorphic on \(\interior{K}\). Then \(\forall\varepsilon>0\), there exists a rational function \(\psi(z)\) with poles in \(E\) such that \[\sup_{z\in K}\abs{\psi(z)-f(z)}<\varepsilon.\] %perhaps add some images in the proof to make it easier to grasp
\end{theorem}
\begin{proof}
    \import{build/visual_output/}{imports.tex}
    \begin{figure}
        \centering
        \import{build/visual_output/}{imports.tex}
        \begin{tikzpicture}
            \node[anchor=center] at (2.8,1) {\small\(K\)};
            \filldraw[thick, even odd rule, pattern=north east lines] {
                \foreach \c [count=\i] in \PreErosionRegionList{
                    plot[smooth cycle] coordinates \c
                }
            };
            \foreach \c [count=\i] in \ErodedRegionList{
                \draw[thin, name path=path \i] plot[smooth cycle] coordinates \c;
            }
            \draw[thin, |-|, line cap=round, shorten >=1pt] (1.98,-1.16) -- (1.93,-0.78) node[midway, anchor=east] {\small\(r\)};
            \draw[thin, |-|, line cap=round, shorten >=1pt] (-0.41,0.02) -- (-0.18,0.36) node[midway, anchor=north west] {\small\(r\)};

            \path (-0.6,-0.4) coordinate (P1)
            (0.94,0.05) coordinate (P2)
            (-2.2,0.2) coordinate (P3);

            \draw[dashed] \tikzcenterarc(P1)(0:360:0.3) \tikzcenterarc(P2)(0:360:0.3) \tikzcenterarc(P3)(0:360:0.3);

            \path
            (P1) node[point,label={[yshift=-2]above:\tiny\(p_1\)}] {}
            (P2) node[point,label={[yshift=2]below:\tiny\(p_2\)}] {}
            (P3) node[point,label={[yshift=-2]above:\tiny\(p_3\)}] {};

            \draw[thin, |-|, line cap=round, shorten >=1pt] (P3) -- ([yshift=-0.3cm]P3) coordinate[midway] (M);
            \draw[thin,-{Stealth}] (-2.8,1.2) -- ([xshift=-1pt]M);
            \node[anchor=south] at (-2.8,1.2) {\tiny\(\tfrac34r\)};
        \end{tikzpicture}
        \caption{The striped region bounds \(K\), while the thin lines bound \(U\).}
        \label{fig:mergelyan_kseterosions}
    \end{figure}Let \(F=\cbraces{p_k}_{1\leq k\leq n}\) contain precisely one point from each connected component of \(\extcomplex\setminus K\) (such that each \(p_k\neq\infty\) is finite). Suppose that \(r\) is chosen such that \(0<\frac34 r<\operatorname{dist}\qty(K, F)\) so that for each \(p_k\in F\) not equal to \(\infty\), \[\overline{D\qty(p_k,\frac34r)}\subset\extcomplex\setminus K.\]
    \begin{figure}
        \centering
        \begin{tikzpicture}
            \filldraw[thick, even odd rule, pattern=north east lines] {
                \foreach \c [count=\i] in \CoveredRegionList{
                    plot[smooth cycle] coordinates \c
                }
            };
            \coordinate (P1) at (-0.6,-0.4);
            \coordinate (P2) at (0.94,0.05);
            \coordinate (P3) at (-2.2,0.2);
            \node[circle, fill=black, inner sep=0.7pt, label={\tiny\(p_1\)}] at (P1) {};
            \node[circle, fill=black, inner sep=0.7pt, label={\tiny\(p_2\)}] at (P2) {};
            \node[circle, fill=black, inner sep=0.7pt, label={\tiny\(p_3\)}] at (P3) {};

            \filldraw[thin, fill opacity=0.3, even odd rule] {
                \foreach \c in \NoCentersRegionList{
                    plot[smooth cycle] coordinates \c
                }
            };
            \filldraw[ultra thin, pattern={Dots[distance=1.5pt,angle=30,radius=0.2pt]}] {
                \foreach \x/\y in \SubcoverDiskCenters{
                    \tikzcenterarc(\x,\y)(0:360:0.5)
                }
            };

        \end{tikzpicture}
        \caption{The striped region represents \(H\) while the unshaded regions represent the set which \(\zeta_k\) can be in. The dotted disks represent the finite subcover of \(H\): notice how every striped region is also dotted.}
        \label{fig:mergelyan_hset}
    \end{figure}Define the extension of \(f\), \(\Phi\), \(U\), and \(H\) as in the previous results. Hence, (see \cref{fig:mergelyan_kseterosions,fig:mergelyan_hset})
    \[\cbraces{D\qty(\zeta_k,\frac54r)}{(\forall)\zeta_k\in\extcomplex\setminus\qty(K\cup\overline{D\qty(p_k,\frac34r)}),1\leq k\leq n}\]
    covers a (compact) \(r\)-neighborhood of \(\extcomplex\setminus K\) (so that each \(\zeta_k\notin K\), and is labeled so that each \(\zeta_k\) is in the same connected component as \(p_k\)) (in the case that \(p_k=\infty\), let the disk inside be the empty set). Thus, the collection also covers \(H\). A finite subcover \(\cbraces{D\qty(\zeta^{(j)}_k,\frac54r)}{1\leq j\leq m_k,1\leq k\leq n}\) covering \(H\) exists by the Heine--Borel Theorem (\cref{thm:heineborel}).

    By the connectivity of each component of \(\extcomplex\setminus K\), there exists a piecewise-linear simple curve \(\gamma_k^{(j)}\) for all \(1\leq k\leq n\), \(1\leq j\leq m_k\), joining \(\zeta_k^{(j)}\) and \(p_k\), which lies entirely within \(\extcomplex\setminus K\). The compact disks \(D\qty(\zeta_k^{(j)},\frac34r)\) are all disjoint from their corresponding \(p_k\) since each \(\zeta_k^{(j)}\notin\overline{D\qty(p_k,\frac34r)}\) by definition.

    \begin{figure}
        \centering
        \centering\vspace{0pt}
        \begin{minipage}{0.48\textwidth}
            \centering
            \vspace{0pt}
            \begin{tikzpicture}
                %\tikztemporarygrid{-3}{3}{-3}{3}
                \draw[thick,dotted]\tikzcenterarc(0,0)(0:360:2);
                \draw[thick, solid] (0,0) node[anchor=north] {\(p_k\)} -- (-0.5,1) -- (0.25,1.5) -- (1,0.8) -- (0.9,0.1) node[anchor=west] {\(\gamma_{k}^{(j)}\)} -- (0.6,-0.4) -- (1.7,-0.7) -- ({1.7*cos(-55)},{1.7*sin(-55)}) -- ({2*cos(-55)},{2*sin(-55)});
                \draw[thick, dashed] ({2*cos(-55)},{2*sin(-55)}) -- ({3*cos(-55)},{3*sin(-55)}) node[point,label={right:{\(\zeta^{(j)}_k\)}}] {};
            \end{tikzpicture}
            \caption{The construction of \(E_k^{j}\).}\label{fig:mergelyan_eset}
        \end{minipage}
        \hfill
        \begin{minipage}{0.48\textwidth}
            \centering
            \vspace{0pt}
            \begin{tikzpicture}
                \filldraw[thin, pattern=north east lines, even odd rule] {
                    \foreach \c [count=\i] in \CoveredDisjointUnionRegionList{
                        plot[smooth cycle] coordinates \c
                    }
                };
            \end{tikzpicture}
            \caption{A concept construction of \(\qty{H_k^{(j)}}\).}\label{fig:mergelyan_hkjsets}
        \end{minipage}
    \end{figure}Hence, the intersection \(\overline{D\qty(\zeta_k^{(j)},\frac34r)}\cap\gamma_{k}^{(j)}\) consists of at least one connected component joining \(\zeta_k^{(j)}\) to a point on \(\partial D\qty(\zeta_k^{(j)},\frac34 r)\). Denote the connected component of this intersection by \(E_k^{(j)}\), satisfying \(\diam E_k^{(j)}\geq\frac34r>\frac r2\) and \(E_k^{(j)}\cap K=\varnothing\).

    Now for each \(j\) and \(k\), \cref{prop:complementbiholomorphism584r4767r2estimates} now provides the existence of a family of holomorphic functions \(\phi_{\zeta,k}^{(j)}:\extcomplex\setminus E_k^{(j)}\to\mathbb{C}\) given with \(\zeta\in D\qty(\zeta_k^{(j)},\frac54r)\) such that
    \begin{equation}
        \abs{\phi_{\zeta,k}^{(j)}(z)}\leq\frac{584}r,\quad\abs{\phi_{\zeta,k}^{(j)}(z)-\frac{1}{z-\zeta}}\leq\frac{4676}{\abs{z-\zeta}^3},\qquad\forall z\in\extcomplex\setminus E_k^{(j)}.\label{eq:mergelyan_familybounds}
    \end{equation}
    Let \(\widetilde{H}_k^{(j)}=H\cap D\qty(\zeta_k^{(j)},\frac54r)\), for each \(j,k\) and construct the disjoint sets \[H_k^{(j)}=\widetilde{H}_k^{(j)}\setminus\qty(\bigcup_{j'<j}\widetilde{H}_k^{\qty(j')}\cup\bigcup_{k'<k}\bigcup_{j'\leq m_{k'}}\widetilde{H}_{k'}^{\qty(j')})\text{ if }j\neq 1,\qquad H_1^{(1)}=\widetilde{H}_1^{(1)}.\]
    Thus the union \[\bigcup_{k=1}^n\bigcup_{j=1}^{m_k}H_k^{(j)}=H\cap\qty(\bigcup D\qty(\zeta_k^{(j)},\frac54r))=H\] since the set of all \(D\qty(\zeta_k^{(j)},\frac54r)\) covers \(H\). Let \[\Psi(z)=\frac{1}{\muppi}\sum_{k=1}^n\sum_{j=1}^{m_k}\int_{H_k^{(j)}}\pdv{\Phi}{\overline{\zeta}}\phi_{\zeta,k}^{(j)}(z)\dd{\xi}\wedge\dd{\eta}\qq{where}\zeta=\xi+\ii\eta,\quad\forall z\notin\bigcup E_k^{(j)}.\]
    Because \[\frac{\Psi(z+\Delta x)-\Psi(z)}{\Delta x}=\frac{1}{\muppi}\sum_{k=1}^n\sum_{j=1}^{m_k}\int_{H_k^{(j)}}\pdv{\Phi}{\overline{\zeta}}\frac{\phi_{\zeta,k}^{(j)}(z+\Delta x)-\phi_{\zeta,k}^{(j)}(z)}{\Delta x}\dd{\xi}\wedge\dd{\eta},\]
    and both \(\pdv{\Phi}{\overline{\zeta}}\) and the integrand is continuous on a set (we only need to consider the term involving \(\phi_{\zeta,k}^{(j)}\) since \(\pdv{\Phi}{\overline{\zeta}}\) is independent from \(z\)) by Cauchy's Estimates and the first bound of \cref{eq:mergelyan_familybounds}, Lebesgue's Dominate Convergence gives that
    \[\pdv{\Psi}{x}=\frac{1}{\muppi}\sum_{k=1}^n\sum_{j=1}^{m_k}\int_{H_k^{(j)}}\pdv{\Phi}{\overline{\zeta}}\pdv{\phi_{\zeta,k}^{(j)}}{x}\qty(z)\dd{\xi}\wedge\dd{\eta},\] and in analogous fashion, \[\pdv{\Psi}{y}=\frac{1}{\muppi}\sum_{k=1}^n\sum_{j=1}^{m_k}\int_{H_k^{(j)}}\pdv{\Phi}{\overline{\zeta}}\pdv{\phi_{\zeta,k}^{(j)}}{y}\qty(z)\dd{\xi}\wedge\dd{\eta}.\]
    Hence, \(\Psi\) is holomorphic on \(\extcomplex\setminus\bigcup_{k=1}^n\bigcup_{j=1}^{m_k} E_{k}^{(j)}\), a neighborhood of \(K\). Since \(\forall z\in\extcomplex\setminus\bigcup E_k^{(j)}\), by \cref{itm:complementbiholomorphism584r4767r2estimates_absolutedifference4676} of \cref{prop:complementbiholomorphism584r4767r2estimates},
    \begin{align*}
        \abs{\Psi(z)-\Phi(z)} & =\abs{\frac{1}{\muppi}\sum_{k=1}^n\sum_{j=1}^{m_k}\iint_{H_k^{(j)}}\pdv{\Phi}{\overline{\zeta}}\phi_{\zeta,k}^{(j)}(z)\dd{\xi}\dd{\eta}-\frac1\muppi\iint_H\pdv{\Phi}{\overline{\zeta}}\frac{\dd{\xi}\dd{\eta}}{z-\zeta}} \\
        & =\frac1{\muppi}\abs{\sum_{k=1}^n\sum_{j=1}^{m_k}\iint_{H_k^{(j)}}\pdv{\Phi}{\overline{\zeta}}\qty(\phi_{\zeta,k}^{(j)}(z)-\frac{1}{z-\zeta})\dd{\xi}\dd{\eta}}                                                           \\
        & \leq\frac1{\muppi}\sum_{k=1}^n\sum_{j=1}^{m_k}\iint_{H_k^{(j)}}\abs{\pdv{\Phi}{\overline{\zeta}}}\abs{\phi_{\zeta,k}^{(j)}(z)-\frac{1}{z-\zeta}}\dd{\xi}\dd{\eta}                                                        \\
        & \leq\frac{1}{\muppi}\sum_{k=1}^n\sum_{j=1}^{m_k}\qty(\iint_{H_k^{(j)}\cap D(z,2r)}+\iint_{H_k^{(j)}\setminus D(z,2r)})\abs{\pdv{\Phi}{\overline{\zeta}}}\abs{\phi_{\zeta,k}^{(j)}(z)-\frac{1}{z-\zeta}}\dd{\xi}\dd{\eta}
    \end{align*}
    The estimates in \cref{itm:integralofcontinuousextensionofholomorphicproperties_antiholomorphicderivativebound} of \cref{prop:integralofcontinuousextensionofholomorphicproperties}, in tandem with those from \cref{eq:mergelyan_familybounds} now give that
    \begin{align*}
        \abs{\Psi(z)-\Phi(z)} & \leq 18704r\omega_f(r)\sum_{k=1}^n\sum_{j=1}^{m_k}\iint_{H_k^{(j)}\setminus D(z,2r)}\frac{1}{\abs{z-\zeta}^3}\dd{\xi}\dd{\eta}                                                           \\
        & \quad+\frac{4\omega_f(r)}{r}\sum_{k=1}^n\sum_{j=1}^{m_k}\iint_{H_k^{(j)}\cap D(z,2r)}\qty(\frac{584}r+\frac{1}{\abs{z-\zeta}})\dd{\xi}\dd{\eta}                                          \\
        & \leq 18704r\omega_f(r)\iint_{\abs{\zeta}>2r}\frac{\dd{\xi}\dd{\eta}}{\abs{\zeta}^3}+\frac{4\omega_f(r)}{r}\iint_{\abs{\zeta}<2r}\qty(\frac{584}r+\frac{1}{\abs{\zeta}})\dd{\xi}\dd{\eta}
    \end{align*}
    through a linear change of variables. Now evaluation via polar coordinates (with \(\rho\ee^{\ii\theta}=\xi+\ii\eta\), \(\dd{\xi}\wedge\dd{\eta}=\rho\dd{\rho}\wedge\dd{\theta}\)) yields a revised upper bound of
    \begin{multline*}
        18704r\omega_f(r)\int_0^{2\muppi}\int_{2r}^\infty\frac{\dd{\rho}\dd{\theta}}{\rho^2}+\frac{4\omega_f(r)}{r}\qty(\frac{584}{r}\operatorname{area}D(0,2r)+\int_0^{2\muppi}\int_0^{2r}\dd{\rho}\dd{\theta})\\
        =18704r\omega_f(r)2\muppi\eval{\frac{1}{\rho}}_{\infty}^{2r}+\frac{9344\muppi\omega_f(r)r^2}{r^2}+\frac{4\omega_f(r)}{r}4r\muppi\\
        =18704\muppi\omega_f(r)+9344\muppi\omega_f(r)+16\muppi\omega_f(r)=28064\muppi\omega_f(r).
    \end{multline*}
    Runge's Theorem (\cref{thm:runge}) provides the existence of some rational function \(\psi\) with poles in \(E\) such that \[\sup_{z\in K}\abs{\psi(z)-\Psi(z)}\leq\muppi\omega_f(r)\] since \(\Psi\) is holomorphic on a neighborhood of \(K\) (to assure this bound is positive, we assume \(f\) is not identically zero, otherwise the assertion is trivial). Therefore, for all \(z\in K\), we have \[\sup_{z\in K}\abs{\psi(z)-\Phi(z)}\leq\sup_{z\in K}\abs{\psi(z)-\Psi(z)}+\abs{\Psi(z)-\Phi(z)}\leq 28065\muppi\omega_f(r).\] Because \(\lim_{r\to0^+}\omega_f(r)=0\), for any \(\varepsilon>0\), there exists a \(r>0\) such that \[\omega_f(r)<\frac{\varepsilon}{28065\muppi}.\]
    Hence for any such \(\varepsilon\), we now construct \(\psi\) in accordance with an \(r\) satisfying \(28065\muppi\omega_f(r)<\varepsilon\).
\end{proof}
