\subsection{Analyticity and Holomorphy}\label{sec:analyticityandholomorphy}
The Cauchy--Goursat Formula (\cref{thm:cauchygoursatformula}) can also be generalized into a result that equates complex integration and differentiation:
\begin{theorem}[\textsc{Cauchy--Goursat}]\label{thm:cauchydifferentiationformula}
    Let \(U\subset\mathbb{C}\) be an open region bounded by a simple closed boundary \(\partial U\), and let \(f:U\to\mathbb{C}\) be holomorphic and continuous over \(\overline{U}\). Then \(\forall z\in U\), \(\forall n\in\mathbb{N}\), \(f^{(n)}(z)\) exists, and
    \begin{equation}
        f^{(n)}(z)=\frac{n!}{2\uppi\ii}\oint_{\partial U}\frac{f(\zeta)}{\paren{\zeta-z}^{n+1}}\ddzeta.\label{eq:cauchydifferentiationformula_statement}
    \end{equation}
    Additionally, since \(U\) is open, \(\forall z_0\in U\), \(\forall r>0\) such that the closed disk \(\overline{D\paren{z_0,r}}\subset U\), \(f\) has the uniformly and absolutely convergent Taylor expansion
    \begin{equation}
        f(z)=\sum_{j=0}^\infty a_j\paren{z-z_0}^j,\label{eq:cauchydifferentiationformula_taylorseries}
    \end{equation}
    where
    \begin{equation}
        a_j=\frac{1}{2\uppi\ii}\oint_{\partial U}\frac{f(\zeta)}{\paren{\zeta-z}^{j+1}}\ddzeta\label{eq:cauchydifferentiationformula_taylorseriescoefficients}
    \end{equation} on \(\overline{D\paren{z_0,r}}\).
\end{theorem}
\begin{proof}
    \(\forall z_0\in U\), \(\forall z\in D\paren{z_0,r}\subset U\), by \cref{thm:cauchygoursatformula}, \[f(z)-f\paren{z_0}=\frac{1}{2\uppi\ii}\oint_{\partial U}\paren{\frac{f(\zeta)}{\zeta-z}-\frac{f(\zeta)}{\zeta-z_0}}\ddzeta=\frac{z-z_0}{2\uppi\ii}\oint_{\partial U}\frac{f(\zeta)\ddzeta}{\paren{\zeta-z}\paren{\zeta-z_0}},\]
    and dividing by \(z-z_0\), the above is equal to \[\frac{f(z)-f\paren{z_0}}{z-z_0}=\frac{1}{2\uppi\ii}\oint_{\partial U}\frac{f(\zeta)\ddzeta}{\paren{\zeta-z}\paren{\zeta-z_0}}.\]
    Since
    \begin{align}
        \frac{f(z)-f\paren{z_0}}{z-z_0}-\frac{1}{2\uppi\ii}\oint_{\partial U}\frac{f(\zeta)\ddzeta}{\paren{\zeta-z_0}^2} & =\frac{1}{2\uppi\ii}\oint_{\partial U}\frac{f\paren{\zeta}}{\zeta-z_0}\paren{\frac{1}{\zeta-z}-\frac{1}{\zeta-z_0}}\ddzeta\nonumber                                      \\
                                                                                                                         & =\frac{z-z_0}{2\uppi\ii}\oint_{\partial U}\frac{f(\zeta)}{(\zeta-z)\paren{\zeta-z_0}^2}\ddzeta.\label{eq:cauchydifferentiationformula_differenceoffirstorderdifferences}
    \end{align}
    Let \(d\) be the distance from \(z_0\) to \(\partial U\), and it follows that \(0<r<d\). Then, since \(\abs{z-z_0}<r\) and \(\abs{\zeta-z_0}\geq d\), \(\abs{\zeta-z}\geq d-r\). Then the absolute value of the integrand of \cref{eq:cauchydifferentiationformula_differenceoffirstorderdifferences} is bounded above by \(\frac{M}{d^2(d-r)}\), where \(M\) is the maximum of \(f(\zeta)\), which exists by \cref{thm:continuousfunctionboundedoncompact}. Then, \[\abs{\frac{z-z_0}{2\uppi\ii}\oint_{\partial U}\frac{f(\zeta)}{(\zeta-z)\paren{\zeta-z_0}^2}\ddzeta}\leq\frac{\abs{z-z_0}}{2\uppi}\frac{M}{d^2(d-r)}\oint_{\partial U}\abs{\ddzeta}.\] As \(z\to z_0\), the difference vanishes, and therefore, \[f'\paren{z_0}=\frac{1}{2\uppi\ii}\oint_{\partial U}\frac{f(\zeta)}{\paren{\zeta-z_0}^2}\ddzeta.\]
    Now inductively assume that \cref{eq:cauchydifferentiationformula_statement} is true for a given \(n=k\in\mathbb{N}\), or \[f^{(k)}(z)=\frac{k!}{2\uppi\ii}\oint_{\partial U}\frac{f(\zeta)}{\paren{\zeta-z}^{k+1}}\ddzeta.\]
    Notice the expansion of the kernel, convergent since \(\abs{z-z_0}<\abs{\zeta-z_0}\):
    \begin{equation}
        \frac{1}{\zeta-z}=\frac{1}{\zeta-z_0}\cdot\frac{\zeta-z_0}{\zeta-z}=\frac{1}{\zeta-z_0}\cdot\frac{1}{1-\frac{z-z_0}{\zeta-z_0}}=\frac{1}{\zeta-z_0}\sum_{j=0}^{\infty}\paren{\frac{z-z_0}{\zeta-z_0}}^j.\label{eq:cauchydifferentiationformula_kernelexpansion}
    \end{equation}
    Then,
    \begin{align*}
        f^{(k)}(z) & =\frac{k!}{2\uppi\ii}\oint_{\partial U}\frac{f(\zeta)}{(\zeta-z)^{k+1}}\ddzeta                                                                           \\
                   & =\frac{k!}{2\uppi\ii}\oint_{\partial U}\frac{f(\zeta)}{\paren{\zeta-z_0}^{k+1}}\paren{\sum_{j=0}^{\infty}\paren{\frac{z-z_0}{\zeta-z_0}}^j}^{k+1}\ddzeta \\
                   & =f^{(k)}\paren{z_0}+\frac{(k+1)!\paren{z-z_0}}{2\uppi\ii}\oint_{\partial U}\frac{f(\zeta)}{\paren{\zeta-z_0}^{k+2}}\ddzeta                               \\
                   & \quad+\order{\abs{z-z_0}^2},
    \end{align*}
    where the remainder terms \(\order{\abs{z-z_0}^2}\) resemble
    \[\paren{z-z_0}^2\frac{k!}{2\uppi\ii}\brackets{k+1+\binom{k+1}{2}}\oint_{\partial U}\frac{f(\zeta)}{\paren{\zeta-z_0}^{k+3}}\ddzeta+\order{\abs{z-z_0}^3}.\]
    The difference quotient is equal to \[\frac{f^{(k)}(z)-f^{(k)}\paren{z_0}}{z-z_0}=\frac{(k+1)!}{2\uppi\ii}\oint_{\partial U}\frac{f(\zeta)}{\paren{\zeta-z_0}^{k+2}}\ddzeta+\order{\abs{z-z_0}}.\] As \(z\to z_0\), the remainder terms vanish, and \[f^{(k+1)}\paren{z_0}=\frac{(k+1)!}{2\uppi\ii}\oint_{\partial U}\frac{f(\zeta)}{\paren{\zeta-z_0}^{k+2}}\ddzeta.\]
    By induction, \cref{eq:cauchydifferentiationformula_statement} is valid. By substituting \cref{eq:cauchydifferentiationformula_kernelexpansion} into \cref{eq:cauchygoursatformula}, we obtain \[f(z)=\frac{1}{2\uppi\ii}\oint_{\partial U}\frac{f(\zeta)}{\zeta-z_0}\sum_{j=0}^{\infty}\paren{\frac{z-z_0}{\zeta-z_0}}^j\ddzeta=\frac{1}{2\uppi\ii}\oint_{\partial U}\sum_{j=0}^{\infty}\paren{z-z_0}^j\frac{f(\zeta)\ddzeta}{\paren{\zeta-z_0}^{j+1}}.\]
    Because \(f(\zeta)\) is continuous over \(\partial U\), it is bounded by a constant \(M\). Additionally, since \(\abs{z-z_0}<\abs{\zeta-z_0}\), the sum is termwise bounded by the convergent series \[\sum_{j=0}^\infty\frac{Mr^j}{\inf_{\xi\in\partial U}\abs{\xi-z_0}^{j+1}}.\]
    By the Weierstrass \(M\)--Test (\cref{thm:weierstrassmtest}), the series uniformly converges, and we can justify
    \begin{gather*}
        \frac{1}{2\uppi\ii}\oint_{\partial U}\sum_{j=0}^{\infty}\paren{z-z_0}^j\frac{f(\zeta)}{\paren{\zeta-z_0}^{j+1}}\ddzeta=\frac{1}{2\uppi\ii}\sum_{j=0}^{\infty}\oint_{\partial U}\paren{z-z_0}^j\frac{f(\zeta)}{\paren{\zeta-z_0}^{j+1}}\ddzeta\\=\sum_{j=0}^\infty a_j\paren{z-z_0}^j,
    \end{gather*}
    which verifies \cref{eq:cauchydifferentiationformula_taylorseries,eq:cauchydifferentiationformula_taylorseriescoefficients}.
\end{proof}
\begin{remark}
    By induction, we have shown that assuming the existence of the first order derivative of a holomorphic function \(f\), the \(n\)-th order derivative of \(f\) exists \(\forall n\in\mathbb{N}\) and is holomorphic over the same region as \(f^{(n-1)}\). Furthermore, if \(f\) is holomorphic, then \(\forall z\in U\), there exists an open disk enclosing \(z\) such that \(f\) has a convergent Taylor series expansion. This property is known as \textit{analyticity}, and \cref{thm:cauchydifferentiationformula} tells us that all holomorphic functions are analytic. Analytic functions can be expanded into power series, which are termwise differentiable, and therefore complex differentiable. Thus, analyticity and holomorphy are logically equivalent, which is a fundamental difference between real and complex functions.
\end{remark}
The differentiation formula above can be thought of as a generalization of \cref{thm:cauchygoursatformula}, and provides similar utility in the evaluation of integrals:
\begin{example}\label{ex:legendrepolynomialintegralformula}
    A \textit{Legendre polynomial} is a polynomial whose explicit equation is given by
    \begin{equation}
        P_n(z)=\frac{1}{2^n n!}\dv[n]{z}({\qty(z^2-1)}^n).\label{eq:legendrepolynomialintegralformula_rodriguesformula}
    \end{equation}
    Prove the integral form \[P_n(z)=\frac{1}{2\uppi\ii}\oint_\gamma\frac{{\qty(\zeta^2-1)}^n}{2^n{(\zeta-z)}^{n+1}}\ddzeta,\] where \(\gamma\) is a simple closed curve enclosing \(z\).
\end{example}
\begin{proof}
    By applying Cauchy--Goursat (\cref{thm:cauchydifferentiationformula}) on \cref{eq:legendrepolynomialintegralformula_rodriguesformula}, we get that \[P_n(z)=\frac{1}{2^{n+1}\uppi\ii}\oint_{\gamma}\frac{{\qty(\zeta^2-1)}^n}{{(\zeta-z)}^{n+1}}\ddzeta,\] as desired.
\end{proof}
\begin{theorem}[\textsc{Cauchy's Estimate}]\label{thm:cauchysestimate}
    For a function \(f:U\to\mathbb{C}\) holomorphic over \(U\subseteq\mathbb{C}\) and \(\forall z_0\in U\) and \(\forall R>0\) such that \(\overline{D\paren{z_0,R}}\subseteq{U}\), \(\forall n\in\mathbb{N}\), \[\abs{f^{(n)}\paren{z_0}}\leq\frac{n!M}{R^n},\]
    where \[M=\max_{z\in\overline{D\paren{z_0,R}}}\abs{f(z)}.\]
\end{theorem}
\begin{proof}
    By the Differentiation Formula (\cref{thm:cauchydifferentiationformula}), \(\forall n\in\mathbb{N}\), \[f^{(n)}\paren{z_0}=\frac{n!}{2\uppi\ii}\oint_{\partial D\paren{z_0,R}}\frac{f(\zeta)}{\paren{\zeta-z_0}^{n+1}}\ddzeta.\]
    Because \(f(z)\) is continuous over the boundary \(\partial D\paren{z_0,R}\), it is bounded by \(M\). Thus,
    \[\abs{f^{(n)}\paren{z_0}}\leq\frac{n!}{2\uppi}\int_0^{2\uppi}\frac{M}{\paren{\ee^{\ii\theta}R}^{n+1}}\ee^{\ii\theta}R\dd{\theta}=\frac{n!M}{R^n},\] as desired.
\end{proof}
\cref{thm:nthderivativeboundedl1norm} will profoundly generalize this statement significantly. The relationship between the derivatives of a holomorphic function and the function itself is an important property of holomorphic functions.
\begin{example}
    Let \(f\) be entire and \(\forall z\in\mathbb{C}\), \(\abs{f(z)}\leq M\ee^{\abs{z}}\). Prove that \(\forall n\in\mathbb{N}\), \(\abs{f(0)}\leq M\) and \[\abs{f^{(n)}(0)}\leq Mn!\qty(\frac{\ee}{n})^n.\]
\end{example}
\begin{proof}
    \(|f(0)|\leq M\) is obviously true by letting \(z=0\). Then \(\forall R>0\), by Cauchy's Estimate (\cref{thm:cauchysestimate}), \[\abs{f^{(n)}(0)}\leq Mn!\frac{\ee^{R}}{R^n}.\]
    By letting \(R=n\), the conclusion follows. In fact, this is the tightest possible inequality. Consider \(\varphi(R)=Mn!\frac{\ee^R}{R^n}\) to be a function of \(R\). It attains its minimum as its derivative vanishes:
    \[\varphi'(R)=Mn!\frac{\ee^RR^n-n\ee^RR^{n-1}}{R^{2n}}=0\Longleftrightarrow R^n=nR^{n-1}\Longleftrightarrow R=n.\] To confirm it as a minimum, we calculate the second order derivative:
    \[\varphi''(R)=Mn!\ee^R\qty(\frac{1}{R^n}-\frac{2n}{R^{n+1}}+\frac{n(n+1)}{R^{n+2}})\implies\varphi''(n)=M(n-1)!\frac{\ee^n}{n^n},\]
    which is positive and convex.
\end{proof}
The following theorem, albeit originally proven by Cauchy in 1844, shows a fundamental difference between holomorphic functions on proper subsets of \(\mathbb{C}\) and entire functions.
\begin{theorem}[\textsc{Liouville}]\label{thm:liouville}
    Any bounded entire function is constant.
\end{theorem}
\begin{proof}
    Let \(f:\mathbb{C}\to\mathbb{C}\) be entire. Then, \(\forall z_0\in\mathbb{C}\), \(\forall R>0\), \(f\) is holomorphic over \(\overline{D\paren{z_0,R}}\). By \cref{thm:cauchysestimate}, \[\abs{f'\paren{z_0}}\leq\frac{M}{R},\] where \(M=\sup_{z\in\mathbb{C}}\abs{f(z)}\). By letting \(R\to\infty\), \(f'\paren{z_0}\) where \(z_0\) is any arbitrary value in \(\mathbb{C}\). Therefore, \(f(z)\) is constant.
\end{proof}
\begin{proof}[Alternative Proof]
    Let \(a,b\in\mathbb{C}\) be distinct and arbitrarily chosen. Let
    \(f:\mathbb{C}\to\mathbb{C}\) be entire and bounded such that \(\abs{f}\leq M\)
    for some \(M>0\). Let \(R>\abs{a},\abs{b}\). Since \(a\neq b\),
    \(\exists\varepsilon>0\) such that
    \(\overline{D(a,\varepsilon)}\cup\overline{D(b,\varepsilon)}=\varnothing\). By
    the Cauchy--Goursat Theorem (\cref{thm:cauchygoursattheorem}), we have \[\oint_{\partial D(0,R)}\frac{f(z)}{(z-a)(z-b)}\ddz=\qty(\oint_{\partial D(a,\varepsilon)}+\oint_{\partial D(b,\varepsilon)})\frac{f(z)}{(z-a)(z-b)}\ddz.\]
    Since \(z\mapsto\frac{f(z)}{z-a}\) is holomorphic on the disk centered at \(b\) and \(z\mapsto\frac{f(z)}{z-b}\) is holomorphic on the disk centered at \(a\), by the Cauchy--Goursat Formula (\cref{thm:cauchygoursatformula}), we have \[\oint_{\partial D(0,R)}\frac{f(z)}{(z-a)(z-b)}\ddz=2\uppi\ii\qty(\frac{f(b)}{b-a}+\frac{f(a)}{a-b}).\]
    On the contrary, we also have
    \begin{align*}
        \abs{\qty(\oint_{\partial D(a,\varepsilon)}+\oint_{\partial D(b,\varepsilon)})\frac{f(z)}{(z-a)(z-b)}\ddz} & \leq M\oint_{\partial D(0,R)}\frac{\abs{\ddz}}{\abs{z-a}\abs{z-b}} \\
                                                                                                                   & =\frac{2\uppi MR}{(R-a)(R-b)}                                      \\
                                                                                                                   & \to 0\qas{R\to\infty}.
    \end{align*}
    We conclude that \[\frac{2\uppi\ii}{b-a}\qty(f(b)-f(a))=0\] for all distinct complex \(a\) and \(b\). Hence, \(f\) is a constant function.
\end{proof}
\begin{theorem}[\textsc{Morera}]\label{thm:morera}
    Let \(U\subseteq\mathbb{C}\) and \(f:U\to\mathbb{C}\) be continuous over \(U\). If for any closed triangular contour \(\gamma\subset U\), \[\oint_{\gamma}f(\zeta)\ddzeta=0,\] then \(f\) is holomorphic over \(U\).
\end{theorem}
\begin{proof}
    Let \(z_0\in U\) be arbitrary. Since \(U\) is open, \(\exists r>0\) such that
    \(\overline{D}=\overline{D\qty(z_0,r)}\subset U\). Define \[F(z)=\int_{z_0}^z f(\zeta)\ddzeta,\] where the path is a straight line segment, and \(F\) is well-defined for \(z\in D\). Now \[F'(z)=\lim_{\Delta z\to 0}\frac{F(z+\Delta z)-F(z)}{\Delta z}=\lim_{\Delta z\to 0}\frac{\qty(\int_a^{z+\Delta z}+\int_z^a+\int_{z+\Delta z}^z+\int_z^{z+\Delta z})f(\zeta)\ddzeta}{\Delta z}.\] Note that the first three integrals sum to form a closed triangular curve and hence vanish by assumption. Therefore, \[F'(z)=\lim_{\Delta z\to 0}\frac{1}{\Delta z}\int_z^{z+\Delta z}f(\zeta)\ddzeta.\] By the continuity of \(f\) at \(z\), for any \(\varepsilon>0\), \(\exists\delta>0\) such that \(\abs{\zeta-z}<\delta\implies\abs{f(\zeta)-f(z)}<\varepsilon\). Then, for \(\abs{\Delta z}<\delta\), \[\abs{\frac{1}{\Delta z}\int_z^{z+\Delta z}f(\zeta)\ddzeta-f(z)}=\abs{\frac{1}{\Delta z}\int_z^{z+\Delta z}(f(\zeta)-f(z))\ddzeta}\leq\varepsilon.\] Thus, \(F'(z)=f(z)\) for all \(z\in D\). Since \(z_0\) was arbitrary, \(f\) is holomorphic over \(U\).
\end{proof}
\begin{theorem}\label{thm:nthderivativeboundedl1norm}
    Let \(U\subseteq\mathbb{C}\) be open, let \(K\subset U\) be compact and \(V\supset K\) be open such that \(\overline{V}\subset U\) (\(V\) is a neighborhood of \(K\) that is relatively compact in \(U\)). Let \(f(z)\) be holomorphic in \(U\). Then there exists a sequence \(\cbraces{c_n}\subset\mathbb{R}\) dependent only on \(K\) and \(V\) (independent of \(f\) and \(z\)) such that \(\forall n\in\mathbb{N}\),
    \begin{equation}
        \sup_{z\in K}\abs{f^{(n)}(z)}\leq c_n\norm{f}_{L^1(V)},\label{eq:nthderivativeboundedl1norm_statement}
    \end{equation}
    where \(\norm{f}_{L^p(V)}\) denotes \[\paren{\int_V\abs{f(z)}^p\ddx\wedge\ddy}^{\flatfrac{1}{p}}.\]
\end{theorem}
\begin{proof}
    Let \(\varphi\in C^\infty\paren{\mathbb{C}}\)  satisfy \(\supp(\varphi)\subset V\) and be identically equal to 1 over some open neighborhood \(W\) of \(K\) relatively compact in \(V\). Since \(f\in C^\infty\paren{U}\), by the Cauchy--Pompeiu Theorem (\cref{thm:pompeiu}) on \(f(z)\varphi(z)\in C^\infty\paren{\overline{U}}\), \[f(z)\varphi(z)=\frac{1}{2\uppi\ii}\paren{\int_{\partial U}\frac{f(\zeta)\varphi(\zeta)}{\zeta-z}\ddzeta-\int_{U}\pdv{f(\zeta)\varphi(\zeta)}{\overline{\zeta}}\cdot\frac{\dd{\overline{\zeta}}\wedge\ddzeta}{\zeta-z}}.\]
    By the product rule, \[\pdv{f(\zeta)\varphi(\zeta)}{\overline{\zeta}}=\pdv{\varphi(\zeta)}{\overline{\zeta}}f(\zeta),\] and since \(\partial U\subset\mathbb{C}\setminus\supp(\varphi)\), the first term vanishes, resulting in \[f(z)\varphi(z)=-\frac{1}{2\uppi\ii}\int_U\pdv{\varphi(\zeta)}{\overline{\zeta}}f(\zeta)\cdot\frac{\dd{\overline{\zeta}}\wedge\ddzeta}{\zeta-z}.\] Let \(K_1\) denote \(\supp\paren{\pdv{\varphi(\zeta)}{\overline{\zeta}}}\), and \(\forall z\in K\), \(\varphi(z)=1\). Therefore, \[f(z)=\frac{1}{2\uppi\ii}\int_{K_1}f(\zeta)\cdot\pdv{\varphi(\zeta)}{\overline{\zeta}}\cdot\frac{\ddzeta\wedge\dd{\overline{\zeta}}}{\zeta-z}.\]
    We can differentiate within the integral as \(f(\zeta)\cdot\pdv{\varphi(\zeta)}{\overline{\zeta}}\) is \(C^\infty\) and bounded over \(K_1\), and thus the integrand is uniformly bounded by an integrable function independent of \(\zeta\):
    \[f^{(n)}(z)=\frac{n!}{2\uppi\ii}\int_{K_1} f(\zeta)\cdot\pdv{\varphi(\zeta)}{\overline{\zeta}}\cdot\frac{\ddzeta\wedge\dd{\overline{\zeta}}}{\paren{\zeta-z}^{n+1}},\]
    and by the triangle inequality,
    \[\abs{f^{(n)}(z)}\leq\frac{n!}{2\uppi}\int_{K_1}\abs{f(\zeta)}\abs{\pdv{\varphi(\zeta)}{\overline{\zeta}}}\frac{\abs{\ddzeta\wedge\dd{\overline{\zeta}}}}{\abs{\zeta-z}^{n+1}}.\]
    Notice that over \(W\), \(\varphi=1\), \(\varphi'=0\), and is disjoint from \(K_1\) (or that \(W\cap K_1=\varnothing\)). Then, the distance between \(W\) and \(K\) is positive and the two are disjoint. Therefore, \(\exists M>0\) such that
    \[\frac{1}{\abs{\zeta-z}}\leq M,\] and thus, \[\abs{\pdv{\varphi(\zeta)}{\overline{\zeta}}}\frac{1}{\abs{\zeta-z}^{n+1}}\] can be bounded by a sequence \(\cbraces{c'_n}\), independent of \(f\) and dependent only on \(n\) and the sets \(K\) and \(V\). Then, \[\abs{f^{(n)}(z)}\leq\frac{n!}{2\uppi}\int_{K_1}c'_n\abs{f(\zeta)}{\abs{\ddzeta\wedge\dd{\overline{\zeta}}}}=\frac{n!}{\uppi}\int_{K_1}c'_n\abs{f(\zeta)}{\abs{\ddx\wedge\ddy}}.\] Because \(K_1\) is compact, it has a finite area \(\mathrm{area}\qty(K_1)\), and we can define a new sequence \(c_n=\frac{n!}{\uppi}c'_n\mathrm{area}\qty(K_1)\) to find that \[\abs{f^{(n)}(z)}\leq c_n\int_{K_1}\abs{f(\zeta)}{\abs{\ddx\wedge\ddy}}\leq c_n\int_{V}\abs{f(\zeta)}{\abs{\ddx\wedge\ddy}}.\]

    The problem now stands to prove that \(\varphi(z)\) exists in the first place, which requires a topological argument to be later discussed in \cref{thm:bumpfunctionexistence}.
\end{proof}
\begin{corollary}\label{cor:nthderivativeboundedsupremum}
    Let \(U\subseteq\mathbb{C}\) be open, let \(K\subset U\) be compact and \(V\supset K\) be open such that \(\overline{V}\subset U\). For any holomorphic function \(f(z)\) in \(U\), there exist constants (independent of \(z\) and \(f\)) \(\cbraces{c_n}\) such that \[\sup_{z\in K}\abs{f^{(n)}(z)}\leq c_n\sup_{z\in V}\abs{f(z)}.\]
\end{corollary}
\begin{proof}
    Starting from \cref{eq:nthderivativeboundedl1norm_statement}, observe that \[c_n\norm{f}_{L^1(V)}\leq c_n\mathrm{area}(V)\sup_{z\in V}\abs{f(z)},\] and we can define a new set of constants equal to \(c_n\mathrm{area}(V)\), which are still independent of \(z\).
\end{proof}
For the next theorem we will briefly introduce the concept of \textit{analytic continuation}.
\begin{definition}[Analytic Continuation]\label{def:analyticcontinuation}
    Let \(U\subseteq\mathbb{C}\) be open, and let \(f:U\to\mathbb{C}\) be holomorphic. Let \(V\subseteq\mathbb{C}\) be open with \(U\subseteq V\). A function
    \(F:U\cap V\to\mathbb{C}\) is an \textit{analytic continuation} of \(f\) to \(V\) if:
    \begin{enumerate}
        \item \(F\) is holomorphic on \(V\), and
        \item \(F\equiv f\) on \(U\).
    \end{enumerate}
\end{definition}
The concept of analytic continuation and its consequent problems and properties will be discussed in more detail in \cref{sec:analyticcontinuation}. For now, we will prove a theorem that is a direct consequence of the Cauchy--Goursat Differentiation Formula (\cref{thm:cauchydifferentiationformula}) and the existence of holomorphic functions with removable singularities.
\begin{theorem}[Riemann]\label{thm:riemannremovablesingularities}
    Let \(D^*\qty(z_0,r)=D\paren{z_0,r}\setminus\cbraces{z_0}\) (known as a punctured disk), and \(f:D^*\paren{z_0,r}\to\mathbb{C}\) be holomorphic and bounded. Then \(f\) can be analytically continued to \(D\paren{z_0,r}\).
\end{theorem}
\begin{proof}
    Define the auxiliary function \[\varphi(z)=
        \begin{dcases}
            \paren{z-z_0}^2f(z) & \qif* z\in D^*\paren{z_0,r}, \\
            0                   & \qif* z=z_0.
        \end{dcases}\]
    \(\varphi(z)\) is bounded and continuously differentiable on \(D\paren{z_0,r}\) and satisfies the Cauchy--Riemann Equations since
    \[\lim_{z\to z_0}\frac{\varphi(z)-\varphi\paren{z_0}}{z-z_0}=\frac{\paren{z-z_0}^2f(z)}{z-z_0}=\lim_{z\to z_0}\paren{z-z_0}f(z)=0,\] meaning that \(\dv{\varphi}{z}\paren{z_0}=0\). For \(z\in D^*\paren{z_0,r}\), \[\varphi'(z)=2\paren{z-z_0}f(z)+\paren{z-z_0}^2f'(z).\] As \(z\to z_0\), \(\varphi(z)\to0\), meaning that \(\varphi\) is holomorphic over \(D\paren{z_0,r}\). By \cref{thm:cauchydifferentiationformula}, \[\varphi(z)=\sum_{j=2}^\infty a_j\paren{z-z_0}^j,\]
    which is convergent over \(D\paren{z_0,r}\). Then we can define \[\widetilde{f}(z)=\frac{\varphi(z)}{\paren{z-z_0}^2}=\sum_{j=0}^\infty a_{j+2}\paren{z-z_0}^j\] over the same disk of convergence. Over the punctured disk, \(\widetilde{f}(z)=f(z)\), and therefore \(\widetilde{f}\) is an analytic continuation of \(f\).
\end{proof}
\subimport{partitions_of_unity/}{index.tex}