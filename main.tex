\documentclass{article}
\usepackage[utf8]{inputenc}
\usepackage{parskip}
\usepackage{tikz}
\usepackage{mathtools}
\usepackage{amssymb}
\usepackage{amsfonts}
\usepackage{amsthm}
\usepackage{cancel}
\usepackage{esint}
\usepackage{graphicx}
\usepackage{tgpagella}
\usepackage[hidelinks]{hyperref}

\allowdisplaybreaks[2]
\usepackage[left=3.5cm,right=3.5cm]{geometry}

\usepackage[english]{babel}
\usepackage{pgfplots}
\usepgfplotslibrary{fillbetween}
\usetikzlibrary{patterns}

\pgfplotsset{compat=1.18}
\pgfplotsset{
    every axis/.append style={
        axis on top=true,
        axis x line=middle,
        axis y line=middle,
        axis equal,
        axis line style={<->,color=black}, 
        xlabel={$x$},
        ylabel={$y$},
    }
}
\newcommand{\dd}{\mathrm{d}}
\newcommand{\supp}{\operatorname{supp}}

\title{Complex Analysis}
\author{Slipper King}
\date{May 2025}

\newtheorem{theorem}{Theorem}[section]
\newtheorem{lemma}{Lemma}[section]
%\renewcommand\qedsymbol{$\blacksquare$}

\theoremstyle{remark}
\newtheorem{example}{Example}[subsection]

\theoremstyle{definition}
\newtheorem{definition}{Definition}[section]

\theoremstyle{remark}
\newtheorem*{remark}{Remark}

\numberwithin{equation}{section}
\newcommand{\reseteqcounter}{\setcounter{equation}{0}}

\begin{document}
\maketitle
\tableofcontents
\section{Prerequisites}
\subsection{Calculus}
\subsection{}
\begin{definition}[Closure of a Set]\label{def:closure}
For a set $A\in\mathbb{C}^n$, define the closure of $A$, or $\overline{A}$ to be the intersection of all closed sets containing $A$. In other words, it is the union of $A$ and every accumulation point.
\end{definition}
\begin{definition}[Compactness of a Set]\label{def:compactsets}
    A set \(X\in\mathbb{C}^n\) is compact if and only if $X$ is closed and bounded.
\end{definition}
\begin{theorem}[Heine-Borel Theorem]\label{theorem:heineborel}
    A set \(X\in\mathbb{C}^n\) is compact if and only if every open cover has a finite subcover.
\end{theorem}
\begin{proof}
    We will first show that any set satisfying the condition is compact.

    First we will show that $X$ is bounded. Suppose that $\forall R>0$, $\exists x\in X$ where \(\|x\|>R\). Consider the collection of open sets \[\mathcal{U}=\{B(0,k)\mid \forall k\in\mathbb{N}\}\] where $B(a,b)$ is the open ball with radius $b$ centered at \(a\). $\mathcal{U}$ forms an open cover of $X$. Then there exists a finite subcover $\{B(0,k_1),\ldots,B(0,k_m)\}$ covering $X$. Then, \[X\subseteq\bigcup_{i=1}^mB(0,k_i)=B(0,\max(k_1,\ldots k_m)).\] By contradiction, \(X\) must be bounded.

    $X$ must also be a closed set. For the sake of contradiction, assume that there exists a point $x_0\in\overline{X}\setminus X$. Since \(x_0\notin X\), the following open collection of sets covers $X$:
    \[\mathcal{U}=\left\{\mathbb{C}^n\setminus\overline{B}\left(x_0,\frac{1}{k}\right)\;\middle|\; \forall k\in\mathbb{N}\right\}.\] By assumption, there exists a finite subcover \(\mathcal{C}=\left\{\mathbb{C}^n\setminus\overline{B}\left(x_0,\frac{1}{k_i}\right)\;\middle|\; i=1,2,\ldots,m\right\}\). Then, \[X\subseteq\mathbb{C}^n\setminus\overline{B}\left(x_0,\frac{1}{\max(k_1,\ldots,k_m)}\right),\]
    and that $X\cap\overline{B}\left(x_0,\frac{1}{\max(k_1,\ldots,k_m)}\right)=\emptyset$. However, by the definition of the accumulation point, every open neighborhood of the accumulation point must intersect $X$. Therefore, by contradiction, $X$ is closed.

    To prove the converse statement, let \(\mathcal{U}\) be an arbitrary open cover of \(X\). By the assumption that \(X\) is bounded, $\exists $\[X\subseteq\]
\end{proof}
\begin{definition}[Support]\label{def:support}
    For a set $X$ and a function $f:X\to\mathbb{C}$, the support, denoted as \(\supp(f)=\overline{\{x\in X\mid f(x)\neq 0\}}\), or the closure of the set for which $f$ is non-zero.
\end{definition}
\begin{remark}
    We are primarily concerned when the support of a function is compact, or if the support is bounded. For smooth functions, functions that are compactly supported are called bump functions.
\end{remark}
\begin{theorem}[Green's Theorem]\label{theorem:complexgreen}
    
\end{theorem}
\begin{theorem}[Pompeiu's Theorem]\label{theorem:pompeiu}
    Let \(U\subseteq\mathbb{C}\) where $\partial U$ is piecewise smooth. Let \(f(z)\in C^1(\overline{U})\). Then $\forall z\in U$, \begin{equation}
        f(z)=\frac{1}{2\pi i}\left(\int_{\partial U}\frac{f(\zeta)}{\zeta-z}\dd\zeta-\int_{U}\frac{\partial f(\zeta)}{\partial\overline{\zeta}}\frac{\dd\overline{\zeta}\wedge \dd\zeta}{\zeta-z}\right).
    \end{equation}
\end{theorem}
\begin{proof}
    Let \(D(z,\varepsilon)\) denote the open disc centered around \(z\) with a radius of $\varepsilon$.
\end{proof}
\end{document}