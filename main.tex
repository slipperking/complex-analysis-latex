\documentclass{article}
\usepackage[utf8]{inputenc}
\usepackage{parskip}
\usepackage{tikz}
\usepackage{mathtools}
\usepackage{amssymb}
\usepackage{amsfonts}
\usepackage{amsthm}
\usepackage{cancel}
\usepackage{esint}
\usepackage{graphicx}
\usepackage{tgpagella}
\usepackage[hidelinks]{hyperref}

\allowdisplaybreaks[2]
\usepackage[left=3.5cm,right=3.5cm]{geometry}

\usepackage[english]{babel}
\usepackage{pgfplots}
\usepgfplotslibrary{fillbetween}
\usetikzlibrary{patterns}

\pgfplotsset{compat=1.18}
\pgfplotsset{
    every axis/.append style={
        axis on top=true,
        axis x line=middle,
        axis y line=middle,
        axis equal,
        axis line style={<->,color=black}, 
        xlabel={$x$},
        ylabel={$y$},
    }
}

\newcommand{\dd}[1]{\mathop{\mathit{d}#1}}
\newcommand{\supp}{\operatorname{supp}}

\title{Complex Analysis}
\author{Slipper King}
\date{May 2025}

\newtheorem{theorem}{Theorem}[section]
\newtheorem{lemma}{Lemma}[section]
%\renewcommand\qedsymbol{$\blacksquare$}

\theoremstyle{remark}
\newtheorem{example}{Example}[subsection]

\theoremstyle{definition}
\newtheorem{definition}{Definition}[section]

\theoremstyle{remark}
\newtheorem*{remark}{Remark}

\numberwithin{equation}{section}
\newcommand{\reseteqcounter}{\setcounter{equation}{0}}

\begin{document}
\maketitle
\tableofcontents
\section{Prerequisites}
\subsection{Calculus}
It is well known that a function $f:(a,b)\to\mathbb{R}$ is differentiable at a point $x\in(a,b)$ if the limit \[\lim_{h\to0}\frac{f(x+h)-f(x)}{h}\] exists, and the value of this limit is the derivative of \(f(x)\), denoted by $f'(x)$ or $\frac{\dd{f}}{\dd{x}}$. The value \(\dd{f}=f'(x)\dd{x}\) is the differential of $f(x)$. Partition $[a,b]$ into \(a=x_0<x_1<x_2<\ldots<x\)
\subsection{Topological Preliminaries}
\begin{definition}[Closure of a Set]\label{def:closure}
For a set $X\in\mathbb{C}^n$, define the closure of $X$, or $\overline{X}$ to be the intersection of all closed sets containing $X$. In other words, it is the union of $X$ and every accumulation point (a point $z\in\mathbb C^n$ is an accumulation point of $X$ if for any open set $U$ containing $z$, \((U\setminus\{z\})\cap X\neq\emptyset\)).
\end{definition}
\begin{definition}[Compactness of a Set]\label{def:compactsets}
    A set \(X\in\mathbb{C}^n\) is compact if and only if $X$ is closed and bounded.
\end{definition}
\begin{theorem}[Bolzano-Weierstrass Theorem]\label{theorem:bolzanoweierstrass}
    Every infinite subset \(A\) of a compact set \(X\subset\mathbb{C}^n\) has an accumulation point in \(X\).
\end{theorem}
\begin{proof}
    Since \(X\) is bounded, there exists a closed cube \(Q\subset\mathbb{C}^n\) such that \(A\subseteq X\subset Q\).

    Bisect \(Q_0=Q\) into \(2^{2n}\) congruent sub-cubes. Since \(A\) is infinite and the sub-cubes are finite in number, at least one of the sub-cubes contains infinitely many points of \(A\), and choose one to be \(Q_1\).

    Bisect \(Q_1\) into \(2^{2n}\) sub-cubes, and choose a sub-cube \(Q_2\subset Q_1\) that contains infinitely many points of \(A\). We then obtain the recursive sequence \[Q_0\supset Q_1 \supset Q_2\supset\ldots.\]

    Because the side lengths shrink to zero and the cubes are nested, the intersection
    \[\bigcap_{k=0}^{\infty} Q_k\]
    consists of exactly one point. Call this point \(z_\infty\in\mathbb{C}^n\).

    For each \(k\), \(Q_k\) contains infinitely many points of \(A\). Because the side length of \(Q_k\) tends to zero, for any \(\varepsilon>0\), \(\exists N\in\mathbb{N}\) such that \(\forall k\geq N\), \(Q_k\subset D(z_\infty,\varepsilon)\) where $D(a,b)$ is the open ball with radius $b$ centered at \(a\). Then, \(D(z_\infty, \varepsilon)\) also contains infinitely many points of \(A\). Therefore, \(z_\infty\) is an accumulation point of \(A\).

    We now show that \(z_\infty\in X\). Suppose for contradiction that \(z_\infty\notin X\). Since \(X\) is closed, \(\mathbb{C}^n\setminus X\) is open, and $\exists\delta>0$ such that \[D(z_\infty,\delta)\subset\mathbb{C}^n\setminus X.\] But then, for sufficiently large \(k\), we have \(Q_k \subset D(z_\infty, \delta)\), and hence \(Q_k \cap X = \emptyset\). This contradicts the construction of \(Q_k\), which ensures that \(Q_k\) contains infinitely many points of \(A \subset X\).
\end{proof}
\begin{theorem}[Heine-Borel Theorem]\label{theorem:heineborel}
    A set \(X\in\mathbb{C}^n\) is compact if and only if every open cover has finite subcovering.
\end{theorem}
\begin{proof}
    We will first show that any set satisfying the condition is compact.

    First we will show that \(X\) is bounded. Suppose that $\forall R>0$, $\exists z\in X$ where \(\|z\|>R\). Consider the collection of open sets \[\mathcal{U}=\{D(0,k)\mid \forall k\in\mathbb{N}\}.\] $\mathcal{U}$ forms an open cover of $X$. Then there exists a finite subcover $\{D(0,k_1),\ldots,D(0,k_m)\}$ covering $X$. Then, \[X\subseteq\bigcup_{i=1}^mD(0,k_i)=D(0,\max(k_1,\ldots k_m)).\] By contradiction, \(X\) must be bounded.

    $X$ must also be a closed set. For the sake of contradiction, assume that there exists a point $z_0\in\overline{X}\setminus X$. Since \(z_0\notin X\), the following open collection of sets covers $X$:
    \[\mathcal{U}=\left\{\mathbb{C}^n\setminus\overline{B}\left(z_0,\frac{1}{k}\right)\;\middle|\; \forall k\in\mathbb{N}\right\}.\] By assumption, there exists a finite subcover \(\mathcal{C}=\left\{\mathbb{C}^n\setminus\overline{B}\left(z_0,\frac{1}{k_i}\right)\;\middle|\; i=1,2,\ldots,m\right\}\). Then, \[X\subseteq\mathbb{C}^n\setminus\overline{B}\left(z_0,\frac{1}{\max(k_1,\ldots,k_m)}\right),\]
    and that $X\cap\overline{B}\left(z_0,\frac{1}{\max(k_1,\ldots,k_m)}\right)=\emptyset$. However, by the definition of the accumulation point, every open neighborhood of the accumulation point must intersect $X$. Therefore, by contradiction, $X$ is closed.

    We then prove the converse. By the assumption that \(X\) is bounded, \(\exists R>0\) such that the $X$ is contained within the closed cube \[Q=\left\{z\;\middle|\; z\in\mathbb{C}^n, \max_{i\in\{1,\ldots,n\}}\left|\Re(z_i)\right|\le R,\max_{i\in\{1,\ldots,n\}}\left|\Im(z_i)\right|\le R\right\}.\] 

    Assume that there exists an infinite open cover \(\mathcal{U}\) of \(X\) without finite subcovering. Bisect $Q_0=Q$ into $2^{2n}$ sub-cubes (for real and complex parts). Choose $Q_1$ such that $Q_1\cup X$ has no finite subcover of $\mathcal{U}$. Under the previous assumptions, this is possible since if every $\text{sub-cube}\cap X$ had finite subcovering, then $Q_0\cap X=X$ would have finite subcovering. Similarly, choose $Q_2$ by bisecting $Q_1$ in a similar way, and recursively obtain a sequence of cubes:
    \[Q_0\supset Q_1\supset Q_2\supset\ldots\]
    Since the side length of each cube tends to 0, \(\bigcap_{i=0}^\infty Q_i\) consists of a single point $z_{\infty}\in\mathbb{C}^n$. By the Bolzano-Weierstrass Theorem, because \(\forall i\in\mathbb{N}\), $Q_i\cap X\neq\emptyset$, select a point \(z_{i}\in Q_i\cap X\), forming a sequence ${z_k}\in X$ convergent to \(z_\infty\in X\) as $X$ is closed. Therefore, $\exists U\in\mathcal{U}$ where $z_\infty\in U$. Since $U$ is open, $\exists\varepsilon>0$ such that $D(z_\infty,\varepsilon)\subset U$. $\exists N\in\mathbb{N}$ such that $\forall k>N$, \(Q_k\subset D(z_\infty,\varepsilon)\). Then taking the intersection with \(X\) on both sides, \[Q_k\cap X\subseteq D(z_\infty,\varepsilon)\cap X\subset U.\] Our original assumption said that each cube had no finite subcovering. However, $U$ covers $Q_k\cap X$, which is a single open set that covers a nonempty subset. Therefore by contradiction, every open cover has finite subcovering.
\end{proof}
\begin{definition}[Support of a Function]\label{def:support}
    For a set $X$ and a function $f:X\to\mathbb{C}$, the support, denoted as \(\supp(f)=\overline{\{z\in X\mid f(z)\neq 0\}}\), or the closure of the set for which $f$ is non-zero.
\end{definition}
\begin{remark}
    We are primarily concerned when the support of a function is compact, or if the support is bounded. For smooth functions, functions that are compactly supported are called bump functions.
\end{remark}
\begin{theorem}[Green's Theorem]\label{theorem:complexgreen}
    
\end{theorem}
\begin{theorem}[Pompeiu's Theorem]\label{theorem:pompeiu}
    Let \(U\subseteq\mathbb{C}\) where $\partial U$ is piecewise smooth. Let \(f(z)\in C^1(\overline{U})\). Then $\forall z\in U$, \begin{equation}
        f(z)=\frac{1}{2\pi i}\left(\int_{\partial U}\frac{f(\zeta)}{\zeta-z}\mathop{\dd{\zeta}}-\int_{U}\frac{\partial f(\zeta)}{\partial\overline{\zeta}}\frac{\dd{\overline{\zeta}}\wedge\dd{\zeta}}{\zeta-z}\right).
    \end{equation}
\end{theorem}
\begin{proof}
    Let \(D(z,\varepsilon)\) denote the open disc centered around \(z\) with a radius of $\varepsilon$.
\end{proof}
\end{document}