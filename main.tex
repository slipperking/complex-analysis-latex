\documentclass{article}
\usepackage[utf8]{inputenc}
\usepackage[fixamsmath]{mathtools}
\usepackage{amssymb, amsfonts, amsthm}
\usepackage{parskip}
\usepackage{cancel}
\usepackage{tikz}
\usetikzlibrary{decorations.markings, calc, arrows.meta, hobby, intersections, decorations.pathreplacing, calligraphy, patterns}
\usepackage{csquotes}
\usepackage{pifont}
\usepackage{physics}
\usepackage[labelfont=bf, textfont=it]{caption}
\usepackage{graphicx}
\usepackage[english]{babel}
\usepackage{xparse}
\usepackage{enumitem}
\usepackage[inkscapelatex=false]{svg}

\setlist[enumerate]{label=(\arabic*)}
\setlist[itemize]{label=\ding{226}}

\usepackage[backend=biber, style=alphabetic, sorting=ynt]{biblatex}
\addbibresource{references.bib}
\nocite{*}

\everymath{\displaystyle}
\allowdisplaybreaks[2]

\newcommand{\tikzhalflengtharrow}[2]{%
    \path (#1) -- (#2) coordinate[pos=0.25] (Pstart) coordinate[pos=0.75] (Pend);
    \draw[-{Stealth}, thick, black] (Pstart) -- (Pend);
}

\newcommand{\ddx}{\dd{x}}
\newcommand{\ddy}{\dd{y}}
\newcommand{\ddz}{\dd{z}}
\newcommand{\ddzeta}{\dd{\zeta}}
\newcommand{\ddt}{\dd{t}}
\newcommand{\supp}{\operatorname{supp}}
\newcommand{\diam}{\operatorname{diam}}
\newcommand{\Log}{\operatorname{Log}}
\newcommand{\Arg}{\operatorname{Arg}}
\newcommand{\Aut}{\operatorname{Aut}}

\usepackage{keytheorems}
\usepackage{hyperref}
\usepackage[noabbrev, nameinlink]{cleveref}

\numberwithin{equation}{subsection}
\numberwithin{figure}{section}
\newkeytheorem{theorem, proposition, lemma}[
    parent=subsection,
    sibling=theorem,
]
\newkeytheorem{corollary}[parent=theorem]
\newkeytheorem{definition,example}[
    style=definition,
    sibling=theorem
]
\newkeytheorem{remark}[
    style=remark,
    numbered=no,
]

\crefformat{section}{\S#2#1#3}
\crefformat{subsection}{\S#2#1#3}
\crefformat{subsubsection}{\S#2#1#3}
\crefrangeformat{section}{\S\S#3#1#4 to~#5#2#6}
\crefmultiformat{section}{\S\S#2#1#3}{ and~#2#1#3}{, #2#1#3}{ and~#2#1#3}

\DeclarePairedDelimiter{\paren}{(}{)}
\let\parendefault\paren%
\renewcommand{\paren}{\parendefault*}
\DeclarePairedDelimiter{\brackets}{[}{]}
\let\bracketsdefault\brackets%
\renewcommand{\brackets}{\bracketsdefault*}
\DeclarePairedDelimiter\ceil{\lceil}{\rceil}
\let\ceildefault\ceil%
\renewcommand{\ceil}{\ceildefault*}
\DeclarePairedDelimiter\floor{\lfloor}{\rfloor}
\let\floordefault\floor%
\renewcommand{\floor}{\floordefault*}

\NewDocumentCommand{\cbraces}{m g}{\ensuremath{\qty{#1\IfNoValueTF{#2}{}{:#2}}}}

\newcommand{\extcomplex}{\widehat{\mathbb{C}}}
\newcommand{\ee}{\mathrm{e}}
\newcommand{\ii}{\mathrm{i}}
\newcommand{\interior}[1]{\overset{\circ}{#1}}
\newcommand{\doubletilde}[1]{
    \rlap{\raisebox{0.45ex}{$\widetilde{\phantom{#1{}}}$}}%
    \widetilde{#1{}}
}

\DeclareMathOperator*{\residue}{Res}
\makeatletter
\renewcommand{\@pnumwidth}{1.65em}
\renewcommand{\@tocrmarg}{2.2em}
\makeatother
\usepackage[lightmath, narrowiints]{kpfonts}

\let\swapRe\Re%
\let\Re\real%
\let\real\swapRe%
\let\swapIm\Im%
\let\Im\imaginary%
\let\imaginary\swapIm%

\title{Complex Analysis}
\author{Slipper King}
\date{May 2025}
\begin{document}
\maketitle
\tableofcontents
Made with \LaTeX.\AddToHookNext{shipout/background}{
    \newcommand{\rep}[1]{{#1}^{#1}_{#1}}
    \newcommand{\rrep}[2]{%
        \ifnum#1>0
        \expandafter\rrep\expandafter{\number\numexpr#1-1\relax}{\rep{#2}}%
        \else
        #2%
        \fi
    }
    \begin{tikzpicture}[remember picture,overlay]
        \node at ([yshift=-0.25\paperheight]current page.west) {\makebox[0pt][l]{\(\smash{\textcolor{lightgray}{\rrep{7}{\textstyle\int}}}\)}
        };
    \end{tikzpicture}
}\newpage
\section{Prerequisites}
\subsection{Topological Preliminaries}
The following definitions are subject to the assumption where the topological space is defined to be \(X=\mathbb{C}^n\). This is satisfactory to the main purpose of our proceeding passage, but it is noteworthy that it can be generalized to more abstract sets.
\begin{definition}[Accumulation Point]\label{def:accumulationpoint}
    A point \(z\in\mathbb{C}^n\) is an \textscsl{accumulation point} of \(X\) if for any open set \(U\) containing \(z\), \((U\setminus\qty{z})\cap X\neq\emptyset\)
\end{definition}
\begin{definition}[Closure]\label{def:closure}
    For a set \(X\in\mathbb{C}^n\), define the \textscsl{closure} of \(X\), or \(\overline{X}\) to be the intersection of all closed sets containing \(X\). In other words, it is the union of \(X\) and its accumulation points.
\end{definition}
\begin{definition}[Interior]\label{def:interior}
    For a set \(X\in\mathbb{C}^n\), the \textscsl{interior} of \(X\), denoted \(\interior{X}\), is the union of all open sets contained in \(X\), or the set of points \(z\in\mathbb{C}^n\) such that there exists an open neighborhood of \(z\) that is fully contained in \(X\).
\end{definition}
\begin{definition}[Compact Set]\label{def:compactsets}
    A set \(X\in\mathbb{C}^n\) is compact iff \(X\) is closed and bounded.
\end{definition}
\begin{definition}[Set Covering]
    A cover \(\mathcal{C}\) of a set \(X\) is a collection of sets \(\cbraces{U_n}\) such that \[\bigcup_{n\in\mathbb{N}}U_n\supseteq X.\] A cover is \textscsl{open} if every set in the collection is open.
\end{definition}
\begin{theorem}[\textsc{Bolzano--Weierstrass}]\label{thm:bolzanoweierstrass}
    Every infinite subset \(A\) of a compact set \(X\subset\mathbb{C}^n\) has an accumulation point in \(X\).
\end{theorem}
\begin{proof}
    Since \(X\) is bounded, there exists a closed cube \(Q\subset\mathbb{C}^n\) such that \(A\subseteq X\subset Q\).

    Bisect \(Q_0=Q\) into \(2^{2n}\) congruent sub-cubes. Since \(A\) is infinite and the sub-cubes are finite in number, at least one of the sub-cubes contains infinitely many points of \(A\), and choose one to be \(Q_1\).

    Bisect \(Q_1\) into \(2^{2n}\) sub-cubes, and choose a sub-cube \(Q_2\subset Q_1\) that contains infinitely many points of \(A\). We then obtain the recursive sequence \[Q_0\supset Q_1\supset Q_2\supset\cdots.\]

    Because the side lengths shrink to zero and the cubes are nested, the intersection
    \[\bigcap_{k=0}^{\infty} Q_k\]
    consists of exactly one point. Call this point \(z_\infty\in\mathbb{C}^n\).

    For each \(k\), \(Q_k\) contains infinitely many points of \(A\). Because the side length of \(Q_k\) tends to zero, for any \(\varepsilon>0\), \(\exists N\in\mathbb{N}\) such that \(\forall k\geq N\), \(Q_k\subset B^n(z_\infty,\varepsilon)\) where \(B^n(a,r)\subset\mathbb{C}^n\) is the \(n\)-dimensional \textscsl{ball} with radius \(r\) centered at \(a=\paren{a_1,a_2,\ldots,a_n}\in\mathbb{C}^n\), or \[B^n(a,r)=\cbraces{\qty(z_1,z_2,\ldots,z_n)\in\mathbb{C}^n}{\sum_{j=1}^n\abs{z_j-a_j}^2<r^2}.\]
    Then, \(B^n(z_\infty,\varepsilon)\) also contains infinitely many points of \(A\). Therefore, \(z_\infty\) is an accumulation point of \(A\).

    We now show that \(z_\infty\in X\). Suppose for contradiction that \(z_\infty\notin X\). Since \(X\) is closed, \(\mathbb{C}^n\setminus X\) is open, and \(\exists\delta>0\) such that \[B^n(z_\infty,\delta)\subset\mathbb{C}^n\setminus X.\] But then, for sufficiently large \(k\), we have \(Q_k\subset B^n\qty(z_\infty,\delta)\), and hence \(Q_k\cap X=\emptyset\). This contradicts the construction of \(Q_k\), which ensures that \(Q_k\) contains infinitely many points of \(A\subset X\).
\end{proof}
\begin{theorem}[\textsc{Heine--Borel}]\label{thm:heineborel}
    A set \(X\in\mathbb{C}^n\) is compact iff if every open cover has a finite subcover.
\end{theorem}
\begin{proof}
    We will first show that any set satisfying the condition is compact.

    First we will show that \(X\) is bounded. Suppose that \(\forall R>0\), \(\exists z\in X\) where \(\norm{z}>R\). Consider the collection of open sets \[\mathcal{U}=\cbraces{B^n(0,k)}{k\in\mathbb{N}}.\] \(\mathcal{U}\) forms an open cover of \(X\). Then by the assumption, there exists a finite subcover in \(\mathcal{U}\), namely \(\qty{B^n(0,k_1),\ldots,B^n(0,k_m)}\) which covers \(X\). Then, \[X\subseteq\bigcup_{i=1}^mB^n(0,k_i)=B^n\qty(0,\max\qty{k_1,\ldots k_m}).\] By contradiction, \(X\) must be bounded.

    \(X\) must also be a closed set. For the sake of contradiction, assume that there exists a point \(z_0\in\overline{X}\setminus X\). Since \(z_0\notin X\), the following open collection of sets covers \(X\):
    \[\mathcal{U}=\cbraces{\mathbb{C}^n\setminus\overline{B^n\qty(z_0,\frac{1}{k})}}{\forall k\in\mathbb{N}}.\]
    There then exists a finite subcover \(\mathcal{C}=\cbraces{\mathbb{C}^n\setminus\overline{B^n\qty(z_0,\frac{1}{k_j})}}{j\in\mathbb{N}_{\leq m}}\). Then, \[X\subseteq\mathbb{C}^n\setminus\overline{B^n\qty(z_0,\frac{1}{\max\qty{k_1,\ldots,k_m}})},\]
    and that \(X\cap\overline{B^n\qty(z_0,\frac{1}{\max\qty{k_1,\ldots,k_m}})}=\emptyset\). However, by the definition of the accumulation point, every open neighborhood of the accumulation point must intersect \(X\). Therefore, by contradiction, \(X\) is closed.

    We then prove the converse. By the assumption that \(X\) is bounded, \(\exists R>0\) such that the \(X\) is contained within the closed cube \[Q=\cbraces{z\in\mathbb{C}^n}{\max_{j\in\mathbb{N}_{\leq n}}\abs{\Re(z_j)}\le R\wedge\max_{j\in\mathbb{N}_{\leq n}}\abs{\Im(z_j)}\le R}.\]

    Assume that there exists an infinite open cover \(\mathcal{U}\) of \(X\) without finite subcovering. Bisect \(Q_0=Q\) into \(2^{2n}\) sub-cubes (for real and complex parts). Choose \(Q_1\) such that \(Q_1\cup X\) has no finite subcover of \(\mathcal{U}\). Under the previous assumptions, this is possible since if every \(\text{sub-cube}\cap X\) had finite subcovering, then \(Q_0\cap X=X\) would have finite subcovering. Similarly, choose \(Q_2\) by bisecting \(Q_1\) similarly, and recursively obtain a sequence of cubes:
    \[Q_0\supset Q_1\supset Q_2\supset\cdots\]
    Since the side length of each cube tends to 0, \(\bigcap_{j=0}^\infty Q_j\) consists of a single point \(z_{\infty}\in\mathbb{C}^n\). By the Bolzano-Weierstrass Theorem (\cref{thm:bolzanoweierstrass}), because \(\forall j\in\mathbb{N}\), \(Q_j\cap X\neq\emptyset\), select a point \(z_{j}\in Q_j\cap X\), forming a sequence \({z_k}\in X\) convergent to \(z_\infty\in X\) as \(X\) is closed. Therefore, \(\exists U\in\mathcal{U}\) where \(z_\infty\in U\). Since \(U\) is open, \(\exists\varepsilon>0\) such that \(B^n(z_\infty,\varepsilon)\subset U\).\ \(\exists N\in\mathbb{N}\) such that \(\forall k>N\), \(Q_k\subset B^n(z_\infty,\varepsilon)\). Then taking the intersection with \(X\) on both sides, \[Q_k\cap X\subseteq B^n(z_\infty,\varepsilon)\cap X\subset U.\] Our original assumption said that for every \(k\), \(Q_k\cap X\) has no finite subcovering. However, \(U\) covers \(Q_k\cap X\), which is a single open set that covers a nonempty subset. Therefore by contradiction, every open cover has finite subcovering.
\end{proof}
\begin{definition}[Support of a Function]\label{def:support}
    For a set \(X\) and a function \(f:X\to\mathbb{C}\), the \textscsl{support}, denoted by \(\supp(f)=\overline{\cbraces{z\in X}{f(z)\neq 0}}\), is the closure of the set for which \(f\) is nonzero.
\end{definition}
\begin{remark}
    A notable classification of functions comes from the compactness of support---more specifically, its boundedness. Compactly supported functions in \(C^\infty\) are commonly referred to as \textscsl{bump functions} (see \cref{sec:partitionsofunity}).
\end{remark}
\subsection{Calculus}
Since traditional complex analysis is the theory of calculus on complex functions, it is only natural that generalizations are made on classical formulas in calculus for complex functions.

It is well known that a function \(f:(a,b)\to\mathbb{R}\) is differentiable at a point \(x\in(a,b)\) if the limit \[\lim_{h\to0}\frac{f(x+h)-f(x)}{h}\] exists, and the value of this limit is the derivative of \(f(x)\), denoted by \(f'(x)\) or \(\frac{\dd{f}}{\ddx}\). The value \(\dd{f}=f'(x)\ddx\) is the differential of \(f(x)\). Partition \([a,b]\) into \(a=x_0<x_1<x_2<\cdots<x_n=b\) such that the length of the intervals \([x_i,x_{i-1}]\) tends to 0 as \(n\to\infty\). If for any such partition, the sum \[\sum_{i=1}^n f(\xi_i)(x_i-x_{i-1})\] tends to the same value \(\forall\xi_i\in[x_{i-1},x_i]\) (as the length of the largest partition approaches 0), then the function can be roughly said to be integrable over \([a,b]\). The full details of Riemann integrability relate to the upper and lower Riemann sums and will not be discussed here. The value of this sum is denoted by \[\int_a^bf(x)\dd{x}.\] The following theorems are the fundamental results of classical calculus:
\begin{theorem}[Fundamental Theorem of Calculus, Differential Form]
    Let \(f(x)\) be a function continuous over \([a,b]\). For \(x\in[a,b]\), define
    \[\Phi(x)=\int_a^x f(t)\dd{t}.\]
    Then \(\Phi(x)\) is differentiable over \([a,b]\), \(\Phi'(x)=f(x)\), and \(\dd{\Phi(x)}=f(x)\dd{x}\).
\end{theorem}
\begin{theorem}[Fundamental Theorem of Calculus, Integral Form]
    Let \(\Phi(x)\) be a function differentiable over \([a,b]\). Let \(f(x)=\Phi'(x)\) over \([a,b]\). Then,
    \[{\int_a^x}f(t)\dd{t}=\Phi(x)-\Phi(a).\]
\end{theorem}
The two forms of the theorem show that differentiation and integration are inverse operations to each other. Operations performed for differentiating oftentimes have a corresponding inverse operation that can be done for integrating. For instance, \[\dv{(f(x)\pm g(x))}{x}=\dv{f(x)}{x}\pm\dv{g(x)}{x}\] corresponds to \[\int(f(x)\pm g(x))\ddx=\int f(x)\ddx\pm\int g(x)\ddx,\]
and \[\dv{x}(f(x)g(x))=f'(x)g(x)+f(x)g'(x)\] corresponds to \[\int f(x)g'(x)\ddx=f(x)g(x)-\int f'(x)g(x)\ddx,\] and \[\dv{f(g(x))}{x}=\dv{f(g)}{g}\cdot\dv{g(x)}{x}\] corresponds to \[\int_a^b f(g(x))g'(x)\ddx=\int_{g(a)}^{g(b)}f(u)\dd{u}.\] Another correspondence is the Mean Value Theorem:
\begin{theorem}[Mean Value Theorem, Differential Form]
    If \(f(x)\) is differentiable over \([a,b]\), then \(\exists c\in[a,b]\) such that \[f(b)-f(a)=f'(c)(b-a).\]
\end{theorem}
\begin{theorem}[Mean Value Theorem, Integral Form]
    If \(f(x)\) is continuous over \([a,b]\), then \(\exists\xi\in[a,b]\) such that \[\int_a^b f(x)\ddx=f(\xi)(b-a).\]
\end{theorem}
A curve is a one-dimensional manifold embedded within a higher dimensional space. They can be parameterized with a vector \(\va{F}(t)=\qty(P(t),Q(t),R(t))\) of one parameter. In the complex plane, a curve is a complex-valued function \(\gamma(t)\) for a real parameter \(\alpha\leq t\leq\beta\). A curve is \textscsl{closed} if \(\gamma(\alpha)=\gamma(\beta)\). It is \textscsl{smooth} if it is continuously differentiable, and its direction is defined to be the direction as \(t\) increases. If it is smooth everywhere except at a finite number of points, it is \textscsl{piecewise smooth}. If it is of finite length, then the curve is said to be \textscsl{rectifiable}. Piecewise smooth curves are rectifiable. A curve is \textscsl{simple} if it is simple (non-self-intersecting), or if \(\gamma\paren{t_1}=\gamma\paren{t_2}\) implies that \(t_1=t_2\). A simple closed curve is also called a \textscsl{Jordan curve}.
\begin{theorem}[Jordan Curve Theorem]\label{thm:jordancurve}
    Let \(\gamma\) be a Jordan curve in \(\mathbb{R}^2\). Then the set \(\mathbb{R}^2\setminus\gamma\) consists of exactly two connected subsets. One of them is the interior, denoted by \(\operatorname{int}(\gamma)\), and is a bounded set, while the other is the exterior, denoted by \(\operatorname{ext}(\gamma)\), which is unbounded. Both of the two sets share the common boundary \(\gamma\).
\end{theorem}
The theorem above seems trivial, but its rigorous proof in topology is extremely complex. The theorem itself can also be stated on \(\mathbb{C}\) instead of \(\mathbb{R}^2\). For a region \(U\), the boundary is denoted \(\partial U\). If the region bounded by any closed curve in \(U\) also lies in \(U\), then it is a \textscsl{simply connected} region. A connected region that is not simply connected is multiply connected. A region bound by 2 non-intersecting Jordan curves is doubly connected, and a region bound by \(n\) non-intersecting Jordan curves is traditionally known as \(n\)-connected. Lastly, any closed curve can degenerate to a single point or slit.

Generalizations of the differential and integral exist for multivariate functions. The partial differentials of \(f(x,y,z)\), \(\pdv{f}{x}\ddx\), \(\pdv{f}{y}\ddy\), and \(\pdv{f}{z}\ddz\) sum up to form the total differential, denoted by \(\dd{f}\). An important result in multivariable calculus allows the calculation of the derivatives of a definite integral with respect to its parameter.
\begin{theorem}[Leibniz Integral Rule]\label{thm:leibnizintegralrule}
    Let \(I\subset\mathbb{R}\) be open, and let \(a:I\to\mathbb{R}\) and \(b:I\to\mathbb{R}\) be differentiable on \(I\) such that \(\forall x\in I\), \(a(x)<b(x)\). Let \[U=\cbraces{(x,t)\in I\times\mathbb{R}}{a(x)\leq t\leq b(x)}.\] Suppose \(f(x,t)\) is a real-valued function on \(U\) such that \(f(x,t)\) and \(\pdv{f}{x}(x,t)\) are continuous on the set \(U\). Then the function \[F(x)=\int_{a(x)}^{b(x)}f(x,t)\dd{t}\] is differentiable on \(I\), and
    \[\dv{F}{x}=\int_{a(x)}^{b(x)}\pdv{f}{x}(x,t)\dd{t} + f(x,b(x))\dv{b}{x} - f(x,a(x))\dv{a}{x}.\]
\end{theorem}
Four main classical theorems exist, relating a function and its line integral in 2 and 3 dimensions, line and surface integrals in 2 and 3 dimensions, and the surface and volume integrals in 3 dimensions:
\begin{theorem}[Gradient Theorem]\label{thm:gradient}
    Let \(C\) be an oriented smooth curve in \(\mathbb{R}^3\) with boundary points \(A\) to \(B\). Then
    \[\int_C\pdv{f}{x}\ddx+\pdv{f}{y}\ddy+\pdv{f}{z}\ddz=f(B)-f(A).\]
\end{theorem}
\begin{theorem}[Green's Theorem]\label{thm:realgreen}
    Let \(U\) be a positively oriented, multiply connected subset of \(\mathbb{R}^2\) with a piecewise smooth oriented boundary \(\partial U\). Suppose that \(P(x,y),Q(x,y)\in C^1\paren{\overline{U}}\). Then,
    \[\oint_{\partial U} P\ddx+Q\ddy=\iint_U\paren{\pdv{Q}{x}-\pdv{P}{y}}\ddx\ddy.\]
\end{theorem}
\begin{theorem}[Stokes' Theorem]\label{thm:kelvinstokes}
    Suppose that \(S\subset\mathbb{R}^3\) is a positively oriented surface with a positively oriented, piecewise smooth boundary curve \(\partial S\). Suppose that \(P(x,y,z), Q(x,y,z), R(x,y,z)\in C^1\paren{\overline{S}}\). Then,
    \begin{multline*}
        \oint_{\partial S}P\ddx+Q\ddy+R\ddz                                                                                                    \\
        =\iint_S\qty(\pdv{R}{y}-\pdv{Q}{z})\ddy\ddz+\qty(\pdv{P}{z}-\pdv{R}{x})\ddz\ddx+\qty(\pdv{Q}{x}-\pdv{P}{y})\ddx\ddy.
    \end{multline*}
\end{theorem}
\begin{theorem}[Gauss' Theorem]\label{thm:divergencegauss}
    Suppose that \(V\subset\mathbb{R}^3\) is a positively oriented region with a positively oriented, piecewise smooth boundary surface \(\partial V\). Suppose that \(P(x,y,z), Q(x,y,z), R(x,y,z)\in C^1\paren{\overline{V}}\). Then,
    \[\oiint_{\partial V}P\ddy\ddz+Q\ddz\ddx+R\ddx\ddy=\iiint_V\qty(\pdv{P}{x}+\pdv{Q}{y}+\pdv{R}{z})\ddx\ddy\ddz.\]
\end{theorem}
In 3-dimensional \(\mathbb{R}^3\) space, define a scalar valued function to be a 0-form, a linear combination of \(\ddx\), \(\dd{y}\), and \(\dd{z}\) to be a 1-form, and a linear combination of \(\dd{y}\wedge\dd{z}\), \(\dd{z}\wedge\ddx\), and \(\dd{x}\wedge\dd{y}\) to be a 2-form, and \(\ddx\wedge\dd{y}\wedge\dd{z}\) to be a 3-form, where \(\wedge\) denotes an anti-commutative and associative product, where for any two differential forms \(\omega_1\) and \(\omega_2\)
\[\omega_1\wedge\omega_2=-\omega_2\wedge\omega_1.\]
Then consequently, for any differential form \(\omega\), \[\omega\wedge\omega=0.\]
We can generalize the operator \(\dd\) to increase the degree of a differential form. For instance,
\[\dd{f}=\pdv{f}{x}\ddx+\pdv{f}{y}\dd{y}+\pdv{f}{z}\dd{z},\]
which is the definition of the total differential. For a 1-form in 3-dimensional space, \(\omega_1=P\ddx+Q\dd{y}+R\dd{z}\), we can define the exterior derivative in a similar way:
\begin{align*}
    \dd{\omega_1} & =\dd{P}\wedge\ddx+\dd{Q}\wedge\dd{y}+\dd{R}\wedge\dd{z}                                                                                                    \\
    & =\qty(\pdv{P}{x}\ddx+\pdv{P}{y}\dd{y}+\pdv{P}{z}\dd{z})\wedge\ddx                                                                                   \\
    & \qquad+\qty(\pdv{Q}{x}\ddx+\pdv{Q}{y}\dd{y}+\pdv{Q}{z}\dd{z})\wedge\ddy                                                                             \\
    & \qquad\qquad+\qty(\pdv{R}{x}\ddx+\pdv{R}{y}\dd{y}+\pdv{R}{z}\dd{z})\wedge\dd{z}                                                                     \\
    & =\qty(\pdv{R}{y}-\pdv{Q}{z})\dd{y}\wedge\dd{z}+\qty(\pdv{P}{z}-\pdv{R}{x})\dd{z}\wedge\ddx+\qty(\pdv{Q}{x}-\pdv{P}{y})\ddx\wedge\ddy.
\end{align*}
Similarly, we can differentiate a 2-form \(\omega=P\ddy\wedge\dd{z}+Q\ddz\wedge\ddx+R\ddx\wedge\ddy\) to get:
\[\qty(\pdv{P}{x}+\pdv{Q}{y}+\pdv{R}{z})\ddx\wedge\ddy\wedge\ddz.\]
The two results above resemble the curl and divergence of \(\qty(P,Q,R)\). A differential form \(\omega\) is \textscsl{closed} if \(\dd{\omega}=0\), and is \textscsl{exact} if there exists \(\eta\) such that \(\omega=\dd{\eta}\).
\begin{lemma}[Poincaré]\label{lem:poincare}
    For any differential form \(\omega\) on an open, contractible set \(U\subseteq\mathbb{R}^n\), if \(\omega\) is closed, then it is also exact.
\end{lemma}
It is true that for any set \(U\subseteq\mathbb{R}^n\), regardless of contractibility, that for a differential form \(\omega\) defined on \(U\), \(\dd\dd\omega=0\). In other words, all exact differential forms are closed. (For a region \(U\), we have \(\partial\partial U=\emptyset\). This is one of many reasons for which the boundary operator is denoted by \(\partial\), in analogy to \(\dd\).)

The implications of this are important: if \(\omega\) is a 0-form, then \(\curl(\grad\omega)=0\), and if \(\omega\) is a 1-form, \(\div(\curl{v})=0\), where \(v\) is the vector of the coefficients of the basis differential forms of \(\omega\) (there are no correlations for higher degree forms since in 3-dimensional space, the highest degree possible for any differential form is 3).

Then, the Fundamental Theorem of Calculus, the Gradient Theorem, Green's, Stokes', and Gauss' Theorems can be generalized into:
\begin{theorem}[Stokes--Cartan]\label{thm:stokescartan}
    For an oriented smooth \(n\)-dimensional compact manifold \(M\) with boundary \(\partial M\), for a smooth differential \((n-1)\)-form \(\omega\) over \(\overline{M}\), \[\int_M\dd{\omega}=\int_{\partial M}\omega.\]
\end{theorem}
Real analysis is the subject dedicated to rigorously defining concepts such as limits, continuity, integrability, convergence, etc. The most widely used definition of a finite limit of a function is the language of \(\varepsilon\)--\(\delta\), which states:
\begin{definition}[Epsilon--Delta]\label{def:epsilondelta}
    Let \(f:U\to\mathbb{R}\) be a function defined over an open set \(U\subseteq\mathbb{R}\) such that \(a\) is an accumulation point of \(U\). We say that \(\lim_{x\to a}f(x)=L\) if \(\forall\varepsilon>0\), \(\exists\delta>0\) such that for all \(x\in U\) with \(0<\abs{x-a}<\delta\), we have \(\abs{f(x)-L}<\varepsilon\).

    Similarly, we define the \textscsl{right-handed limit} \(\lim_{x\to a^+}f(x)=L\) if for every \(\varepsilon>0\), there exists \(\delta>0\) such that for all \(x\in U\) with \(0<x-a<\delta\), we have \(\abs{f(x)-L}<\varepsilon\).

    Likewise, the \textscsl{left-hand limit} \(\lim_{x\to a^-}f(x)=L\) exists if for every \(\varepsilon>0\), there exists \(\delta>0\) such that for all \(x\in U\) with \(-\delta<x-a<0\), we have \(\abs{f(x)-L}<\varepsilon\).
\end{definition}
We also have the definition of the limit of a sequence:
\begin{definition}[Epsilon--N]\label{def:epsilonn}
    Let \(\qty{a_n}_{n\in\mathbb{N}}\subset\mathbb{R}\) be a sequence. If \(\exists a_\infty\in\mathbb{R}\) such that \(\forall\varepsilon>0\), \(\exists N\in\mathbb{N}\) such that \(\forall n>N\), \(\abs{a_n-a_\infty}<\varepsilon\), then \(\qty{a_n}\) \textscsl{converges} to \(a_\infty\).
\end{definition}
\begin{theorem}[Cauchy Criterion]\label{thm:cauchycriterionsequenceconvergence}
    Let \(\qty{a_n}_{n\in\mathbb{N}}\subset\mathbb{R}\) be a sequence. Then \(\qty{a_n}\) is convergent iff \(\forall\varepsilon>0\), \(\exists N\in\mathbb{N}\) such that \(\forall n,m>N\), \(\abs{a_n-a_m}<\varepsilon\).
\end{theorem}
\begin{proof}
    Assume \(\qty{a_n}\) is convergent. Then \(\forall\varepsilon>0\), \(\exists N\in\mathbb{N}\) such that \(\forall n,m>N\), \(\abs{a_n-a_\infty}<\frac{\varepsilon}{2}\) and \(\abs{a_m-a_\infty}<\frac{\varepsilon}{2}\) for some \(a_\infty\in\mathbb{R}\). It follows that \[\abs{a_n-a_m}\leq\abs{a_n-a_\infty}+\abs{a_m-a_\infty}=\varepsilon.\]

    Conversely, \(\qty{a_n}\) is bounded (fixing \(N\), \(\forall n>N\), \(\abs{a_n-a_{N+1}}<\varepsilon\)). By the Bolzano--Weierstrass Theorem (\cref{thm:bolzanoweierstrass}), \(\qty{a_n}_{n\in\mathbb{N}}\) has a subsequence \(\qty{a_{n_k}}_{k\in\mathbb{N}}\) that converges to \(a_\infty\). Therefore, \(\forall\varepsilon>0\), \(\exists N\in\mathbb{N}\) and \(\exists M\in\mathbb{N}\) such that \(\forall k>M\), \(n_k>N\), and \(\forall n>N\), \(\abs{a_n-a_{n_k}}<\frac{\varepsilon}{2}\) and \(\abs{a_{n_k}-a_\infty}<\frac{\varepsilon}{2}\). Then \[\abs{a_n-a_\infty}\leq\abs{a_n-a_{n_k}}+\abs{a_{n_k}-a_\infty}<\varepsilon.\] Hence, \(\qty{a_n}\) converges to \(a_\infty\).
\end{proof}
\begin{definition}[Limit Superior]\label{def:limsup}
    For a number sequence \(\qty{a_n}\subset\mathbb{R}\), if \(\exists a\in\mathbb{R}\) such that:
    \begin{enumerate}
        \item \(\forall\varepsilon>0\), \(\exists N\in\mathbb{N}\) such that \(\forall n>N\), \(a_n<a+\varepsilon\),
        \item \(\forall\varepsilon>0\), \(\forall N\in\mathbb{N}\), \(\exists n>N\) such that \(a_n>a-\varepsilon\),
    \end{enumerate}
    then the \textscsl{superior limit} of \(\qty{a_n}\) is \(a\), denoted by \(\varlimsup_{n\to\infty}a_n=\limsup_{n\to\infty}a_n=a\).
\end{definition}
\begin{definition}[Limit Inferior]\label{def:liminf}
    For a number sequence \(\qty{a_n}\subset\mathbb{R}\), if \(\exists a\in\mathbb{R}\) such that:
    \begin{enumerate}
        \item \(\forall\varepsilon>0\), \(\exists N\in\mathbb{N}\) such that \(\forall n>N\), \(a_n>a-\varepsilon\),
        \item \(\forall\varepsilon>0\), \(\forall N\in\mathbb{N}\), \(\exists n>N\) such that \(a_n<a+\varepsilon\),
    \end{enumerate}
    then the \textscsl{inferior limit} of \(\qty{a_n}\) is \(a\), denoted by \(\varliminf_{n\to\infty}a_n=\liminf_{n\to\infty}a_n=a\).
\end{definition}
\begin{lemma}
    A number sequence \(\qty{a_n}\) is convergent iff if \(\varlimsup_{n\to\infty} a_n=\varliminf_{n\to\infty}a_n\).
\end{lemma}
\begin{proof}
    We first prove that \(a=\lim_{n\to\infty}a_n\) implies \(\varlimsup_{n\to\infty}a_n=\varliminf_{n\to\infty}a_n=a\).
    By \cref{def:epsilonn}, \(\forall\varepsilon>0\), \(\exists N\in\mathbb{N}\) such that \(\forall n>N\), \[\abs{a_n-a}<\varepsilon\Longleftrightarrow a-\varepsilon<a_n<a+\varepsilon.\]
    Then from \cref{def:limsup,def:liminf}, we have that \(\varlimsup_{n\to\infty}a_n\geq a\) and \(\varliminf_{n\to\infty} a_n\leq a\). By the second conditions, we get \(\varlimsup_{n\to\infty}a_n\leq a\) and \(\varliminf_{n\to\infty} a_n\geq a\). Therefore, \[\varlimsup_{n\to\infty}a_n=\varliminf_{n\to\infty}a_n.\]

    For the converse, assume \(\varlimsup_{n\to\infty}a_n=\varliminf_{n\to\infty}a_n\). Since \(\exists N_1\in\mathbb{N}\) such that \(\forall n>N_1\), \(a_n<a+\varepsilon\).\ \(\exists N_2\in\mathbb{N}\) such that \(\forall n>N_2\), \(a_n>a-\varepsilon\). Then \(\forall n>\max\qty{N_1,N_2}\), \(\abs{a_n-a}<\varepsilon\), as expected.
\end{proof}
\begin{definition}[Continuity]\label{def:continuity}
    A function \(f:U\to\mathbb{R}\), defined on an open set \(U\subseteq\mathbb{R}\) containing a point \(a\in U\), is said to be continuous at \(a\) iff if \[\lim_{x\to a}f(x)=f(a).\]
\end{definition}
It is important to note that in the case of multiple \emph{explicit} variables, a distinction is made between (separate) continuity (where there are two \(\delta\)'s on which variable varies, and does not guarantee a single \(\delta\) for when both variables vary simultaneously) and \textscsl{joint} continuity (where a single \(\delta\) controls both variables at once). To illustrate this, let \(\qty(x_0,y_0)\) be fixed. The former is commonly written as \[\forall\varepsilon>0,\exists\delta>0\text{ such that }\forall\abs{x-x_0}<\delta,\abs{f\qty(x,y_0)-f\qty(x_0,y_0)}<\varepsilon\] in conjunction with \[\forall\varepsilon>0,\exists\delta>0\text{ such that }\forall\abs{y-y_0}<\delta,\abs{f\qty(x_0,y)-f\qty(x_0,y_0)}<\varepsilon,\] whereas the latter is expressed as \[\forall\varepsilon>0,\exists\delta>0\text{ such that }\forall\abs{\qty(x-x_0,y-y_0)}<\delta,\abs{f\qty(x,y)-f\qty(x_0,y_0)}<\varepsilon.\]
\begin{theorem}\label{thm:continuousfunctionboundedoncompact}
    Any continuous function on a compact set \(K\) is bounded on \(K\).
\end{theorem}
\begin{proof}
    Suppose for the sake of contradiction that \(f:U\to\mathbb{R}\) is continuous and unbounded on compact \(K\). Then for each \(n\in\mathbb{N}\), there exists \(x_n\in K\) such that \(|f(x_n)|>n\). The sequence \(\cbraces{x_n}\) lies in \(K\), which is compact, so by the Bolzano--Weierstrass Theorem (\cref{thm:bolzanoweierstrass}), \(\cbraces{x_n}\) has an accumulation point in \(K\). In other words, there exists a convergent subsequence \(\cbraces{x_{n_k}}\) with \(\lim_{k\to\infty} x_{n_k}\in K\).

    Since \(f\) is continuous, \(\lim_{k\to\infty}f\qty(x_{n_k})=f\qty(\lim_{k\to\infty}x_{n_k})\), which is well-defined because \(\lim_{k\to\infty} x_{n_k}\in K\). However, this contradicts \(\abs{f\qty(x_{n_k})}>n_k\to\infty\), hence \(f\) must be bounded on \(K\).
\end{proof}
\begin{theorem}[\textsc{Extreme Value}]\label{thm:extremevalue}
    A continuous function \(f(x)\) defined on a compact set \(K\) attains its infimum and supremum in \(K\).
\end{theorem}
\begin{proof}
    Assume that \(f\) never attains its supremum \(M\). Then, \(f(x)<M\). Define the auxiliary function \(\psi(x)=\frac{1}{M-f(x)}\), which is strictly positive and continuous as the denominator never reaches \(0\). By \cref{thm:continuousfunctionboundedoncompact}, \(\psi(x)\) is bounded with some value of \(\mu>0\) satisfying \(\psi(x)\leq\mu\).\ \(f(x)\) also has the representation \(M-\frac{1}{\psi(x)}\), and therefore, \[f(x)\leq M-\frac{1}{\mu},\] which means that \(M\) is not the supremum. Similarly, assume that \(f\) never attains its infimum \(m\). Then \(f(x)>m\). Let \(\psi(x)=\frac{1}{f(x)-m}\), which is strictly positive and continuous as the denominator never reaches \(0\). By \cref{thm:continuousfunctionboundedoncompact}, \(\psi(x)\) is bounded with some value of \(\mu>0\) satisfying \(\psi(x)\leq\mu\).\ \(f(x)\) also has the representation \(m+\frac{1}{\psi(x)}\), and therefore, \[f(x)\geq m+\frac{1}{\mu},\] which means that \(m\) is not the infimum.
\end{proof}
\begin{definition}[Uniform Continuity]\label{def:uniformcontinuity}
    A function \(f:U\to\mathbb{R}\), defined on a set \(U\subseteq\mathbb{R}\), is uniformly continuous iff \(\forall\varepsilon>0\), \(\exists\delta>0\) such that \(\forall x,y\in U\) where \(|x-y|<\delta\), \(\abs{f(x)-f(y)}<\varepsilon\).
\end{definition}
\begin{example}
    The function \(f(x)=\frac{1}{x}\) is not uniformly continuous over \((0,1)\).
\end{example}
\begin{proof}
    If \(\exists\varepsilon>0\) such that \(\forall\delta>0\), \(\exists x,y\in(0,1)\) satisfying both \(|x-y|<\delta\) and \(\abs{f(x)-f(y)}\geq\varepsilon\), then \(f\) is not uniformly continuous over \((0,1)\).

    Let \(\varepsilon=1\) and \[x=\frac{1}{n},\quad y=\frac{1}{n+1}.\] Then \(\forall\delta>0\), \(\exists n>1\) where \(\abs{x-y}<\delta\), since \(\lim_{n\to\infty}\abs{x-y}=0\). Additionally, \(\abs{f(x)-f(y)}=1\geq\varepsilon\). This satisfies the negation, and thus, \(f(x)=\frac{1}{x}\) is not uniformly continuous over \((0,1)\).
\end{proof}
\begin{theorem}[\textsc{Heine--Cantor}]\label{thm:heinecantor}
    A continuous function on a compact set \(K\) is uniformly continuous on \(K\).
\end{theorem}
\begin{proof}
    Suppose for contradiction that \(f:K\to\mathbb{R}\) is continuous but not uniformly continuous on compact \(K\). Then \(\exists\varepsilon>0\) such that \(\forall\delta>0\), \(\exists x,y\in K\) with \(|x-y|<\delta\) and \(|f(x)-f(y)|\geq\varepsilon\).

    In particular, \(\forall n\in\mathbb{N}\), let \(\delta=\frac{1}{n}\). Then \(\exists x_n,y_n\in K\) such that \(|x_n-y_n|<\frac{1}{n}\) and \(|f(x_n)-f(y_n)|\geq\varepsilon\). Since \(K\) is compact, the sequence \(\cbraces{x_n}\) has a convergent subsequence \(x_{n_k}\to x_{n_\infty}\in K\). Because \(\qty|x_{n_k}-y_{n_k}|<\frac{1}{n_k}\to 0\), we also have \(y_{n_k}\to x_{n_\infty}\).

    By continuity, \(f(x_{n_k})\to f\paren{x_{n_\infty}}\) and \(f(y_{n_k})\to f\paren{x_{n_\infty}}\), so \(\qty|f(x_{n_k})-f(y_{n_k})|\to 0\), contradicting \(\qty|f(x_{n_k})-f(y_{n_k})|\geq\varepsilon\). Therefore, \(f\) is uniformly continuous on \(K\).
\end{proof}
A more general version of this proof can be achieved by using the Heine--Borel Theorem (\cref{thm:heineborel}) analogously to the proof of \cref{thm:heinecantorfamily}.
\begin{definition}
    A function \(f\) is Lipschitz continuous over \(U\) if \(\exists M\in\mathbb{R}_{\geq0}\) such that \(\forall x,y\in U\), \(abs{f(x)-f(y)}\leq M\abs{x-y}\). The smallest possible \(M\) satisfying the above condition is known as the Lipschitz constant.
\end{definition}
Lipschitz continuity is an important concept in real analysis and the theory of differential equations. It is a strong form of uniform continuity.
\begin{proposition}
    All Lipschitz continuous functions on \(U\) are uniformly continuous on \(U\).
\end{proposition}
\begin{proof}
    Let \(M>0\) be the Lipschitz constant. Then \(\forall\varepsilon>0\), let \(\delta=\frac{\varepsilon}{M}\). It then follows that \(\forall x,y\in U\) such that \(\abs{x-y}<\delta\), \(\abs{f(x)-f(y)}\leq M|x-y|<\varepsilon\).
\end{proof}
\begin{proposition}\label{prop:c1lipschitz}
    A \(C^1\) function on a compact set \(K\) is Lipschitz continuous on \(K\).
\end{proposition}
\begin{proof}
    Let \(f:K\to\mathbb{R}\) be \(C^1\). By \cref{thm:continuousfunctionboundedoncompact}, since \(K\) is compact and \(f'\) is continuous, \(\exists M>0\) such that \(\forall x\in K\), \(|f'(x)|\leq M\).

    By the Mean Value Theorem, \(\forall x,y\in K\), \(\exists c\) between \(x\) and \(y\) such that \(f(x)-f(y)=f'(c)(x-y)\). Then, \(|f(x)-f(y)|=|f'(c)||x-y|\leq M|x-y|\), which means \(f\) is Lipschitz continuous with Lipschitz constant less than or equal to \(M\).
\end{proof}
We will now give statements relating to operations on Riemann integrals:
\begin{lemma}\label{lem:integralwithparameterproperties}
    Let \(f(x,u)\) be a continuous complex function on the rectangle defined by \(a\leq x\leq b\), \(\alpha\leq u\leq\beta\) (where \(a<b\), \(\alpha<\beta\)). Then,
    \begin{enumerate}
        \item For an arbitrary point \(u_0\in [\alpha,\beta]\), we have \[\lim_{u\to u_0}\int_a^b f(x,u)\ddx=\int_a^b f\qty(x,u_0)\ddx.\]
        \item The order of integration can be switched: \[\int_a^b\int_\alpha^\beta f(x,u)\dd{u}\ddx=\int_\alpha^\beta\int_a^b f(x,u)\ddx\dd{u}\]
    \end{enumerate}
\end{lemma}
\begin{proof}
    The second part is self-explanatory as it can be interpreted as an area integral. Let \[\varphi(u)=\int_a^b f(x,u)\ddx.\]
    Since \(f(x,u)\) is continuous on a compact set in both \(x\) and \(u\), by \cref{thm:heinecantor}, \(\forall\varepsilon>0\), \(\exists\delta>0\) such that \(\forall\qty(x_1,u_1),\qty(x_2,u_2)\in[a,b]\times[\alpha,\beta]\) satisfying \[\abs{\qty(x_1-x_2,u_1-u_2)}<\delta,\]
    we have \(\abs{f(x_1,u_1)-f(x_2,u_2)}<\frac{\varepsilon}{b-a}\). Then for \(\abs{u-u_0}<\delta\), the integral difference can be bounded by \[\abs{\varphi(u)-\varphi\qty(u_0)}\leq\int_a^b\abs{f(x,u)-f\qty(x,u_0)}\ddx<\varepsilon.\qedhere\]
\end{proof}
\section{Complex Prerequisites}
\subsection{The Extended Complex Plane and its Spherical Representation}
All complex numbers form a field that extends the real number field. A complex number \(\alpha+\ii\beta\) can be visualized on a rectangular plane as the point \((\alpha,\beta)\), with two axes: the real axis and the imaginary axis. It is well known that any complex number also has the polar form \(r\ee^{\ii\theta}=r\paren{\cos\theta+\ii\sin\theta}\).

The infinity point, \(\infty\), extends \(\mathbb{C}\) with \(\extcomplex=\mathbb{C}\cup\qty{\infty}\). The following arithmetic operations are defined: \(\forall a\in\mathbb{C}\), \(a+\infty=\infty+a=\infty\), and \(\forall b\in\mathbb{C}\setminus\qty{0}\), \(b\cdot\infty=\infty\cdot b=\infty\) and \(\frac{a}{\infty}=0\).

Let \(S^2=\cbraces{\qty(x_1,x_2,x_3)\in\mathbb{R}}{x_1^2+x_2^2+x_3^2=1}\). There exists a \textscsl{stereographic projection} of \(S^2\) onto \(\extcomplex\). For every point other than \((0,0,1)\), there is a corresponding complex number
\begin{equation}\label{eq:extcomplexformula1}
    z=\frac{x_1+\ii x_2}{1-x_3}.
\end{equation}
This correspondence between \(\mathbb{C}\) and \(S^2\setminus\qty{(0,0,1)}\) is injective. In fact, the inverse can be solved for:
\[\abs{z}^2=\frac{1-x_3^2}{\paren{1-x_3}^2}=\frac{1+x_3}{1-x_3},\]
which results in \[x_3=\frac{\abs{z}^2-1}{\abs{z}^2+1},\]
and consequently,
\[x_1=\Re\paren{z}\paren{1-x_3}=\frac{z+\overline{z}}{\abs{z}^2+1},\quad x_2=\Im\paren{z}\paren{1-x_3}=\frac{z-\overline{z}}{\ii\abs{z}^2+\ii}.\]
By letting \(\infty\) correspond to \(\paren{0,0,1}\), the bijection is complete. The sphere \(S^2\) is also called the \textscsl{Riemann sphere}. The region given by the disk \(\abs{z}<1\) is given by \(x_3<0\), and the region \(\abs{z}>1\) is given by \(x_3>0\).

We will now give a geometric visualization of this projection. Let \(z=x+\ii y\). Then we obtain that \(x=\frac{x_1}{1-x_3}\) and \(y=\frac{x_2}{1-x_3}\). Therefore, \[x:y:1=x_1:x_2:1-x_3.\]
It follows that the points \(0\), \(\qty(x,y,1)\), and \(\qty(x_1,x_2,1-x_3)\) are collinear in \(\mathbb{R}^3\). Under the linear map \(\va{v}\mapsto\va{v}\mqty(\dmat[0]{1,1,-1})+\qty(0,0,1)\), we get that \(\qty(0,0,1)\), \(\qty(x,y,0)\), and \(\qty(x_1,x_2,x_3)\) are collinear. In other words, this correspondence is a central projection with center \(\qty(0,0,1)\), projecting the points from \(S^2\setminus\qty(0,0,1)\) onto \(\mathbb{C}\). Let this center correspond to \(\infty\). In this representation, \(\infty\in\extcomplex\) is no longer considered to be special.

For the purpose of notation, we will define \(\mathbb{H}^+=\cbraces{z\in\mathbb{C}}{\Im(z)>0}\).
\subsection{Complex Differentiation}\label{sec:complexdifferentiation}
For \(U\subseteq\mathbb{C}\) and a complex function \(f:U\to\mathbb{C}\), \(f(z)\) is \textscsl{complex differentiable} at \(z\in U\) if the following limit exists, regardless of the direction \(\Delta z\) approaches 0 at:
\[\lim_{\Delta z\to0}\frac{f(z+\Delta z)-f(z)}{\Delta z}.\]
We can consider \(f(z)\) to be a bivariate function \(f(x,y)\) for \(z=x+\ii y\). Two main cases we are concerned with are when \(\Delta z\) approaches 0 from the real and imaginary axes:
\[\lim_{\substack{\Delta z\to0\\\Delta z\in\mathbb{R}}}\frac{f(z+\Delta z)-f(z)}{\Delta z}=\lim_{\substack{\Delta z\to0\\\Delta z\in\mathbb{R}}}\frac{f(z+\ii\Delta z)-f(z)}{\ii\Delta z}.\] Expressing \(f(z)\) as \(f(x,y)=u(x,y)+\ii v(x,y)\),
\[\lim_{\substack{\Delta z\to0\\\Delta z\in\mathbb{R}}}\frac{f(z+\Delta z)-f(z)}{\Delta z}=\lim_{\substack{\Delta z\to0\\\Delta z\in\mathbb{R}}}\frac{f(x+\Delta z,y)-f(x,y)}{\Delta z}=\pdv{u}{x}+\ii\pdv{v}{x},\]
\[\lim_{\substack{\Delta z\to0\\\Delta z\in\mathbb{R}}}\frac{f(z+\ii\Delta z)-f(z)}{\ii\Delta z}=-\ii\lim_{\substack{\Delta z\to0\\\Delta z\in\mathbb{R}}}\frac{f(x,y+\Delta z)-f(x,y)}{\Delta z}=\pdv{v}{y}-\ii\pdv{u}{y}.\]
By comparing the real and imaginary parts, we obtain necessary conditions for complex differentiability:
\begin{equation}
    \pdv{u}{x}=\pdv{v}{y}\qand\pdv{v}{x}=-\pdv{u}{y}\label{eq:cauchyriemanneqs1}
\end{equation}
By multiplying the second equation by \(\ii\) and adding it to the first, we obtain the logical equivalence with:
\begin{equation}
    \pdv{f}{x}=-\ii\pdv{f}{y}\label{eq:cauchyriemanneqs2}
\end{equation}
\cref{eq:cauchyriemanneqs1,eq:cauchyriemanneqs2} are known as the \textscsl{Cauchy--Riemann equations}. Although this condition is necessary, it is not sufficient. Consider the function \(f(z)=\sqrt{\abs{\Re(z)\Im(z)}}\). Let \(z=x+\ii y\), \(x=\alpha t\), and \(y=\beta t\). Then
\[\lim_{z\to0}\frac{f(z)-f(0)}{z-0}=\lim_{z\to0}\frac{f(z)}{z}=\lim_{t\to0}\frac{\sqrt{\abs{\alpha\beta t^2}}}{\alpha t+\ii\beta t}=\frac{\sqrt{\abs{\alpha\beta}}}{\alpha+\ii\beta}.\]
The derivative along \(\alpha=1\), \(\beta=0\) (or the real axis) vanishes. Along \(\alpha=0\), \(\beta=1\) (or the imaginary axis), it also vanishes. However, the limit is different for any other pair of \(\alpha\) and \(\beta\), or the consequent direction of approach.
\begin{definition}[Holomorphy]\label{def:holomorphy}
    A function \(f:U\to\mathbb{C}\) is said to be \textscsl{holomorphic} at \(z_0\in U\) if it is complex differentiable on a neighborhood of \(z_0\). If \(f(z)\) is holomorphic for every point in an open connected set \(U\), then it is said to be holomorphic over \(U\). A function is holomorphic over compact set \(K\) if it is holomorphic on a neighborhood of \(K\).
\end{definition}
Weierstrass provided the following classification:
\begin{definition}
    A function is \textscsl{entire} if it is holomorphic over \(\mathbb{C}\).
\end{definition}
For the purpose of the following contents, a \textscsl{region} or \textscsl{domain} will denote a nonempty, open, connected subset of the complex plane.
\begin{theorem}\label{thm:holomorphycondition}
    Let \(U\subseteq\mathbb{C}\) be open, and let \(f:U\to\mathbb{C}\) be a function. Then \(f\) is holomorphic on \(U\) iff \(f\in C^1(U)\) and satisfies the Cauchy--Riemann equations.
\end{theorem}
\begin{proof}
    The first part is to prove that any holomorphic function on \(U\) has continuous first-order partial derivatives in \(U\). This requires an argument that will be covered later (specifically in \cref{sec:analyticityandholomorphy}), which claims that the complex derivative of any holomorphic function is also holomorphic over the region.

    For the second part, let \(f(z)=f(x,y)=u(x,y)+\ii v(x,y)\). Assume that \(u,v\in C^1\qty(\qty{z_0})\) and satisfy the Cauchy--Riemann equations at \(z_0=x_0+\ii y_0\). Let \[\alpha=\pdv{u}{x}\paren{x_0,y_0}=\pdv{v}{y}\paren{x_0,y_0},\quad\beta=\pdv{v}{x}\paren{x_0,y_0}=-\pdv{u}{y}\paren{x_0,y_0}.\]

    Then because \(u,v\in C^1\paren{U}\) have continuous partial derivatives, \(\forall x+\ii y\in U\):
    \[u\paren{x,y}-u\paren{x_0,y_0}=\alpha\paren{x-x_0}-\beta\paren{y-y_0}+o\paren{\abs{\Delta z}},\]
    \[v\paren{x,y}-v\paren{x_0,y_0}=\beta\paren{x-x_0}+\alpha\paren{y-y_0}+o\paren{\abs{\Delta z}},\]
    where \(\abs{\Delta z}=\sqrt{{\Delta x}^2+{\Delta y}^2}\) and \(o\paren{\abs{\Delta z}}\) denotes a value with higher infinitesimal order to \(\abs{\Delta z}\), or that \(\lim_{\Delta z\to0}\frac{o\paren{\abs{\Delta z}}}{\abs{\Delta z}}=0\). Then letting \(\Delta z=x-x_0+\ii\paren{y-y_0}\),
    \[f\paren{z}-f\paren{z_0}=\alpha{\Delta z}+\ii\beta{\Delta z}+o\paren{\abs{\Delta z}}+o\paren{\abs{\Delta z}},\]
    \[\frac{f\paren{z}-f\paren{z_0}}{z-z_0}=\alpha+\ii\beta+\frac{o\paren{\abs{\Delta z}}}{\abs{\Delta z}}\frac{\abs{\Delta z}}{\Delta z}.\]
    Taking the limit as \(\Delta z\to0\), the high order infinitesimals on the right-hand side vanish, and \[\lim_{\Delta z\to0}\frac{f\paren{z}-f\paren{z_0}}{z-z_0}=\alpha+\ii\beta.\qedhere\]
\end{proof}
We will prove later in \cref{sec:analyticityandholomorphy} that the complex derivative of a holomorphic function \(f(z)=u(z)+\ii v(z)\) is holomorphic. Under this assumption, \(f(z)\) has continuous second-order partial derivatives, and therefore, \[\pdv[2]{u}{x}{y}=\pdv[2]{u}{y}{x},\quad\pdv[2]{v}{x}{y}=\pdv[2]{v}{y}{x},\] and by the Cauchy--Riemann equations, \[\pdv[2]{u}{x}=\pdv[2]{v}{x}{y},\quad\pdv[2]{u}{y}=-\pdv[2]{v}{y}{x},\]
and \[\pdv[2]{v}{x}=-\pdv[2]{u}{x}{y},\quad\pdv[2]{v}{y}=\pdv[2]{u}{y}{x}.\]
Adding the equations, \[\pdv[2]{u}{x}+\pdv[2]{u}{y}=0,\quad\pdv[2]{v}{x}+\pdv[2]{v}{y}=0.\]
This type of equation is called the \textscsl{Laplace equation}, which is a basic example of an elliptic partial differential equation. Define the operator (the \textscsl{Laplacian}) \[(\Delta=)\laplacian=\div{\grad}=\pdv[2]{}{x}+\pdv[2]{}{y}.\] A function \(u\) satisfying the Laplace equation \(\laplacian u=0\) is a \textscsl{harmonic function}. Thus, the real and complex parts of a holomorphic function are harmonic functions.
\begin{proposition}\label{prop:realvaluedholomorphicfunctionconstant}
    Let \(U\subseteq\mathbb{C}\) be open and connected and \(f:U\to\mathbb{R}\) be holomorphic. It follows that \(f\) is constant over \(U\).
\end{proposition}
\begin{proof}
    Since \(f(x,y)=u(x,y)+\ii v(x,y)\) is real-valued, \(v(x,y)\equiv0\) on \(U\). Then by the Cauchy--Riemann equations on \(U\), \(\pdv{u}{x}=\pdv{v}{y}=0\). Similarly, \(\pdv{u}{y}=-\pdv{v}{x}=0.\) Therefore, \(f(z)=u(z)\) is constant.
\end{proof}
\subsubsection{Wirtinger Derivatives}
We have previously introduced the concept of expressing a complex function as a function of \(x\) and \(y\). It can also be expressed in terms of \(z\) and \(\overline{z}\), where \(z=x+\ii y\) and \(\overline{z}=x-\ii y\). Then \(\abs{z}^2=z\overline{z}\), \(x=\frac{z+\overline{z}}{2}\), and \(y=\frac{z-\overline{z}}{2\ii}\). By the rules of the derivative, it is only natural that we define
\begin{equation}
    \pdv{z}=\pdv{x}\pdv{x}{z}+\pdv{y}\pdv{y}{z}=\frac{1}{2}\qty(\pdv{x}-\ii\pdv{y})\label{eq:wirtingerderivative1}
\end{equation} and
\begin{equation}
    \pdv{}{\overline{z}}=\pdv{}{x}\pdv{x}{\overline{z}}+\pdv{}{y}\pdv{y}{\overline{z}}=\frac{1}{2}\qty(\pdv{}{x}+\ii\pdv{}{y}).\label{eq:wirtingerderivative2}
\end{equation}
If \cref{eq:wirtingerderivative1} is set equal to 0, then it is the equivalent form of the homogeneous Cauchy--Riemann Equations. Then for a holomorphic function \(f(z)\), the Wirtinger derivative \(\pdv{f}{z}=\dv{f}{z}\).

In terms of \(u\) and \(v\), the two derivatives of a function \(f(z)\) are equal to:
\[\pdv{f}{z}=\frac{1}{2}\paren{\pdv{u}{x}+\ii\pdv{v}{x}-\ii\pdv{u}{y}+\pdv{v}{y}},\]
and
\[\pdv{f}{\overline{z}}=\frac{1}{2}\paren{\pdv{u}{x}+\ii\pdv{v}{x}+\ii\pdv{u}{y}-\pdv{v}{y}}.\] If \(f\) is holomorphic,
\begin{equation}\label{eq:holomorphicderivativedecomposition}
    \dv{f}{z}=\pdv{u}{x}+\ii\pdv{v}{x}=\pdv{v}{y}+\ii\pdv{v}{x}=\pdv{u}{x}-\ii\pdv{u}{y}=\pdv{v}{y}-\ii\pdv{u}{y}.
\end{equation}
On the contrary, by the rules of the derivative,
\begin{equation*}
    \pdv{x}=\pdv{z}\pdv{z}{x}+\pdv{}{\overline{z}}\pdv{\overline{z}}{x}=\pdv{z}+\pdv{\overline{z}}
\end{equation*} and
\begin{equation*}
    \pdv{}{y}=\pdv{}{z}\pdv{z}{y}+\pdv{}{\overline{z}}\pdv{\overline{z}}{y}=\ii\pdv{}{z}-\ii\pdv{}{\overline{z}}.
\end{equation*}
The Laplacian is equal to
\begin{align}
    \Delta=\pdv[2]{}{x}+\pdv[2]{}{y} & =\paren{\pdv{z}+\pdv{}{\overline{z}}}^2+\paren{\ii\pdv{z}-\ii\pdv{}{\overline{z}}}^2                                               \nonumber \\
    & =\pdv[2]{}{z}+\pdv[2]{}{\overline{z}}+2\pdv[2]{}{z}{\overline{z}}-\pdv[2]{}{z}-\pdv[2]{}{\overline{z}}+2\pdv[2]{}{z}{\overline{z}}\nonumber \\
    & =4\pdv[2]{}{z}{\overline{z}}.\label{eq:laplaciancomplexform}
\end{align}
Under this definition, we can derive the chain rule:
\begin{theorem}[Chain Rule]\label{thm:wirtingerchainrule}
    Let \(\Omega\subseteq\mathbb{C}\) is a region such that \(g\in C^1(\Omega)\) and \(f\in C^1(g(\Omega))\). It follows that
    \begin{align*}
        \pdv{z}(f\circ g)&=\qty(\pdv{f}{z}\circ g)\pdv{g}{z}+\qty(\pdv{f}{\overline{z}}\circ g)\pdv{\overline{g}}{z}\\
        \pdv{\overline{z}}(f\circ g)&=\qty(\pdv{f}{z}\circ g)\pdv{g}{\overline{z}}+\qty(\pdv{f}{\overline{z}}\circ g)\pdv{\overline{g}}{\overline{z}}.
    \end{align*}
\end{theorem}
\begin{proof}
    Write \(z=x+\ii y\). Let
    \[g(z)=\xi(x,y)+\ii\eta(x,y),\qquad\zeta=\xi+\ii\eta\] so that \(\zeta=g(z)\) with \(\xi=\xi(x,y),\eta=\eta(x,y)\). Let \(f\) be regarded as a \(C^1\) function of the real variables \(\xi,\eta\); equivalently we may view \(f\) as \(f\qty(\zeta,\overline{\zeta})\) where \(\overline{\zeta}=\xi-\ii\eta\). The composition is \(h(z)=f\circ g(z)=f\qty(\xi(x,y),\eta(x,y))\).

    Using the real chain rule (provided by the continuous differentiability), we have
    \[\pdv{h}{x}=\pdv{f}{\xi}\pdv{\xi}{x}+\pdv{f}{\eta}\pdv{\eta}{x},\qquad\pdv{h}{y}=\pdv{f}{\xi}\pdv{\xi}{y}+\pdv{f}{\eta}\pdv{\eta}{y}.\]
    Hence, \[\pdv{h}{z}=\frac{1}{2}\qty[\pdv{f}{\xi}\qty(\pdv{\xi}{x}-\ii\pdv{\xi}{y})+\pdv{f}{\eta}\qty(\pdv{\eta}{x}-\ii\pdv{\eta}{y})].\]
    Now recall
    \[\pdv{f}{\zeta}=\frac{1}{2}\qty(\pdv{f}{\xi}-\ii\pdv{f}{\eta}),\qquad\pdv{f}{\overline\zeta}=\frac{1}{2}\qty(\pdv{f}{\xi}+\ii\pdv{f}{\eta}).\]
    Thus,
    \[\pdv{f}{\xi}=\pdv{f}{\zeta}+\pdv{f}{\overline\zeta},\qquad\pdv{f}{\eta}=\ii\qty(\pdv{f}{\zeta}-\pdv{f}{\overline\zeta}).\]
    Then by substitution,
    \begin{align*}
        \pdv{h}{z}&=\frac{1}{2}\qty[\qty(\pdv{f}{\zeta}+\pdv{f}{\overline\zeta})\qty(\pdv{\xi}{x}-\ii\pdv{\xi}{y})+\ii\qty(\pdv{f}{\zeta}-\pdv{f}{\overline\zeta})\qty(\pdv{\eta}{x}-\ii\pdv{\eta}{y})]\\
        &=\pdv{f}{\zeta}\frac{1}{2}\qty[\qty(\pdv{\xi}{x}-\ii\pdv{\xi}{y})+\ii\qty(\pdv{\eta}{x}-\ii\pdv{\eta}{y})]+\pdv{f}{\overline\zeta}\frac{1}{2}\qty[\qty(\pdv{\xi}{x}-\ii\pdv{\xi}{y})-\ii\qty(\pdv{\eta}{x}-\ii\pdv{\eta}{y})].
    \end{align*}
    The terms in brackets equal \(\pdv{g}{z}\) and \(\pdv{\overline{g}}{z}\). Thus,
    \[\pdv{h}{z}=\qty(\pdv{f}{\zeta}\circ g)\pdv{g}{z}+\qty(\pdv{f}{\overline\zeta}\circ g)\pdv{\overline g}{z}.\]
    Renaming the variables yields
    \[\pdv{z}\qty(f\circ g)=\qty(\pdv{f}{z}\circ g)\pdv{g}{z}+\qty(\pdv{f}{\overline z}\circ g)\pdv{\overline{g}}{z}.\]
    A similar calculation using \cref{eq:wirtingerderivative2} gives
    \[\pdv{\overline{z}}(f\circ g)=\qty(\pdv{f}{z}\circ g)\pdv{g}{\overline{z}}+\qty(\pdv{f}{\overline z}\circ g)\pdv{\overline{g}}{\overline{z}}.\]
    These are exactly the proclaimed identities.
\end{proof}
\subsection{Elementary Functions}
Univariate functions formed by compositions, sums, products, and powers of finitely many functions of the following form are known as \textscsl{elementary functions}:
\begin{enumerate}
    \item Power functions including polynomials, rational functions, and their inverses.
    \item Trigonometric functions, hyperbolic functions, and their inverses
    \item Exponential functions and their inverses.
\end{enumerate}
Power functions are easily extendable to the complex plane by simply changing the real variable to a complex variable. The other two functions have to be redefined and reinterpreted for the complex plane. It is well known that the exponential function can be expanded as
\begin{align*}
    \ee^x & =\frac{x^0}{0!}+\frac{x^1}{1!}+\frac{x^2}{2!}+\frac{x^3}{3!}\cdots                            \\
    & =\frac{x^0}{0!\ii^0}+\ii\frac{x^1}{1!\ii^1}-\frac{x^2}{2!\ii^2}-\ii\frac{x^3}{3!\ii^3}+\cdots \\
    & =\cos(\frac{x}{\ii})+\ii\sin(\frac{x}{\ii}).
\end{align*}
This is better written as
\begin{equation}
    \ee^{\ii x}=\cos(x)+\ii\sin(x),\label{eq:eulersformula}
\end{equation} which is the famous \textscsl{Euler Formula}. Then for any complex number \(z=x+\ii y\), \(\ee^z=\ee^{x+\ii y}=\ee^x\paren{\cos(y)+\ii\sin(y)}\). Then trigonometric functions and exponential functions can be written in terms of each other:
\begin{align*}
    \sin(z)  & =\frac{\ee^{\ii z}-\ee^{-\ii z}}{2\ii}, & \cos(z)  & =\frac{\ee^{\ii z}+\ee^{-\ii z}}{2}, & \tan(z)  & =\frac{\ee^{\ii z}-\ee^{-\ii z}}{\ii\paren{\ee^{\ii z}+\ee^{-\ii z}}} \\
    \sinh(z) & =\frac{\ee^z-\ee^{-z}}{2},              & \cosh(z) & =\frac{\ee^z+\ee^{-z}}{2},           & \tanh(z) & =\frac{\ee^z-\ee^{-z}}{\ee^z+\ee^{-z}}x.
\end{align*}
Hence, the following relationships are derived:
\[\sin(z)=-\ii\sinh(\ii z),\quad\cos(z)=\cosh(\ii z),\quad\tan(z)=-\ii\tanh(\ii z).\]
The complex logarithm, denoted \(w=\log(z)\), is the solution to \(z=\ee^w\). We can then define the inverse trigonometric and hyperbolic functions.

We can also define the power function for non-integer powers with \(w=z^\alpha=\ee^{\alpha\log(z)}\). Then, power functions can all be written in terms of exponential functions and logarithms. Letting \(x=\piup\) in \cref{eq:eulersformula} yields \(\ee^{\ii\piup}=-1\). Furthermore, we can see that exponentiation with an imaginary number is a rotation:
\begin{theorem}[De Moivre]\label{thm:demoivre}
    \(\forall x\in\mathbb{R}\), \(\forall n\in\mathbb{N}\),
    \[\paren{\cos(x)+\ii\sin(x)}^n=\cos(nx)+\ii\sin(nx).\]
\end{theorem}
Since all elementary functions can be written in terms of exponential functions and complex logarithms, we will first study the exponential function.
\begin{enumerate}
    \item The exponential function \(\ee^z\) never vanishes as \(\abs{\ee^z}=\ee^x>0\).
    \item Since \(\ee^{2\piup\ii}=1\), it is periodic over \(2\piup\ii\).
    \item It is also an entire function with \(\paren{\ee^z}'=\ee^z\).

        Write \(\ee^z=\ee^{x+\ii y}=\ee^x\paren{\cos(y)+\ii\sin(y)}\) where \(x,y\in\mathbb{R}\). Let \(u(x,y)=\Re\paren{\ee^z}=\ee^x\cos(y)\) and \(v(x,y)=\Im\paren{\ee^z}=\ee^x\sin(y)\). The first order derivatives are respectively:
        \[\pdv{u}{x}=\ee^x\cos(y),\quad\pdv{u}{y}=-\ee^x\sin(y),\]\[\pdv{v}{x}=\ee^x\sin(y),\quad\pdv{v}{y}=\ee^x\cos(y),\]
        and indeed, the condition described by \cref{thm:holomorphycondition} is satisfied.
    \item For any two complex numbers \(z_1\) and \(z_2\), \(\ee^{z_1}\ee^{z_2}=\ee^{z_1+z_2}\).

        In fact, most real exponentiation rules are identical to those in the complex number field. Previously we claimed the periodic properties of \(\ee^z\). For \(U\subseteq\mathbb{C}\), a holomorphic function \(f:U\to\mathbb{C}\) is \textscsl{univalent} over \(U\) if it is injective over \(U\). This means that the solutions \(z_1\) and \(z_2\) satisfying \(f\paren{z_1}=f\paren{z_2}\) will also always satisfy \(z_1=z_2\).
    \item The region \(\ee^z\) is univalent over:

        Let \(z_1=x_1+\ii y_1\), \(z_2=x_2+\ii y_2\), \(x_1,y_1,x_2,y_2\in\mathbb{R}\) satisfy \(\ee^{z_1}=\ee^{z_2}\). Then, \[\ee^{x_1}\ee^{\ii y_1}=\ee^{x_2}\ee^{\ii y_2}.\] The moduli are equal, and therefore \(x_1=x_2\), and by the periodic nature of the exponentiation of imaginary numbers, \(y_1-y_2=2\piup k\), where \(k\in\mathbb{Z}\). To satisfy the univalence over a region \(U\), \(y_1-y_2\neq2\piup k\), we can select \(U\) to be any horizontal strip \(2\piup k\leq\Im(z)< 2\piup (k+1)\) (or \(2\piup k<\Im(z)\leq 2\piup (k+1)\)). Similar to the exponential function, any belt region with thickness \(2\piup\) is a region over which \(\log\) is univalent.
\end{enumerate}
Next we examine the complex logarithm.
\begin{enumerate}
    \item From of the periodicity of \(z=\ee^w\), \(\log\) is a multi-valued function (infinite-valued).
    \item Let \(z=r\ee^{\ii\theta}\) and \(w=u+\ii v\), where \(r,\theta,u,v\in\mathbb{R}\). Then,
        \[r\ee^{\ii\theta}=\ee^{u+\ii v},\]
        and \(\ee^u=r\), meaning that \(u=\log(r)\), \(v=\theta+2\piup k\), where \(k\in\mathbb{Z}\). Then,
        \[w=\log(r)+\ii(\theta+2\piup k),\]
        and using the modulus-argument notation,
        \[\log(z)=\log\abs{z}+\ii\arg(z),\]
        where \(\arg(z)\) is the multi-valued argument function. We denote the \textscsl{principal branch} of the argument function by \(\Arg:\mathbb{C}\setminus\qty{0}\to(-\piup,\piup]\). The principal branch of \(\log(z)\), or \(\Log(z)\), can be defined such that \(\Im\paren{\Log}\in(-\piup,\piup]\).
\end{enumerate}
The functions \(\sin\) and \(\cos\), through their exponential form, still satisfy the properties such as their derivatives, periodicity being \(2\piup\), parity, sum and difference, and the fundamental identities \(\sin^2(z)+\cos^2(z)=1\), \(\sin(z)=\cos(\frac{\piup}{2}-z)\). However, due to the extension, some properties do not hold. For instance, \(\sin(z)\) and \(\cos(z)\) are unbounded, as along the imaginary axis, they resemble their hyperbolic counterparts, which are unbounded along the real line.

We now examine the regions over which they are univalent. Consider \(\cos(z)=\frac{\ee^{\ii z}+\ee^{-\ii z}}{2}\). Define the auxiliary functions \[\xi(z)=\ii z,\quad\zeta(\xi)=\ee^\xi,\quad w(\zeta)=\frac{\zeta+\frac{1}{\zeta}}{2}.\] Then, \(\cos(z)=(w\circ\zeta\circ\xi)(z)\).

\(\xi\) is clearly univalent on \(\mathbb{C}\), as it is a linear map (specifically, a rotation by \(\frac{\piup}{2}\) radians followed by scaling).\ \(\zeta\) is univalent on any domain \(U\subset\mathbb{C}\) such that for all \(\xi_1,\xi_2\in U\), \(\xi_1-\xi_2\neq2\piup\ii k\) for any \(k\in\mathbb{Z}\). That is, \(U\) must not contain any pair of points differing by a nonzero integer multiple of \(2\piup\ii\). If \(\xi_1=\ii z_1\) and \(\xi_2=\ii z_2\), then this translates to \(z_1-z_2\neq2\piup k\) for \(k\in\mathbb{Z}\).\ \(w(\zeta)=\frac{\zeta+\frac{1}{\zeta}}{2}\) is univalent on regions excluding pairs \((\zeta_1,\zeta_2)\) such that \(\zeta_1=\frac{1}{\zeta_2}\). In terms of \(z\), this condition becomes \(\ee^{\ii z_1}\ee^{\ii z_2}\neq1\), or equivalently, \(z_1+z_2\neq2\piup k\) for any \(k\in\mathbb{Z}\).

Combining these constraints, we conclude that \(\cos(z)\) is univalent on any vertical strip in the complex plane of width \(\piup\), such as a region of the form \[\cbraces{z\in\mathbb{C}}{k\piup<\Re(z)<(k+1)\piup,k\in\mathbb{Z}}.\] Let us now consider the specific region \(\cbraces{z\in\mathbb{C}}{0<\Re(z)<\piup}\), and analyze how it is mapped under \(\cos(z)\).
\begin{enumerate}
    \item \(\xi(z)=\ii z\) maps the region \(\cbraces{z\in\mathbb{C}}{0<\Re(z)<\piup}\) to \(\cbraces{\xi\in\mathbb{C}}{0<\Im(\xi)<\piup}\).
    \item \(\zeta(\xi)=\ee^\xi\) maps this region to the upper half plane \(\Im(\zeta)>0\) since \(0<\Arg(\zeta)<\piup\) and \(0<\abs{\zeta}\).
    \item \(w(\zeta)=\frac{\zeta+\frac{1}{\zeta}}{2}\) maps \(\Im\paren{\zeta}>0\) to \(\mathbb{C}\setminus\paren{(-\infty,-1]\cup[1,\infty)}\).
\end{enumerate}
Thus, the composition \(\cos(z)=w\circ\zeta\circ\xi\) is univalent on the strip \[\cbraces{z\in\mathbb{C}}{0<\Re(z)<\piup},\]
and the image of this strip under \(\cos\) is \(\mathbb{C}\setminus\paren{(-\infty,-1]\cup[1,\infty)}\). We will now analyze the inverse cosine function, denoted \(\arccos(z)\).

Consider \(z=\frac{\ee^{\ii w}+\ee^{-\ii w}}{2}\). Then,
\begin{align*}
    \paren{\ee^{\ii w}}^2+1 & =2z\ee^{\ii w}                \\
    \ee^{\ii w}             & =\frac{2z\pm\sqrt{4z^2-4}}{2} \\
    w                       & =-\ii\log(z\pm\sqrt{z^2-1}).
\end{align*}
Then \(\arccos\) is also a multi-valued function. We can also define \(\arcsin(z)=\frac{\piup}{2}-\arccos{z}\).

Lastly, we will examine the power function. Let \(\alpha=u+\ii v\) where \(u,v\in\mathbb{R}\). Then, \[z^\alpha=\exp(\alpha\log(z))=\exp((u+\ii v)\paren{\log\abs{z}+\ii\arg\paren{z}}),\]
and in polar form, \[z^\alpha=\exp(u\log\abs{z}-v\arg\paren{z})\exp(\ii\paren{v\log\abs{z}+u\arg\paren{z}}).\]
Let \(r_k=\exp\paren{u\log\abs{z}-v\arg(z)}\) and \(\theta_k=v\log\abs{z}+u\arg(z)\). Then, \(z^\alpha=r_k\ee^{\ii\theta_k}\), where \(k\in\mathbb{Z}\). Analyzing the coefficient of \(v\) in the exponent of \(r_k\), \(z^\alpha\) is multi-valued if \(v\neq0\).

Then assuming \(v=0\), \(z^\alpha=\abs{z}^u\exp(\ii u\arg(z))\). Doing casework on \(\alpha\),
\begin{enumerate}
    \item If \(\alpha=u\in\mathbb{Z}\), then \(u\) can be absorbed into \(k\), and \(z^\alpha\) is single valued.
    \item If \(\alpha=u\in\mathbb{Q}\) with reduced fractional form \(\frac{p}{q}\), where \(p,q\in\mathbb{Z}\), \(q>0\), and \(\gcd(p,q)=1\), then the multivalued function \(z^\alpha\) is given by
        \[z^\alpha=\abs{z}^{\frac{p}{q}}\exp\qty(\ii\frac{p}{q}(\Arg(z)+2\piup k))=\abs{z}^{\frac{p}{q}}\exp(\ii\frac{p}{q}\Arg(z))\exp(2\ii\frac{p}{q}\piup k),\]
        for \(k\in\mathbb{Z}\). These values are periodic with period \(q\), since
        \[\exp(2\ii\frac{p}{q}\piup(k+q))=\exp(2\ii\frac{p}{q}\piup k)\exp(2\ii\piup p)=\exp(2\ii\frac{p}{q}\piup k),\]
        as \(\exp(2\piup\ii p)=1\) for integer \(p\). To prove there are exactly \(q\) distinct values, consider \(k=0,1,\dots,q-1\). The exponential factors are \(\exp(2\ii\frac{p}{q}\piup k)\). These are distinct if, for \(0\leq j<k\leq q-1\),
        \[\exp(2\ii\frac{p}{q}\piup j)\neq\exp(2\ii\frac{p}{q}\piup k),\]
        which holds unless \(\frac{p}{q}(k-j)\in\mathbb{Z}\), or if \(q\) divides \(p(k-j)\). Since \(\gcd(p,q)=1\), \(q\) must divide \(k-j\), but \(\abs{k-j}<q\) and \(k-j\neq 0\), a contradiction. Thus, \(z^\alpha\) has exactly \(q\) distinct values.
    \item If \(\alpha=u\in\mathbb{R}\setminus\mathbb{Q}\), then \(z^\alpha\) is infinite valued.
\end{enumerate}
Lastly, there exist series representations of functions using power functions (Taylor series) and trigonometric functions (Fourier series). There does not exist another representation using exponential functions as trigonometric functions can be written in terms of them.
\subsection{Complex Power Series}
Power series in real analysis can be generalized into complex series. Particularly, concepts such as uniform convergence are the same in complex analysis:
\begin{definition}[Uniform Convergence]
    For a set \(U\subseteq\mathbb{C}\), a function sequence \(\cbraces{f_n(z)}\) \textscsl{uniformly converges} to a function \(f(z)\) on \(U\) iff \(\forall\varepsilon>0\), \(\exists N\in\mathbb{N}\) such that \(\forall n>N\), \(\forall z\in U\), \(\abs{f_n(z)-f(z)}<\varepsilon\).
\end{definition}
\begin{remark}
    The definition above is equivalent to the following definition.

    For a set \(U\subseteq\mathbb{C}\), a function sequence \(\cbraces{f_n(z)}\) uniformly converges to \(f(z)\) iff
    \[\lim_{n\to\infty}\sup_{z\in U}\abs{f_n(z)-f(z)}=0.\]
    (Informally, we will use the notation \(f_n(z)\rightrightarrows f(z)\) to represent uniform convergence.)
\end{remark}
\begin{theorem}[Cauchy Criterion]\label{thm:cauchycriterionuniformconvergence}
    For a set \(U\subseteq\mathbb{C}\), a function sequence \(\cbraces{f_n(z)}\) uniformly converges to a function \(f(z)\) iff \(\forall\varepsilon>0\),\(\exists N\in\mathbb{N}\) such that \(\forall n,m>N\), \(\forall z\in U\), \(\abs{f_n(z)-f_m(z)}<\varepsilon\).
\end{theorem}
\begin{proof}
    \(\forall\varepsilon>0\), \(\exists N\in\mathbb{N}\) such that \(\forall n,m>N\), \(\forall z\in U\), \[\abs{f_m(z)-f(z)}<\frac{\varepsilon}{2},\quad\abs{f_n(z)-f(z)}<\frac{\varepsilon}{2}.\]
    Then,
    \[\abs{f_m(z)-f_n(z)}\leq\abs{f_n(z)-f(z)}+\abs{f_m(z)-f(z)}<\varepsilon.\]

    For the converse, refer to the analogous proof in \cref{thm:cauchycriterionsequenceconvergence}.
\end{proof}
Function series are defined to be a sequence formed by the partial sums of function sequences. There are many ways to verify the uniform convergence of a function series. Perhaps the most widely known is the Weierstrass \(M\)--Test.
\begin{theorem}[Weierstrass \(M\)--Test]\label{thm:weierstrassmtest}
    Let \(U\subseteq\mathbb{C}\) be a region and \(\cbraces{f_n}\) be a function sequence on \(U\)

    If \(\exists\qty{M_n}\subset\mathbb{R}_{\geq 0}\) such that \(\forall n\in\mathbb{N}\), \(\forall z\in U\), \(\qty|f_n(z)|\le M_n\) and the series \(\sum_{n=1}^\infty M_n\) converges, then the series \(\sum_{n=1}^\infty f_n(z)\) converges uniformly and absolutely on \(U\).
\end{theorem}
\begin{proof}
    By the convergence of \(\sum_{n=1}^\infty M_n\), \(\forall\varepsilon>0\), \(\exists N\in\mathbb{N}\) such that \(\forall m\geq n>N\), \[\abs{M_m+M_{m-1}+\cdots+M_{n+1}}<\varepsilon.\] Since \(M_j\) bounds \(f_j(z)\), it follows that \[\abs{f_m(z)+f_{m-1}(z)+\cdots+f_{n+1}(z)}\leq\abs{M_m+M_{m-1}+\cdots+M_{n+1}}<\varepsilon,\]
    and the result follows from \cref{thm:cauchycriterionuniformconvergence}.
\end{proof}
The concept of uniform convergence is generalized to improper integrals with parameters, and the same theorems from series have a corresponding counterpart.

In both complex and real analysis, the concept of \textscsl{power series}, a unique type of function series, is of trivial importance. Similar to real power series, complex series have the form \[\sum_{n=0}^\infty a_n z^n,\] where \(\cbraces{a_n}\) are constants.

Let \(D(a,r)=B^1(a,r)=\cbraces{z\in\mathbb{C}}{\abs{z-a}<r}\) denote the \textscsl{open disk} centered at \(a\) with radius \(r\). For simplicity, from now on we will have \(\mathbb{D}\) denote the unit open disk, or \(D(0,1)\). We will now observe the convergence of power series.
\begin{theorem}[Abel's Theorem]\label{thm:abelradius}
    For a power series \(f(z)=\sum_{n=0}^\infty a_nz^n\), there exists a constant \(R\in\mathbb{R}_{\geq0}\cup\cbraces{\infty}\), known as the \textscsl{radius of convergence} such that:
    \begin{enumerate}
        \item \(f\) absolutely converges on \(D(0,R)\), and \(\forall 0\leq\rho<R\), uniformly converges on \(\overline{D(0,\rho)}\).\label{itm:abelradius_absoluteanduniformconvergence}
        \item \(f(z)\) diverges when \(\abs{z}>R\).\label{itm:abelradius_divergence}
        \item \(f\) is holomorphic over \(D(0,R)\) and \(f'(z)\) can be obtained by termwise differentiation, or \(f'(z)=\sum_{n=1}^\infty na_n z^{n-1}\), which also has the radius of convergence \(R\).\label{itm:abelradius_differentiation}
    \end{enumerate}
\end{theorem}
The disk \(\abs{z}<R\) is known as the \textscsl{disk of convergence}, a direct generalization of the \textscsl{interval of convergence} for real series. There are many ways to determine the radius of convergence:
\begin{theorem}[\textsc{Cauchy--Hadamard}]\label{thm:cauchyhadamard}
    The radius of convergence of the power series in the form of \(\sum_{n=0}^\infty a_n z^n\) can be determined by
    \begin{equation}
        R=\frac{1}{\varlimsup_{n\to\infty}\sqrt[n]{\abs{a_n}}}. \label{eq:cauchyhadamard}
    \end{equation}
\end{theorem}
Of course, a convergence radius of \(0\) implies that the series is divergent everywhere except for possibly at \(0\), and a convergence radius of \(\infty\) means that the series absolutely converges everywhere.
\begin{proof}[Proof of \cref{thm:abelradius}]
    We will prove that the value in \cref{eq:cauchyhadamard} satisfies the criteria in \cref{thm:abelradius}.

    Assume \(\abs{z}<R\). Then, \(\forall\rho\in(\abs{z},R)\), and consequently, \(\frac{1}{\rho}>\frac{1}{R}\). By \cref{def:limsup,eq:cauchyhadamard}, \(\exists N\in\mathbb{N}\) such that \(\forall n>N\), \(\sqrt[n]{\abs{a_n}}<\frac{1}{\rho}\) and \(\abs{a_n}<\frac{1}{\rho^n}\). It follows that \(\abs{a_n z^n}<\frac{\abs{z}^n}{\rho^n}<1\) for all \(n>N\). Then, \(\sum_{n=0}^\infty\qty|a_nz^n|\) converges.

    Let \(\rho'\in(\rho, R)\). Similarly, \(\exists N'\in\mathbb{N}\) such that \(\forall n>N'\), \(\sqrt[n]{\abs{a_n}}<\frac{1}{\rho'}\), and \(\abs{a_n}<\frac{1}{\rho'^n}\). Then \(\qty|a_n z^n|<\qty|a_n\rho^n|<\frac{\rho^n}{\rho'^n}<1\). By the Weierstrass \(M\)--Test (\cref{thm:weierstrassmtest}), the \(\sum_{n=0}^\infty\abs{a_n z^n}\) is uniformly bounded (for \(n>N'\)) by the convergent series \(\sum_{n=0}^\infty a_n\rho^n\), and thus uniformly converges on \(\abs{z}<\rho\). This proves~\ref{itm:abelradius_absoluteanduniformconvergence}.

    Assume that \(\abs{z}>R\).\ \(\forall\rho\in\qty(R,\abs{z})\), \(\frac{1}{\rho}<\frac{1}{R}\). By \cref{def:limsup}, \(\forall N\in\mathbb{N}\), \(\exists n_N>N\) such that \(\sqrt[n_N]{\abs{a_{n_N}}}>\frac{1}{\rho}\). It follows that \(\abs{a_{n_N}z^{n_N}}>\frac{\abs{z^{n_N}}}{\rho^{n_N}}>1\). Since \(\forall N\in\mathbb{N}\), \(\abs{\sum_{k=0}^{n_N}a_k z^k-\sum_{k=0}^{n_N-1}a_k z^k}>1\), by the Cauchy Criterion (\cref{thm:cauchycriterionsequenceconvergence}), \(\sum_{n=0}^\infty a_n z^n\) is divergent. Thus,~\ref{itm:abelradius_divergence} is satisfied.

    To prove~\ref{itm:abelradius_differentiation}, first observe that \(\sum_{n=1}^\infty na_n z^n\) and \(\sum_{n=1}^\infty a_n z^n\) have the same convergence radius since \(\varlimsup_{n\to\infty}\sqrt[n]{n}=1\). For \(z\in D(0, R)\), let \(f(z)=S_n(z)+R_n(z)\), where \[S_n(z)=\sum_{k=0}^{n-1}a_k z^k,\quad R_n(z)=\sum_{k=n}^\infty a_k z^k.\]
    Let \(f_1(z)=\lim_{n\to\infty}S_n'(z)=\sum_{n=0}^\infty na_n z^{n-1}\). Let \(\rho<R\) be positive and \(\abs{z_0}<\rho\). Then we aim to show that \[\lim_{z\to z_0}\frac{f(z)-f\qty(z_0)}{z-z_0}-f_1\paren{z}=0.\]
    By analyzing the difference,
    \begin{align}
        \frac{f(z)-f\qty(z_0)}{z-z_0}-f_1\paren{z} & =\qty[\frac{S_n(z)-S_n\qty(z_0)}{z-z_0}-S'_n(z)]\nonumber                                                \\
        & \quad+S'_n(z)-f_1(z)+\frac{R_n(z)-R_n\qty(z_0)}{z-z_0}.\label{eq:abelradius_differentiationintermediate}
    \end{align}
    Since \(S'_n(z)\to f_1(z)\) as \(n\to\infty\), it follows that \(\forall\varepsilon>0\), \(\exists N\in\mathbb{N}\) such that \(\forall n>N\), \(\abs{S'_n(z)-f_1(z)}<\frac{\varepsilon}{3}\). Since \[\frac{R_n(z)-R_n\qty(z_0)}{z-z_0}=\sum_{k=n}^\infty a_k\frac{z^k-z_0^k}{z-z_0}=\sum_{k=n}^\infty a_k\qty(z^{k-1}+z^{k-2}z+\cdots+z_0^{k-1})\]
    for \(z\neq z_0\), \(\abs{z}<\rho<R\), \[\qty|\sum_{k=n}^\infty a_k\qty(z^{k-1}+\cdots+z_0^{k-1})|\leq\sum_{k=n}^\infty\abs{a_k}\qty(\abs{z^{k-1}}+\cdots+\abs{z_0^{k-1}})<\sum_{k=n}^\infty\abs{a_k}k\rho^{k-1}.\] Since \(\sum_{k=1}^\infty ka_k\rho^{k-1}\) is absolutely convergent, \(\sum_{k=n}^\infty\abs{a_k}k\rho^{k-1}\) is the remainder term of a convergent series. Then, \(\exists N'\in\mathbb{N}\) such that \(\forall n>N\), \(\qty|\sum_{k=n}^\infty\abs{a_k}k\rho^{k-1}|<\frac{\varepsilon}{3}\).

    Finally, for a fixed \(n>\max\qty{N,N'}\), \(\exists\delta>0\) such that \(\forall z\in D\qty(z_0,\delta)\setminus\qty{z_0}\), \[\qty|\frac{S_n(z)-S_n\qty(z_0)}{z-z_0}-S'_n(z)|<\frac{\varepsilon}{3}.\]
    From \cref{eq:abelradius_differentiationintermediate}, we get:
    \[\qty|\frac{f(z)-f\qty(z_0)}{z-z_0}-f_1\paren{z}|<\varepsilon,\] confirming~\ref{itm:abelradius_differentiation}.
\end{proof}
Obviously, a substitution of \(z=\zeta-a\) where \(a\in\mathbb{C}\) translates the disk of convergence to \(D(a,R)\).

The subsequent results on uniform convergence hold for complex functions:
\begin{theorem}[Uniform Limit]\label{thm:uniformlimit}
    Let \(\cbraces{f_n(z)}\) be continuous on \(U\subseteq\mathbb{C}\) and uniformly convergent to \(f(z)\). Then \(f(z)\) is continuous on \(U\).
\end{theorem}
\begin{proof}
    By continuity, \(\forall n\in\mathbb{N}\), \(\forall z_0\in U\), \(\forall\varepsilon>0\), \(\exists\delta>0\) such that \(\forall z\in D\paren{z_0,\delta}\subseteq U\), \(\abs{f_n(z)-f_n\paren{z_0}}<\frac{\varepsilon}{3}\). Additionally, \(\exists N\in\mathbb{N}\) such that \(\forall n>N\), \(\forall z\in U\) (\(n\) is independent of \(z\)), \(\abs{f_n(z)-f(z)}<\frac{\varepsilon}{3}\). It follows that \(\abs{f_n\paren{z_0}-f\paren{z_0}}<\frac{\varepsilon}{3}\). By the triangle inequality,
    \begin{align*}
        \abs{f(z)-f\paren{z_0}} & \leq\abs{f(z)-f_n(z)}+\abs{f_n(z)-f_n\paren{z_0}}+\abs{f_n\paren{z_0}-f\paren{z_0}}                                   \\
        & <\frac{\varepsilon}{3}+\frac{\varepsilon}{3}+\frac{\varepsilon}{3}=\varepsilon,\quad\forall z\in D\paren{z_0,\delta}.
    \end{align*}
    Then the continuity of \(f\) is satisfied.
\end{proof}
Lastly, the sufficient criteria to pass a limit through an integral:
\begin{theorem}\label{thm:limitintegralswitch}
    Let \(\gamma\) be a rectifiable curve on which the function sequence \(\cbraces{f_n}_{n\in\mathbb{N}}\) is continuous on. If \(\cbraces{f_n(z)}\) uniformly converges to \(f\), then \[\lim_{n\to\infty}\int_{\gamma}f_n(z)\ddz=\int_{\gamma}f(z)\ddz.\]
\end{theorem}
\begin{proof}
    Since \(\cbraces{f_n(z)}\) uniformly converges to \(f(z)\) on \(\gamma\), \(\forall\varepsilon>0\), there exists \(N\in\mathbb{N}\) such that for all \(n