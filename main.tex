\documentclass{article}
\usepackage[utf8]{inputenc}
\usepackage{parskip}
\usepackage{tikz}
\usepackage[most]{tcolorbox}
\usepackage{amsmath}
\usepackage{amsfonts}
\usepackage{esint}
\usepackage{graphicx}
\allowdisplaybreaks[2]
\usepackage[left=3.5cm,right=3.5cm]{geometry}

\usepackage{pgfplots}
\usepgfplotslibrary{fillbetween}
\usetikzlibrary{patterns}

\pgfplotsset{compat=1.18}
\pgfplotsset{
    every axis/.append style={
        axis on top=true,
        axis x line=middle,
        axis y line=middle,
        axis equal,
        axis line style={<->,color=black}, 
        xlabel={$x$},
        ylabel={$y$},
    }
}
\definecolor{pinkplester}{rgb}{0.733, 0.02, 0.502}
\newcommand\getcurrentref[1]{%
 \ifnumequal{\value{#1}}{0}
  {??}
  {\the\value{#1}}%
}
\newtcbtheorem[number within = section]{Theorem}{Theorem}{
    opacityback=1,
    colback=blue!7!white,
    colframe=pinkplester,
    fonttitle=\bfseries
}{thm}


\newtcbtheorem[no counter]{Solution}{Solution}{
    colback=white,
    colframe=blue!25,
    fonttitle=\bfseries
}{prf}

\title{Complex Analysis}
\author{Slipper King}
\date{July 2024}

\begin{document}

\maketitle

\section{Complex Numbers}
The value of $\sqrt{-1}$ is denoted as $i$ and is the fundamental unit of complex numbers. It can be combined with elements from $\mathbb R$ to form $\mathbb C$. Complex numbers have the form $a+bi$ where $a,b\in \mathbb R$. Complex numbers $a+bi$ and $c+di$ have the same fundamental operations:
\begin{align*}
    (a+bi)+(c+di)=(a+c)+(b+d)i&(a+bi)(c+di)=(ac-bd)+(ad+bc)i
\end{align*}
\[\left.\frac{a+bi}{c+di}=\frac{(a+bi)(c-di)}{(c+di)(c-di)}=\frac{(ac+bd)+(bc-ad)i}{c^2+d^2}\right|c,d\neq 0\]
The operation described in the division above uses the operation of multiplying the fraction by its conjugate on both sides; the conjugate of a complex number $z$ is denoted as $\overline{z}$ and is the complex number with the property that $\Re(z)=\Re\left(\bar{z}\right)$ and $\Im(z)=-\Im\left(\overline{z}\right)$ where $\Re$ and $\Im$ denote the real and imaginary parts of the complex number. The following identities hold for complex $a$ and $b$:
\begin{align*}
    \overline{\bar{a}}=a&&\Re(a)=\frac{a+\bar a}{2}&&\Im(a)=\frac{a-\bar a}{2i}\\
    \overline{a+b}=\bar a+\bar b&&\overline{ab}=\bar a \cdot \bar b && \overline{\left(\frac{a}{b}\right)}=\frac{\bar a}{\bar b}
\end{align*}
Imaginary units and real units do not combine and can be described as orthogonal to each other in the complex plane, where the $y$-axis denotes the complex part and the $x$-axis denotes the real part. The conjugate is equal to the reflection of a point across the real axis, and the absolute value of a complex number is the real number describing the distance to the origin.
We can derive easily that:

\end{document}
